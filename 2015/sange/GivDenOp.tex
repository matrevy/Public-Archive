\documentclass[a4paper,11pt]{article}

\usepackage{revy}
\usepackage[utf8]{inputenc}
\usepackage[T1]{fontenc}
\usepackage[danish]{babel}


\revyname{MatematikRevy}
\revyyear{2015}
\version{1.0}
\eta{3:15 minutter}
\status{Færdig}

\title{Giv Den Op}
\author{Lotte}
\melody{Cisilia - Vi To Datid Nu}

\begin{document}
\maketitle

\begin{roles}
\role{X}[Jakob] Instruktør
\role{Y}[Jasmin] Koreograf
\role{S}[Sofie] Sanger
\role{D1}[Sanne] Danser
\role{D2}[Josefine] Danser
\role{D3}[Lise] Danser
\role{D4}[Anna Munk] Danser
\role{K1}[Christine] Kor 
\role{K2}[Rikke] Kor
\end{roles}

\begin{song}


\sings{S} Uh-u-uh
(Hey Hey)
Ok
(Hey)
Uh-u-uh
( Hey hey)


\sings{S} Hey
Der er slet ingen som dig
Du' altid til at regne med og gør det let som en leg
Du kan unikke ting, ja
Og det gør dig faktisk ret speciel 
Jeg syns du' fantastisk 
Selvom du' irrationel 


\sings{S} I tusind år har man prøvet at nærme sig
Ved sammenligning forsøgt at bestemme dig
Men du er som bekendt
Unik og transcendent
Der' ingen repetition i din decimal
Alle forsøg på at find' det har vær't fatal'
For du er som bekendt
Unik og transcendent


\sings{S} Det kan godt bli' lidt monoton'
At du bedst ka' beskrives som re-la-tion
Et forhold uden li-i-ig'
Og stadig meget at si-i-ig'
Selv givet en definition 
Kan vi kun tale om approximation
Så det' på tid' at vi-i-i
Gi'r den op for pi-i-i
(Ja, giv den op for pi)


\sings{S} De, forsøgte med kvadrat
Det var bare ikke i Egypten det tog fart
Nej det var Arkimedes
Der fandt noget int'ressant
Han kunne komme tæt på
Ved hjælp af en mange-kant


\sings{S} I tusind år har man prøvet at nærme sig
Ved sammenligning forsøgt at bestemme dig
Men du er som bekendt
Unik og transcendent


\sings{S} Ka' godt bli' lidt monoton'
At det bedst ka' beskrives som en re-la-tion
Et forhold uden li-i-ig'
Og stadig meget at si-i-ig'
Selv givet en definition 
Kan vi kun tale om approximation
Så det' på tid' at vi-i-i
Gi'r den op for pi-i-i

Ja, ja, ja, ja, ja


\sings{S} Så kom Le’nard Euler
Han var en rigtig gøgler
Han elsked trigonometri, så han fandt på nøgler
Han kom med e i pi’te
Plus én og kunne li’ det 
For det gi’r nul og det var sejt at det kunne gi’ det
for pi det er det bedste
nummer på min liste   
måske at det var heldigt
om end vældigt tilfældigt
men uanset hvad
så vi da bar lig’glad
fordi i cirkler Archimedes sad


\sings{S} Ka' godt bli' lidt monoton'
At det bedst beskrives som en re-la-tion
Et forhold uden li-i-ig'
Og stadig meg't at si-i-ig'
Selv givet en definition 
Kan vi kun tale om approximation
Så det' på tid' at vi-i-i
Gi'r den op for pi-i-i


\sings{S} Giv den op for pi-i-i
Vi gør det, vi gør det 
Vi gør det
Man kan ikk' sammenlign' 
Jeg ville ønske du var min
(Så giv den op for pi)
Ih-ihh
Vi gør det, vi gør det 
Vi gør det
Kan ikk' sammenlign'
Jeg ville ønske du var min
Åh pi-i-i





\end{song}
\end{document}