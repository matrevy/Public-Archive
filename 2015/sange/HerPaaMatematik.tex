
\documentclass[a4paper,11pt]{article}

\usepackage{revy}
\usepackage[utf8]{inputenc}
\usepackage[T1]{fontenc}
\usepackage[danish]{babel}


\revyname{MatematikRevy}
\revyyear{2015}
\version{1.0}
\eta{2 minutter}
\status{Færdig}

\title{Her På Mat'matik}
\author{Jasmin}
\melody{Fætter Mikkel}

\begin{document}
\maketitle

\begin{roles}
\role{X}[Jakob] Instruktør
\role{S1}[Anna Munk] Sanger
\role{S2}[Marianne] Sanger
\role{S3}[Ljørring] Sanger
\role{U}[Stig] Ukulele
\end{roles}

\begin{song}


\sings{S} Her på mat’matik
Skal man være kvik
Sætte pris på elegance
Ha’ intuition
Også ambition
Gribe fat i hver en chance

\sings{S} For vi elsker ren logik (klap-klap)
algebra og statistik (klap-klap)
Alt der er komplekst
Analyse, vækst
det er her vi har det bedst (klap-klap)

\sings{S} Fra det første år
Kurserne består
Fællesskabet ej forglemme
Der bli’r undervist
Masser kage spist
Det er her hvor vi har hjemme

\sings{S} For vi elsker ren logik (klap-klap)
algebra og statistik (klap-klap)
Alt der er komplekst
Analyse, vækst
det er her vi har det bedst (klap-klap)





\end{song}
\end{document}










På “Fætter Mikkel”

Her på mat’matik
Skal man være kvik
Sætte pris på elegance
Ha’ intuition
Også ambition
Gribe fat i hver en chance

For vi elsker ren logik (klap-klap)
algebra og statistik (klap-klap)
Alt der er komplekst
Analyse, vækst
det er her vi har det bedst (klap-klap)

Fra det første år
Kurserne består
Fællesskabet ej forglemme
Der bli’r undervist
Masser kage spist
Det er her hvor vi har hjemme

For vi elsker ren logik (klap-klap)
algebra og statistik (klap-klap)
Alt der er komplekst
Analyse, vækst
det er her vi har det bedst (klap-klap)
