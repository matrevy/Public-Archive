\documentclass[a4paper,11pt]{article}

\usepackage{revy}
\usepackage[utf8]{inputenc}
\usepackage[T1]{fontenc}
\usepackage[danish]{babel}

\revyname{MatematikRevy}
\revyyear{2015}
\version{1.0}
\eta{$4.0$ minutter}
\status{Udkast færdigt}

\title{Når jeg' væk}
\author{Matematikrevyens manusgruppe}
\melody{A. P. Carter and Luisa Gerstein $-$ Cups (den fra Pitch Perfect)}

\begin{document}
\maketitle

\begin{roles}
\role{X}[William] Instruktør
\role{S}[Lotte] Forsanger
\role{D1}[Lutzen] Danser
\role{D2}[Josefine] Danser
\role{D3}[Bettina] Danser
\role{D4}[Sanne] Danser
\role{D5}[Patrick] Danser
\role{D6}[MaltheRus] Danser
\role{D7}[Rikke] Danser
\role{D8}[Ljørring] Danser
\end{roles}

\begin{props}
\prop{Kommer senere...}[Finder]
\end{props}

\begin{song}
\sings{S}
Nu har jeg afleveret, nu er det slut
Og voksen-livet venter her
Men spørgsmålet er, om I vil savne mig
og om jeg kunne have ydet mer’

\sings{S}
Når jeg’ væk, når jeg’ væk
Hvad skal I gøre, når jeg’ væk?
Ja, hvem skal nu vær’ den sjove,
den kloge og den grove?
Hvad skal I gøre, når jeg’ væk?

\sings{S}
Når jeg’ væk, når jeg’ væk
Hvad skal I gøre, når jeg’ væk?
Hvem skal nu arrangere fester,
eller være tragtens mester?
Hvad skal I gøre, når jeg’ væk?

(evt. pause)

\sings{S}
Nu bli’r det nye folk, der skal slå til
så hvem skal dog underholde her
for ingen er jo ligeså gode - overalt på den klode
Men måske vil jeg også savne jer

\sings{S}
Når jeg’ væk, når jeg’ væk
Hvad skal I gøre, når jeg’ væk?
Hvem skal nu Caféen? vælte,
Eller russers hjerter smelte?
Hvad skal I gøre, når jeg’ væk?

(evt. lang pause med koppelyde) - folk kommer ind

\sings{S}
Alle: Når du/jeg’ væk, når du/jeg’ væk
Forsanger: Hvad skal jeg gøre, når jeg’ væk?
Men jeg kommer nok tilbage
Når der’ fredagssøsterkage
Det skal jeg måske gøre, når jeg’ væk

\sings{S}
Når jeg’ væk, når jeg’ væk
Hvad skal jeg gøre, når jeg’ væk?
Men vi ses jo nok i byen
Og til mat’matikrevyen
For jeg blir aldrig rigtig væk

\end{song}

\end{document}
