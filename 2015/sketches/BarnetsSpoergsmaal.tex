\documentclass[a4paper,11pt]{article}

\usepackage{revy}
\usepackage[utf8]{inputenc}
\usepackage[T1]{fontenc}
\usepackage[danish]{babel}
\usepackage{amssymb}

\revyname{Matematik Revyen}
\revyyear{2015}
% HUSK AT OPDATERE VERSIONSNUMMER
\version{1.0}
\eta{3 minutter}
\status{Færdig}

\title{Barnets Spørgsmål}
\author{Lise, William}

\begin{document}
\maketitle

\begin{roles}
\role{X}[Jakob] Instruktør
\role{B}[Iris] Barn
\role{F}[Stolberg] Forælder
\end{roles}

%\begin{props}
%\end{props}

%\begin{mics}
%\mic{HS1}[] ???
%\end{mics}
  
\begin{sketch}
\scene{Lys op. En seng står på scenen. B ligger i den. P kommer ind.}

\says{F} Undskyld at jeg kommer sent hjem. Jeg ville bare lige sige godnat inden at jeg skal ind i
soveværelset og.... sove. Har du i øvrigt haft en god dag i skolen? Har du lært noget 
spændende og så videre?

\says{B} Ja... Og nej.

\says{F} Nåh. Hvad da?

\says{B} Altså vi har lært hvordan man skærer en lagkage ud i $8$ lige store stykker. Og at $1$ delt $2$ er $0.5$, som er en halv. Og vi har også lært nogle andre brøker. Men der er noget, som jeg ikke forstår. Hvad sker der, hvis man sætter nul ind nederst?

\says{F} Det må man ikke.

\says{B}[Uskyldigt] Hvorfor ikke?

\says{F} Fordi det kan man ikke sige!

\says{B} \act{Tænker lidt og siger uskyldigt} Hvorfor ikke?

\says{F} Øøøøøhh.... Fordi at så skulle der findes et tal, således at $\textnormal{tæller}=0*\textnormal{brøkværdien}$. Og det kan ikke lade sig gøre. \act{Vender sig for at gå ud}

\says{B}[Uskyldigt] Hvorfor ikke?

\says{F} Hmmm... Jamen hvis vi nu for eksempel havde $1/0=x$, så skulle det jo være tilfældet, hvis vi ganger med nul på begge sider af lighedstegnet , så har vi at $1=0*x=0$. og Vi kan jo ikke have at $0$ og $1$ er det samme. \act{Vender sig og går mod døren}

\says{B} Hvorfor ikke?

\says{F} Altså for det første har 1 eksisteret i meget længere tid end nul. Man opdagede nemlig først 0 omkring år 650. 

\says{B}[Uskyldigt] Hvorfor det?

\says{F} Uh... Det kommer jo egentlig også an på hvordan man tænker på nul. Er det en pladsholder, en værdi eller noget man kan lægge til andre tal og gange med osv. \act{Eftertænksomt} Så på en måde er der mange forskellige nul’er, så man må i virkeligheden spørge sig selv: hvad er nul? 

\says{B} Hvad er nul?

\says{F} Jamen, det er jo det additive neutrale elementet i legemet af komplekse tal. \act{Vender sig og går mod scene exit}


\says{B} Hvorfor det?

\says{F} Det er fordi $x+0=x$ for alle $x$ i $\mathbb{C}$. \act{Vender sig og går mod scene exit}


\says{B}[Lige før P når af scenen] Hvorfor det?

\says{F} Hvis du tænker på de komplekse tal som polynomiumsringen af de reelle tal i en variabel, hvor man har taget kvotienten med hensyn til $x^2-1$. Så er det jo fordi ækvivalensklassen af polynomier, der er $0$ sender ethvert $x$ ind i idealet frembragt af ($x^2-1$). 

\says{B} Nåårh, det kunne du jo bare have sagt fra starten af.

\scene{Lys ned.}
\end{sketch}
\end{document}

