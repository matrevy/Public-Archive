\documentclass[a4paper,11pt]{article}

\usepackage{revy}
\usepackage[utf8]{inputenc}
\usepackage[T1]{fontenc}
\usepackage[danish]{babel}
\usepackage{amssymb}

\revyname{Matematik Revyen}
\revyyear{2015}
% HUSK AT OPDATERE VERSIONSNUMMER
\version{1.0}
\eta{2 minutter}
\status{Færdig}

\title{Gud og Mogens 3}
\author{Anna, William, Kristian og Michael}

\begin{document}
\maketitle

\begin{roles}
\role{X}[Jakob] Instruktør
\role{G}[Marianne] Gud
\role{F}[Stolberg] Fortæller
\role{M}[Emil] Mogens
\end{roles}

%\begin{props}
%\end{props}

%\begin{mics}
%\mic{HS1}[] ???
%\end{mics}
  
\begin{sketch}
\scene{Lys op. G og M står i en side af scenen.}


\says{G} Okay - nu har jeg skrevet ned hvad der skal ske. Og jeg vil gerne på forhånd snakke om nogle af de misforståelser, der måske kan forekomme.

\says{M} Okey dokey

\says{G} Der står på et tidspunkt at faraoen slår nogle nyfødte drenge ihjel. Her er det meget vigtigt og jeg kan ikke understrege det her nok. Det betyder ikke at ALLE nyfødte drenge skal slås ihjel.

\says{M} OK. ikke ALLE drenge \act{Tænker alle, pånær 1}

\says{F} \act{Pause} Og alle drengebørn blev slået ihjel, pånær Moses, som blev gemt i en kurv i floden.

\says{G} Og i stykket hvor Moses er kommet op til Faraoen, må du gerne skræmme ham Farao lidt. Men lad vær med at gå amok. 

\says{M} \act{Bevæger sig baglæns ud.} Ja okay, jeg gør det bare lidt vildere, hvis han ikke forstår det.

\says{G} Ja, okay, men husk: Vi er ikke ude på at slå nogen ihjel her.

\says{F} Alle køer g afgrøder blev slået ihjel og ødelagt. Alting var mørkt at der var myg, fluer, bylder og græshopper over det hele. Alle førstefødte egyptere blev slået ihjel og alle førstefødte dyr blev også slået ihjel.

\says{G} Moooogens.




\scene{Lys ned.}
\end{sketch}
\end{document}

