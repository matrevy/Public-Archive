\documentclass[a4paper,11pt]{article}

\usepackage{revy}
\usepackage[utf8]{inputenc}
\usepackage[T1]{fontenc}
\usepackage[danish]{babel}

\revyname{MatematikRevy}
\revyyear{2015}
\version{1.0}
\eta{$4$ minutter}
\status{Udkast}

\title{Kendte Danskere Fortolker Matematik}
\author{Martin og Stolberg}

\begin{document}
\maketitle

\begin{roles}
\role{X}[Shake] Instruktør
\role{S}[Martin] Voice-over
\role{JL}[Christoffer] Jørgen Leth
\role{Ya}[Rolf] Yahya Hassan
\role{DM}[Stine] Dronning Magrethe II
\role{L}[Erik] Lars Løkke
\role{B}[Malthe] Brian Nielsen
\end{roles}

\begin{props}
\prop{Kommer senere...}[]
\end{props}

\begin{sketch}
\says{S} I et forsøg på at udbrede interessen for matematik hos de unge har Undervisningsministeriet i samarbejde med MATH iværksat et initiativ hvor kendte danskere fremfører deres fortolkninger af store danske matematiske værker.

\scene{Jørgen Leth kommer ind på scenen med en Kalkulus. Han sætter sig til rette i en lænestol og slår omhyggeligt op på en side i bogen.}

\says{JL} Læg mærke til hvordan $\varepsilon$-pølsen smyger sig tæt omkring grafen for f, ligesom når mine filippinske au-pairs smyger sig tæt ind til min blege, liderlige krop alt imens min potensfunktion vokser og vokser.

\scene{Jørgen Leth går ud. Yayha Hassan kommer ind}

\says{Ya} MIG, JEG SKAL VISE F KONTINUERT. EPSILON GIVET. MIG, JEG SKAL VÆLGE DELTA. MEN DELTA ER IKKE TRIVIEL. MIG, JEG SKAL FINDE ET TAL. ET EPSILON GIVET ARBITRÆRT. ET EPSILON, ALLAH HAR PLANER FOR DET. HAM, INSTRUKTOR SIGER DET SKAL SKE. MIG, JEG BLIVER VED. GÆTTER SOM EN RUS. DROPPER UD SOM EN RUS.

\scene{Yayha Hassan går ud. Dronning Margrethe kommer ind.}

\says{DM} Nu er det DisRus. Bag os ligger en gymnasietid med sorger og glæder. For mange nok flest sorger. En ny tid begynder, og med denne følger et væld af følelser og bevistyper. Der vil blive stiftet bekendtskab med modstrid, induktion såvel som medstrid. Men i hvirvelstrømmen af det nye, må vi ikke glemme vores rødder. Om dette så er roden af 4, 2 sågar -1. Uanset jeres baggrund lyder der her fra Prinsgemalen og os, et glædeligt Analyse 1. Gud bevare Danmark.

\scene{Dronning Margrethe går ud. Lars Løkke kommer ind.}

\says{L} Så vi har altså valgt at lave et system, hvor man skal kunne betegne mindre, mere end man gjorde før. Det er det, vi har valgt, og det fører selvfølgelig til, at det, der er skarpt mindre og afviger meget og nu afviger lidt mindre, ja, det afviger så mere mindre end det, der er svagt mindre og afviger mindre, men altså så afviger mindre mindre.

\scene{Lars Løkke går ud. Brian Nielsen kommer ind.}

\says{B} Nøh, nø'h, h'jøøø , Nhjøøeaædf , HÆnds''døøsdfne epsilon mmma naa vaa delta
bla bla bla slår rigtig hårdt hvhgkgkjg QED.
\end{sketch}

\end{document}
