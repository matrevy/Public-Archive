\documentclass[a4paper,11pt]{article}

\usepackage{revy}
\usepackage[utf8]{inputenc}
\usepackage[T1]{fontenc}
\usepackage[danish]{babel}

\revyname{Matematik Revyen}
\revyyear{2015}
% HUSK AT OPDATERE VERSIONSNUMMER
\version{1.}
\eta{4 minutter}
\status{Skrevet}

\title{Bamse og kylling og tælling}
\author{Matematikrevyens manusgruppe}

\begin{document}
\maketitle

\begin{roles}
\role{X}[William] Instruktør
\role{B}[Jasper] Bamse
\role{R}[Stolberg] Kylling
\role{L}[Anna Munk] Luna
\role{A}[Kristian] Aske
\end{roles}

%\begin{props}
%\prop{Rip/Rap/Rup kasket i et}
%\end{props}

%\begin{mics}
%\mic{HS1}[] ???
%\end{mics}
  
\begin{sketch}
\scene{Intro-melodien fra Bamses billedbog spilles - der er balloner og Lunas telt på scenen til at starte med. Bamse kommer ind og slår ballonerne væk. Kylling gar efter.}
\says{B} Dumme balloner.
\scene{Luna kommer ud af sit telt.}
\says{B} Hej Luna.
\says{L} Hej Bamse, hej Kylling. Jeg er ved at sy.
\says{B} Nåååhj. Hvor langt er du nået?
\says{L} Jeg har syet $\frac{3}{4}$ af den her kjole.
\scene{Mærkeligt syet kjole hvor der mangler noget i midten (eller lignende)
}
\says{K} 3 fjer!
\says{B} Nej Dumme Kylling! Det betyder altså ikke 3 fjer! Det betyder, at vi skal DELE 3 fjer.
\says{L} Tal pænt, Bamse.
\says{B} Undskyld, Luna.
\says{L} $\frac{3}{4}$ er en slags tal. Skal jeg fortælle jer lidt om tal?
\says{B+K} Jaaah!
\says{B} Tihvertifald!
\says{L} Kender I en halv? Fx. en halv lagkage.
\says{B} Ja, det er alt for lidt!
\scene{Luna tager en kage fra sin store frakke frem.}
\says{B} Hvis I nu skulle dele den her kage jer to.
\scene{Bamse skærer den i en kvart og en $\frac{3}{4}$. Og skal til at tage det store stykke.
}
\says{K} Bamse!
\says{L} Bamse, det er ikke pænt gjort over for kylling.
\says{B} Jamen det er fordi jeg også vil skære et stykke til dig.
\scene{Bamse skal til at skære det lille stykke over i to.}
\says{L} Ej Bamse! Hvis vi nu skærer her, så har vi 4 dele, der er lige store. Så kan vi tage 3 af delene og så har vi $\frac{3}{4}$. På samme måde kan man dele kagen op i flere eller færre stykker, bare de er lige store.
\says{B+k} Nåååååhr!
\scene{Bamse samler sin honningkrukke op. Kylling gestikulerer og siger lyde, der indikerer at han tæller.}
\says{B} Arj, Kylling, man kan altså tihvertifald ikke tælle til $\frac{3}{4}$!
\says{L} Det kan man faktisk godt. Prøv engang at se her (tager et skilt frem fra teltet, der viser det vha. en slags spiral.)
\says{B} Nå! Men i min honningkrukke, der er der 0 dele honning, så hvis jeg tager 0 dele, så tog jeg 0/0-dele.
\says{K} Bamse, L'Hospital.
\says{B} Hospital! Nej, Kylling, jeg er ikke syg.
\scene{Aske kommer ud fra Lunas frakke med 3 boller
}
\says{A} Jeg hørte at nogen var syge og som måske kunne bruge nogle boller.
	En til dig, Luna.
	En til dig, Kylling.
	Og en til… \act{tøver} mig.
\says{B} Ej Aske, din forskrupne klapspraglerprut, Så gider jeg bare heller ikke røre ved dine boller, nåh.
\scene{Bamse krydser armene, men han meget gerne vil have en bolle. Han griber ud efter Kyllings bolle og smadrer den ind i sin mund, der ikke kan åbne.}
\says{B} Krumme, krummme, krumme!
\says{A} Der er fordi at 3 faktisk et meget særligt tal, det er nemlig et primtal.
\says{K} a' hvaffor noget?
\says{A} Det er et tal, som kun 1 og tallet selv går op i. Man kan i øvrigt skrive sammensatte tal, som et produkt af primtal.
\scene{
Aske tager 25 boller frem
}
\says{A} Kan I fortælle fortælle mig hvad primtalsopløsningen af 25 er
\says{K} 5 og 5
\says{B} Det kan godt være at du var først, kylling, men jeg fandt bare en ANDEN løsning, nåh!
\says{B} Jeg tager lige $\frac{24}{25}$-dele af bollerne.
\says{K} Nej.
\says{B} Jo.
\says{K} Nej, nej.
\says{B} Uendeligt gange Jo.
\says{K} Uendelige +1 Nej.
\says{B} Det er det altsåså ikke noget som at der hedder.
\says{L} Faktisk er uendeligt plus 1 et ordinaltal, som er større end uendelig.
\says{B} Nåh, men så 2 plus uendligt gange Jo.
\says{A} Faktisk er 2+uendeligt og uendligt det samme ordinaltal.
\says{B} Ej, så gider jeg bare slet ikke matematik mere, jeg HADER matematik.
\scene{Bamse går ind i sit hus, og idet han smækker med døren falder skorstenen ned. Lys ned}
\end{sketch}
\end{document}

