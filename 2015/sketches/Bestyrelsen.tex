\documentclass[a4paper,11pt]{article}

\usepackage{revy}
\usepackage[utf8]{inputenc}
\usepackage[T1]{fontenc}
\usepackage[danish]{babel}

\revyname{MatematikRevy}
\revyyear{2015}
\version{1.0}
\eta{$3.0$ minutter}
\status{Udkast færdigt}

\title{Bestyrelsen}
\author{Martin}

\begin{document}
\maketitle

\begin{roles}
\role{X}[NB] Instruktør
\role{Bx}[Rolf] Bestyrelsesmedlem
\role{By}[Martin] Bestyrelsesmedlem
\role{Bz}[Malthe] Bestyrelsesmedlem
\role{N1}[Michael] Ninja eller ES01indbygger
\role{N2}[Freja] Ninja eller ES01indbygger
\role{N3}[Stolberg] Ninja eller ES01indbygger
\role{N4}[Sofie] Ninja eller ES01indbygger
\role{N5}[AnnG] Ninja eller ES01indbygger
\role{N6}[Jasper] Ninja eller ES01indbygger
\role{N7}[Erik] Ninja eller ES01indbygger
\end{roles}

\begin{props}
\prop{Kommer senere...}[Skaffer]
\end{props}

\begin{sketch}
\scene{Forslag til forbedringer fra Michael og Kristian:

Giv bestyrelsesmedlemmerne en personlighed:
Det kunne være, at bestyrelsen synes de er meget bedre, end alle andre; men overhovedet ikke er det!  [Det er nok bedre, hvis det foregår subtilt. De skal ikke sige det direkte, men mere tale om, hvor vigtigt det er for et svagt folk at have en stærk leder. Arrogance kan være ret sjovt, hvis det laves subtilt!
De taler, om at man hele tiden skal huske at give lunser til folket for at opretholde bestyrelsens enorme popularitet.

Formanden er sådan rigtig statsmands-agtig og taler, om hvordan alt skal ske hurtigt og, at bestyrelsen skal være handlekraftig og visionær! Det bør selvfølgelig være formanden, der kommer for sent.

Det kunne sikkert også være ret sjovt, at formanden er sådan en rigtig leder-type, men uduelig! De andre er undersåtter (forhold som Pinky og Brain). Fx. kunne det være de andre bestyrelsesmedlemmer, som undskyldte for, at de kom for tidligt, nu når formanden kom for sent. I øvrigt har de noget at bage en kage som undskyldning for, at de kom for tidligt.}

\scene{Etdelt - bestyrelsesmøde i den ene side, ES01 i den samme. Mens bestyrelsen snakker, sker ændringerne samtidig i ES01.}

\says{B1}[Kommer hen til de andre med en kage.] Hej Venner, jeg har taget kage med, undskyld jeg kommer lidt sent. Hvad var det, I gerne ville tale om?

\says{B2} Jo, kunne det ikke være herrenice med en sofa i S01? Så kan vi sidde og spille spil og rigtig hygge!

\says{B1+B3}: Jo, det er en god idé!

\scene{Sofa kommer ind, som de sætter sig i. Evt. skilt med teksten  6 måneder senere ... .}

\scene{Dum rus åbner vinduet}

\says{B1} Hvad skal vi for resten gøre ved vinduerne, de er svære at lukke, og da jeg kom ind i morges, stod det ene vindue bare helt åbent, der var herrekoldt.

\scene{Dum rus prøver at lukke det igen}

\says{B3} Hvorfor kan folk ikke finde ud af at lukke vinduerne ordentligt?

\says{B1} Måske mangler vi at fortælle dem, hvordan man gør. Det er jo lidt ulogisk, den måde håndtagene fungerer på, og hængslerne er ikke alt for gode.

\says{B2} Kan vi ikke hænge sedler op, der minder folk om at lukke vinduerne.

\says{B1} Jah, joh ... Det virker lidt for besværligt, men vi kunne evt. tape vinduerne til, så de ikke kan åbnes.

\says{B2} Ej, så bliver der jo alt for varmt.

\says{B3} Vi kunne bare give personen karantæne fra S01, hvis vedkommende gentagne gange bryder reglen om at lukke vinduerne efter sig.

\says{B3} Karantæne, ja, det er noget, der virker!

\scene{Person trækker dum rus ud af scenen}

\says{B1} Men det har vi jo ikke magt til...

\scene{Person trækker høfligt dum rus ind igen}

\says{B3} Magt? Jeg har en idé! Vi siger, at S01 er en forening, og at vi er bestyrelsen. Så har vi ubegrænset magt!

\scene{Evt. torden-effekt som ved den onde rusvejleder i Rusmus-sketchen. Måske en grundlovsændring}

\scene{Person trækker dum rus ud af scenen igen}

\says{B1+B2} Ja mand!

\scene{Person sætter skilt med ny bestyrelse op på bagtæppet.}

\scene{B1-3 får pludselig et lumskt smil på læberne. Lyset bliver mere dystert. B3 går i gang med at skrive manifest.}

\says{B3}  Ved grove overtrædelser af reglementet har Bestyrelsen uindskrænket magt til at udvise synderen i så lang tid, det måtte behage dem. 

\says{B2}  Der skal være navn og dato; og cpr.nr. på alle madvarer; ellers spiser vi dem. 

\says{B3} Altså russerne!

\says{B1}  Tag din egen opvask - ellers gør den dagsansvarlige det. 

\scene{Person med skilt hvor der står dagsansvarlig tager opvask og putter det ned i sin taske og går hjem}

\says{B2}  Det er ikke tilladt at overnatte eller sove eller lukke øjnene i ES01. 

\scene{Personer på scenen tager tændstikker i øjnene}

\says{B1}  Man skal stå op og hilse med hånd-/armtegn, når et medlem af Bestyrelsen træder ind 

\scene{B1-3  strækker  sig så de hejler. Personer på scenen giver dem røde armbind på.}

\says{B3} Det er alle de russer, der er begyndt at strømme ind på studiet. De er for mange og for uduelige. De snylter på det ES01, som VI har bygget op gennem generationer af matematikere! Lad os lade dem vide, at de ikke er velkomne i S01!

\says{B2}  Det pålægges alle brugere af ES01, nye som gamle, til enhver tid at bære den af Bestyrelsen valgte uniform. \act{Personer giver dem kasketter på.} Enhver rus skal bære tydeligt mærke, der signalerer, at vedkommende er en rus. 

\scene{Person stempler russerne i panden med ordet rus.}

\scene{På et tidspunkt heromkring rulles et stort billede af Inger Støjberg ned på væggen.}

\says{B3} Hvorfor egentlig lukke dem ind til at starte med? \act{Michael og Kristian synes dét er grinern!}  ES01's overlevelse er alt for vigtig til, at vi kan risikere at have russer herinde. Vi sørger for, at de slet ikke kan komme ind overhovedet!

\scene{Person smider russerne ud af ES01 - stadig flere på scenen bliver stemplet som rus og smidt ud, ind til næsten alle er smidt af scenen}

\says{B1} Hvordan gør vi det? Hvis de er indskrevet på matematikstudiet, kan vi ikke gøre noget.

\scene{Alt går i stå i ES01}

\says{B3} Vi indrykker annoncer i alle gymnasiers studenterblade, hvor vi skriver, at de ikke er velkomne! Og hvis de kommer ind alligevel. så stiller vi os op på svalegangen og spytter på dem.

\scene{God slutning.}

\end{sketch}

\end{document}