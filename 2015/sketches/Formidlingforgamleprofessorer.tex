\documentclass[a4paper,11pt]{article}

\usepackage{revy}
\usepackage[utf8]{inputenc}
\usepackage[T1]{fontenc}
\usepackage[danish]{babel}

\revyname{Matematik Revyen}
\revyyear{2015}
% HUSK AT OPDATERE VERSIONSNUMMER
\version{1.}
\eta{4 minutter}
\status{Skrevet}

\title{Formidling om formidling}
\author{Matematikrevyens Manusgruppe}

\begin{document}
\maketitle

\begin{roles}
\role{X}[William] Instruktør
\role{D}[AnnG] Formidler
\role{P}[Anna Munk] Ny professor
\end{roles}

%\begin{props}
%\prop{Rip/Rap/Rup kasket i et}
%\end{props}

%\begin{mics}
%\mic{HS1}[] ???
%\end{mics}
  
\begin{sketch}
\scene{Beskrivelse:
D er meget energisk og entusiastisk. Han forklarer positivt (ikke ironisk) om de mange fordele ved bl.a. at være svær at forstå som forelæser. Han er ikke selv svær at forstå.
P Starter med at være ok taler, men bliver efterhånden mere og mere kedelig osv.
}
\scene{Lys op, når Professoren begynder at gå ind på scenen. (efter ca. 1 skridt)}
\says{D} Hej P. Og velkommen til didaktikkurset. 
Ja, du er jo blevet bedt om at komme her af ledelsen, fordi du, selvom du lige er blevet professor, ikke helt lever op til forventningerne på det didaktiske punkt.
\says{P} Ja, tak. Ja jeg forstår det ikke helt, jeg har prøvet at følge et formidlingskursus engang. Her lærte vi at bruge meget kropssprog. Du ved, store armbevægelser, når man taler.
\says{D} Store armbevægelser?! Hvad havde du selv forestillet dig? Du står i et stort auditorium, hvor over 100 studerende sidder og skal undervises. Sæt dig ned.
Det er meget vigtigt ikke at bruge for store armbevægelser, det tager fokus væk fra det væsentlige. Hvad har du ellers forsøgt dig med?
\says{P} Jo, altså jeg prøver bl.a. at være sjov og energisk hele tiden.
\says{D} \act{ryster på hovedet} Sjov, siger du… Har du nogensinde prøvet at fortælle en sjov kommentar, som de studerende ikke reagerede på.

\says{P} Ja, det sker tit, jeg fortæller jo ret mange af dem.
\says{D} Og hvordan føltes det så, når sådan en “joke” falder FULDSTÆNDIG til jorden?

\says{P} Det føltes vel ikke særlig godt, men…

\says{D} Ikke noget MEN! Nu skal jeg fortælle dig, hvad du skal gøre: Sørg for at være overordentligt kedelig, hele tiden. På denne måde sørger du for at når du en sjælden gang kommer til at sige noget sjovt, så vil det få en garanteret reaktion, hvis ikke andet, så bare pga. den store kontrast!

\says{P} Jamen.. Hvordan sørger man for at være så kedelig? Du skal huske på at jeg plejer at være lektor og lidt ny til det her professor-noget.

\says{D} Sørg for at tænke over tonelejet.

\says{P} Altså, variere det?

\says{D} NEJ! NEJ! NEJ!

\says{D} Det, du skal gøre er at tale monotont. Men hvor skal du så lægge niveauet, tænker du nu måske. Her har jeg en god huskeregel: Det skal være så kedeligt som muligt, uden at de falder i søvn.

\says{P} Hvorfor dog det?

\says{D}  Jamen… Hvis de nu sover, så kan de jo pludseligt vågne op og være friske i hovedet og det kan vi jo ikke have.

\says{P} Hør..
\says{D} Bare husk de 3 K’er. Kedelig, cute og kikset.
\says{P} Ej, prøv lige at stop. Det her virker lidt langt ude! Hvorfor skal jeg tale sådan her? \act{taler kedeligt/robotagtigt/monotont}

\says{D} Okay, godt spørgsmål - det har jeg hørt før. Men det er jo bare, fordi du ikke har tænkt det til ende.
For det første skal de studerende jo have noget at snakke om, en pudsig forelæser giver sammenhold og DET er godt for studiemiljøet.
Og så skal en professor jo også adskille sig fra en folkeskolelærer.

\says{P} Jeg tror altså stadig ikke at jeg er overbevist.

\says{D} Og så skal en professor jo også adskille sig fra en gymnasielærer.

\says{P} Nu er jeg overbevist. Hvad ellers?

\says{D} Det er også virkeligt godt med en såkaldt “catchphrase”. Kan du prøve at sige et ord?

\says{P} Ja.
\says{D}Godt. Prøv at gentag det ord så meget som overhovedet muligt.

\says{P} Ja, ja…. ja,ja… ja. \act{li’som Jønne}

\says{D} Glimrende, den lyd har Søren Jøndrup i øvrigt også taget! Lyden skal du bare sige, når folk taler. Og det gør ikke noget at de ikke er færdige med at tale. Hvis “ja” ikke virker, kan du også prøve med “earh” \act{Ernstlyd med tilhørende Ernstarme},ligesom Studieleder Hansen.
Det næste, der sker i en god forelæsning er den obligatoriske “I er meget velkommen til at stille spørgsmål.” Prøv at sig det engang.

\says{P} Ja, ja. I er meget velkommen til at stille spørgsmål…. ja.

\says{D}  STOP, stop, stop. Du kommer til at lyde som om du mener det. Hvordan ville du egentlig reagere, hvis nogen stillede et spørgsmål?

\says{P}  Ja, jeg ville nok svare på det.

\says{D}  Aj, det er ikke helt rigtigt. Den rigtige fremgangsmåde er følgende: Vent 5 sekunder, knib derefter øjnene sammen og drej lidt på hovedet, så skal du stirre intenst på den studerende og gå derefter gerne bare videre.

\says{P}  Ja. Og hvad så med. ja. selve matematikken…

\says{D} Jo.

\says{P} \act{afbrydende} Ja.
\says{D} \act{fortsat} Det er selvfølgelig vigtigt at de studerende får sig nogle succesoplevelser.

\says{P} ja.
\says{D}Men de skal heller ikke tro at de er noget, vel?

\says{P}ja.
\says{D} Derfor er det vigtigt at du vælger et par detaljer, som er meeeget nemme og som alle forstår. Dem kan du så virkeligt gå i detaljer med. Det kan fx være et gange og additionsproblem eller noget, man som minimum har terpet en del for et par kurser siden.
Når du så kommer til de svære detaljer, der godt kunne behjælpes med lidt forklaring, så er her at der skal spares tid.

\says{P}  Ja. Hvad så hvis de studerende afbryder og spørger ind til det.
\says{D} Jamen så siger du bare at det har de allerede lært i et senere kursus, og hvis de ikke hopper på den, så henvis altid til Lineær Algebra kurset.

\says{P} ja, smart ja. Hvad så med hensyn til afslutningen? er der en særlig kotume?

\says{D}Sjovt du spørger. Hvordan vil du normalt slutte en samtale af?

\says{P} Vi ses igen eller sådan noget, ja.

\says{D}Okay, men her kan du næsten regne ud det gør vi ikke, vel?
\says{P} Ja, nej, ja, ja.
\says{D} Sig en lidt afsluttende kommentar i stil med: Det var så det for i dag:

\scene{Alle på scene står helt stille, så det ligener at sketchen er slut et kort stykke tid.}

\says{D} Og så vent en rum tid og begynd på det næste, som fx en motiverende historie og slut derefter af med at sige at det når vi IKKE på dette kursus.

\says{P}Hvad med farvekridt?
\says{D}  Generelt er det en god idé at bruge forskelligt farvet kridt…. så længe at der ikke er noget system i brugen. fx må et enkelt ord godt have 4 farver, selvom det kun har 3 bogstaver.

\says{P}  Ja. Jeg har virkelig lært meget.
\says{D} Ja. Du føler dig allerede mer professor-agtig, ikke. Du har måske bar et spørgsmål tilbage: Hvordan undgår jeg akavede afslutninger.

\says{P} Ja.

\says{D} Det når vi IKKE på dette kursus.

\scene{Lys ned}
\end{sketch}
\end{document}
