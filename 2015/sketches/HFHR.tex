\documentclass[a4paper,11pt]{article}

\usepackage{revy}
\usepackage[utf8]{inputenc}
\usepackage[T1]{fontenc}
\usepackage[danish]{babel}

\revyname{MatematikRevy}
\revyyear{2015}
\version{1.1}
\eta{$1$ minut}
\status{Udkast}

\title{Hvis Filosofi Havde Revy}
\author{Stolberg}

\begin{document}
\maketitle

\begin{roles}
\role{X}[Shake] Instruktør
\role{R1}[Marie] ReplikFilosof1
\role{R2}[Rolf] Filosof
\role{F3}[Malthe] Filosof
\role{F4}[Johanna] Filosof
\role{F5}[Ljørring] Filosof
\role{F6}[Iris] Filosof
\end{roles}

\begin{props}
\prop{}[]
\end{props}

\begin{sketch}
\scene{F1 og F2 sidder på scenen}. 
\says{R1} \act{Kommer ind på scenen} Hej allesammen! Velkommen til Amager Fælled og til første manusmøde Nå men, lad os bare komme igang. \act{Sætter sig ned}

\scene{Alle sætter sig og begynder at tænke. Nogle finder pibe frem, imens folk spekulerer og kigger rundt i lokalet}.
\says{F6} \act{Mimer hun får en god ide og skal til, at sige den, men siger alligevel intet}
 \scene{Dæmp lys.}

\scene{Lys op. Alle sidder stadig og tænker. }
\says{R2} \act{kommer småløbende ind på scenen, kommer hen til de andre}
\says{R2} Undskyld, jeg kommer for sent \act{Han sætter sig og går i gang med at tænke videre.}

\says{F3} Kællinger. Og kampagner
\scene{Dæmp lys.}

\scene{Lys op. Alle tænker stadig. }
\says{R1} \act{Kigger på sit ur og rejser sig} Så tror jeg, det er nok for i dag! Tak for et konstruktivt møde, jeg kan mærke det bliver en fantastisk revy!

\end{sketch}
\end{document}
