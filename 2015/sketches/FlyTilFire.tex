\documentclass[a4paper,11pt]{article}

\usepackage{revy}
\usepackage[utf8]{inputenc}
\usepackage[T1]{fontenc}
\usepackage[danish]{babel}

\revyname{MatematikRevy}
\revyyear{2015}
\version{1.0}
\eta{$5.0$ minutter}
\status{Udkast færdigt}

\title{Fly til fire}
\author{Anna, Michael, Jakob, Pelle, Kristian, William og NB}

\begin{document}
\maketitle

\begin{roles}
\role{X}[NB] Instruktør
\role{A}[Michael] Onkel Anders
\role{Per}[Emil] Lille Per
\role{Ole}[Erik] Ole
\role{Mie}[Marianne] Mie
\role{Pet}[Stolberg] Peter
\role{Søs}[Iris] Søs
\role{Far}[Christoffer] Far
\role{Stew}[AnnG] Stewardesse
\end{roles}

\begin{props}
\prop{Kommer senere...}[Skaffer]
\end{props}

\begin{sketch}
\scene{Sketchen skal primært foregå ombord i et fly. Inden da skal familien og dennes nære venner boarde i Københavns Lufthavn. Det skal dog vise sig, at der i løbet af hele sketchen popper mange løselige, uløselige problemer op.}

\scene{Boarding: De kommer ind af sideindgangen i aud.}

\says{Far} Kom så børn. Flyet til Mallorca \act{udtal l’erne} letter om 2 minutter.

\says{Søs} Jamen hvor er Per?

\scene{Per kommer glad trallende ind efter de andre med sin elefant under armen.}

\says{Per} Bommelom... Her kommer Gardens allerstørste mand... Jeg er lige her Lille Søs.

\says{Far} Skynd dig lidt, Per.
 
\says{Per} Ja, ja, lille far. Jeg skynder mig alt jeg kan, men Bodil er ikke så hurtig.

\scene{Lige inden de går ind bag tæppet og op til scenen, skal de checke ind ved den lækre stewardesse. 
Anders griner fjoget, da han checker ind.
Mie begynder at græde, da hun har glemt sit pas. Hun er den anden sidste, der skal boarde.}

\says{Mie} Åhh. Nej! Jeg har glemt mit pas!!

\says{Per} Bare rolig, Mie. Du må godt låne mit.

\scene{Per spurter tilbage til Mie og rækker sit pas til hende. Mie kommer igennem uden problemer. Til sidst kommer Far.}

\says{Stew}	De må ikke have beholdere på mere end 100 ml med ombord, Hr.

\scene{Modvilligt giver faren stewardessen flasken med snaps.
Bag bagtæppet brokker Far sig.}

\says{Far} Åhr. Må man nu heller ikke have mere end en kvart liter snaps med? Nu troede jeg lige, jeg skulle have en god tur til Mallorca

\scene{Far tager så en pibe frem, som han skal til at ryge.}

\says{Stew} De må ikke ryge pibe ombord, Hr.

\says{Far} Ej, det er da da også for galt

\says{A} Huhu. Rolig nu, Lille Far, gør nu ikke et problem ud af ingenting 

\scene{Herefter hiver han to cigarer frem og giver ham en cigar, som de tænder.}

\scene{Én af gangen finder de langsomt deres pladser med tumult. Per længst til venstre, så Søs, Onkel Anders, mellemgang, Far, Ole , Mie.}

\says{Pet}[Over radioen] Dette er deres kaptajn - gør klar til at lette.

\scene{Når de letter, sidder de alle og læner sig tilbage i stolen, som om de letter. (Lette-lyd fra teknikken) 
Alle er på, undtagen far. Han sidder og læser avis, indtil nogen pointerer, at de jo er ved at lette}

\says{Ole} Men Lille Far, vi letter jo!

\says{Far} Orv, det er også rigtigt lille Ole

\scene{Herefter følger Far også trop. Til sidst kaster de armene op i luften og siger wuh.
Piloten (som er Peter (Søses kæreste)) siger noget i mikrofonen.
Peter træder frem.}

\says{Pet} Velkommen ombord. Dette er jeres kaptajn!

\says{Alle} Peter!!!

\says{Søs} Jeg vidste slet ikke, at du kunne flyve.

\says{Pet} Det kan jeg da heller ikke. Heldigvis er autopiloten for længst blevet opfundet.

\scene{Peter træder op på scenen. Her trækker han en (urimelig stor) buket blomster frem, som han rækker ind over Lille Per og Ole til Søs (evt. Lille Per og Ole nyser).}

\says{Pet} Jeg var lige en tur rundt i flyet og har plukket dem her til dig, min egen. 

\scene{Herefter ranker Peter sig, stikker en finger i munden og mærker vinden.}

\says{Pet} Det ser ud til vi støder ind i noget turbulens.

\scene{Alle griber fat i deres stol, ryster og hopper med stolen. Alle siger bumlelyde, og især Onkel Anders lyder skinger (og morsom!), undtagen Peter der står robust i midtergangen, med hænderne i siden og ryggen rank. Turbulensen slutter, men Onkel Anders fortsætter et stykke længere end de andre.}

\says{Far} Onkel Anders, det er forbi.

\says{A} Nåååååh, hoho. Jamen jeg ved sådan set heller ikke, hvad jeg har gang i.

\scene{Peter trækker op i ærmet og aflæser sit ikke eksisterende armbåndsur, hvorefter han højtidligt, henvendt til publikum, kan konkludere}

\says{Pet} Det viser sig der slet ikke var noget turbulens, der var tale om en fejlmåling. Jeg må tilbage til cockpittet.

\scene{Peter sætter sig på sin plads forrest på scenen og styrer flyet med et rat. Onkel Anders og Far har efterhånden pulset længe på deres cigarer, så Mie hoster og vifter røgen væk.}

\says{Mie} Puuh, det er blevet røget. Lille Per, vil du ikke åbne et vindue?

\says{Per} Det kan du tro, lille Mie!

\scene{Lille Per møver sig forbi Søs, Onkel Anders, Far, Ole og til sidst Mie og siger Undskyld lille (indsæt navn) for hver han passerer.}

\says{Per} Undskyld lille Søs. Undskyld lille Onkel Anders. Undskyld lille Far. Undskyld lille Ole. Undskyld lille Mie.

\scene{Nu prøver han ihærdigt at åbne et (imaginært) vindue ved siden af scenen (lige ved udgangen). Dette virker ikke og han kravler op på en stol for at stå bedre, med ordene:}

\says{Per} Iiiih, altså, for syv sytten!

\says{Far}[Kigger pludseligt op fra avisen] SÅ PER!

\says{Per} Satans også. Det må du undskylde, lille far.

\says{Far} Ja, det må du nok sige.

\scene{I samme øjeblik får Lille Per vinduet op, og stikker hovedet ud af vinduet. Suselyde fra Texnikken.}

\says{Per} Jodeleyhoohoo

\says{Far} Nej, Per, det er alt for koldt! \act{Ingen reaktion} Per! \act{Vender sig} Jamen Lille Per, hvad laver du derude, kan du straks komme ind. Og luk vinduet efter dig!

\scene{Per trækker hovedet ind igen, og lukker vinduet efter sig. En stewardesse kommer kørende med en madvogn gennem gangen, og deler mad ud. Ole kigger skeptisk på maden, lugter til det og trækker hovedet tilbage i afsky.}

\says{Ole} Ad! Det ka’ DU da gøre bedre, kære Søs!

\scene{Søs rejser sig op, tager et forklæde frem (og tager det på) samt tager en alt for lille stegepande frem (eventuelt den fra S01) samt et helt kilogram flæsk i skiver.}

\says{Søs} Jamen, jeg går da lige fluks ud og laver en omgang stegt flæsk.

\scene{Søs møver sig forbi rækken for at komme ud.}

\scene{Lille Per klapper Søs kærligt (på sådan en barnemåde) i numsen.}

\says{Søs}[delvist forarget, delvist forlegen]  “Eej Per!  Du ved, at jeg har ondt bag i...    Af at sidde så længe ned…  På Peters varme skød... “

\scene{Mie sidder lige og piller ved sine… fletninger. Mie begynder at græde og hulke.}

\says{Ole} Men kæreste, Mie dog! Hvad tuder du dog sådan over?

\says{Mie}[vrælende] Der var nogen, der hev mig sådan i fletningerne.

\says{Ole} Men Mie, det er jo dig selv, der sidder og hiver i... fletningerne.

\says{Mie} Ih, ja - det er da også rigtigt. Sikke fjollet og hjælpeløs jeg er - ligesom alle andre piger.

\scene{Søs kommer nu gående ind - med stegepanden. Denne gang med et færdiglavet farsbrød på den stadig alt for lille stegepande.}

\says{Søs} Tal for dig selv, lille Mie. Alle piger og kvinder er selvstændigt tænkende individer med en plads her i samfundet.

\says{Far} Så Søs! Ja din plads er i køkkenet.

\says{Søs} Jovist, søde Far!

\scene{Søs vandrer tilbage mod bagtæppet. Her får hun pludselig øje på fru Sejersen blandt de andre passagerer.}

\says{Søs} Jamen, fru Sejersen! Skal de også til Mallorca?

\says{FS} Ja, jeg tænkte jeg trængte til en ferie fra jer uvorne unger! Men det fik I da også fint spoleret!

\says{A} Fru Sejersen! Er de også med ombord? Sikke et sammentræf!

\says{FS} Jamen, det er jo det, jeg lige har sagt, kære Onkel Anders.

\says{A}[Onkel Anders-lyde] Brruuuuhuhuhuhurgggrrrrr 

\says{Søs} Men Fru Sejersen, DE kan da være mig behjælpelig i køkkenet.

\scene{Søs og Fru Sejersen vandrer tilbage bag bagtæppet - til køkkenet.
Peter rejser sig efterfølgende op.}

\says{Pet} Jeg tror lige jeg går \act{løfter øjenbrynene} med.

\scene{Peter går med. Derefter høres pruttelyde.}

\says{Ole} Peter har vist glemt at slukke mikrofonen.

\scene{Lille Per løber på tværs af gangen, så flyet tipper fra side til side (De sidder og tipper med på scenen). Bodil triller efter Per ved hjælp af en snor som holdes fast i begge ender.}

\says{Far} Så Pe... \act{afbrydes}

\says{Pet}[Over mikrofonen] Ja... Og vi er nu fremme ved destinationen. Vi håber de har haft en behagelig rejse og at de vil nyde deres ophold her i Syrien. (Lyd af toilet der skyller ud).

\says{Alle} SYRIEN?!?!

\says{Per} Nå, men det skal da nok blive hyggerligt alligevel.

\says{Far} Så Per! Det er nemmelig rigtigt!

\scene{Søs træder ud - nu med en færdiglavet flæskesteg på den stadig alt, alt for lille stegepande. Og udbryder}

\says{Søs} Deeeeet...

\end{sketch}
\begin{song}

\sings{Alle}
...Er sommer, det er sol og det er Syrien!
alle hjerter er så glade og fri
håb ej på, at du bli’r hentet af valkyrien,
vor familie, den kan klare en krig

\end{song}
\begin{sketch}

\says{Alle} hahahahahahahahhaha

\scene{Lys ned}

\end{sketch}

\end{document}


Noter:
Måske kunne det hele gøres til en sikkerhedsintro. Man kunne eventuelt få Mie til at græde, da hun ikke kan huske sikkerhedsinstrukserne - “Men så gør vi da det bare sammen”, råber de andre gladeligt.
Foto, hvor de rykker sammen, så alle kan være på billedet.
Lille Per kommer gående/løbende ned gennem gangen. “Hej for dig og hej for mig…” Pludselig springer en terrorist ud foran Lille Per! Ole kommer straks til undsætning og hænger terroristen op i kraven på knagen. Der sker ikke andet, end at terroristen slår ud i luften. Evt. kommer der meget belejligt en stumtjener ind.
Slutningen kan være den rigtige pilot siger at vi ikke er lettet, på grund af tåge. Peter kommer ind med rattet mens han laver bil-prutte-lyde. Der er da også dejligt i Danmark. Jeg har egentlig ikke tid til at holde ferie. Der er jo stadig toldfrit i lufthavnen. Jeg har da for resten også et farsbrød i ovnen derhjemme.
