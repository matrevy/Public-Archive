\documentclass[a4paper,11pt]{article}

\usepackage{revy}
\usepackage[utf8]{inputenc}
\usepackage[T1]{fontenc}
\usepackage[danish]{babel}


\revyname{MatematikRevyen}
\revyyear{2017}
% HUSK AT OPDATERE VERSIONSNUMMER
\version{1.0}
\eta{$1$ minut og $46$ sekunder}
\status{færdig}

\title{Begynderfejl}
\author{Ulrik}

\begin{document}
\maketitle

\begin{roles}
\role{X}[Alexander] Instruktør
\role{E}[Kristian] Eksaminator
\role{Cen}[Phillip] Censor
\role{C}[Toke] Chewbacca
\end{roles}

\begin{props}
\prop{Tavle med kridt}
\prop{Bord}
\prop{To stole}
\end{props}


\begin{sketch}

\scene{Lys op.}

\scene{Eksaminator og censor sidder parat, og Chewbacca træder ind i lokalet.}

\says{E} Goddag, \act{kigger ned på listen over eksamensdetalgende} Chew… bacca,
og velkommen til Analyse 0 eksamen. Hvilke spørgsmål var det du trak?
\says{C} Roooo… \act{AV wookie 1 afspilles}
\says{E} Javel, ja.
\says{Cen} Gå endelig bare i gang.
\scene{Chewbacca går straks i gang med at skrive på tavlen ved at banke sin pote ind i tavlen.}
\says{C} Rooo… \act{AV wookie 2 afspilles samtidig}
\says{Cen} Han er da én af de mere veltalende studerende, vi har haft oppe i dag.
\says{E} Ja… det tør siges. Sig mig engang, nu refererer du til Rolles sætning. Kan du fortælle mig, hvad den siger?
\act{AV wookie 3 afspilles}
\says{Cen} Ja, det er nok vældigt fint. Men hvorfor bruger du ikke induktion?
\says{C} \act{vredt} Rooo! \act{hæver armen} \act{AV wookie 4 afspilles}
\says{E} Den er altså god nok… induktion er slet ikke pensum i dette kursus.
\says{Cen} Det var dog åndssvagt.
\says{C} Rooo! \act{AV wookie 5 afspilles}
\says{Cen} Selvfølgelig. Fortsæt endelig.
\scene{Chewbacca vender tavlen, hvor selve brugen af middelværdisætningen
på det $n$’te Talorpolynomium er udførligt dokumenteret,
men der er differentieret efter både udviklingpunktet a og variablen x.}
\says{E} Ej, se lige det var hende der var inde lige før, he he hun består i hvertfald på udseendet. Ja ja det er fint ikke så mange detaljer.
\says{C} Rooo! Rooo! \act{AV wookie 6 afspilles}
\says{Cen} Bernstein polynomierne? Dem kender jeg slet ikke.
\says{C} Roo! \act{AV wookie 7 afspilles}
\says{Cen} Ahh, de kan da ikke bare sådan approksimere hvad som helst i $L^p$?
\says{C} Roo! \act{AV wookie 8 afspilles}
\says{Cen} Ahh, ja selvfølgelig jeg er et fjols.
\says{E} Det må jeg sige, sjældent har en Analyse 0 eksaminand været så vidende.
\scene{E kigger nervøst over på Chewbacca.}
\says{E} …Men… vi kan jo desværre ikke give dig 12.
\says{C} \act{vredt} Roo? \act{AV wookie 9 afspilles}
\says{E} Jo ser du…
\scene{E rejser sig og går over til tavlen og peger på fejlen.}
\says{E} Altså… du har differentieret efter både variable og udviklingspunkt. Helt ærligt Chewbacca… det er altså en Wookie mistake.

\scene{Lys ned.}

\end{sketch}
\end{document}
