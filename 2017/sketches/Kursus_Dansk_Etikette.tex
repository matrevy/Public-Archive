\documentclass[a4paper,11pt]{article}

\usepackage{revy}
\usepackage[utf8]{inputenc}
\usepackage[T1]{fontenc}
\usepackage[danish]{babel}


\revyname{MatematikRevyen}
\revyyear{2017}
% HUSK AT OPDATERE VERSIONSNUMMER
\version{1.0}
\eta{$n$ minutter}
\status{Sandsyligvis færdig}

\title{Kursus i Dansk Etikette}
\author{Jakob '11, Marie '15}

\begin{document}
\maketitle

\begin{roles}
\role{X}[Jakob] Instruktør
\role{L}[Khalil] Lærer 
\role{E1}[Emil] Elev 
\role{E2}[Mads] Elev
\role{E3}[Nadia] Elev 
\end{roles}

\begin{props}
\prop{Whiteboard med dansk flag i hjørnerne}
\end{props}


\begin{sketch}

\scene{Lys op}

\says{L} Velkommen tilbage til vores anden time i: "Kursus i Dansk Etikette."
Lad os starte med at genopfriske lidt af det, vi lærte sidste time. Er der en af jer,
der kan fortælle mig, hvad man gør, hvis en børnehaveklasse træder ind i bussen?

\scene{E1 og E2 rækker hånden op}

\says{L} \act{Peger på E2 og siger} Ja?

\says{E2} Man tilbyder dem alle sammen slik for at være søde?

\says{L} Ja, det er jo så %\act{kunstpause, hiver megafon op og peger den mod ansigtet på E2}
\act{Koldt} Helt forkert! %\act{Trækker kort vejret og siger roligt og glad} 
Andre forslag?

\scene{E1 rækker hånden i vejret}

\says{L} Ja, \act{Peger på E1} dig der. 

\says{E1} \act{Bedrevidende} Flygter ud af bussen ved det nærmeste stop.

\says{L} \act{Dansk og stolt med stort smil} Korrekt. Hvad lærte vi ellers?

\says{E1} \act{Rækker hånden op} Man skal snakke om vejret minimum 6 gange dagligt.

\says{E3} \act{Rækker hånden op og siger ivrigt} Det eneste vigtige der skete i ‘92 var, at Danmark vandt EM i fodbold.

\says{L} \act{Skråler og gestikulerer til eleverne, at de skal synge med}. OG DET VAR DANMARK,
OG DET VAR DANMARK \act{Tager megafon op for munden} OLÉ, OLÉ, OLÉ

\scene{Eleverne synger halv-hjertet med.}

\says{L} Nå, nok om vort elskede fædreland.
%Lad os komme i gang med dagens undervisning!
Forestil jer, at I sidder blandt jeres ægte danske venner. Diskussionen bevæger sig ind på politik.
I bliver spurgt, hvem i stemmer på. %, hvis I boede i Danmark. 
Hvad svarer I?

%(Hvis der er en asiat i revyen til at spille E3 indskydes følgende replik: \says{E3} Er der mere end et parti?)

\scene{E1’s hånd ryger i vejret}

\says{L} \act{peger på E1} Ja? \act{Mens E1 siger sin replik nærmer L sig opmuntrende mens han gentager “ja” og gestikulerer til E1 at han skal fortsætte. Når E1 siger “Dansk Folkeparti” stopper L brat.}

\says{E1} \act{Slesk og spytslikkende} Jeg skal jo gøre noget rigtig dansk.
Så jeg ville, som du jo nok også ville gøre \act{peger på læreren}, stemme på Dansk Folkeparti.



\scene{Kunstpause}

\says{L} \act{Mod E1} Du må ALDRIG sige at du stemmer på Dansk Folkeparti!

\says{E2} Men, hvad nu hvis man rent faktisk stemmer på Dansk Folkeparti?

\says{L} \act{Mod E2} Du må aldrig SIGE at du stemmer på Dansk Folkeparti!

%\scene{L skriver DF på sit lærred og streger det over, venter lidt, skriver DF lidt større en gang til og streger det over}

\says{L} \act{Vender hurtigt og råber gennem megafonen} \textbf{Popquiz!} Hvad er de 3 mest danske sportsgrene?

\scene{L peger på E1}

\says{E1} \act{Svarer med det samme} Fodbold!

\scene{L peger på E2}

\says{E2} \act{Svarer med det samme} Den bold med den hånd 

\scene{L peger på E3}

\says{E3} \act{Tænker sig lidt om} Kolding?

\says{L} Ja, du mener curling. Vil i have nogen flæskesvær? \act{Tilbyder som belønning} 

%Flot! Lad os gå videre. Dig dér \act{peger på E3}, kom her op!

\says{E3} Er det Halal?

\says{L}[Let tøvende] Ja... Det kan vi godt sige. \act{Peger på $E3$} Dig der, kom her op. 

\scene{E3 bevæger sig frygtsomt op ved siden af L}

\says{L} Forestil dig nu, at du kommer ind i 4A. Der er en ledig plads ved siden af
ham/hende dér \act{peger ved siden af E1}, to ledige pladser dér \act{peger på E3’s tomme plads}
og en ledig plads ved siden af ham/hende dér \act{peger ved siden af E2}. Hvor ville en ægte dansker sætte sig?

\says{E3} \act{Nervøst} Jeg ville jo nok sætte mig ved siden af en ægte dansker. For som ægte dansker er jeg jo bange for den fætter.
%Så ville der jo også være et sæt ledige pladser, hvis et par kom ind i bussen. \act{får lidt mere selvtillid} For som dansker er jeg jo super solidarisk.

\says{L} %\act{Holder, let truende, megafonen op. Siger langsomt og ondt smilende} 
Er du \act{kunstpause} helt sikker?

\says{E3} \act{Skælvende} Ja?

\says{L} \act{Koldt} Det er så forkert. %\act{Råber ind i megafonen, der peger ind i hovedet på E3} FORKERT! \act{Peger mod pladsen}
%SÆT DIG! \act{Fjerner megafonen og siger forpustet} 
\act{Almindeligt} Hvis du frivilligt sætter dig ved siden af en fremmed person er du \act{råber mod E3} FUCKING MÆRKELIG!

\says{E1} \act{Rækker hånden i vejret} Undskyld, jeg har et spørgsmål.

\says{L} \act{Trækker vejret dybt og smiler} Ja?

\says{E1} Det er meget dansk at cykle. Hvad er det vigtigt at huske, når jeg cykler? \act{Tager noter mens L svarer}

\says{L} \act{Tegner på tavlen situationen mens der fortælles. Entusiastisk} Godt spørgsmål.
Hvis du kommer kørende på en christianiacykel er det vigtigt, at du så ofte som muligt
stopper op på midten af cykelstien for at snakke med en af dine bekendte på fortovet.
Dernæst skal man, hvis man cykler en gruppe af 10 mountainbikeryttere,
naturligvis \textbf{aldrig} ringe med klokken før man overhaler.

\says{L} Men den allervigtigste situation, man som ægte dansker bør tage stilling til,
er, hvad man gør, når 3 personer der lige har overhalet én holder for rødt.
Det, man selvfølgelig gør, som ægte dansker, er at presse sig vej igennem cyklerne foran en,
for så at stoppe midt i fodgængerfeltet. På den måde skal alle cyklisterne nu overhale en \textbf{igen}.
\act{Siger optimistisk} Samtidig holder man jo så optimalt i vejen for alle fodgængerne.

\scene{Tegning af en cykelsti. L tegner slalom sti imellem cyklerne og tegner et kryds foran dem}

\says{L} Nå timen er ved at være ovre. Her til sidst, skal vi så ikke slutte af med sammen at synge Danmarks nationalsang?
\says{E2} \act{Spørgende} Ah, den Yndige Land?  %"Der er et Yndigt Land"?

\says{L} \act{Råbende gennem megafonen} FORKERT! \act{Skråler og ud mod publikum gennem megafon}
OG DET VAR DANMARK, OG DET VAR DANMARK, OLÉ, OLÉ, OLÉ. \act{fortsættes ad libitum}.

\scene{Skrålen gentages mens lys fader}

\end{sketch}
\end{document}
