\documentclass[a4paper,11pt]{article}

\usepackage{revy}
\usepackage[utf8]{inputenc}
\usepackage[T1]{fontenc}
\usepackage[danish]{babel}


\usepackage{amsmath}
\usepackage{amsfonts}
\usepackage{amssymb}
\usepackage{amsthm}


\revyname{MatematikRevyen}
\revyyear{2017}
% HUSK AT OPDATERE VERSIONSNUMMER
\version{3.0}
\eta{$3$ minutter}
\status{Sandsyligvis færdig}

\title{Helvede}
\author{Dem der ikke selv texer sketches}

\begin{document}
\maketitle

\begin{roles}
\role{X}[Freja] Instruktør
\role{X}[Lasse] Instruktør
\role{K}[Janne] Kvinde 
\role{R}[Mads] Receptionist
\role{F}[Toke] Forelæsertype
\role{S1}[Aurora] Statist1
\role{S2}[Phillip] Statist2
\role{S3}[Frederikke] Statist3
\end{roles}

\begin{props}
\prop{}
\end{props}


\begin{sketch}



\scene{K falder ind på scenen. R står og læser i et magasin ved bagtæppet, og i midten er en stor dør.}
\says{R} Velkommen til Helvede! Vi tænder op under dit efterliv.
\says{K} Du går bare i gang mit underliv er knastørt.
\says{R}[træt] Ikke underliv. Efterliv! Efterliv! Åh, jeg hader det slogan.
\says{K}[forfærdet] Efterliv? Er jeg død?
\says{R} Ja, du er død. Velkommen til Helvede.
\says{K} What?! Hvordan kan jeg være kommet i Helvede? Jeg har betalt 50 kr. til Amnesty hver måned de sidste 5… måneder.
\says{R} Ved du ikke, at alle pengene går til administration. Du er kommet i Helvede og sådan er det. Stik mig så din rapport.
\says{K} Rapport? Hvad er det for noget?
\says{R} Din rapport er det stykke papir, der skal fortælle mig, hvilken afdeling jeg skal sende dig til. Men hvis du ikke har den med, må jeg vel bare vise dig rundt. Så kan du selv vælge, hvor du føler, du passer bedst ind.
\scene{R og K går over til døren. K trækker i håndtaget, og døren åbner ind til et lokale fyldt med lidende fysikere.}
\says{R} Her har vi afdelingen for folk, der har divideret med 0. Det er primært fysikere. Det er det blodigste sted i Helvede. Vi har ladet dem mærke på egen krop, hvordan det føles kun at have et element tilbage i sit legeme.
\says{K} For søren - Jeg mener - for helvede da!
\scene{R værdstætter ikke joken, og trækker igen i håndtaget. Døren åbner op til folk der lider til Partyalarm (spilles på AV)}
\says{K}[råbt] Hvad sker der derinde?
\says{R}[råbt] Det er blot afdelingen for folk, der ikke lukker døren efter sig kl. 17. Så kan de få lov at party'e lidt til PartyAlarmen.
\scene{Døren lukker igen og alarmen stopper.}
\says{K} Åh, gud…
\says{R}[tørt] Nej...
\scene{F træder forvirret ind på scenen, og går over til R.}
\says{F} Hej, undskyld mig… Hvor er afdelingen for folk, som bruger en våd tavlesvamp til deres tavler?
\says{R} Ja, det er så nede ad gangen, forbi afdelingen for cykelturister og telefonsælgere, til højre ad gangen med matematikrevyens manusgruppe, og til venstre ved lokalet for hvemend der fandt på, at gøre en debat ud af ananas på pizza. 
\says{F} Ah, tak!
\scene{K ser lettere forfærdet ud}
\says{K} Ej… Jeg kan altså ikke rigtig se mig selv i NOGEN af de her grupper. 
\says{R} Ok, jamen, så kan du bare vende tilbage til livet.
\says{K} Kan man virkelig det?
\says{R} Ja, selvfølgelig. Vi vil jo ikke putte folk ned i en eller anden kasse, hvor de ikke føler, de hører til. Det ville være totalt politisk ukorrekt. Udgangen er her. 
\scene{R trækker i håndtaget og der åbnes op til mørke.}
\says{K} Ok, nice.
\scene{K går ind ad døren.}
\says{K} Det her ligner altså ikke en udgang… 
\scene{Ond latter på AV. En masse hænder dukker op og lukker døren bag K. Kvindeskrig på AV.}
\says{R} Og det er det der sker, når man lader opvasken stå i studenterkøkkenet.
\scene{Lys ned.}
\says{R}[sagt mest til sig selv] Hvor mange gange skal jeg sige det...

\end{sketch}

\end{document}

