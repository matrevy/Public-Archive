\documentclass[a4paper,11pt]{article}

\usepackage{revy}
\usepackage[utf8]{inputenc}
\usepackage[T1]{fontenc}
\usepackage[danish]{babel}


\revyname{MatematikRevyen}
\revyyear{2017}
% HUSK AT OPDATERE VERSIONSNUMMER
\version{1.1}
\eta{4 minut og 15 sekunder}
\status{færdig}

\title{Belejringsproblemer}
\author{Ulrik '14}

\begin{document}
\maketitle

\begin{roles}
\role{X}[Alexander] Instruktør
\role{X}[Lasse] Instruktør
\role{T}[Janne] Skuespiller i tårn
\role{F}[Marius] Vildfaren revyst
\role{B1}[Stolberg] Ridder 1
\role{B2}[Mikkel] Ridder 2
\role{B3}[Mads] Ridder 3
\role{C}[Toke] Chewbacca
\end{roles}

\begin{props}
\prop{Rekvisit}[Person, der skaffer]
\end{props}


\begin{sketch}
\scene{Lys op.}

\scene{T sidder i et tårn i den ene side af scenen iført et passende fjollet kostume.}
\scene{F står på scenen}
\says{T} Marius, du ødelægger nummeret…
\scene{F opdager, at lyset er oppe, og der er publikum.}
\says{F} Åh… fuck. Hurtigt, hurtigt: Hvad går det her ud på?
\says{T} Ud på?
\says{F} Ja, for helvede. Er det en sang? Er det en sketch? Hvad skal der ske?
\scene{T ser forvirret ud.}
\says{T} Det ved jeg da ikke.
\says{F} Hvad mener du med, at du ikke ved det? Har du ikke læst manus?
\says{T} Jeg har sgu da kun læst mine egne replikker.
\says{F} Nå… men dem kan vi måske bruge til at finde ud af, hvad det her går ud på.
\says{T} Jeg har slet ikke nogen replikker.
\says{F} Og så valgte du bare ikke at læse noget?
\says{T} Præcis.
\says{F} klaphat.
\says{T} Hey… jeg er i det mindste på scenen, når jeg burde være.
\says{F} Low blow! Men kan vi ikke bare gætte? Bandet spiller ikke, så det er nok en sketch!
\says{T} Hm… sandt. Og det handler sikkert om… et eller andet internt, som nogle af de gamle røvhuller har skrevet.
\says{F} Så… folk griner, hvis bare jeg siger… cohomologi?
\says{T} Eller… Ækvivariant?
\scene{Uanset om der reageres på replikkerne, fortsætter sketchen som uændret.}
\says{F} Ingen respons. Nå. Fair nok. Så er det sikkert noget med dumme ordspil.
\says{T} Så pointen er måske, at nutidens pres på de studerende er tårnhøjt?
\scene{F ser utilfreds ud.}
\says{F} Av. Nej… det gør vi ikke. Ikke på vilkår. Kunne det ikke bare være noget mindre kontroversielt… altså… en sketch om, at vi ikke kan lide russer, eller sådan noget?
\says{T} Vi kunne også bare kigge rundt i stedet for at gætte… der må være nogle rekvisitter eller sådan noget.
\says{F} Se, det var konstruktivt.
\scene{F kigger bag tårnet og kommer tilbage ud med en rambuk}
\says{F} Altså… Jeg har fundet den her dims, hvad end det her så er.
\scene{F holder en meget lang rambuk af billige materiale, som han skruer hovedet af.}
\says{T} Tja… jeg har ingen anelse om, hvad det dér skal forestille.
\scene{Tre folk iført fes og ridderrustninger kommer løbende ind på scenen med sværd. T giver sig straks til at genindtage sin position fra tidligere.}
\says{B1} I Mogens Dams navn vil vi…
\says{B1} stopper, fordi han bliver prikket på skulderen af B2
\says{B1}[irriteret] Hvad er der nu?! Jeg gør det altså præcis, som jeg skulle.
\says{B3} Jojo, men hvad fanden laver Marius her?
\scene{B3 peger over på F.}
\says{B1}[mere irriteret] Ej altså, for helvede Marius. Du er først med i næste nummer!
\says{F} Ja tak for info. Nu er jeg altså her.
\says{B2} Jamen, for himlens skyld, har du ikke læst manus?
\scene{F rødmer og kigger ned i Jorden.}
\says{B3} Har du… slet ikke læst manus? Noget af det?
\says{F} ...Måske.
\says{B1} Nå… så skrub dog af scenen og lad sketchen fortsætte. Eller… mens du går ud, så ræk os lige rambukken.
\says{F} Mener du den her?
\scene{F samler rambukken, der er i stykker op.}
\says{B3} Jamen… Jesus Christ, har du smadret rambukken?
\says{F} I mit forsvar så er det altså en rimelig ringe rambuk, hvis man bare sådan kan pille den fra hinanden. Hvad skal I overhovedet vælte?
\says{B1+B2+B3} Tårnet for helvede!
\says{F} Tårnet? Hvad?! Det dér?
\scene{F gestikulerer mod kæmpetårnet.}
\says{B1+B2+B3} Ja!
\says{F} Det kan I altså ikke være bekendt. Der er jo endnu en fucking forestilling i morgen! Tænk på den stakkel i rekvisitten, der skal bruge hele natten på at reparere det!
\scene{F kigger mig bestemt på Jakob ude i publikum.}
\says{F} Sig mig Jakob, synes du det her er sjovt?! Først farvekridtkostumer, så en mammut og nu det her!
\scene{I det kommer Chewbacca løbende ind på scenen.}
\says{B1} Nej for helvede! Det er ikke din sketch, det her!
\scene{Chewbacca kigger bedrøvet ned, laver en trist wookielyd, vender sig om og forlader scenen igen. I mellemtiden vandrer B2 om bag tårnet og finder et detonationsapparat.}
\says{B2} Se! Vi kan bare bruge Plan H!
\says{F}[Overrasket] Plan H?
\says{B1+B3} Ahh, Plan H, pissegod idé.
\says{F} Hør nu her: I vil vel ikke sprænge tårnet? Vi må slet ikke have åben ild på scenen.
\says{B1}[Suk] Hør nu, lille pus, flyt dig nu, før du kommer til skade og lad de rigtige revyster om at lave revy. Spring det!
\scene{B2 griner højt og diabolsk og der kommer røg på scenen, men der sker ikke så meget mere. Indsæt AV.}
\says{B3} Ej, hvor er det hele altså noget lort! Kom, sketchen er ødelagt.
\scene{B1, B2 og B3 stille på række og begynder at ridde ud.}
\says{F} Vent. Hvad fanden gik sketchen overhovedet ud på?
\says{B1} Vi er da åbenlyst fysikrevyster.
\says{F} ...fysikrevyster. Hvordan det?
\scene{Alle tre peger de på deres respektive fes.}
\says{F} Og det skulle man kunne genkende det på. Hvad fanden er joken?
\says{B2} Altså… før du ødelagde det hele var joken, at fysikrevyen har store dumme rekvisitter, som de bare spilder på det rene ingenting!
\says{F} Og… skulle det være sjovt?
\says{B3} Hvad har det nu med noget at gøre?
\says{F} Men… skal folk ikke grine?
\says{B1} Publikum er fulde! De griner af hvad som helst!
\says{F} Ahh, come on. Kan i ikke finde på noget bedre?
\scene{B1+B2+B3 holder samråd}
\says{B1} Okay okay, \act{rømmer sig} så meningen med sketchen er at presset på de studerende er tårnhøje.
\says{F} Ej, ikke igen.
\scene{Lys ned.}
\end{sketch}
\end{document}