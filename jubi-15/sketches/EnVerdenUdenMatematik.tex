\documentclass[a4paper,11pt]{article}

\usepackage{revy}
\usepackage[utf8]{inputenc}
\usepackage[T1]{fontenc}
\usepackage[danish]{babel}

\revyname{Matematikrevy}
\revyyear{2008}
\version{1.0}
\eta{$6.5$ minutter}
\status{Færdig}

\title{En Verden Uden Matematik}
\author{Ukendt}

\begin{document}
\maketitle

\begin{roles}
\role{X}[NB] Instruktør
\role{M}[Christoffer] Matematiker
\role{F1}[Michael] Fysiker
\role{F2}[Rasmus] Fysiker
\role{K1}[Brandt] Kemiker
\role{K2}[Lise] Kemiker
\role{D1}[Jonas] Datalog
\role{D2}[Toke] Datalog
\role{N}[Shake] Nano
\role{H}[Stig] Husmor
\role{Da}[Marius] Dansklærer
\end{roles}

% \begin{props}
% \prop{Rekvisit}[Person, der skaffer]
% \end{props}

\begin{sketch}
\scene{MÅSKE: En lidt kæk forelæser går frem og tilbage mellem scenens sider, alt imens eksemplerne udspiller sig i den modsatte af hvor han går hen bag mellemtæppet. Tæppet skal køre flydende, i samme tempo i begge sider og ca. i samme tempo som ham. De to fysik-eksempler i starten får ikke tæppet for i mellemtiden, men fryser dog mens forelæseren snakker. Dette foregår i højre side af scenen for publikum, og efter dette skifter eksemplerne side efter hver. Gælder for alle, at de fryser, mens forelæseren taler. Der bliver sat markeringer og streger der viser hvor på tavlen man må skrive, hvor man må stå for at publikum kan se, samt hvor langt tæppet skal trækkes hen.}

\says{M} Godaften. I dag skal vi se på hvordan verden ville se ud uden matematik. Vi ser på hvordan specifikke situationer ville blive påvirket, hvis man fjernede bestemte dele af matematikken. Ja. Lad os starte med at fjerne integralregningen og se hvad fysikerne så ville gøre hvis de skulle finde arealet under en kurve.

\scene{Man ser fysikere tælle kvadrater under en graf på en tavle. Pludselig indser hans ven, at der er plads til en ekstra kasse et sted, hvorefter de begynder at tælle forfra. Efter replikken 'Kaffepause' går forelæseren i gang igen.}

\says{M} Ja, det gjorde jo ikke nogen forskel. Lad os prøve at tage skrapere midler i brug. Lad os se hvad der sker hvis vi fjerner al differentialregning, når kemikerne skal finde ud af hvordan de får mest udbytte af deres reaktion

\says{K1} Vi har denne ligning for udbytte, og vi skal nu finde den maksimale værdi.

\says{K2} Lad os prøve med x = 2. Det giver 5... Det lyder ret maksimalt.

\says{M} Kemikere. Ja, de har jo lige fået en revy, så vi bliver nødt til at nævne dem. Vi kan nu prøve at fjerner talteori og algebra. Disse personer skal sende en hemmelig besked.

\scene{Tekst på tavlen: Koden til DIKU er Coca Cola}

\says{D1} Hmm, hvad hvis vi oversætter den til Klingon?

\says{D2} Nej, det er der for mange der forstår. Hvad med fransk?

\says{D1} Ja, det må være det mest sikre.

\says{M} Man bliver helt tørstig. Nanoteknologi!!!, hvad ville der ske, hvis man fjernede potensregning.

\says{N} Resultatet; det endelige resultat, konklutionen på tre års arbejde bliver 0,00000... ahh, lad os bare sige 0.

\says{M} De har altid manglet potens. Selv i madlavning ville det at fjerne en ellers så indviklet ting som udvalgs-aksiomet medføre store forandringer

\scene{Tekst på tavlen: 'Juice: Pres 2 Appelsiner'.}

\says{H} Æv, nu har jeg kun én appelsin

\says{M} Og til sidst, ja det var da på tide: Selv et så fuldstændigt ligegyldigt fag som dansk bliver påvirket hvis vi fjerner \emph{al} matematik.

\scene{Dansklæreren står med en bog, som hun bladrer rundt i.}

\says{Da} Og hvis i vil slå op på side... PIS!
\end{sketch}

\end{document}
