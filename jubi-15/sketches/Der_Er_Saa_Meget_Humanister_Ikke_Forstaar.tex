\documentclass[a4paper,11pt]{article}

\usepackage{revy}
\usepackage[utf8]{inputenc}
\usepackage[T1]{fontenc}
\usepackage[danish]{babel}

\revyname{Matematikrevy}
\revyyear{2007}
\version{1.0}
\eta{$1$ minutter}
\status{Færdig}

\title{Der er så meget humanister ikke forstår}
\author{Ukendt}

\begin{document}
\maketitle

\begin{roles}
\role{X}[William] Instruktør
\role{Sp}[Eigil] Speaker
\role{F}[NB] Forelæser
\role{H}[Toke] Humanist
\end{roles}

% \begin{props}
% \prop{Rekvisit}[Person, der skaffer]
% \end{props}

\begin{sketch}
\scene{Scenen er delt i to af en dør. Der er en forelæsning igang på den ene side - på den anden side står en humanist og lytter ved døren.}

\says{F} Så stryger vi $n$'erne på begge sider.

\scene{Lyset går og drømme-agtig musik lyser - vi skifter til humanistens tankeverden - lyset kommer igen.}

\says{H} Så stryger vi $n$'erne på begge sider.

\scene{Nu står der en person på den anden side af døren og stryger nogle ænder på begge sider.}

\says{Sp} Al-gebra, anal-yse og tyngdekraft -- der er så meget humanister der ikke forstår.
\end{sketch}

\end{document}
