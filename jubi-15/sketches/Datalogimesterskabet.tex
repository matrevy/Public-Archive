\documentclass[a4paper,11pt]{article}

\usepackage{revy}
\usepackage[utf8]{inputenc}
\usepackage[T1]{fontenc}
\usepackage[danish]{babel}

\revyname{Matematikrevy}
\revyyear{2016}
\version{1.0}
\eta{$9$ minutter}
\status{Færdig}

\title{Datalogimesterskabet}
\author{Ulrik '14, Julie '15}

\begin{document}
\maketitle

\begin{roles}
\role{Sp}[aDA!] Speaker
\role{xXx}[Jonas] xXxJegBollerDinMor1337xXx (pwnie og 1337 h4ck0r)
\role{F}[Eigil] FrækFyr68 (Ifølge hans mor, en veritable sex machine)
\role{T}[Toke] Twilight Sparkle (Rus)
\role{J}[Lise] Julie (pigedatalog)
\role{D}[Anne G] Delbert Datastrøm (Gammelt røvhul, kommer lige fra mors kælder)
\role{K}[Stine L] Kvinde
\end{roles}

% \begin{props}
% \prop{Rekvisit}[Person, der skaffer]
% \end{props}

\begin{sketch}
%Datalogdiscipliner:
%Mur af tændstiksæsker v
%Formå at komme forbi en kvinde (vel at mærke UDEN at interagere med hende) v
%Skal fejle i et induktionsbevis v
%Tegn et træ v
%Forfejl et induktionsbevis v
%Få dispensation til et fjerde eksamensforsøg i Algoritmer og Datastrukturer
%Synge i morgen er verden vor v
%Coladrikning om kap.
\scene{Lys op. I højre side af scenen sidder Sp bag et kommentatorbord, eller står ved en pult. I venstre side af scenen står en tavle og midt på scenen er en ca. 5 cm høj mur af tændstiksæsker.}

\says{Sp} Godaften mine damer og herre. Velkommen til endnu en hæsblæsende kulmination på et helt års datalogi. Det er jo i dag, at vi i Matematikrevyen skal overvære dette års Datalogmesterskab LIVE fra Store UP1, hvor vejret er fornemt, og lyset skruet ned til datalogvenlige niveauer. \act{Kunstpause, tager sig til øret} Og  jeg hører netop i min øresnegl, at deltagerne er ved at være klar.

\scene{xXx træder ind på scenen med en pepsiflaske i hånden som han tager den sidste slurk af. Spot på xXx.}

\says{Sp} Første mand i manegen er pwnie-legenden xXxJegBollerDinMor1337xXx \act{udtales 'trippel x jeg boller din mor tretten syvogtredive trippel x}, der med en kampvægt på lige over 200 kg plus cola har domineret supersværvægtskategorien hele året. Hvis vindforholdende ikke er for voldsomme, skulle han nok kunne hive en sikker sejr i land.

\scene{F træder ind på scenen. Spot på F.}

\says{Sp} Vores næste deltager er kendt fra Arto som Frækfyr68, og han er ifølge sin mor en veritabel 'sex machine' \act{Laver gåseøjne}, som både er fuldt funktionel og anatomisk korrekt. Han udtaler selv, at han i tæt løb med sin bror, lige akkurat missede Frækfyr69.

\scene{T træder ind på scenen. Spot T.}

\says{Sp} Næste levende billede er en rus der går under navnet Twilight Sparkle. Han er oppe imod et hårdt felt i dag, men studiet har endnu ikke knækket hans kamplyst, og hvem siger i øvrigt, at man bliver en bedre datalog af at læse på DIKU?

\scene{J træder ind på scenen og retter på falsk overskæg.}

\says{Sp} Næste mand er...  \act{tøvende} Ju-lie? \act{småforvirret} Julie? Det er ikke et datalognavn.

\scene{Delbert vader sådan lidt klamt ind på scenen. Spot på Delbert.}

\says{Sp} Sidst, men ikke mindst, har vi det residerende gamle røvhul, Delbert Datastrøm, der her efter sit niende forsøg på at vinde titlen måske burde overveje at foretage sig noget andet med sit liv.

\scene{Datalogerne står lidt og kigger rundt på hinanden. De vinker lidt fjollet. Skærmer sig evt mod lyset.}

\says{Sp} Og vi er klar til start. Foran os venter et væld af discipliner, som kun den mest datalogede datalog kan komme igennem. Modsat DIKUs siddende studienævn, tror vi nemlig ikke på gratis point. Nu venter vi bare på at startsignalet går...

\scene{Lyd af startsignal der går.}

\says{Sp} Og Julie er UDE, da hun er en pige.  \act{Peger mod en sceneudgang}

\scene{J er dybt utilfreds og river det falske overskæg af. Hun stormer demonstrativt af scenen.}

\says{Sp} Delbert ligger godt ben i men er desværre ikke lige så frisk, som han har været. xXxJegBollerDinMor1337xXx er stadig forpustet efter at have trådt op på scenen og han har derfor valgt at støtte sig til Twilight Sparkle, der som så mange andre russer før ham er blevet fanget af en ældre datalog.

\scene{xXx griber fat i kraven på Twilight Sparkle der forsøger at slippe fri. FrækFyr68 styrter videre til tavlen, hvor han giver sig i kast med kridtet.}

\says{Sp} I mellemtiden er Delbert nået frem til aftenens første disciplin: at Forfejle et Induktionsbevis. Lige efter ham følger FrækFyr68. Men åh nej, det går jo alt for godt. FrækFyr68 får stillet betingelserne op og tjekket tilfældet n=0. Og man kan se at han godt ved det er for tæt på at være rigtigt.

\scene{Cue: På 'Ældre datalog' skal Delbert nå tavlen. På 'Forfejle et Induktionsbevis skal FrækFyr68 nå tavlen.}

\says{Sp} Samtidig er Delbert, strålende af selvtillid, igang med manuelt at tjekke samtlige tilfælde fra n=0 op til 5, og han er derfor videre. Sjældent har Store UP1 set så ringe eksempler på induktion.

%Twilight Sparkle og FrækFyr68 ligger dog nøjagtig skulder ved skulder, idet de kommer til dagens første udfordring: Tændstiksmuren.

%\scene{FrækFyr68 hopper fint over, men Twilight Sparkle snubler ind over tændstiksæskerne.}

\scene{Delbert rykker videre til døren til venste for scenen, hvor han giver sig til at skråle I Morgen Er Verden Vor. Twilight Sparkle smider xXx arm over sine skuldre og hjælper ham hen til tavlen.}

\says{Sp} Imens har xXxJegBollerDinMor1337xXx og Twilight Sparkle langt om længe kæmpet sig over til tavlen, og Delbert Datastrøm er fornemt i gang med at gjalde I Morgen Er Verden Vor. Det er en fryd at opleve så tonedøv en fortolkning. I mellemtiden har FrækFyr68 langt om længe fundet ud af ikke at foretage induktionsskridtet korrekt, og med sine medkonkurrenter lige i halen skynder han sig videre for at synge. Bemærk, hvor selvsikkert Twilight Sparkle ikke den fjerneste anelse har om, hvad induktion er, og han er fluks videre. xXxJegBollerDinMor1337xXx skal dog lige hidkalde sig den dårlige idé, før han selv kan komme med. Imens er Delbert på vej videre til den tredje disciplin.

\scene{Cues: På 'tonedøv fortolkning' begynger FrækFyr68 at finde ud af ikke at foretage induktionsskridtet korrekt. På 'synge' begynder Twilight Sparkle at se forvirret ud. På 'fluks videre' hidkalder xXx sig den dårlige ide. På "Twilight Sparkle ikke den fjerneste anelse..." bliver Delbert færdig med at synge og begynder at løbe mod 'Kvinden'. }

\scene{Når Delbert når 'kvinden' stopper han op og stirrer lamslået på det hunkønsvæsen, der står foran højre dør.}

\says{Sp}  \act{Pause} Men, åh nej... Delbert stopper målløs op da han når den ubestridt sværeste disciplin i hele konkurrencen - han skal forbi "Kvinden" \act{Tryk på Kvinden}. Mangt en datalog har givet sig i kast med denne udfordring over årene, men kvindens mytiske status på datalogisk institut gør det mere end umuligt at nærme sig dem uden at falde måbende sammen.

\scene{Cues: På 'mytiske status' bliver FrækFyr68 færdig med at synge og løber (den lange vej, high-fiver måske folk på vejen) videre. På 'mere end umuligt' bliver xXx ligeledes færdig med at synge og løber efter. }

\scene{Delbert står paralyseret, mens de andre konkurrenter får sunget færdig og når frem til. Undtagen T, som hiver sin mobil frem.}

\says{Sp} FrækFyr68 og xXxJegBollerDinMor1337xXx er blevet færdige med at synge, mens Twilight Sparkle febrilsk prøver at google teksten til I Morgen Er Verden Vor. Hans rusvildledere må være meget skuffet. Men hvad er dog det? FrækFyr68 ser ud til at ville gøre et forsøg på at komme forbi "Kvinden" \act{tryk på Kvinden}.

\scene{F træder frem og prikker til Kv. }

\says{F}[Nervøst] Bry-bry-bry-bryster \act{Ryster på hovedet og siger} Um, øh, h-hej.

\says{Sp} Og vi er allerede nået til aftenens anden diskvalifikation, for ingen ægte datalog ville snakke med nogen eller noget, der ikke har et Y-kromosom. %FrækFyr68 må derfor nyde resten af forestillingen fra sidelinjerne.
Hvad skuer mit øje? "Kvinden" gengælder FrækFyr68's tilnærmelser og han falder til jords.

\scene{Cues: på 'Ægte datalog' begynder 'Kvinden' at gengælde FrækFyr68 tilnærmelser. }

\scene{F ser lidt slukøret ud, men får rent faktisk et smil og et kys på kinden fra pigen. F besvimer. Kv tager ham i benene, slæber ham ud af døren og lukker døren igen under næste replik.}% og de løber smilende ud af UP1 sammen.}

\says{Sp} Hvilken rørende, sød og gennemført udatalogisk afslutning på hans deltagelse. Samtidigt åbner det fuldstændigt for løbet igen og Delbert lægger sig stærkt i spidsen på vej mod scenen, skarpt efterfulgt af xXxJegBollerDinMor1337xXx, der ligger godt i svinget. Sidst og absolut mindst kommer Twilight Sparkle der febrilsk har fået stammet sig igennem en næsten vederstyggeligt toneren udgave af I Morgen Er Verden Vor. %Han burde nok have hørt lidt bedre efter inden forestillingen.

\scene{Cues: På 'Samtidigt åbner...' løber xXx og Delbert videre (Delbert forrest). På 'Delbert lægger...' bliver Twilight Sparkle færdig med at synge og begynder at løbe mod scenen. }

\scene{På vej mod scenen river xXx Delbert ned så han vælter. }

\says{Sp} Vores deltagere er tydeligvis fysisk udfordret. Jeg har ikke set tre ansigter så røde, siden jeg blottede mig for en flok ruspiger sidste år. Vores supersværvægtsmester er dog lige en kaliber eller to hurtigere end sine konkurrenter og har nået næste disciplin: Tændstiksæskemuren. Han tager et enkelt kig på muren, og det er tydeligt at han ikke kan overskue udfordringen ved at krydse så høj en barriere.
Imens indhentes han af Delbert, der selvsikkert- \act{kuntspause} ikke kommer over muren.

\scene{Cues: På 'supersværvægtsmester' skal xXx nå muren. På 'barriere kommer Delbert ind på scenen' . På 'han af Delbert' hopper Delbert mod muren. }

\scene{Delbert maveplasker muren.}

\says{Sp} Twilight Sparkle gir et friskt hop i fuld gallop hen over muren, mens xXxJegBollerDinMor1337xXx endelig har besluttet bare at gå uden om. De når alle samtidig aftenens næstsidste datalogiske udfordring: At sende en dispensationsansøgning om et fjerde eksamensforsøg til algoritmer og datastrukturer.

\scene{Cues: På 'Twilight Sparkle gir...' skal xXx rejse sig og begynde at gå uden om.
På 'mens xXxJegBollerDinMor1337xXx...' rejser Delber sig og løber videre.
På 'datastrukturer' skal Twilight Sparkle begynde at være fortvivlet over han ikke kan finde den korrekte formular.}

%Og hvad er det?! Twilight Sparkle giver op og hælder resten af cola'en ud. Dette er selvfølgelig dybt ureglementeret og uhørt, men jeg kan ikke se det mindste, vi kan stille op overfor denne grumme usportslighed. Og stakkels Delbert Datastrøm drikker fortsat stille og roligt af sin flaske,

%mens hans konkurrenter er nået til

%\scene{Delbert formår endelig at få gjort sin cola færdig, men er blevet træt og går derfor langsomt tilbage til scenen.}

%\says{K} Og det bliver et nydeligt spring fra FrækFyr68, men Twilight Sparkle snubler lige ind i muren og bliver straks overhalet af Delbert Datastrøm med et flot skridt hen over muren. Twilight Sparkle ligger dog i vejen for xXxJegBollerDinMor1337xXx, der slet ikke kan overskue, hvad han skal gøre ved denne udfordring.

\says{Sp} Det tegner ikke godt, Twilight Sparkle kan overhovedet ikke finde den korrekte formular på KU-net. Derimod finder xXxJegBollerDinMor1337xXx selvsikkert den rette ansøgning, men hvad sker der? Den bliver afvist! Det viser sig nemlig, at xXxJegBollerDinMor1337xXx, modsat langt størstedelen af DIKU's befolkning allerede har bestået algoritmer og datastrukturer. Han må derfor selvfølgelig forlade mesterskabet, og dysten står nu mellem et gammelt røvhul og en rus, der trods massivt pres endnu ikke er droppet ud.

\scene{Cues: På 'Ku-net' skal xXx begynde at hæve armene i vejret i sejr. På 'massivt pres' skal Delbert begynde at bevæge sig hen for at aflevere. Han når dog ikke at aflevere før efter han har råbt 'DNUR'. På 'droppet ud' skal Twilight Sparkle begynde at finde papirer frem, juble, og febrilsk udfylde papirene.}

\says{Alle} \act{Råber mod TwilightSparkle}  DNUR!!!

\says{Sp} Og mens Twilight Sparkle endelig har fundet de korrekte dokumenter, stryger Delbert Datastrøm igennem dispensationsansøgningen - studienævnet kender ham jo så godt i forvejen.

\scene{Cues: På 'forvejen' skal Twilight Sparkle spæne over og aflevere.}

\scene{T jubler åbenlyst og begynder at udfylde formularer, mens D bare kommer og indtaster sin svenske nummerplade.}

\says{Sp}  Og det gamle røvhul ligger dermed i spidsen hen mod målstregen, hvor den sidste udfordring venter: At tegne... et træ. Og de er i gang. De skribler og skribler, men Delbert Datastrøms gamle hånd er ikke lige så hurtig på en tavle, som de begejstrede men upræcise fakter fra vores rus her. Det ser ud til at blive tæt løb her på falderebet, og de bliver færdige præcis samtidigt! Men hvad er nu det? Twilight Sparkle har tegnet... et rigtigt træ, og vi må derfor erklære Delbert Datastrøm SEJRHERREN af datalogimesterskaberne 2016. Så kan han måske endelig komme videre med sit liv.

\scene{Cues: på 'præcis samtidig' skal både Twilight Sparkle og Delbert smide armene i vejret for at indikere de er færdige}

\scene{Delbert bliver overrakt Datalogmesterskabsbæltet af O.}

%\says{K} Det var alt for i år, men husk, at der stadig er masser af fjollede dataloger, som du kan observere i din hverdag.

\scene{Folk forlader scenen og lyset fader.}
\end{sketch}

\end{document}
