\documentclass[a4paper,11pt]{article}

\usepackage{revy}
\usepackage[utf8]{inputenc}
\usepackage[T1]{fontenc}
\usepackage[danish]{babel}

\revyname{Matematikrevy}
\revyyear{2017}
\version{1.0}
\eta{$1.5$ minutter}
\status{Færdig}

\title{Bevis Det}
\author{Ukendt}

\begin{document}
\maketitle

\begin{roles}
\role{X}[MaWeK] Instruktør
\role{M}[Christoffer] Mand
\role{K}[Stine L] Kvinde
\role{D}[Freja E] Drømmepige
\role{H1}[Arnvig] Hjælper
\role{H2}[Kristian] Hjælper
\role{H3}[Marius] Hjælper
\role{H4}[René] Hjælper
\end{roles}

% \begin{props}
% \prop{Rekvisit}[Person, der skaffer]
% \end{props}

\begin{sketch}
\scene{Lys op.}

\says{K} Jeg ved ikke, hvordan jeg skal kunne se dig i øjnene. Hvordan skal jeg nogensinde kunne stole på dig igen!?

\says{M} \act{Går ned på knæ og trygler.} Men skat, jeg elsker dig jo virkelig. Der må da være noget jeg kan gøre?

\says{M} \act{Vender ryggen til og slår armene over kors.} Hvis du virkelig elsker mig, så bevis det!

\scene{Kvinden forlader scenen med faste skridt, samtidig begynder 'Eye of the Tiger' at spille. Hjælperne kommer på scenen med rekvisitter (kalkulus, kaffe, papir, bord, stol, tavle). Lige når sangen slår an tager han imod kalkulus og slår op. Han modtager kaffe, drikker, giver den væk i modsat retning. Han går over scenen modtager en kuglepen, et bord bliver sat foran ham, han begynder at sætte sig, og en stol bliver placeret under ham i samme bevægelse. I takt til musikken flyver papirerne væk under hans pen, han drikker mere kaffe. Han rejser sig til en tavle (der er blevet kørt ind). Borde og stole forsvinder. Han skriver ekstremt hurtigt på tavlen, vender sig væk mens han skriver og drikker mere kaffe. Han får et kridt mere og skriver med begge hænder. Slutter med en firkant. Han stiller sig midt på scenen med triumferende udbredte arme, imens han få en jakke(sætsjakke) på. Kæresten kommer ind igen. Han vender sig og sætter på knæ i samme bevægelse foran hans kæreste. Hun holder sig for munden, musikken stopper. Han rækker ind i sin inderlomme, trækker et stykke papir frem, og folder det ud foran hende.}

\says{M} Desværre, jeg har fundet et modeksempel.

\scene{Drømmepige går ind på scenen, manden lægger armen om hende, og de forlader sammen scenen. Lys ned.}
\end{sketch}

\end{document}
