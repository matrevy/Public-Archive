\documentclass[a4paper,11pt]{article}

\usepackage{revy}
\usepackage[utf8]{inputenc}
\usepackage[T1]{fontenc}
\usepackage[danish]{babel}

\revyname{Matematikrevy}
\revyyear{2011}
\version{1.0}
\eta{$6.5$ minutter}
\status{Færdig}

\title{Niels Bohr Science Park}
\author{Ukendt}

\begin{document}
\maketitle

\begin{roles}
\role{X}[Shake] Instruktør
\role{A}[Michael] Arkitekt
\role{J}[Anne G] Journalist
\end{roles}

% \begin{props}
% \prop{Rekvisit}[Person, der skaffer]
% \end{props}

\begin{sketch}
\scene{Beskrivelse}

\scene{Vi befinder os udenfor Niels Bohr Science Park. Båndet er lige blevet klippet over, og den er blevet indviet}

\says{J} Den nye Niels Bohr Scince Park er netop indviet. Fornavn Efternavn er hovedarkitekt på projektet. Vil du ikke introducere os for de nye tiltag?

\says{A} Jo, selvfølgelig! Lad os starte ved auditorierne. Forestil dig StUP1 og Lille UP1 slået sammen. *Gør sådan her med hænderne* Vi har kaldt det Mega UP1! Vi har placeret forelæseren i midten af lokalet, og de studerende sidder på begge sider af ham.

\says{J} Men hvad med dem, der sidder bag ham. Kan de overhovedet se noget?

\says{A} SAGTENS! Vi har installeret en roterende tavle i midten af lokalet. På denne måde behøver forelæseren ikke at flytte sig, men kan bare blive stående og skrive.

\says{J} Javel. Men er det det eneste auditorium, I har?

\says{A} JA! Med et samlet auditorium kan vi skabe en synergieffekt mellem alle studierne på Science KU.

\says{J} Men hvis I kun har et auditorium, hvordan skal alle så nå at få deres forelæsninger?

\says{A} SIMPELT! Vi indfører konceptet multiforelæsninger. Med to sider på tavlen kan det lade sig gøre at have to forelæsninger på en gang. Men du skal høre om kantinen! *skifter plantegning* Vi har prøvet at afskaffe de monopoltilstande, der normalt hærger en kantines priser. Derfor har vi fire kantiner, som ligger i hver sit hjørne af lokalet. Desuden har vi minimeret pladsforbruget ved at forene kantinernes maddepoter og kemilaboratorierne. Forestil dig bare, at ostene kan opbevares stinkskabet, og kantinernes køleskabe kan jo også bruges til sølvnitrat og nitroglycerin.

\says{J} Øøøøh, det lyder spændende. Hvor er toiletterne henne?

\says{A} Dem har vi gemt væk, for toiletter er så grimme.

\says{J} ...Men hvor er de?

\says{A} Der er desværre ikke nogen pigetoiletter. Da vi designede NBSP havde vi kun dataloger at spørge til råds, så det eneste, vi har bygget, er handicaptoiletter. Men du skal til gengæld høre om vores nye energispareplan! Vi har fundet ud af, at det er datalogerne, som bruger mest strøm i løbet af deres uddannelse, så vi har koblet en masse motionscykler til deres computere, så de selv kan forsyne dem med energi.

\says{J} I kan da aldrig få en datalog op på en motionscykel!

\says{A} Jojo, for der er indbygget håndfri cola!

\scene{A hiver datalogølhat frem}

\says{A} For hvert tråd i pedalerne kommer der en lille smule cola ud af en slange, som kan monteres på hovedet.

\says{J} Jeg kan se, at bygningen er malet i en skrigende rød farve. Hvorfor dette ekstreme valg?

\says{A} ÅRH! Det er ikke bare malet rødt; vi har udnyttet naturvidenskaben og lagt nanopartikler ud over hele bygningen, så den afspejler Henrik Busch? humør. Når den er rød, er det fordi, han er sur.

\says{J} ...Så bygningen har kun én farve?

\says{A} Nej nej, den kan også blive MERE rød!

\says{J} Jeg synes, at bygningen virker meget høj. Er der nogen elevatorer?

\says{A} NEJ! Der er trapper! Masser af trapper! Vi har endda lavet trapper, der begynder og slutter på samme etage, og igen, ved hjælp af naturvidenskaben, har vi fået konstrueret verdens første uendelige trappe

\scene{Penrose trappe på en plantegning}

\says{J} Hvad så med markiserne? Hvordan fungerer de?

\says{A} Vi har lært af fejltagelserne på HCØ, så de fungerer nu helt efter hensigten... Til gengæld er der ingen vinduer. De hæmmer datalogernes arbejdsindsats.

\says{J} Hvad er der gjort for at forbedre studiemiljøet?

\says{A} Jo, kender du fysikernes legestue? Vi har taget fysikernes legestue og slået det sammen med kemikernes bollerum. Det er synergi. En sexgynge er et udmærket eksempel på er harmonisk oscillator, som fysikerne kan observere.

\says{J} Studenterrevyerne er jo efterhånden blevet en meget integreret del af miljøet på Naturvidenskab. Er der et sted, hvor vi kan afholde revy?

\says{A} Det kan du tro! Vi har bygget en flot tribune ude på plænen. Folk elsker jo udendørs forestillinger.

\says{J} Men Matematik Revyen ligger jo i december!

\says{A} Netop!

\scene{Lys ud}
\end{sketch}

\end{document}
