\documentclass[a4paper,11pt]{article}

\usepackage{revy}
\usepackage[utf8]{inputenc}
\usepackage[T1]{fontenc}
\usepackage[danish]{babel}

\revyname{Matematikrevy}
\revyyear{2017}
\version{1.0}
\eta{$4$ minutter}
\status{Færdig}

\title{Helvede}
\author{Ukendt}

\begin{document}
\maketitle

\begin{roles}
\role{X}[William] Instruktør
\role{R}[Marius] Receptionist
\role{S}[aDA!] Studerende
\role{F}[Michael] Forelæser
\role{H1}[Kristian] Helvedes Statist
\role{H2}[Lise] Helvedes Statist
\role{H3}[Stig] Helvedes Statist
\role{G1}[Loke] Gospelkor
\role{G2}[NB] Gospelkor
\role{G3}[Tina] Gospelkor
\end{roles}

% \begin{props}
% \prop{Rekvisit}[Person, der skaffer]
% \end{props}

\begin{sketch}
\scene{K falder ind på scenen. R står og læser i et magasin ved bagtæppet, og i midten er en stor dør.}

\says{R} Velkommen til Helvede! Because you are worth it.

\says{G} Worth it.

\says{R} Ja, du er død.

\says{K} What?! Hvordan kan jeg være kommet i Helvede? Jeg har betalt 50 kr. til Amnesty hver måned de sidste 5... uger.

\says{R} Ved du ikke, at alle pengene går til administration? Du er kommet i Helvede og sådan er det. Stik mig så din rapport.

\says{G} Rapport

\says{K} Rapport? Hvad er det for noget?

\says{R} Din rapport er det stykke papir, der skal fortælle mig, hvilken afdeling jeg skal sende dig til. Men hvis du ikke har den med, må jeg vel bare vise dig rundt. Så kan du selv
vælge, hvor du føler, du passer bedst ind.

\says{K} Men hvor længe skal jeg være her?!

\says{R} Du skal kun være her indtil Niels Bohr-bygningen er færdig.

\says{G} Aldrig

\says{K} Men, men ...

\scene{R og K går over til døren. K trækker i håndtaget, og døren åbner ind til et lokale fyldt med lidende fysikere.}

\says{R} Her har vi afdelingen for folk, der skriver sine opgaver i maple. Det er primært mindre begavede studerende, som alligevel ender med at droppe ud. Det er det grimmeste sted i Helvedet. Vi har straffet dem ved at få dem til at knække et uløst matematisk problem, som er formuleret af en fem-årig på indersiden af en tændstiksæske. 

\says{K} Åh, gud...

\says{R}[tørt] Nej...

\scene{R værdstætter ikke joken, og trækker igen i håndtaget. Døren åbner op til folk der lider til Partyalarm (spilles på AV)}

\says{K}[råbt] Hvad sker der derinde?

\says{R}[råbt] Det er blot afdelingen for folk, der ikke lukker døren efter sig kl. 17. Så kan de få lov at party'e lidt til PartyAlarmen.

\says{G} synger alarmlyde.

\scene{Døren lukker igen og alarmen stopper.}

\says{K} For søren - Jeg mener - for helvede da!

\scene{F træder ind på scenen i en ubåddragt med lidt blod på buksebenet og en sort sæk på skulderen, hvor der stikker noget af en fod ud, og går over til R.}

\says{F} Undskyld mig, men hvor er afdelingen for folk der har sejlet for hurtigt i Køge Bugt?

\says{R} Ja, det er så nede ad gangen, forbi afdelingen for cykelturister og elløbehjulsbenyttere, til højre for Kristian Peter Poulsen, hen ad gangen med matematikrevyens manusgruppe, og til venstre ved lokalet for folk, der tager sokker af til sidst.

\says{F} Ah, tak!

\scene{K ser lettere forfærdet ud}

\says{K} Ej... Jeg kan altså ikke rigtig se mig selv i NOGEN af de her grupper.

\says{R} Ok, jamen, så kan du bare vende tilbage til livet.

\says{K} Kan man virkelig det?

\says{R} Ja, selvfølgelig. 

\says{G} Selvfølgelig.

\says{R} Vi vil jo ikke putte folk ned i en eller anden kasse, hvor de ikke føler, de hører til. Det ville være totalt politisk ukorrekt. Udgangen er her.

\scene{R trækker i håndtaget og der åbnes op til mørke.}

\says{K} Ok, nice.

\scene{K går ind ad døren.}

\says{K} Det her ligner altså ikke en udgang...

\scene{Ond latter på AV. En masse hænder dukker op og lukker døren bag K. Kvindeskrig på AV.}

\says{R} Og det er det der sker, når man ikke bruger varepsilon.
\says{R} Og det er det der sker, når man råber 'det sagde hun også i går'


\scene{Lys ned.}
\end{sketch}

\end{document}
