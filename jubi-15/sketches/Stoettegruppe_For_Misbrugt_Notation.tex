\documentclass[a4paper,11pt]{article}

\usepackage{revy}
\usepackage[utf8]{inputenc}
\usepackage[T1]{fontenc}
\usepackage[danish]{babel}

\revyname{Matematikrevy}
\revyyear{2017}
\version{1.0}
\eta{$6$ minutter}
\status{Færdig}

\title{Støttegruppe for Misbrugt Notation}
\author{Ulrik '14, Erik '15}

\begin{document}
\maketitle

\begin{roles}
\role{X}[Freja S] Instruktør
\role{S}[Michael] Støtteperson / Fysiker
\role{Lig}[Toke] $=$ (Lighedstegn)
\role{Pil}[Brandt] $\Rightarrow$ (Implikationspil)
\role{Del}[Jonas] $\subset$ (Delmængde)
\role{Dot}[Lise] $\langle \cdot , \cdot \rangle$ (Indre produkt)
\role{Dif}[Christoffer] $d / dt$ (Differentialoperator)
\end{roles}

% \begin{props}
% \prop{Rekvisit}[Person, der skaffer]
% \end{props}

\begin{sketch}
\scene{Lys op.}

\scene{S sidder i en halvcirkel af puder midt på scenen. Ind træder Lig. S rejser sig og kommer over og giver Lig et kram.}

\says{S} \act{rolig, støttende stemme} Hej med dig. Hvad så, hvad er der galt med dig?

\says{Lig} Jo, jo... ser du... der var den her... dumme dumme kandidatstuderende, der ville drille nogle russer... Og så \emph{brugte} han mig til... til at vise... at 1... er lig 2! \act{Lig snøfter}

\says{S} Så er du kommet til det rette sted. Jeg har nemlig lavet en støttegruppe.

\says{Lig}  Men... men så var der også den her professor. Og han... han skrev, at hans gruppe var lig med de hele tal, selv om det kun, var en isomorfi. \act{hulker igen}

\says{S} Jaaah... men altså isomorfi er jo også en ækvivalensrelation, sååå... mon ikke det går?

\scene{Lig hulker endnu dybere, lægger sig ned og græder, mens S nærmer sig glubsk.}

\scene{Pil kommer baglens og meget nervøs ind på scenen. Spjætter hver gang nogen rører hende.}

\says{S} Hej, du... kom bare og sæt dig her ved siden af mig. Du skal ikke være bange, her bliver alle misbrugt... Øh.. har været misbrugt.\act{Trækker Pil hen ved siden af sig}

\scene{Pil hopper væk derfra.}

\says{Pil} Jeg.... jeg har haft en lidt skidt dag.... det hele startede midt i august. Det var sommer, og solskin, og alle de søde, små børn var lige begyndt i gymnasiet. Alt lovede godt. Men så... så skulle eleverne... skrive opgave. Og så... kunne de ikke helt overskue, hvilken vej jeg skulle vende... Og så... så... så satte de bare biimplikationer alle steder.

\says{S} Dobbelt så godt - så vender du to veje! \act{Tager fat om Pil, der springer væk.}

\says{Pil} \act{Lettere forvirret over terapeutens kommentar fortsætter} Nej, det er jo forkert og... hvis bare... det kun var eleverne... så ville det nok gå alligevel. \act{Ser angst ud ved minderne og kan slet ikke få oplevelsen over sine læber, så lighedstegn fortsætter.}

\says{Lig}[imellem hulkene] Var...var...deres lærer også med til det? Det føles bare aldrig sjovt at blive udnyttet på så grov maner. \act{Går hen og krammer Pil grædende.}

\scene{Pil skuler frem for sig og klapper Lig på ryggen. Del kommer ind på scenen og kigger forvirret rundt.}

\says{S} Det lyder slemt. Mon ikke du bare skal have et kram? \act{Bliver distraheret af Del, der kommer ind på scenen og smutter hen mod Del.} Nååh, hvad er der sket med dig, min \textit{lille ven}? \act{Løfter øjnbrynene lumskt, mens han krammer Del ind til sin ene side; hvorefter han fører Del ned på gulvet, så de sidder ned.}

\says{Del} Nej... øh... måske... øh... det er mig egentlig lidt uklart.

\says{S} Har du fortrængt det? Kom lidt tættere på mig, så skal vi nok finde frem til \emph{noget} \act{Løfter øjnbrynene lumskt.}

\says{Del} Altså... forleden dag, til forelæsning, der tabte jeg min streg, og det gjorde mig i grunden lidt forvirret. Altså... hvad betyder jeg overhovedet uden min streg? Og det var ikke kun mig, der var forvirret. Faktisk vidste ingen rigtig, hvad jeg var lige pludselig. Var der... ægte inklusion... var det det samme.... altså... hvad skete der overhovedet? \act{Del gestikulerer mod sit kostume, hvor der ikke er nogen streg under inklusionen.}

\says{Pil} Bare rolig, Del. Jeg lover, vi skal nok finde din streg.

\says{Lig} Du kan bare få en af mine streger, der er ingen, der bruger mig rigtigt alligevel. De fatter alle sammen minus.

\scene{Del ser endnu mere forvirret ud}

\scene{Dot kommer forarget ind på scenen.}

\says{Dot} Nej, nu er det også for galt!

\says{S} Så, så, såh. Rolig nu. Hvad er der sket med dig?

\says{Dot} Jo ser du... jeg er et fuldstændigt rent og pænt indre produkt, og der er ting, man har lov at stikke ind i mig, og der er bestemt ting man ikke har lov sådan at stikke ind i mig. Og jeg har fået nok af at blive behandlet, som om jeg bare finder mig i hvad som helst. Funktionaler, distributioner og andet skidt og pak. Jeg finder mig ikke i det!

\says{S} Nej... det kan jeg godt se. Kom sæt dig ned i cirklen og prøv at sætte nogel ord på, hvordan du har det?

\says{Dot} Jeg har prøvet alt! Men så konjugerer de mig bare, indtil mit fortegn passer. De tænker kun på sig selv og deres behov. Hvad hvis jeg synes, at $x$ hørte til i den ene indgang og ikke i den anden?

\says{S} Det er nemlig vigtigt med samtykke, ikke sandt?

\says{Dot} \act{lavmælt} Jo.

\says{S} Det var vist et Ja, jeg hørte der. Jeg kan altså også godt lide rene og pæne produkter. \act{Ser på Dots to prikker og rager lidt på dem.}

\scene{De sidder og ser foraget på, hvad der sker. Ind kommer Dif arrigt.}

\says{Dif} Og I tror, I har det slemt? \act{stiller sig arrigt helt fremme på scenen. Husk en kunstpause.}

\says{Dif} Helt seriøst ligner jeg måske en brøk??

\says{S} Så, så, jeg skal nok tage mig af dig! \act{Tager fat om Dif og går ud med ryggen mod publikum, så man kan se, at der står fysiker på ryggen.}

\scene{Lys ned.}
\end{sketch}

\end{document}
