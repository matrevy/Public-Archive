\documentclass[a4paper,11pt]{article}

\usepackage{revy}
\usepackage[utf8]{inputenc}
\usepackage[T1]{fontenc}
\usepackage[danish]{babel}

\revyname{Matematikrevy}
\revyyear{2004}
\version{1.0}
\eta{$2$ minutter}
\status{Færdig}

\title{Nedskæringer}
\author{Ukendt}

\begin{document}
\maketitle

\begin{roles}
\role{X}[Eigil] Instruktør
\role{V}[Christoffer] TV-vært
\role{A}[Anne G] Albert Nielsen
\role{T}[Brandt] Flemming Topsøe
\end{roles}

% \begin{props}
% \prop{Rekvisit}[Person, der skaffer]
% \end{props}

\begin{sketch}
\scene{N, A og T sidder i et TV-studie. Der spilles en jingle. Lys op.}

\says{N} Godaften. Der er varslet nye nedskæringer på universiteterne. På matematik skal der skæres et rent ud sagt irrationelle beløb svarende til samtlige af fru Jensens hofter. Og Albert Nielsen, hvad siger du som expert?

\says{A} Efter min overbevisning vil det ikke betyde det store. Det har alligevel vist sig, at stort set alt hvad der foregår på matematik ikke har offentlighedens interesse. For eksempel er det da et eksempel på absurd overforbrug at have overtælleligt mange tal.

\says{N} Du siger altså der bør rationaliseres?

\says{A} Ja og hvorfor stoppe der? Det er tydeligt at erhvervslivet...

\says{Bandet} Hil erhvervslivet!

\says{A} ... for eksempel slet ikke er interesseret i negative tal. Når vi skal til at prioritere velfærdsressourcer, kunne det også være en ide at kigge på, om vi overhovedet har brug for tal større end atten fantasilioner.

\says{N} Ja, eller tal mindre end $10^{-8}$?

\says{A} Ja, hvem har brug for nano? De studerende har ikke brokket sig, de har alligevel kun brug for tallene mellem 00 og 13.

\says{N} Og Flemming Topsøe, hvad siger du til de nedskæringer?

\says{T} Ja, hvis jeg tager mine topologiske briller på, så ser det tragisk ud. Og hvis jeg tager mine mængdelærebriller på, ser det heller ikke for godt ud. Men hvis jeg tager brillerne helt af kan jeg ikke se noget galt.

\says{N} Vi siger tak herfra.

\scene{Lys ned.}
\end{sketch}

\end{document}
