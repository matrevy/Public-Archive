\documentclass[a4paper,11pt]{article}

\usepackage{revy}
\usepackage[utf8]{inputenc}
\usepackage[T1]{fontenc}
\usepackage[danish]{babel}

\revyname{Matematikrevy}
\revyyear{2015}
\version{1.0}
\eta{$n$ minutter}
\status{Færdig}

\title{Kendte Danskere Fortolker Matematik 1}
\author{Martin '11, Stolberg '14}

\begin{document}
\maketitle

\begin{roles}
\role{X}[Freja S] Instruktør
\role{L}[Kristian] Lars Løkke
\end{roles}

% \begin{props}
% \prop{Rekvisit}[Person, der skaffer]
% \end{props}

\begin{sketch}
\says{Band} Kendte Danskere Fortolker Matematik. Dagens gæst: Lars Løkke Rasmussen.

\scene{Spot på Lars.}

\says{L} Så vi har altså valgt at lave et system, hvor man skal kunne betegne mindre, mere end man gjorde før. Det er det, vi har valgt, og det fører selvfølgelig til, at det, der er skarpt mindre og afviger meget og nu afviger lidt mindre, ja, det afviger så mere mindre end det, der er svagt mindre og afviger mindre, men altså så afviger mindre mindre.

\scene{Lys ned.}
\end{sketch}

\end{document}
