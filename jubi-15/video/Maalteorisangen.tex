\documentclass[a4paper,11pt]{article}

\usepackage{revy}
\usepackage[utf8]{inputenc}
\usepackage[T1]{fontenc}
\usepackage[danish]{babel}

\revyname{Matematikrevy}
\revyyear{2006}
\version{1.0}
\eta{$1.5$ minutter}
\status{Færdig}

\title{Målteorisangen}
\author{Ukendt}
\melody{Folk og Røvere i Kardemomme By -- Røversangen}

\begin{document}
\maketitle

\begin{roles}
\role{X}[Kristian] Instruktør
\role{S1}[Arnvig] Sanger (Flemming)
\role{S2}[Kaspian] Sanger (Tage)
\role{S3}[Michael] Sanger (Christian)
\end{roles}

\begin{song}
\sings{S1-3} Vi lister os afsted på tå
Når vi skal integrere
Vi måler kun på, det vi må
Alt andet la'r vi være
\sings{S2} Jeg måler på en gammel spand
\sings{S3} Jeg måler på din tissemand
\sings{S1-3} Og ellers så måler vi det, vi nu kan
Både Flemming og Tage og Christian

\sings{S1-3} Vi starter med et Radon mål
Og fjerner dig fra støtten
Og pluds'lig har du mål på nul
Hvor før du havde sytten
Du tro'de, at du var nog't stort
Nu er du bar' en lille lort
Men det må vi gerne fordi vi har spurgt
Både Flemming og Tage og Christian

\sings{S1} Og Kurzweil-Henstock integralet
\sings{S2} Det er helt umuligt
Halvtreds procent det rene pral
Halvtreds procent ubrug'ligt
\sings{S2-3} Og Riemann han blev pluds'lig væk
\sings{S1} Måske var det en fejl i \TeX
\sings{S1-3} Men så kan vi bruge den kære Lebesgue
Både Flemming og Tage og Christian

\sings{S2}[Tales] Ej! Du må ikke måle på min tissemand!
\sings{S3}[Tales] Bare rolig, jeg bruger tællemålet
\sings{S1} Men hov, det kan jo ikke gå!
Den her kan du ik' måle på!
\sings{S1-3} For udvalgsaxiomet det stoler vi på
Både Flemming og Tage og Christian
\end{song}

\end{document}
