\documentclass[a4paper,11pt]{article}

\usepackage{revy}
\usepackage[utf8]{inputenc}
\usepackage[T1]{fontenc}
\usepackage[danish]{babel}

\revyname{Matematikrevy}
\revyyear{2015}
\version{1.0}
\eta{$n$ minutter}
\status{Færdig}

\title{Kendte Danskere Fortolker Matematik 2}
\author{Martin '11, Stolberg '14}

\begin{document}
\maketitle

\begin{roles}
\role{X}[Freja S] Instruktør
\role{J}[Christoffer] Jørgen Leth
\end{roles}

% \begin{props}
% \prop{Rekvisit}[Person, der skaffer]
% \end{props}

\begin{sketch}
\says{Band} Kendte Danskere Fortolker Matematik. Dagens gæst: Jørgen Leth.

\scene{Jørgen Leth kommer ind på scenen med en Kalkulus. Han sætter sig til rette i en rekvisit, der allerede står på scenen, og slår omhyggeligt op på en side i bogen.}

\says{J} Jeg lægger mærke til, hvordan $varepsilon$-pølsen smyger sig tæt omkring grafen for $f$. Denne flotte grafs kurver minder mig om Mont Ventoux. Det gør den. Og det gør noget ved mig. Jeg mærker min potensfunktion vokse og vokse.

\scene{Lys ned.}
\end{sketch}

\end{document}
