\documentclass[a4paper,11pt]{article}

\usepackage{revy}
\usepackage[utf8]{inputenc}
\usepackage[T1]{fontenc}
\usepackage[danish]{babel}

\revyname{Matematikrevy}
\revyyear{2011}
\version{1.0}
\eta{$6$ minutter}
\status{Færdig}

\title{Bevis for Skæringssætningen}
\author{Ukendt}
\melody{Queen -- Bohemian Rhapsody}

\begin{document}
\maketitle

\begin{roles}
\role{X}[Freja S] Instruktør
\role{S}[Loke] Sanger
\role{K1}[Rikke L] Kor (Sopran)
\role{K2}[Stine L] Kor (Sopran)
\role{K3}[AK] Kor (Alt)
\role{K4}[Tina] Kor (Alt)
\role{K5}[MaWeK] Kor (Tenor)
\role{K6}[Stig] Kor (Tenor)
\role{K7}[Rasmus] Kor (Bas)
\role{K8}[Sommer] Kor (Bas)
\role{B}[Lise] Bevisopskriver
\end{roles}

\begin{song}
\sings{K} Se på min tegning
Grafen går denne vej
Det er da klart nok
Hvor det nulpunkt må gemme sig
Ja, her og her
Der kan være fler' end ét
Se på det største, det er da ligetil
Det er supremum for mængden $D$
Hvad er $D$? Lad os se
Det skal vær' de $x$, hvor $f(x)$ er mindre end nul, end nul

\sings{S} Sæt $c$ lig $\sup(D)$
Vælg $x_n$ i mængden $D$ højst $1 / n$ fra $c$
Følgen $x_n$ går mod $c$
Så derfor konkluderer vi nu at
Følgen $f(x_n)$
Konverger' mod $f(c)$
Så $f(c)$ er derfor svagt mindre end nul
Det var den ulighed, kan vi også få den anden?

\sings{S} $c + 1 / n$
Det kalder vi $x_n$ for $b$ større end $x_n$
Følgen kaldet $x_n$, den går mod $c$
Så vi konkluderer derfor ligesom før
Følgen $f(x_n)$
Går mod $f(c)$
Så $f(c)$ er derfor svagt større end nul

\sings{S} Nu har jeg næsten fået klaret mit bevis
\sings{K} Pas nu på, pas nu på, du har glemt en detalje
Du skal også vise Hjælpesætning 5.1.10
(Du er givet) Du er givet
(Du er givet) Du er givet
Du er givet $\varepsilon$ større end nul

\sings{S} Da er vor afstand fra $f(x_n)$
\sings{K} Hen til $f(c)$ mindre end $\varepsilon$
Hvis blot $x_n$ kun er $\delta$ fra $c$

\sings{S} Så er jeg færdig her, følgen konverger'
\sings{K} Bevis det! Nej, det må da være klart (soleklart)
Bevis det! Det må da være klart (soleklart)
Bevis det! Det må da være klart (soleklart)
Må da være klart (soleklart)
Må da være klart -- trivielt!
\sings{K} Nej, nej, nej, nej, nej, nej, nej
\sings{S} Åh, hvorfor ikke, hvorfor ikke, hvorfor ikke trivielt?
\sings{K} Funktionen $f$ er kontinuert i punktet $c$, i $c$, i $c$

\sings{S} Da $x_n$ konvergerer mod $c$ ses igen
At $x_n$ højst er $\delta$ fra $c$ for stort $n$
Hvad mere? Vi kan nu konkludere
At $f(x_n)$ har grænseværdi $f(c)$

\sings{K} (Hvad nu? Hvad nu?)

\sings{S} $f(c)$ er større
Eller lig med nul
$f(c)$ er mindre
$f(c)$ må være lig nul

\sings{S} Hvilket skulle vises
\end{song}

\end{document}
