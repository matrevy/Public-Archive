\documentclass[a4paper,11pt]{article}

\usepackage{revy}
\usepackage[utf8]{inputenc}
\usepackage[T1]{fontenc}
\usepackage[danish]{babel}

\revyname{Matematikrevy}
\revyyear{2013}
\version{1.0}
\eta{$3$ minutter}
\status{Færdig}

\title{Ode til Analyse 1}
\author{Freja '10}
\melody{Shout -- Jeg Tror, Det Kaldes Kærlighed}

\begin{document}
\maketitle

\begin{roles}
\role{X}[MaWeK] Instruktør
\role{Y}[Nanna] Koreograf / Sanger (S1)
\role{S2}[Rikke L] Sanger
\role{S3}[Rikke M] Sanger
\role{D1}[AK] Danser
\role{D2}[Freja E] Danser
\role{D3}[Jasmin] Danser
\role{D4}[Julie] Danser
\role{D5}[aDA!] Danser
\end{roles}

\begin{song}
\sings{S1} Hvordan gi'r vi mening til
Et uend'ligt summespil?
Tallene bli'r fler' og fler'
Tjek om delsum konverger'

\sings{S2} Sum af $n$'te-del i $p$
Forholdstest og rottesten
Vi lærte om en dejlig ven
\sings{S1-3} Den hedder Grænsesammenligningstesten

\sings{S1-3} Denne sang den er til dig
Åh Jan Philip Solovej
Når funktionerne de ændrer sig
Så ta'r vi det skridt for skridt
Lært i Analyse 1
Lad $n$ gå mod uendeligt

\sings{S3} Nu blev faget mer' abstrakt
Vi fandt en sum for $\pi$ eksakt
Lærte om vor tavlesvamp
\sings{S1-3} Valget tør og våd er ikke mer' en kamp

\sings{S1-3} Denne sang den er til dig
Åh Jan Philip Solovej
Når funktionerne de ændrer sig
Så ta'r vi det skridt for skridt
Lært i Analyse 1
Lad $n$ gå mod uendeligt

\sings{S1} Er du nu lang langt væk eller så nær?
\sings{S2-3} Er du nu lang langt væk eller så nær?
\sings{S2} Er metrikken nul, så du er her?
\sings{S1-3} Wowyeahhh

\sings{S1-3} Denne sang den er til dig
Åh Jan Philip Solovej
Når funktionerne de ændrer sig
Så ta'r vi det skridt for skridt
Lært i Analyse 1
Lad $n$ gå mod uendeligt

\sings{S1-3} Denne sang den er til dig
Åh Jan Philip Solovej
Når funktionerne de ændrer sig
Så ta'r vi det skridt for skridt
Lært i Analyse 1
Lad $n$ gå mod uendeligt

\sings{S3} Lad $n$ gå mod uendeligt
\end{song}

\end{document}
