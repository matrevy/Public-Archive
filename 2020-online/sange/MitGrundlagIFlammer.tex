\documentclass[a4paper,11pt]{article}

\usepackage{revy}
\usepackage[utf8]{inputenc}
\usepackage[T1]{fontenc}
\usepackage[danish]{babel}

\revyname{Matematikrevy}
\revyyear{2020}
\version{1.0}
\eta{$4$ minutter}
\status{Færdig}

\title{Mit Grundlag i Flammer}
\author{Nynne '17}
\melody{Saybia: ``The Day After Tomorrow''}

\begin{document}
\maketitle

\begin{roles}
\role{X}[Marius] Instruktør
\role{Y}[Jasmin] Koreograf / Danser
\role{S}[KE] Sanger
\role{D2}[AK] Danser
\end{roles}

\begin{song}
\sings{S1} Alting er gået fint,
Her på mit studie
Første halvandet år

Jeg har forstået alt
Helt uden problemer
Uden at gå i stå

\sings{S1} Men så ramte blok tre
Og alting blev strammer'
Er forvirret og svag
Som ramt af en hammer

Det er værre end fnat
Når tvivlen den rammer
Mikkel Willum har sat
Mit grundlag I flammer

\sings{S1} I sku' ha sagt det før
Føler mig trådt på
Føler mig fucking dum

Jeg har jo spildt min tid
På svage aksiomer
Tro'de jeg vidste nog't
Var så klog

\sings{S1} Men så ramte blok tre
Og alting blev strammer'
Er forvirret og svag
Som ramt af en hammer

Det er værre end fnat
Når tvivlen den rammer
Mikkel Willum har sat
Mit grundlag I flammer

\sings{S1} I VT
Mit liv blev knust i VT
Fundamentet skred i VT
Kan jeg få det igen?

\sings{S1} Humaniora, du' den
Der smadred' min verden
Kommer aldrig igen
Jeg klarer ik' smerten

Så jeg lægger mig fladt
Når tvivlen den rammer
Mikkel Willum har sat
Mit grundlag I flammer

Mit grundlag brænder nu
\end{song}

\end{document}
