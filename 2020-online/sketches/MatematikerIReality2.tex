\documentclass[a4paper,11pt]{article}

\usepackage{revy}
\usepackage[utf8]{inputenc}
\usepackage[T1]{fontenc}
\usepackage[danish]{babel}

\revyname{Matematikrevy}
\revyyear{2020}
\version{1.0}
\eta{$2$ minutter}
\status{Færdig}

\title{Matematiker i Reality 2}
\author{Sommer '17, Stine '18, Victoria '19, Nynne '17, KE '18}

\begin{document}
\maketitle

\begin{roles}
\role{X}[Toke] Instruktør
\role{A}[Rune] Algebraiker
\role{B}[Anna] Pige
\role{C}[] Undertekster
\end{roles}

\begin{sketch}
\scene{Algebraiker står og svarer på spørgsmål, men snakker bare algebra og bliver undertekstet til noget andet:}

\says{B}[Leger med sit hår] Så, hvad leder du efter i en pige?

\says{A} Det er klart, at $G$ er endeligt frembragt hvis $G$ er endelig (vi kan da tage $S = G$). Men også er f.eks. $Z$ endeligt frembragt, for vi har jo $Z = \langle 1 \rangle$.

\says{C} Jeg søger bare en super sød pige, som jeg kan have det hyggeligt med.

\says{B} Ja, det er selvfølgelig vigtigt. Sådan har jeg det også med algebraikere. Hvordan ville du beskrive dig selv?

\says{A} En endeligt frembragt, abelsk gruppe er selvfølgelig en abelsk gruppe, der er endeligt frembragt. Specielt er enhver endelig, abelsk gruppe endeligt frembragt, abelsk.

\says{C} Jeg synes også, jeg er en, man kan have det hyggeligt med.

\scene{Cikade-akavethed}

\says{B} Nå, hvad synes du så om alt det med Morten Østergaard?

\says{A} Om jeg tror på choice? Næh.

\says{C} Han er en idiot.
\end{sketch}

\end{document}
