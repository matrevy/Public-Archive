\documentclass[a4paper,11pt]{article}

\usepackage{revy}
\usepackage[utf8]{inputenc}
\usepackage[T1]{fontenc}
\usepackage[danish]{babel}

\revyname{Matematikrevy}
\revyyear{2020}
\version{1.0}
\eta{$3$ minutter}
\status{Færdig}

\title{Støt Revysterne}
\author{Stine '18, KE '18, Sommer '17}

\begin{document}
\maketitle

\begin{roles}
\role{XB}[Stine] Instruktør / Bunde revyst
\role{A}[Anna] Speaker
\role{C}[Rune] Nøgen revyst
\role{D}[Toke] Revyst der panikker i skoven
\role{St1}[Maja] Statist
\role{St2}[Sommer] Statist
\role{St3}[Sunniva] Statist
\end{roles}

\begin{sketch}
\says{A} År efter år har revyster latterliggjort dem selv foran resten af matematikstudiet i håb om at opnå en form for anerkendelse i det barske miljø. I deres grænseløse bestræbelser på at få en latter, endog bare et smil frem på en medstuderendes læber, er der blevet sat uundgåelige og permanente spor i deres sjæle.

\says{A} Her er B. Han er revyst. I sin dagligdag kan han måske virke som alle andre - men arrene stikker dybt. Sommetider er symptomerne for stærke til at undertrykke:

\scene{Revysten sidder i kantinen i sin egen verden og hører mange samtaler flyde ind i hinanden. Pludselig hører han “tøjet” klart fra en sætning, og det giver genklang i hans hoved. Genklangen bliver til et vedvarende tilråb, og han ser forlegen og småirriteret ud. Han “giver sig” og begynder at tage tøjet af. Flash forward til at de andre i kantinen griner ad ham og peger på ham, fordi han har taget sit tøj af. De synes, han er grim og dum (og grim og dum\ldots, så klap i hænderne)!}

\scene{Revysten er ude at gå en tur med sin ven, og så nærmer de sig nogle træer. Revysten begynder at høre “TRÆET TRÆET”-chant, og spørger sin ven om vedkommende også kan høre det, hvilket vedkommende svarer nej til. Revysten prøver at agere normal, men som chantet bliver højere og højere som de nærmer sig, må revysten til sidst lukke øjnene og løbe væk.}

\scene{Revysten sludrer med sine venner med en kop kaffe i hånden. Revysten tager sin kop op til munden for at tage en tår. Vi cutter til revystens POV, hvor han pludselig ser, at hans medstuderende klapper i takt og råber “kan *revyst*, kan *revyst*, kan *revyst* drikke ud?” Kameraet peger opad, og revysten drikker ud. Vi ser igen selskabet fra 3. persons synsvinkel. De medstuderende kigger forvirret over på revysten, som er ved at bunde sin kop kaffe. Revysten har det tydeligvis ikke godt, men han bliver ved. Vi ser igen revyens POV, og han hører stadig “kan *revyst*, kan *revyst*, kan *revyst* drikke ud?”. Til sidst ser vi 3. persons synsvinkel, og revysten er virkelig medtaget for har bundet sin kaffe uden grund. Revysten bliver ked af det, for nu skete det igen.}

\says{A} Revysterne vil sætte en stopper for misbrug af stakkels studerende til underholdning. Men vi har brug for din hjælp! For bare x kroner om måneden kan du være med til at stoppe revyst-mishandling. Din støtte gør en forskel! En gang i timen finder en revyst på en dårlig ordspils-joke. Vi har allerede omvendt mange. Din støtte hjælper - tak!
\end{sketch}

\end{document}
