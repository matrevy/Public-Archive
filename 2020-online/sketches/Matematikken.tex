\documentclass[a4paper,11pt]{article}

\usepackage{revy}
\usepackage[utf8]{inputenc}
\usepackage[T1]{fontenc}
\usepackage[danish]{babel}

\revyname{Matematikrevy}
\revyyear{2020}
\version{1.0}
\eta{$3$ minutter}
\status{Færdig}

\title{Matematikken}
\author{KE '18, Sommer '17}

\begin{document}
\maketitle

\begin{roles}
\role{X}[Nynne] Instruktør
\role{A}[KE] Matematikekspert 1
\role{B}[Sommer] Matematikekspert 2
\end{roles}

\begin{sketch}
\scene{De to eksperter udtaler sig om matematik i et populærvidenskabeligt program, som skal appellere til ikke-matematikere.}

\says{A} Matematik - sikke et koncept.

\says{B} Altså abstrakt matematik er så stort og fantastisk, at det bare giver mig sommerfugle i maven og så kilder det op gennem blyanten, og jeg får gåsehud, og lige pludselig er verden ikke tredimensionel længere men et abstrakt vektor-rum i overtælleligt mange dimensioner. Jeg tror, origo ligger på HCØ et sted, men jeg er ikke sikker.

\says{A} Hvis du kunne proppe alle tal ned i et rør, så ville du få et meget langt rør uhmm, som nok ville være længere end tavlerne i Aud 1, for når du propper tal ned i et rør, så bliver de længere end normalt og så er der ikke plads til dem alle sammen. Medmindre røret er meget langt, men sådan nogle findes ikke.

\says{B} Når jeg skal lære russerne om Analyse, siger jeg bare, at de skal forestille sig nogle virkelig små, græske bogstaver. Og når bogstaverne bliver mindre og kommer tættere på hinanden, sker der magiske ting. Hvis de kommer tæt nok på hinanden, så må vi differentiere, og det er virkelig godt at kunne.

\says{A} Hver gang jeg støder på en funktion, tænker jeg automatisk “er den veldefineret?” Og, du ved, nogle mennesker er bare fuldstændigt li- geglade med det.

\says{B} Prøv at sætte hvert tal på en dråbe vand. Hvis du tog en spand og fyldte den op, ville du have mange tal, men ikke uendeligt mange. Hvis du tog alle verdenshavene, ville du have flere tal, men stadig ikke uendeligt mange. Uendelig er så stort, at mennesker bare ikke kan håndtere det. Havet kan heller ikke, åbenbart.

\says{A} $\infty$ er så stor, at vi øhmm\ldots\ at man bliver nødt til at bruge Maple for at nå derud. Der er så langt, at man ikke rigtig kan se det for sig, for man skal forbi virkelig virkelig virkelig virkelig virkelig virkelig mange rigtig sjove tal på vejen.

\says{B} Da jeg var barn, troede jeg, der var hundrede tal, men nu er der, sådan, i hvert fald en million.

\says{A} Altså, jeg er virkelig stolt af at kunne formidle matematik til russer. Nogle gange siger jeg til dem: “Vi skal alle sammen dumpe, og vi skal dumpe sammen. Og når vi gør, så holder jeg om dig og trøster dig.”

\scene{Kunstpause}

\says{A}[I russens sted] “Må jeg gå på druk med de ældre studerende?”

\says{B} Nej, rus! Nej!

\says{A} Forestil dig et egern med 2 nødder, og put så n+1 tal ned i n skuffer, og put så skufferne i et skab, så er nødderne\ldots\ uhm\ldots

\says{B} Når du kigger op på en tavle og ser en funktion, så er den der egentlig ikke. Det er faktisk mig der står i din have og underviser dig i matematik, og din far kommer ud og spørger “hvad laver du i min have?” og jeg siger “jeg underviser dit barn i den fantastiske matematiske verden. I dag lærer vi om analysens fundamentalsætning. Jeg gør dette hver nat med dit barn.”
\end{sketch}

\end{document}
