\documentclass[a4paper,11pt]{article}

\usepackage{revy}
\usepackage[utf8]{inputenc}
\usepackage[T1]{fontenc}
\usepackage[danish]{babel}

\revyname{Matematikrevy}
\revyyear{2020}
\version{1.0}
\eta{$3$ minutter}
\status{Færdig}

\title{The Regning}
\author{KE '18}

\begin{document}
\maketitle

\begin{roles}
\role{X}[KE] Instruktør
\role{A}[Regine] Rus
\role{BC}[Rune] Instruktor 1 + 2
\role{D}[Victoria] Stresset rus
\end{roles}

\begin{props}
\prop{Rekvisit}[Trehjulet cykel]
\end{props}

\begin{sketch}
\scene{Vandrehallen, stueetagen Rus kommer gennem vandrehallen på sin trehjulede cykel. Hun drejer til højre ned i E-bygningen. Hun sætter den trehjulede cykel i elevatoren, der går ned ad den lille trappe og venter på elevatoren. Så åbner hun døren ind til kontorerne og tager cyklen med. Rus cykler nu på cyklen og drejer om hjørnet i E-bygningens gange. Hun drejer om hjørnet igen og igen, som om E-bygningen bare fortsatte for evigt. Pludselig drejer hun om et hjørne og ser to instruktorer.}

\says{B og C}[Manisk] Kom og regn med os, Rus. Kom og regn med os. For evigt og altid!

\scene{Filmen klipper mellem instruktorerne og det samme sted, men hvor instruktorerne står med den stressede rus, som regner på opgaver. Rus bliver fortvivlet og holder hænderne for øjnene}

\says{A}[Fortvivlet, taler med sig selv] Rusvejleder, jeg er bange.

\says{A}[Som om, hendes finger var vejlederen] Rolig nu. Husk, hvad Fabien
sagde. Det er bare ligesom løsningen på $x^2 + 1 = 0$. Det er ikke reelt.

\scene{Da hun åbner øjnene igen, ser hun instruktorerne stå med den samme rus, men nu er russen glad og frisk; han sidder med terninger og slår med dem for at få svarene på sine MC-opgaver. Filmen slutter med et film-cover med titlen “The Regning”, men hvor h e og g gradvist bliver udvisket eller dækket, så der står “TeRning”}
\end{sketch}

\end{document}
