\documentclass[a4paper,11pt]{article}

\usepackage{revy}
\usepackage[utf8]{inputenc}
\usepackage[T1]{fontenc}
\usepackage[danish]{babel}

\revyname{Matematikrevy}
\revyyear{2020}
\version{1.0}
\eta{$3$ minutter}
\status{Færdig}

\title{Rejsen gennem Vandrehallen}
\author{Marius '17, Sommer '17, KE '18}

\begin{document}
\maketitle

\begin{roles}
\role{X}[Sommer] Instruktør
\role{A}[Rune] Studerende
\role{B}[Maja] Studerende
\role{C}[] Baggrundsperson
\role{M}[] Mor
\role{F}[] Far
\role{L}[] Læge
\role{RV}[] Rusvejleder
\role{P}[] Professor
\role{Æ}[] Ægtefælle
\end{roles}

\begin{sketch}
\scene{Sydenden, en gruppe mennesker sidder ved et bord. De har lavet en aflevering færdig, men person A mangler lige én sidste ting...}

\scene{B sætter sig ved A}

\says{B} Så har jeg lige klipset min opgave, så er den sgu klar til aflevering

\says{A} Ah pis, det har jeg glemt. Jeg smutter lige ned og får det ordnet i receptionen inden øvelsestimen.

\says{B} Receptionen? Det når du jo aldrig; vi har time lige om lidt, og det tager 100 år at gå i gennem vandrehallen!

\says{A} Pjat med dig, jeg skal nok nå det, vi ses til timen!

\scene{A tager et papir med fra bordet og går af sted}

\scene{Her begynder så rejsen gennem vandrehallen. I starten virker det ikke så slemt, og A går raskt af sted. A går så forbi person Cs fødsel}

\says{L} Tillykke! Det blev en dreng!

\says{M} Jeg tror jeg vil kalde ham C

\scene{Klipper tilbage til A der ikke er kommet meget videre, og gennem forskellige kameravinkler virker det til at A ikke kommer nogen vegne. A går så forbi Cs konfirmation.}

\says{F} Så C, i dag er du blevet konfirmeret. Nu er du trådt ind i de voksnes rækker

\says{C} Ej faaaar\ldots

\scene{A er kommet lidt længere, og er begyndt at se ældre ud. Igen kan man lege med kameravinkler. A går forbi C som rus}

\says{RV} Hey mit navn er RV og jeg er din rusvejleder

\says{C} Hey RV, mit navn er C

\scene{Leg med kameravinkler. A kommer forbi C som kandidat, hvor mor og far også er til stede}

\says{P} Tillykke C, du er nu kandidat

\says{M\&F} Tillykke C vi er så stolte

\scene{Leg med kamera, A er nu blevet markant ældre (Skæg og let-krum ryg). A kommer forbi C der nu bliver gift}

\says{P} Og vil du C, tage Æ som din ægtefælle?

\says{C} Ja

\scene{Leg med kamera, A går nu virkelig dårligt. A går nu forbi Cs begravelse, hvor Æ og evt. børn er til stede og græder. På en gravsten skal der stå C.}

\scene{A kommer nu endelig frem til receptionen og er frisk igen.}

\scene{A gør sig klar til at klipse, og kigger på det papir han har gået med i hånden hele tiden. Det går nu op for ham at det er en ansøgning til adgang til S01 (Eller et andet papir), og ikke hans aflevering. A får flashback til episoden ved bordet, hvor han tager S01 papiret, i stedet for det papir ved siden af (Hvor der med store bogstaver står AFLEVERING på). Flasher tilbage til A der facepalmer og evt. udråber shit eller andet bandeord.}

\end{sketch}

\end{document}
