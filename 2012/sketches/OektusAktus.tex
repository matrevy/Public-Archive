\documentclass[a4paper,11pt]{article}

\usepackage{revy}
\usepackage[utf8]{inputenc}
\usepackage[T1]{fontenc}
\usepackage[danish]{babel}

\revyname{MatematikRevy}
\revyyear{2012}
% HUSK AT OPDATERE VERSIONSNUMMER
\version{1.0}
\eta{$3$ minutter}
\status{Skal Redigeres}

\title{Øktus og Aktus}
\author{Maling og Silvia}

\begin{document}
	\maketitle
	
\begin{roles}
\role{I}[NB] Instruktør
\role{K}[Jenny] Aktus
\role{B}[Freja] Øktus
\role{F}[Emil] Fortæller (Voice-Over)
\role{N1}[JOEP] Ninja
\role{N2}[Silvia] Ninja
\role{N3}[Mathias] Ninja
\end{roles}
	
\begin{props}
\prop{\$ Hakke}[]
\prop{£ Hakke}
\end{props}
	
%	\begin{mics}
%		\mic{HS1} Skuespiller
%	\end{mics}
	
	\begin{sketch}
		
\scene{Scenen består af en række tænder der står nederst på scenen, de dækker benene og lidt af maven på K og B, så det ligner de er små. Temaet begynder og lyset går op på scenen, de står henholdsvis med en \$-hakke og en £-hakke og hakker skiftevis i takt til musikken (den ene backbeat, den anden til slaget). De nynner/laller til. Imens de hakker falder der guldmønter ud af tænderne. Da temaet stopper lægger de hakkerne fra sig og tørrer sveden af panden synkront. Der er en bunke ædelmetaller/sten og måske rav hvor fx 1+1=2 stør på klumperne}

\says{F} Den stakkels matematik havde haft det hårdt på det seneste. Den var inficeret af Øktus og Aktus, og var blevet særdeles uren.
\says{B} Øjjj nej, Aktus. Nu synes jeg vi har hugget og hakket og hugget og hakker heeele dagen. Nu synes jeg vores bunke guld er stor nok.
\says{K} Næh, du Øktus. Vi må have den MEGET større endnu. Du må huske på vi bruger flere og flere penge for hver dag der går, sådan som vi vælter os guldkæder og damer. Hæng du bare i gamle Øktus!
\says{B} Ja ja, så hænger jeg da i.
\scene{De hugger lidt videre.}
\says{B} Duuu, Aktus. 
\says{K} Ja, hvad er der nu!? 
\says{B} Se lige den her! 
\scene{Han holder en guldklump med Black-Scholes-ligningen på frem}
\says{K} Årh! Black-Scholes ligningen!!  Over kodyl fedt mand. Låg den over i bunken, den kan bruges!
\says{B} Ja, i modsætning til den her
\scene{B holder en stor sten frem med "galoisteori" skrevet på.}
\says{K} Fy da føj for den lede. Den er bare tung og ubrugelig og helt uden værdi. AAAD!
\scene{De hugger lidt videre.	Pludselig kommer der et skarpt lys på scenen.}
\says{K+B} OOOH Han åbner munden!!
\scene{ Det vælter ind med roulade, citronmåner og dvs. andre billigere kager.}
\says{B+K} HuuuuRrAAAA så er der kagesøster!!!
\scene{De spiser på en ulækker klam måde. Lady og vagabonden fra hver side.}
\says{B} Det' ik' særlig bling bling, men hold kæft det smager godt!
		 
		 %\says{B} Jeg tænkte på, skulle vi ikke fordele os lidt. Så bliver der meget bedre plads når vi bliver flere på studiet.
		 %\says{K} Du må bruge forstanden lidt, Øktus! Vi er stadig så lille et studie, at vi må holde sammen!
		 %\says{B} Hvad mener du? Har det ikke altid været sådan her?
		 %\says{K} Jeg kan huske dengang at matematikken her var ren. Føj for en ulykke!

\says{B+K} Hej hurra hurra hurra \\
For det guld som vi to har \\
Mat'matikken bruges vidt, til at skaffe penge tit \\
Statistik vi gerne ta'r, hos mat-øk og aktuar \\
SS, Liv og Jura ja, Mikro Makro ha ha ha \\
Tra la la la la (hakker igen) \\
Tra la la la la (hakker videre) \\
Tra la la la la la la la la la la la la (hakker videre) \\
\says{F} "Aaaav, det gør ondt!", sagde Matematikken
\says{B} Hør, hvad var det?
\says{K} Det er bare matematikken der brokker sig.
\scene{De hugger lidt videre.}	
\scene Et tågehorn lyder
\says{K} Så der fyraften!
\says{B} Jahhh. 	 
\says{B+K} Vi vil ha Guld Tuborg! Vi vil have Guld Tuborg!
\scene{Der kommer et kraftigt lys på scenen}
\says{B+K} Jaaah se! nu kommer den!
\says{K} Måske har det hjulpet at vi råbte! 
\scene{En kæmpe tandbørste med "DET FÆLE ARBEJDSMARKED" trykt på siden kommer ind og skubber dem væk, de lægger sig ned bag tænderne så de ikke kan ses, imens de skriger som om de bliver skyllet ud}

\says{F} Det var måske lidt synd for Øktus og Aktus, men nu var Matematikken igen ren og levede lykkeligt til sine dages ende.
\scene{Tæppe}
\end{sketch}
\end{document}

