\documentclass{article}
\usepackage[utf8]{inputenc}
\usepackage[danish]{babel}
\usepackage{revy}
\revyname{MatematikRevy}
\revyyear{2012}
\version{1.0}
\eta{3 minutter 20 sek}
\status{Skal Redigeres}

\title{Hvad Kan Man Bruge Matematik til?}
\author{Diverse}

\begin{document}
\maketitle

\begin{roles}
\role{I}[NB] Instruktør
\role{M}[Jasper] Matematiker, tidligere MATØKer
\role{St}[Soeren] Studievejleder
\end{roles}

\begin{sketch}
\says{M} Goddag.
\says{St} Hej Jasper. Velkommen til studievejledningen. 
\says{M} Jeg har haft et år på Matematik-Økonomi. Men det var lige tanden for svært for mig, så for at halvere arbejdsbyrden, så nøjes jeg med ren matematik.
\says{St}[hånligt] Hehehe 
\says{M} ...Men jeg har lige nogle spørgsmål..
\says{St} Okay - spørg bare løs.
\says{M} Hvornår vil man nogensinde få brug for skuffeprincippet i virkeligheden?
\says{St} Hm, du har en kommode, ikke?
\says{M} Jo!
\says{St} Og en bandana?
\says{M} Øh, ja..
\says{St} Okay, hvis du nu har 8 skuffer i din kommode - eller 3 skuffer. Så VED du altså, at når du har kigget de 2 af skufferne at dit bandana ligger i den tredje. Smart ikke!?
\says{M} Men jeg har den jo i lommen/på hoved... Hvad så med skæringssætningen?
\says{St} Jo, lad os tage din økonomi som eksempel.
\says{M}[Bryder ind] Økonomi!? Jeg er på SU...
\says{St} Ja, jo. Men lad os ANTAGE at du engang havde penge og at dit
pengeforbrug er kontinuert.
\says{M} Altså ingen Caféen??
\says{St} Præcis. Hvis du så nu står og skylder penge væk - så ved du at
på et tidspunkt har haft præcist 0 kroner.
\says{M} Nåja, det giver da god mening. Cirkeldelingspolynomier de MÅ da
være ubrugelige!
\says{St} Nej, på ingen måde! Forstil dig at du har 6 venner, inklusiv
dig selv selvfølgelig. Du har lige været nede for at købe en pakke bacon med 6 stykker i.
\says{M}[Skeptisk] Ja!?
\says{St}[Illustrerer] Du kan nu lægge baconen i en perfekt cirkel og ved hjælp af
cirkeldelingspolynomiet kan du skære baconcirklen på en sådan måde at
der bliver netop ét stykke bacon til hver
\says{M} Ja, men nu har jeg jo haft Lineær Algebra...
\says{St} Jamen, det er sådan noget med matricer.
\says{M} ...og Algebra 1... 
\says{St} Det er noget med symmetri i grupper 
\says{M}   og Kombinatorisk Spilteori - hvad kan jeg bruge det til?
\says{St}[Meget glad, nærmest syngende] Sudoku.
\says{M} Men man kan jo ikke leve af at løse sudokuer?
\says{St} Det kommer helt an på, om du har et andet job ved siden af. F.eks. eksamensvagt.
\says{M} Men hvad så med Anvendt Satistik?
\says{St} Det kan du øh - det kan du ikke bruge til noget.

\end{sketch}
\end{document}