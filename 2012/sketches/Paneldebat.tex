\documentclass[a4paper,11pt]{article}

\usepackage{revy}
\usepackage[utf8]{inputenc}
\usepackage[T1]{fontenc}
\usepackage[danish]{babel}


\revyname{MatematikRevy}
\revyyear{2012}
% HUSK AT OPDATERE VERSIONSNUMMER
\version{1.5}
\eta{5 minutter}
\status{Færdig}

\title{Paneldebat}
\author{Diverse}

\begin{document}
\maketitle

\begin{roles}
\role{I}[KOEK] Instruktør
\role{S}[Kristian] Tidligere Kranfører, Sven-Erik Skov (Alti' Alti' manden)
\role{H}[Tomas] Hitler
\role{M}[Jenny] Matematiker
\role{D}[Lilli] Debatstyrer
\role{B}[JOEP] Bertel Haarder
\role{L}[Emil] Lyskurv
\end{roles}

\begin{props}
\prop{Udfoldelig fodgængerovergang}[]
\prop{Rød lampe}[]
\prop{Øl}[]
\prop{Pistol}[]
\prop{Cigaret}[]
\end{props}


\begin{sketch}

\scene{}
\says{D} Godaften og velkommen tilbage til Paneldebatten. Jeg er aftenens debatstyrer og mit job er at styre diskussionen.
\says{H}[afbryder] ICH bin der Fürher.
\says{D}[Fortsætter ufortrødent] Vi har forsøgt at sammensætte et panel der afspejler alle sider af samfundet. I aften har vi med os: Adolf Hitler tidligere tysk Kansler, Bertel Haarder tidligere Uddannelsesminister, Sven-Erik Skov - tidligere kranfører fra Kolling, samt dig, Jytte Nexø, postdoc i matematik fra Københavns Universitet. 

Bertel, nu har du været undervisningsminister i sammenlagt 17 år, føler du ikke at du har et medansvar, når vi hører gymnasielærere der siger at eleverne ikke har den fornødne viden, når de går ud af folkeskolen?
\says{B} Der er jo flere ting end bare undervisningsministeriet, der har en effekt på et barns uddannelse. Det er et utroligt kompliceret emne og der er fantastisk mange faktorer der spiller ind.
\says{H} Terpen macht frei!
\says{M} Ja til en vis grænse, men niveauet skal jo tilpasses så det passer sammen gennem hele systemet.
\says{H} Nein, nein, nein. Das niveau skal dass højt vären. \act{Heiler}
\says{B} Nej! Vi er kommet for højt op. Vi skal længere ned. \act{Hiver armen ned til et lavere niveau}
\says{D} Flere og flere elever prøver at snyde. Hvad er din holdning til det, Sven-Erik?
\says{S} Det er selvføgelig et problem, men nu går jeg jo ikke i skole mere og dengang jeg gik i skole, der snød jeg alti' alti'. Det var da godt de ik' fangede mig da jeg snød allermest.
\says{M} Man må da ikke snyde.
\scene{Tager en øl frem imens han siger...}
\says{S} Der' sgu så meg' man ik' må. Skål.
\says{D} Store dele af kritikken mod folkeskolen går på at brøkregnereglerne ikke sætter sig fast. Hvad siger du, Anders?
\says{M} Det kunne måske - eventuelt - have noget at gøre med at de ikke synes at reglerne giver mening, men forsøger at lære dem udenad. Det er jo egentlig bare at gange med den omvendte.
\says{D} Sven-Erik, kender du reglen, om at gange med 0.
\says{S} Ja, jeg ganger med det, det gør jeg alti' alti', ogs' når jeg dividerer. Det er jeg li'gla' me'.
\says{M} Man må jo ikke dividere med nul. Det giver jo ingen veldefineret mening.
\scene{S tager en smøg frem}
\says{S} Der' sgu så meg' man ik' må.
\scene{S tager en lighter frem, forsøger at tænde den, men M tager den fra ham} 
\says{D} Adolf, et af de større problemer i de danske folkeskoler er at børnene mangler disciplin. De pjækker, afleverer ikke deres opgaver til tiden og er generelt uopdragne. Hvordan klarede du det i din tid som kansler?
\says{H} Wir haben Sommerlejr introduziert, zu fremme die Kinders konzentration.
\says{S} Man skal bare give dem nogle tæv. Det gør jeg alti' alti'.
\says{M} Man må jo ikke slå børnene.
\scene{Tager et sammenfoldeligt fodgængerfelt og folder det ud, og Hitler tager en rød lyskurv op, mens der siges...}
\says{S} Der er sgu så meget man ikke må.
\scene{S går over for rødt.}
\says{D} Adolf, der er meget tale om at eksamen skal være mere blød. Mindre nazi om man så må sige.
\says{H} Das gymnazium ist deevaluert. Dass ist eine platz für das eliten. Wir müssen atgangskrav einführen! Meine atgangskrav est Blau eine, lysen hår, normalen hutfarve, UND grosse Br...
\says{D} Og det er netop adgangskrav som vi skal til at tale om.
\scene{Hitler vandrer ud i vrede.}
\says{D} Det er jo et stort problem når eleverne ikke opfylder adgangskravene til deres ønskede studier.  Hvad har du at sige til det Bertel?
\says{B} Jeg aner det ikke.
\says{D} Jamen synes du ikke at du har et medansvar når eleverne ikke placerer sig, der hvor de burde på karakterskalaen?
\says{B} Jeg anede slet ikke der var en karakterskala!
\scene{Bertel ranter og vandrer ud i vrede.}
\says{S} Vi kan jo bare bøje reglerne, det gør jeg alti' alti'. Det er jeg li'gla' me'
\says{M} Man kan altså ikke altid bare bøje reglerne.
\scene{S tager en pistol frem og skyder}
\says{S} Det kan man alti alti.
\scene{Lys ud.}

\end{sketch}
\end{document}
