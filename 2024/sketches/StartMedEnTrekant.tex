\documentclass[a4paper,11pt]{article}

\usepackage{revy}
\usepackage{amssymb}
\usepackage{amsmath}
\usepackage{commath}
\usepackage{amsfonts}

\revyname{MatematikRevyen}
\revyyear{2024}
\version{0.1}
\eta{3 minutter}
%\status{$\neg$færdig }

\title{Start med en trekant}
\author{Louie 24', Louise 23', Malthe '20, Magnus '21, Hugo '21, Villads '19}

\begin{document}
\maketitle

\begin{roles}
\role{I}[KE] Instruktør
    \role{S1}[Philip] Studerende 1, den lumre
    \role{S2}[Kisser] Studerende 2, gør ALT for et 12-tal
    \role{S3}[Louise] Studerende 3, synes det er for meget
\end{roles}

\begin{sketch}
\scene Louise og Kisser sidder i kantinen og skal lige til at begynde på en aflevering. Philip kommer ind fra sidetæppet.
\says{S1} (\textit{Sagt lidt lummert}) Nårh, skal vi se at få kigget på den aflevering?
\says{S2, S3}[træt] Åh ja, det kan vi godt.
\says{S2} Lad os se. Vi skal åbenbart bevise Goursats lemma. Men der er heldigvis givet et hint.  Der står at vi skal \textit{starte med en trekant}? (rystet i kursiv)
\says{S3} Ahhva?! Er det ikke lidt upassende? Har I prøvet det før?
\says{S1} (\textit{lidt tøvende}) Hmmm, det var jo den ene gang hvor mentoren ikke kom til mentormødet...
\says{S3} Ej, det kan simpelthen ikke være det vi skal. 
\says{S1} Den er altså god nok, der står at vi skal vise Coochie-Sætningen.
\scene Philip og Kisser stopper op og tjekker hinanden ud. Måske en lille smule småflirt.
\says{S3} Cauchy! Det hedder Cauchy.
\says{S1} Jeg har altså altid læst det som Coochie.
\says{S2} Nå, lad os komme i gang. Vi skal nok bruge noget vi har lært...
\says{S3} Altså per trekantsuligheden kommer det aldrig til at virke. Der vil altid være ulighed i en trekant.
\says{S1} (\textit{Belærende, selvsikkert og lummert}) Ahh, ikke helt. Hvis vi bare alle er \textit{positive} så er der faktisk lighed i trekantsuligheden!
\says{S2} Jeg har en anden idé. Hvis nu jeg er $f$ og du Philip er $g$, så tager vi $f$ bolle $g$ og bruger kædereglen. 
\says{S1} God ide! Det er sjovt, du siger det.
\scene S2 tager håndjern frem.
\says{S1} Prøv lige at komme med din hånd, Kisser.
\scene Louise sidder med armene over kors, mens Kisser får håndjern på (overrasket, men accepterer det)
\says{S3} Det bliver uden mig!
\says{S2} Måske vi skal bruge noget fra algebra 1... jeg føler virkelig at jeg havde styr på grupper efter det kursus.
\says{S1} Ja! Måske noget med cykler? Vi kunne lave en tre-cykel. Først mig, så dig, så dig. Og så mig igen.
\scene{Philip hiver et nyt sæt håndjern frem og peger mod Louise. Louise læner sig væk fra Philip, falder ud af stolen og rejser sig op.}
\says{S3} (\textit{Louise er meget oprevet}) Ej, det er simpelthen for ulækkert. Jeg finder en ny gruppe og så må i klare jer uden mig!
\scene Louise går i protest
\says{S2} Hmmm, hvad gør vi så, nu er vi jo kun 2.
\says{S1} (\textit{Meget lummert}) Hvad med forelæseren i SS?
\says{S2} Hvad med hende?
\says{S1} Kan du ikke huske forelæsningerne sidste år? Det var altid megafrækt...
\says{S2} (\textit{lamslået}) Øhhhhh? 
\says{S1} Kom nu! Hvad er det værste, der kan ske
\scene{Philip bruger håndjernene til at rive Louise med ud af scenen}
\scene{Lys ned!}
% Altså Henrik sagde til forelæsningen at hans dør altid står åben...
% \says{S2} Og han snakker altid om sin epsilon-pølse.
% \scene lys ned! 
\end{sketch}

\end{document}

