\documentclass[a4paper,11pt]{article}

\usepackage{revy}
\usepackage[utf8]{inputenc}
\usepackage[T1]{fontenc}
\usepackage[danish]{babel}


\revyname{MatematikRevyen}
\revyyear{2024}
\version{0.1}
\eta{$3$ minut}
%\status{Færdig}

\title{En instruktor siger op}
\author{Villads 19'}

\begin{document}
\maketitle

\begin{roles}
\role{I}[Snow] Instruktør
\role{I}[Christian] Instruktor
\role{F}[Baldur] Forelæser og instruktorkoordinater
\end{roles}


\begin{sketch}
\scene{C sidder på sit kontor og arbejder da I banker på døren og kommer ind}

\says{I} Hej F! Jeg vil gerne tale om min stilling som instruktor!

\says{F} Ja, hvad kan jeg hjælpe med? 

\says{I} Nå, men der er såmænd ligetil, chef. Jeg siger op!

\says{F}[Noget forvirret] Men... har du ikke kun haft  én øvelsestime?

\scene{I er stolt og retter ryggen}

\says{I} Jo!

\says{F} Men hvad er så grunden?

\says{I} Prhhh, ja, jeg ved næsten ikke hvor jeg skal starte... der er så meget. Altså for det første skulle jeg gå hele vejen over til DIKU i regnvejr. Altså hvis jeg i det mindste ikke skulle undervise datøkkere... 

\scene{F er lidt skeptisk}

\says{F} Okay... er det alt?

\says{I} Og så skal jeg løse opgaver!? På tavlen? Helt selv, foran en masse studerende. Helt ærligt, det er jo stort set som An0-eksamen om igen.

\says{F} Men det er vel ikke noget problem for en dygtig studerende som dig?

\scene{I lader lidt som ingenting og kigger væk}

\says{F} Vel?

\scene{Der er stille i et sekund eller to}

\says{I} Og så spørger de om hints til afleveringen! *sagt lidt som om I skammer sig* altså jeg kunne jo ikke engang løse dem sidste år, så hvordan skulle jeg kunne nu, der er jo ikke engang vejledende besvarelser? Og det der med at skulle vide ting, det ved jeg ikke helt om er mig!

\scene{F begynder at blive provokeret}

\says{F} Men hvorfor i helvede har du så søgt om at blive instruktor?? 

\scene{I retter ryggen}

\says{I}For at få gratis kaffe og print sir! \emph{(kort pause)} Men nu har de fjernet printerne og kaffemaskinen virker ikke. Så du må forstå at det ikke giver mening for mig længere!

\says{F}[sagt i en lidt hård tone som om F er ved at være træt] Men i det mindste så må det da være meningfyldt at undervise en masse interesserede studerende?

\says{I} Interesserede? Altså når ikke siger noget er der bare helt stille. Det er da mega akavet, det er næsten ligesom på Barwin. Og altså, tror du ikke det går ud over mig at være sammen med så mange deprimerede mennesker? Jeg bliver jo helt deprimeret selv.

\scene{F er oppe i det røde felt og rejser sig op under replikken}

\says{F} Deprimeret? Ved du overhovedet hvad det vil sige at være deprimeret?
At stå op klokken 6:30 om morgen, for at undervise 300 studerende, den ene halvdel sidder på deres telefoner, og den anden halvdel af dem skriver sketches til revyen hvor du bliver nedgjort. Det er deprimerende. Larmen fra folk der installerer brandalarmer og borer i loftet dagen lang, uge efter uge, det er deprimerende. Og forsknin-
gen... og du snakker om vejledende besvarelser!... jeg har siddet og revet håret ud af hovedet over et problem i 8 år, og der er fandeme ikke nogen vejledende besvarelser! Og så her til morgen er der et eller andet 21-årigt geni fra Indien der ligger en kort og elegant løsning op i et preprint. Det er deprimerende.

\scene{Der er stilhed i et par sekunder }

\says{F} Ved du hvad... Jeg tror jeg forstår dig. Jeg siger op!

\scene{Lys ned!}

\end{sketch}


\end{document}