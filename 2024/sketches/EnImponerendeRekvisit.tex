\documentclass[a4paper,11pt]{article}

\usepackage{revy}
\usepackage[utf8]{inputenc}
\usepackage[T1]{fontenc}
\usepackage[danish]{babel}


\revyname{MatematikRevyen}
\revyyear{2024}
% HUSK AT OPDATERE VERSIONSNUMMER
\version{0.1}
\eta{$3$ minutter}
%\status{Færdig}

\title{En Imponerende Rekvisit}
\author{Frida 21'}

\begin{document}
\maketitle

\begin{roles}
\role{I}[Stine] Instruktør 
\role{I}[Niklas] Instruktør
\role{S}[Kisser] Rekvisitperson
\role{N}[Hugo] Ninja
\role{N}[Philip] Ninja
\role{N}[Thea] Ninja
\end{roles}

\begin{props}
\item En stor drage
\end{props}


\begin{sketch}
\scene{R står på scenen}

\scene{Lys op (I kommer ind på scenen)}

\says{I} Hva Kisser, du er i rekvisitten ikke?. 

\says{R} Jo det er jeg. 

\says{I} Godt, for jeg er kommet i tanke om en rekvisit jeg mangler.

\says{R} Ja. Det er lidt sent du kommer med det. Vi er jo allerede igang med forestillingen \emph{(R laver gestus ud om publikum)}

\says{I} Jojo, men I er jo så gode, så vil du ikke nok. Please \emph{(I sætter sig ned på knæ og beder)}

\says{R} Okay så. Hvad vil du have?

\says{I} Jeg vil gerne have en drage. Den skal være grøn med skæl og have syleskarpe klør. Og det ville være så fedt hvis den kunne baske med vingerne. Sådan RAAARRRH \emph{(I begynder at baske med armene)}

\says{I} Og ild. Der skal være masser af ild. Store flammer. 

\says{R} Men der må jo ikke være ild på scenen..

\says{I} Nårh nej.. \emph{(I klør sig bag hovedet)}

\says{I} Men kan du ikke bare lave noget der ligner?

\says{R} Så måske noget af stof?

\says{I} Det ved jeg ikke. Det er jo ikke mig der er i rekvisitten.

\says{R} Okay. Så du du vil have en drage, der er grøn og med skæl og har skarpe klør, og du vil have at den skal kunne baske med vingerne og spy ild, right? \emph{(R er meget ironisk)}

\says{R} Er det alt, eller skal den også kunne brøle? \emph{(I fanger ikke ironien)}

\says{I} Orv. Det ville være fedt. 

\says{R} Og du er sikker på at det er alt? helt sikker?

\says{I} Ja

\says{R} Jeg ser hvad jeg kan gøre 

\says{I} Fedt. Kan du skynde dig? Den skal bruges lige om lidt. 

\scene{R forlader scenen. I står og ser fornøjet ud. Der lyder høje voldsomme lyde, som store brag og en motorsav omme backstage.}

\says{I} Øhm, er alt okay derude? 

\scene{R stikker hovedet ind af sidetæppet med sod på kinderne}

\says{R}[Ironisk og lidt forpustet] Det er så fint!

\scene{R stikker hovedet ud igen. Der kommer flere voldsomme lyde før R endelig kommer ind på scenen igen med flænger i tøjet og rodet hår. R ser smadret men stolt ud over sit arbejde.}

\says{R} Det lykkedes. Den er måske ikke optimal, men jeg har gjort det, jeg kunne. 

\scene{Tæppet går fra, og en stor flot drage kommer til syne}

\says{R} Tadaa! \emph{(R er meget tilfreds med sig selv. I ser ikke lige så begejstret ud)}

\says{I} Glemte jeg at sige at den skal kunne være i en taske? \emph{(I tager sin ryksæk frem)}

\scene{Tæppe}
\end{sketch}
\end{document}