\documentclass[a4paper,11pt]{article}

\usepackage{revy}
\usepackage{amssymb}
\usepackage{amsmath}
\usepackage{commath}
\usepackage{amsfonts}

\revyname{MatematikRevyen}
\revyyear{2024}
\version{0.1}
\eta{? minutter}
%\status{Færdig}

\title{Den Store Klagedyst}
\author{Augusta '20}

\begin{document}
\maketitle

\begin{roles}
\role{I}[Sommer] Instruktør
\role{T}[Frigg] Tim Vladimir
\role{X}[Thor] Deltager 1 - den vrede
\role{Y}[Niklas] Deltager 2 - den aktivistiske
\role{S}[Thea] Speaker (ikke på scenen)
\end{roles}

\begin{props}
    \prop{1} 2 x store skåle 
    \prop{2} 2 x store skeer
    \prop{3} 2 x kager som deltagerne bager 
    \prop{4} Ingrediens - ligner en blok smør med teksten "høflighed" på
    \prop{5} Ingrediens - ligner mel med teksten "vrede" på
    \prop{6} Ingrediens - ligner sukker men med teksten "aktivisme" på
    \prop{7} Kæmpe skraldespand
    \prop{8} 3 borde
    \prop{9} 2 x kageforme
\end{props}


\begin{sketch}
\scene Der står tre borde på scenen side om side. Bag hvert bord står en deltager. Foran dem i midten står et bord med klassiske kageingredienser. Ingredienserne med tekst på står bag bordet sådan at man ikke kan se dem. Tim Vladimir står forrest på scenen.

\says{T} Godaften og velkommen til Den Store Klagedyst!

\scene Deltagerne og Tim (og publikum) klapper

\says{T} Deltagere! \textit{*gestikulerer mod de 2 deltagere*} Som I nok husker, var det Y som blev sidste uges mesterklager for sin klage over at begge vandautomater i vandrehallen aldrig virker på samme tid. I denne uge skal I kæmpe om at kreere den lækreste klage over... den gradvise tilbageføring af vandrehallen! \\
I har helt frie tøjler og I kan benytte jer af alle ingredienserne på bordet. \\
Så er der vist ikke andet at sige end klar, parat, klag!  

\scene Nu går alle deltagere i gang med at bevæge sig rundt bag bordene. Der spilles bagedysten-musik i baggrunden. Der kommer (ekstra) lys på deltager Y sådan at det er tydeligt at det er denne deltager speakeren snakker om. Y går hen til bordet med ingredienser, hiver selvsikkert fat i "aktivisme" og går tilbage til sin plads med det. 

\says{S} Y's klage har aktivisme som hovedingrediens. Klagens bund består af at dreje alle bordene i kantinen 90 grader. Som en lækker ganache skal skillevægge bæres tilbage til der hvor de plejede at stå. Til sidst skal klagen pyntes med en glace.   

\scene Y rører i sin skål. Der kommer nu (ekstra) lys på deltager X. 

\says{S} X har valgt at kaste sig ud i en klage-mail til arkitekten. Mailen skal bestå af en bitter bund og syrligt fyld. Til sidst skal klagen pyntes med et \textbf{tykt} lag af høflighed.

\scene Imens det næste sker, hælder Y sin klage i en form og sætter den i ovnen.

\scene X går op til bordet med ingredienser.
X tager fat i "høflighed" men begynder at tøve. X tænker sig om en ekstra gang og ser mere vred ud nu. X sætter "høflighed" ned igen og hiver i stedet "vrede" op. X bærer den tilbage til bordet og begynder at hælde i skålen. 


\says{S} X har glemt alt om den høflige pynt og laver nu i stedet en fondant bestående udelukkende af vrede.

\scene X hælder fortsat vrede ned i skålen. T bevæger sig over mod X
\says{T} Hej X
\says{X} Hej Tim
\says{T} Nu passer du på med alt den vrede ik?
\says{X} Jojo, jeg følger bare min mors opskrift, hun var Rødstrømpe!
\says{S} I X's klage indgår der vrede. I vrede indgår der bl.a. nordrenalin, og tilføjer man for meget, kan man risikere at ens klage bliver enormt sur. 
\says{T} \act{Lidt stille, ud mod publikum} Jeg frygter altså at X har puttet \textit{lidt} for meget vrede i sin klage...i det mindste bliver den ikke tør!

\scene X sætter sin klage i ovnen. T begynder at bevæge sig over imod Y's side af scenen

\scene Y hiver kagen ud af ovnen, men den falder fra hinanden og Y kæmper at holde sammen på den.

\says{T} \act{Kigger ud mod publikum, som om han taler til et kamera} Uuuuh, åh nej, det ser ud til at Y er kommet til at give den lidt for meget gas med aktivismen! 
\says{S} Ja, ganske rigtigt er Y kommet til at tilføje lidt for meget af det farlige stof, Aktivisme. At balancere Aktivisme er blot en af de ting professionelle klagere lærer i løbet af deres uddannelse. Putter du for lidt i, kommer din klage til at virke som et mediestunt som ingen tager seriøst, men putter du for meget i, som i Y's tilfælde, bliver klagen usammenhængende og ligner nu mere hærværk.  

\says{T} Der er 2 minutter tilbage! 

\scene Deltagerne begynder at bevæge sig hurtigere. X hiver kagerne ud af ovnen (dvs op fra bag bordet) og begge deltagere pynter med sprøjteposer. Der kommer lys på X.

\says{S} X har indset sin fejl, ved at have puttet for meget vrede i. Han/hun gør nu et sidste forsøg på at rette op på sin finish ved at tilføje et "på forhånd tak" til sidst. 


\scene Deltagerne pynter fortsat.

\says{T} 3, 2, 1, træd væk fra jeres klager!  

\scene Deltagerne springer tilbage og venter i stilhed. Der går et par sekunder.

\says{T} Tak for i aften! Se med i næste uge når-

\says{Y} \textit{*afbryder*} Hallo! Hvem er vinderen?

\says{T} Nååå ja... vinderen! Ja... nu skal I se!

\scene T går over bag tæppet og hiver en kæmpe skraldespand frem.

\says{T} Hvis I nu tager jeres klager og smider dem herned  så \textit{lover} vi at administrationen ser på dem inden for et år! \\

\scene Lys ned.

\end{sketch}

\end{document}

