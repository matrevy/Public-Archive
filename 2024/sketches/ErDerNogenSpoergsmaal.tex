\documentclass[a4paper,11pt]{article}

\usepackage{revy}
\usepackage[utf8]{inputenc}
\usepackage[T1]{fontenc}
\usepackage[danish]{babel}


\revyname{MatematikRevyen}
\revyyear{2024}
% HUSK AT OPDATERE VERSIONSNUMMER
\version{0.1}
\eta{$3$ minutter}
%\status{Færdig}

\title{Er der nogen spørgsmål?}
\author{Sirius 23', Carl 21', Louise 23’, Malthe 20', Villads 19'}

\begin{document}
\maketitle

\begin{roles}
\role{I}[Stine] Instruktør
\role{M}[Johan] Mikael Rørdam
\role{S1}[Anni] Den forvirrede
\role{S2}[Elinborg] Den irriterede 
\role{St1}[Inga] Statist
\role{St2}[Lise] Statist
\role{St3}[Niklas] Statist
\role{St4}[Lea] Statist
\end{roles}


\begin{sketch}
\scene{Vi er til forelæsning. M står ved tavlen og S1, S2 og S3 ser på.}

\says{M} Og hvis vi bemærker at $\lambda$ er tællelig additiv, har vi vist eksistens af Lebesguemålet. Er der nogen spørgsmål?

\says{S1} \emph{(Hvisker til S2)} Spørgs-målet? Har vi haft om det mål?

\says{S2} Shhh, jeg prøver lige at følge med.

\says{M} Nej... Så er vi færdige med det gøremål.

\says{S1} Først spørgs-målet og nu gøre-målet? Jeg føler mig virkelig bagud, forstår du noget S2?

\says{S2}\emph{(Lidt irriteret)} Ej gider du lige at tie stille, jeg kan slet ikke koncentrere
mig når du sidder og snakker hele tiden.

\says{S1} Undskyld, jeg tror bare ikke jeg helt forstår, hvad man kan bruge målteori til.

\says{S2} Altså min søster læser Jura, og hun snakker hele tiden om søgsmål så tror faktisk det er ret anvendt.

\says{S1} Ahhh, jeg forstår, min storebror er også hele tiden på bodega og kommer i slagsmål.

\scene{S2 ser forvirret ud.}

\says{M} Hov, hov ro på! Hvis ikke I er stille når vi ikke i mål med dagens program.

\scene{S1 kigger forvirret over på S2.}

\says{S1} I-mål?!? Hvad er nu det for noget?

\says{S2}\emph{(Meget kraftigt shhhh)} Shhhh! Hvis du nu bare var stille og fulgte med, så ville du nok ikke være så forvirret over det her kursus!

\says{M} \emph{(Skriver på tavlen, og tegner grafen for en funktion)}: Nå, nu hvor vi har vist eksistensen af Lebesguemålet, er der så nogen af jer der kan regne det her integrale ud?

\says{S1}\emph{(Rækker selvsikkert hånden op og svarer før de bliver valgt)} Vent nu tror jeg jeg forstår det! Kan vi ikke bare bruge øjemålet?

\says{M} \emph{(Roder nervøst rundt i sine noter og mumler for sig selv)}: Øjemålet - hvordan er det nu det er defineret?

\scene{Tæppe}
\end{sketch}
\end{document}