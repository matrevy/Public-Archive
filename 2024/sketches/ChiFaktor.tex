\documentclass[a4paper,11pt]{article}

\usepackage{revy}
\usepackage[utf8]{inputenc}
\usepackage[T1]{fontenc}
\usepackage[danish]{babel}


\revyname{MatematikRevyen}
\revyyear{2024}
% HUSK AT OPDATERE VERSIONSNUMMER
\version{0.1}
\eta{$~5$ minutter}
%\status{Færdig}

\title{$\chi$-faktor}
\author{Sommer '17, Stine '18 og Carl '21}

\begin{document}
\maketitle

\begin{roles}
\role{I}[Stine] Instruktør
\role{V}[Inga] Vært på $\chi$-faktor
\role{S}[Baldur] Studerende, der gerne vil være den næste instruktor
\role{D1}[Christian] Veteran-dommeren, han har set hundredevis af instruktorer komme og gå
\role{D2}[Louise] Den moderlige dommer, som altid tager højde for at instruktorerne har det godt
\role{D3}[Thais] Den unge hot-shot dommer, der ikke kan lade være med at få alting til at handle om ham selv
\end{roles}

\begin{sketch}
\scene{Der starter med at være slukket lys på scenen. Sketchen starter ude i sidegangen på trappen, hvor der er spot på V og S}
\says{V} Hej og velkommen tilbage til $\chi$-faktor. Jeg står her med S, og S vil du ikke fortælle lidt om dig selv.
\says{S} Jo hej, jeg hedder S, jeg er 23 år gammel og jeg kommer fra Køge.
\says{V} Og S, hvorfor tror du at du har det der skal til for at blive den næste instruktor?
\says{S} Altså det har bare været min drøm hele livet. Jeg har altid følt mig lidt anderledes, og de andre i gymnasiet drillede mig altid med, at i stedet for at feste, ville jeg bare hjem og rette fiktive afleveringer.
\says{V}[Næsten ved at græde] Wow det lyder godt nok hårdt. Hvordan har din familie haft det med det?
\says{S} Jamen de har altid støttet mig helt enormt, og jeg har faktisk taget dem alle sammen med i dag, og de sidder ude bagved og hepper på mig
\says{V}[Meget begejstret og næsten med baby-sprog] Ej, det er så vigtigt med sådan noget support. Jamen S, jeg vil ikke sige så meget mere, jeg håber du har hvad der skal til for at blive den næste instruktor, og så krydser vi fingre for at dommerne er søde mod dig. Pøj pøj derinde!
\says{S} Tak.
\scene{Spot slukkes nu, og i stedet kommer der lys på scenen. Her sidder de 3 dommere ved et bord. Der står en tavle på scenen, og S kommer ind på scenen}
\says{D1, D2 og D3} \act{De siger det sådan lidt oven i hinanden} Hej med dig / velkommen / heeej
\says{S} Hej.
\says{D1} Hvad hedder du?
\says{S} Jeg hedder S
\says{D2} Hej S. Hvad har du forberedt til os i dag?
\says{S} Jeg har øvet mig på min introduktion til en første time i Mat-intro
\says{D3} Okay, du går bare i gang
\scene{S virker selvsikker, og nikker op til bandet. En føleren sang går i gang (F.eks. Breaking Free fra High School Musical, lige inden der kommer sang på, tager S en stor indånding, men går så i stedet i gang med at snakke og musikken cutter fra}
\says{S} Hej hold 7, mit navn er S, og jeg skal være jeres instruktor i MatIntro. Min KU-mail er \act{vender sig mod tavlen og tager et stykke kridt op, men taber det så}
\says{S} \act{Hvisker} Shit.... \act{Prøver at samle kridtet op, men taber det igen, er tydeligt rystet}
\says{D2} Er... Er du okay?
\says{S} Ja... \act{Prøver at genvinde situationen, men er tydeligvist lidt hevet ud af den} såøh min KU-nummer... øhm.... Ej jeg undskylder virkelig... Er-er det okay jeg prøver igen?
\says{D3} Jaja det er helt naturligt, bare tag en dyb indånding, og så prøv igen.
\scene{S tager en dyb indånding, og nikker så til bandet, der spiller det samme igen, og cutter samme sted}
\says{S} Hej S, mit navn er hold 7 og jeg skal være.. ej fuck... det sker altså aldrig derhjemme...
\says{D2} Såså, det kan også være enormt nervepirrende at skulle starte en matintro øvelsestime.
\says{D3} Ja altså jeg kan huske da jeg selv var instruktor. Ja okay dér gik det rimelig fucking godt, men nogle af de andre instruktorer havde enormt svært ved det, men altså alt endte jo fint
\says{S}\act{Snøfter lidt, men kigger fortrøstningsfuld op på D3} Hvad gjorde de andre instruktorer?
\says{D3} Ja altså de sagde op, og alle eleverne rykkede ind på mit hold, men det var fint, der er ikke den udfordring jeg ikke kan klare.
\says{S} Nå... Må jeg godt prøve én gang til?
\says{D1} Altså nu bliver jeg lige nødt til at afbryde. Jeg beklager Buldar..
\says{D2} Han hedder altså Baldur..
\says{D1} Ah Balder, men tro mig når jeg siger, jeg har set hundredvis af unge studerende som dig, som tror de har lige hvad der skal til, men som altid knækker under presset. Jeg tror at en kattekilling ville have mere held med at instruere frømandskorpset end du ville have foran et hold førsteårs-studerende!
\says{D2} Ej, D1, det kan du da ikke bare sige
\says{D1} Jamen jeg beklager, men altså det er fint nok at tøve lidt her til audition, men hvad gør vi i en live-situation, når Bulgur her ikke engang kan kende forskel på sin KU-nummerplade og en kageliste? Det bliver et nej fra mig
\scene{S bliver nu meget ked af det}
\says{D3} Altså jeg sikker på med lidt øvning, så kunne du måske en dag blive lige så god som mig... okay med meget øvning... men det er en chance jeg ikke tør tage, så det bliver desværre også et nej fra mig.
\scene{S er nu knust. Hans drøm er gået i vasken}
\says{D2} Jeg er ked af det S... Jeg ville ønske der var noget jeg kunne gøre, men du har bare ikke hvad der skal til for at blive instruktor i et af vores mere profilerede kurser...
\scene{S er på knæ og græder. D2 får tydeligvist meget ondt af ham.} 
\says{D2} Okay jeg har måske en idé, men jeg kan ikke love for meget
\scene{S kigger med fornyet håb op mod D2}
\says{D2} Der er andre som dig, der viste potentiale, men ikke var helt stærk nok solo... Så jeg tænkte...
\says{D1} Åh nej, please fortæl mig du ikke skal til at forslå en gruppe...
\says{D2} Vi laver sgu da en gruppe, og så kan I få lov til at være lektiehjælpere i kantinen!
\scene{D1 og D3 himler med øjenene / er IKKE imponerede, imens mister S alt resterende håb og græder igen ned i sine hænder}
\scene{Tæppe}
\end{sketch}
\end{document}