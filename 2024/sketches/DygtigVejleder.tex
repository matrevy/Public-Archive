\documentclass[a4paper,11pt]{article}

\usepackage{revy}
\usepackage[utf8]{inputenc}
\usepackage[T1]{fontenc}
\usepackage[danish]{babel}


\revyname{MatematikRevyen}
\revyyear{2024}
\version{0.1}
\eta{$3$ minut}
%\status{Færdig}

\title{Dygtig vejleder}
\author{Malthe 20' \& Hugo 21'}

\begin{document}
\maketitle

\begin{roles}
\role{I}[Stine] Instruktør
\role{S}[Louie] Studerende
\role{V}[Hugo] Vejleder
\end{roles}


\begin{sketch}
\scene{En studerende står ude foran døren til et kontor}

\says{S} Ej, hvor jeg glæder mig til min første vejledning. Min vejleder er bare sååå dygtig, jeg skal bare lære alt om hvordan han arbejder.

\scene{S banker på døren og træder ind på kontoret}

\says{V} Kom ind. 

\scene{V sidder i skrædderstilling oppe på sit bord og har meget hippie-vibes}

\says{S} Før vi går i gang vil jeg bare lige sige, at jeg prøvede at læse din seneste artikel, og jeg synes det var virkeligt interessant at se, hvordan du brugte ideer fra homotopiteori til at...

\scene{S bliver afbrudt af V}

\says{V} Det første du skal lære er at matematik ikke er interessant. Det er sandt eller falskt.

\says{S} Nå ja okay,... øhm. Det er et \emph{(kort pause)} sandt specialeemne vi fik aftalt sidst, men jeg ved slet ikke hvor jeg skal starte?

\says{V} Det vigtige er ikke hvor du starter, men hvor du ender.

\says{S} Hvordan ender jeg så? 

\says{V} Med Q.E.D. Eller den der firkant.

\says{S} Okay... Hvordan får jeg overblik over processen? Det er jo et stort projekt. 

\says{V}[Flyvsk] Forestil dig det hele på én gang - det gør jeg altid. Og æbler.

\says{S} Æbler?

\says{V} Ja, det er efterår. Gå efter elstar og cox orange, de er gode nu. Aldrig discovery; jeg prøvede det engang, matematikken blev sat tilbage med en måned.

\says{S} Virkelig?

\says{V}[Bestemt] Ja. Og Karl-Johan svampe.

\says{S}[Hiver notesbog frem og skriver hektisk ned] Men øhm, jeg prøvede faktisk at gå i gang med at læse, men jeg forstod ikke mere end de første par linjer af teorien. Ville du kunne forklare mig noget af...?

\says{V} Teori? Det er bare sådan noget med definitioner og sætninger hele tiden, det bliver lidt ensformigt i længden, synes du ikke? Når du forestiller dig matematikken skal det slå gnister.

\says{S} I mig?

\says{V}[Bestemt] Ja. Og i tallene \emph{(Pause)}. Når man læser et bevis skal man lukke øjnene og drømme; drømme om den næste linje i beviset. Og så den næste. Det er lige som et induktionsbevis.

\says{S} Skal jeg så åbne øjnene og tjekke efter?

\says{V} Ja! Øh, jeg mener nej. Du fortsætter. Indtil du kan forestille dig de næste $n$ linjer, og så de næste $n+1$ sider.

\says{S}[Skriver ned i notesbog] Så man læser kun den første linje??

\says{V} Det jeg prøver at sige er... Find på en rigtig god titel.

\end{sketch}


\end{document}