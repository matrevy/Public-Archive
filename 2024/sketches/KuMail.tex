\documentclass[a4paper,11pt]{article}

\usepackage{revy}
\usepackage[utf8]{inputenc}
\usepackage[T1]{fontenc}
\usepackage[danish]{babel}


\revyname{MatematikRevyen}
\revyyear{2024}
\version{0.1}
\eta{$3$ minut}
\status{Færdig}

\title{KuMail}
\author{Carl '21, Hugo '21, Malthe '20 \& Baldur '20}

\begin{document}
\maketitle

\begin{roles}
\role{I}[KE] Instruktør
\role{K}[Lise] Kumail entusiast
\role{V}[Hugo] Ven
\end{roles}

\begin{props}
\item 2 computere (en mac, en windows)
\end{props}
\begin{sketch}
\scene{K og V sidder ved et bord. K har sin computer fremme. V leder efter sin i tasken. V tager den lille drage op fra rekvisitsketchen. V kan ikke finde sin computer i tasken.}

\says{V} Ugh, jeg har glemt min computer. Må jeg ikke lige låne din hurtigt?

\says{K}[tøvende] Argh, jeg er ikke så meget for at låne min computer ud... Hvad skal du?
% Herfra ligesom normalt:
\says{V} Jeg skal bare lige tjekke noget på absalon hurtigt. Hvad kan du have på den jeg ikke må se?

\scene{Stilhed, intest øjnkontakt mellem K og V}

\says{K} Okay, så 

\scene{K giver computeren til V. Der kommer en computer op på AV, der er åbent på Google}

\says{V} Hmm, hvordan er det nu man gør på Windows...

\scene{V trykker på lidt forskellige knapper. Browserhistorikken åbner på AV. Den er fyldt med google søgninger på 'kumail'}

\says{V} Hov! det var vist browserhistorikken.

\scene{K får en chok og ser nervøs ud}

\says{V} ... Hvorfor ... er der udelukkende søgninger på "kumail"?

\says{K}[griner nervøst] Jooh, hehe, deeeet.. Det bare fordi. \textit{(K ligner en der lige har fået en god idé og siger mere selvsikkert)} At jeg har lidt problemer med min bachelor, så jeg har skrevet meget med min vejleder.

\says{V} Ahh okay, men hvis det er en side, du bruger meget, så kan du jo gøre det til en 'favorit'. Lad mig lige vise dig det.

\says{K}[uroligt] Nej, nej, nej, det behøver du virke..

\says{V}[afbryder] Se her! Vi søger bare på kumail på Google og så

\scene{V når kun at
skrive 'ku' før Google foreslår "kumail shirtless", som der er blevet søgt på før. V kigger spørgende hen på K.}

\says{K}[meget presset] Øhhh, autokorrekt?

\says{V} ... fra hvad?

\scene{K ser virkelig nervøs ud. K prøver at tage sin computer tilbage, men V prøver at stoppe det. I 'kampen' kommer de til at klikke på en af søgninger i historikken og AV viser 'kumail' billeder på Google. V siger hurtigt (gerne oven i publikum)}

\says{V}[nysgerrigt] Hold da op, hvad er det, du har søgt på?

\scene{V bladrer igennem billederne og forhindrer K adgang til computeren}

\says{K} Ej, det er altså privat, vil du ikke lade være? Hugo, stop så, giv mig den tilbage!!

\scene{K giver op på Hugo}

\says{K} Teknik. Sluk for ham. Lys ned, lys ned!

\scene{Lys ned!}

% \scene{K ser virkelig nervøs ud. K prøver at tage sin computer tilbage, men V prøver at stoppe det. I 'kampen' kommer de til at klikke på en af søgninger i historikken og AV viser 'kumail' billeder på Google. Stilhed.}

% \says{V}[anerkendende] Hmm, ikke dårligt.

% \scene{Tæppe}


\end{sketch}




\end{document}