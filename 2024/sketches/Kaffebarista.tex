\documentclass[a4paper,11pt]{article}

\usepackage{revy}
\usepackage[utf8]{inputenc}
\usepackage[T1]{fontenc}
\usepackage[danish]{babel}


\revyname{MatematikRevyen}
\revyyear{2024}
% HUSK AT OPDATERE VERSIONSNUMMER
\version{0.1}
\eta{$2$ minutter}
%\status{Færdig}

\title{Kaffebarista}
\author{Frida '21 \& Carl '21}

\begin{document}
\maketitle

\begin{roles}
\role{I}[KE] Instruktør
\role{K}[Lise] Kaffebarista
\role{S1}[Thor] Studerende
\role{S2}[Louise] Studerende
\role{S3}[Elinborg] Studerende
\role{S4}[Minh] Studerende
\end{roles}

\begin{props}
\prop{Rekvisit}[5 kopper]
\prop{Rekvisit}[Bord]
\prop{Rekvisit}[5 stole]
\prop{Rekvisit}[lille blok (tjenerblok)]
\end{props}


\begin{sketch}
\scene{K og S1-S4 sidder ved et bord. K kigger på sit ur}

\says{K} Nå. Så er kl. 14. Tid til at hente kaffe. Er der nogen af jer, der vil med op?

\says{S1} Hmm. Jeg sidder lige midt i en opgave. Kunne du overtales til at tage en kaffe med til mig?

\says{K} Ja, klart. Hvad for en slags?

\says{S1} Bare en sort kaffe.

\scene{K går ud af scenen}

\says{S1} Ej, ved du hvad? Kan den ikke blive en americano i stedet?

\scene{K stikker hovedet tilbage ind}

\says{K} Jo, det kan det godt. Jeg er tilbage om lidt.

\says{S2} Hey, kom lige tilbage!

\scene{K kommer tilbage og har nu barista-outfit på. Frem med notesblokken}

\says{K} Ja, hvad skulle det være?

%\says{S1} Bare en Americano tænker jeg, den er god her et par timer efter frokost.

%\says{K}[skriver ned] Så gerne. Skal I andre have noget med?

\says{S2} Altså, nu du alligevel skal derop, så tænkte jeg på, om du ikke vil give mig halvdelen af en hot milk, så et espresso shot og så en halv hot milk igen?

\says{K}[service-minded] ... mener du en café latte?

\says{S2}[afvisende] Nej nej nej, det smager meget bedre, hvis det bliver blandet på den her måde end den forbehandlede latte, maskinen har.

\says{K}[lidt mindre service-minded] okay, så en americano og en caffe latte.

\says{S2} Nej, altså, hører du ikke efter?

\says{K}[afbrydende] Jaja, en caffe latte på din måde.

\scene{K går hurtigt ud af scenen.}

\says{S3} Hey hej, stop lige!

\scene{K kommer tilbage ind og går stille over til S3 og S4}

\says{K} Ja, hvad med jer?

\scene{S1 giver K forklæde og omvendt barret (barista-hat) på. K har "accepteret sit liv" som henter af kaffe til den halve kantine.}

\says{S3} Jeg vil gerne have en dobbelt espresso.

\says{K}[afbryder] Tak! En normal bestilling

\says{S3}[fortsætter] ...men kun den første halvdel.

\says{K}[forvirret] Det er da bare et enkelt espresso shot?

\says{S3} Jeg ved ikke hvad jeg skal sige, det smager bare anderledes på den måde

\scene{K sukker og vender sig mod S4}

\says{K}[opgivende] Hvad med dig? 

\says{S4}[selvsikkert] Jeg skal have en irish coffee!

\scene{K frustreret smider sit baristahue/hat på gulvet}

\says{K} NEJ! Det kan maskinen ikke lave. Bestil nu bare noget normalt...

\says{S4} Jo jo! Hvis du trykker på americano og cappuccino samtidig og så hurtigt to gange på kaffe, så putter den lidt whiskey i.

\says{S1}[afbryder ivrigt] Også sådan en til mig!!

\scene{Latterpause}

\says{S2}[begejstret] Jeg har lige læst manualen igennem. Kan jeg få en Espresso Martini? 

\says{K} Mener du det?

\says{S2} Jaja, du skal bare trykke på ... \textit{(S2 læser op fra sin computer/telefon/manualbog og trykker i luften på en usynlig kaffemaskine)} kaffe, kaffe, latte. Så hold mælk inde i 2 sekunder. Gå så over på den anden maskine, skift sprog til russisk og bestil en kakao der. Til sidst så tryk på single og double espresso samtidig på den første maskine. Det koster godt nok 1000 yen, men du anmoder mig bare.

\scene{Version 1: K stirrer dødt ud i luften, river siden af blokken og går ud af scenen.}

\scene{Version 2: K stirrer dødt ud i luften, og alt andet lys slukker end et spot, som zoomer ind på K.}

\scene{Lys ned}

\end{sketch}
\end{document}
