\documentclass[a4paper,11pt]{article}

\usepackage{revy}
\usepackage[utf8]{inputenc}
\usepackage[T1]{fontenc}
\usepackage[danish]{babel}


\revyname{MatematikRevyen}
\revyyear{2024}
% HUSK AT OPDATERE VERSIONSNUMMER
\version{0.1}
\eta{$~3$ minutter}
%\status{Færdig}

\title{Hvorfor er der 2 af dem?}
\author{Sommer '17}

\begin{document}
\maketitle

\begin{roles}
\role{I}[Sommer] Instruktør
\role{S1}[Thais] Matematikstuderende 
\role{S2}[Lise] Matematikstuderende med navneskilt hvor S1s navn står på (Det skal være stort/TYDELIGT)
\role{U}[Emma] Studerende som er usikker på hvem der er hvem (Iført revy t-shirt)
\end{roles}
\begin{sketch}
\scene{Lys op. U og S1 sidder ved et bord med en matematik bog og kigger på aflevering.}
\says{U} Altså jeg siger bare, det tager N+1 tørringer, for at finde ud af man egentlig kun havde brug for N
\says{S1} Altså jeg siger ikke du tager fejl, men jeg har altså VIRKELIG svært ved at se hvorfor det er relevant for vores AdVec-aflevering
\scene{De sidder lidt videre i stilhed, pludselig kommer S2 ind på scenen, som gerne må ligne S1 så meget som muligt.}
\says{S2} Undskyld jeg kommer for sent U, jeg blev lige- \act{Afbryder sig selv, da S2 ser S1, og vender sig så ud mod publikum}
\says{S2} Wow, hvad foregår der her?
\says{U} \act{rejser sig op, og er meget chokeret} Vent... Hvorfor er der 2 af jer?
\says{S1} Hvad snakker du om?
\says{U} Har du måske en identisk tvilling du ALDRIG har fortalt mig om?
\says{S1} Nej?
\says{U} Hvordan forklarer du så \act{peger hen mod S2, og bliver en smule forarget} AT DER ER 2 AF JER?
\says{S2} Så U, bare slap af træk vejret. Jeg er sikker på at der er en logisk forklaring på det her, ingen grund til at hæve stemmen eller tage nogle forhastede beslutninger
\says{U}\act{Trækker pistol frem ud af lommen} Nu forholder alle sig i ro! Jeg bliver nødt til at finde ud af hvem den rigtige S er.
\says{S1} Hvad mener du? Jeg er tydeligvis den rigtige S... Hør, hvorfor stiller du os ikke et spørgsmål, som du ved den rigtige mig ville vide?
\says{U} Okay okay okay... Hvad er mit navn?
\says{S1 og S2 i kor} U. 
\says{U}\act{Kigger ned af sig selv, iført revy t-shirt med navn på} Ja okay, det viste bare I begge kan læse... Okay hvad er dit fulde navn?
\says{S1 og S2 i kor} Stine Marie Langhede (Brug U's rigtige "fulde" navn, men smid et dumt mellem navn ind, som f.eks. Stine Marie Langhede)
\says{U} Vent... dit mellemnavn er Marie?
\says{S1} \act{Nærmest lidt skuffet} Ja?.. 
\says{S2} Okay du bliver nødt til at spørge om noget, som du selv kender svaret på!
\says{U}Okay okay... \act{Tager en hånd bag ryggen} Hvor mange fingre holder jeg oppe bag ryggen?
\says{S2} What? Det er ikke fair for nogen af os?
\says{S1} Ja jeg ser ikke hvordan det hjælper?
\says{U} \act{Tager hånden frem igen} Puh det altså bare virkelig svært at finde på gode spørgsmål... Okay hvis du vandt i Lotto, hvad ville du så bruge pengene på?
\says{S2} Jeg er overbevist om at det her er en samtale vi aldrig har haft før
\says{S1} Ja jeg kan i hvert fald ikke huske den
\says{U} Måske ikke, men jeg føler jeg kender dig godt nok til at vide hvad du ville svare.
\says{S2} Så du efterlader det her til din egen subjektive fornemmelse af hvad jeg ville bruge penge på
\says{S1} Jeg ville nok tage min familie ud på ferie og ellers donere en masse til velgørenhed.
\says{U} Aha! Det ville S aldrig gøre! S ville investere i Hagoromo kridt og luksus tavlesvampe!
\says{S2} Hmm nej, jeg ejer allerede nok kridt, og i takt med inflationen falder luksus tavlesvampe nok i værdi. Familieferie lyder skønt og meget mere som mig.
\says{U} Vent hvorfor hjælper du ham?
\says{S2} Jeg vil gerne have du finder ud af jeg den rigtige på fair vis
\says{S1} Ah ja, det ville jeg også
\says{U} Hold op med at blive venner! Okay vi var til forelæsning tidligere, hvad handlede den om?
\scene{S1 og S2 står begge og tripper lidt og kigger på hinanden og er tydeligt i tvivl}
\says{S2} Vi.... definerede hvad en gruppe er?
\says{U}\act{Står lige og tænker}... pis det har jeg sgu selv glemt... Hvad er din yndlingsfarve?
\says{S1} Grøn
\says{S2} Rød
\says{U} Min er orange, jeg ved ikke rigtig hvor det efterlader os... Okay jeg har det! Vi har lavet mange afleveringer sammen, hvad er nogle af mine svagheder?
\says{S1} Du uduelige til at tænke på stedet
\says{S2} Rimelig dårlig til at løse problemer
\says{U} \act{Lidt såret} Øhm ok.. og nogle styrker?
\says{S2} Du er øhm.... 
\says{S1} Din... æhm.... Du er meget....
\says{U} Wow...
\says{S2} Bare spørg os om noget helt konkret! Ikke en holdning, ikke noget hypotetisk!
\says{U} Hvad er kvadratroden af 4?
\says{S2} Som ikke alle og enhver ved...
\says{U} Okay fair point... Løs Riemann hypotesen.
\says{S2} Føler at vi har slået fast at du også skal kunne svaret...
\says{U} You never know
\says{S1} Come on... Vi ved begge du brugte 3 forsøg på MatIntro
\says{U} Det er altså virkelig svært!... Okay hvad med at I stiller I \textit{mig} et spørgsmål som kun \textit{jeg} jeg kan svaret på, og så kan vi se om svarene matcher?
\says{S1} Okay, hvad er vores yndlingsdrink nede på Cafeen?
\says{U} Det kan jeg ikke huske
\says{S2} Hvad er koden til vores skab hvor vi gemmer alkohol til festerne?
\says{U} Ingen idé...
\says{S1} Okay nu \textit{jeg} i tvivl om du er den rigtige U
\says{U} Ej, jeg er altså den rigtige U, jeg er bare dårlig til at huske sådan tilfældige fun facts om vores venskab
\says{S2} Hvordan er koden til vores skab "fun facts"?
\says{U} I know, det bare... sådan med aflevering.. og jeg meget stresset for tiden...
\says{S1} Vent... Han har ret... Det ER ikke en fun fact. Han spurgte dig ikke om det, for at bevise at han selv vidste det, han spurgte dig om det, i håb om at du selv ville afsløre hvad koden til vores skab er! Det er dét han har været ude efter hele tiden! Han er den falske S!
\scene{Retter pistolen mod S2, før han til sidst vender den mod S1 og skyder}
\says{S2} \act{Meget lettet} Wow! Hvordan regnede du ud det var mig der var den rigtige S?
\says{U} \act{Sænker pistolen ned igen} Alle og enhver burde vide jeg aldrig ville kunne følge den slags logik
\says{S2} \act{Går lidt væk, ud mod publikum, og kigger ikke på U længere} Haha ja, tænk du brugte 3 forsøg på det kursus...
\says{U} 4. \act{U Hæver pistolen op igen og sigter på S2} Jeg brugte 4 forsøg, det fortalte jeg dig da godt.
\says{S2} Huh, det vidste jeg ikke... vent! \act{Vender sig mod U, men det for sent}
\scene{Lys ned, men der bliver spillet samme skudlyd som da S1 blev skudt}
\end{sketch}
\end{document}