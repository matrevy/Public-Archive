\documentclass[a4paper,11pt]{article}

\usepackage{revy}
\usepackage[utf8]{inputenc}
\usepackage[T1]{fontenc}
\usepackage[danish]{babel}


\revyname{MatematikRevyen}
\revyyear{2024}
% HUSK AT OPDATERE VERSIONSNUMMER
\version{0.1}
\eta{$~5$ minutter}
\status{Færdig}

\title{Den Matemagiske Familie}
\author{Sommer '17, Stine '18 og Carl '21}

\begin{document}
\maketitle

\begin{roles}
\role{I}[Sommer] Instruktør
\role{Y}[Inga] Dr. Y, Matemagikernes leder
\role{S}[Philip] Sandsynligs Søren
\role{G}[Thais] Geometri Gert (Headset vigtigt!)
\role{T}[Elinborg] Talteori Trine
\end{roles}
\begin{sketch}
\scene{Lys op. S, G og T står tilbage på scenen, iført noget lidt matchende X-men agtigt tøj. Forrerst på scenen står Dr. Y.}
\says{Y} Matematikken er i udvikling. Jeg, Dr. Y, har brugt de sidste 10 år af mit liv, på at udforske grene af matematikken, som ingen turde nærme sig. Her er jeg stødt på nogle helt specielle individer, som fra fødslen tilegnede sig en gave, der adskilte dem fra alle andre. \act{Kunstpause} Matemagikere. De blev behandlet som monstre, udstødt af det samfund der skulle have passet på dem, bare fordi de ikke var som andre matematikere. Men her, i min matemagiske familie, har de fået et nyt hjem, hvor de frit kan udforske deres evner. Selv er jeg familiens tankelæser. Lad mig demonstrere!
\scene{En ninja kommer ind med et stort skilt og noget at skrive med ind og giver det til Y}
\says{Y}\act{Skriver noget ned på skiltet} Okay... Publikum! Råb noget!
\scene{Publikum råber forhåbentligt et eller andet}
\says{Y} Aha! \act{vender skiltet, hvorpå der står "Publikum råber noget dumt"} Der kan I bare se! Men nok om mig! Lad mig præsentere jer for min familie. Se nu Sandsynligheds Søren!
\scene{S begynder at gå mod Y}
\says{Y} I en hver situation kender Søren øjeblikketligt sandsynligheden for SAMTLIGE udfald! \act{Y stikker hånden i lommen} Søren, hvad er sandsynligheden for... \act{Han trækker nu en mønt ud af lommen} AT DENNE MØNT RAMMER PÅ PLAT!?
\says{S} \act{Søren tager sig til hovedet} HMMMMMM..... 50\%!
\scene{Dr. Y står med mønten som om han skal til at flippe den, men tager den så tilbage i lommen}
\says{Y} Ja det er rigtigt, er det ikke vildt, mine damer og herre giv ham en hånd!
\scene{Publikum reagerer forhåbentligt lidt, men ellers går det nok også}
\says{Y} Nååå... Jeg kan mærke at publikum gerne vil se noget \textit{lidt} vildere! Søren! Hvad er sandsynligheden for... at om 30 sekunder... styrter  en kæmpe meteor ned i Store UP1 og dræber ALLE herinde?!
\says{S}\act{S tager sig igen til hovedet og tænker endnu mere højlydt} HMMMMMMMMM... 50 / 50! Enten sker det... eller også sker det ikke!
\says{Y} Hvor er det vildt! 
\scene{Dr. Y prøver at gejle publikum op, men er tydeligvist meget urolig, og kigger op mod loftet, ned på sit ur, og rundt i rummet igen, ned på sit ur, og ånder lettet ud}
\says{Y} \act{Tørrer sved af panden} Phew! Det var godt nok heldigt! Nå, men det må også være nok af det for nu, tak til Søren! Lad mig nu vise jer vores nyeste medlem. Geometri Gert, kom her!
\scene{G går mod Dr. Y}
\says{Y} Geometri Gert, vil du ikke være sød at fortælle publikum, hvad din helt fantastiske evne er?
\says{G} Jeg... Jeg kan se den 4. dimension!
\says{Y} Wow! Den 4. dimension! Hvordan bærer du dig ad med det?
\says{G} Jo altså... \act{G trækker ét par 3D briller ud af lommen} De her tager mit syn fra 2D... \act{Tager dem på} til 3D... og de her \act{G trækker endnu et par ud af lommen og tager dem på} tager mig til 4D!
\says{Y} Wow! Kan du prøve at beskrive hvad du ser?
\says{G} Jamen... Det er meget svært at forklare... okay prøv at forestil dig du har x-aksen... \act{Peger den ene arm vandret ud til siden} og at du har y-aksen \act{Peger den anden arm lodret op i luften} og z-aksen \act{Peger den første arm vandret frem mod publikum}... men så har du bare også... \act{Begynder at ryste med hele kroppen} W! \act{Skal blive ved med at ryste med kroppen}
\says{Y} Spændende!... Men hvad sker der hvis vi nu... \act{Dr. Y trækker selv et par 3D briller ud af lommen, og giver Gert dem på}
\says{G} NEJ! \act{Gert stikker nu også det ene ben ud, og ryster endnu vildere over det hele} DET ER ALT FOR MEGET! \act{Gert falder på knæ, men ryster stadig over det hele}
\scene{Dr. Y går også ned på knæ, og holder om Gert, mens han begynder at tage 3D brillerne af ham}
\says{Y} Sshhh Gert... Jeg pressede dig, det skulle jeg ikke have gjort... Sshhh, vi prøver igen i morgen...
\scene{Gert, stadig meget rystet, får kravlet tilbage på scene. Dr. Y rejser sig i mellemtiden op}
\says{Y}\act{Rømmer sig lige} Nå men... Sidst, skal I få lov til at møde den allerførste Matemagiker jeg stødte på. Den dag i dag, er hun stadig den matemagiker jeg er stødt på der besidder de vildeste evner! Talteori Trine!
\scene{T begynder nu at gå mod Dr. Y}
\says{Y} Trine kan, relativt hurtigt, vurdere om et vilkårligt tal er... LIGE... ELLER ULIGE! \act{Dr. Y er helt VILDT begejstret og prøver at få publikum med på hvor vildt det er}
\says{Y} Lad mig nu demonstrere! TeXnikken, kan vi få et tal op på AV?
\scene{Der kommer et sådan ret lavt tal op, gerne sådan under 100}
\says{Y} Trine... er dette tal... LIGE.... ELLER ULIGE?
\scene{T strækker sine arme mod vejret, og tænker tydeligvis meget, før hun til sidst, nærmest trækker svaret ned til sig}
\says{T} Det er.... lige!
\says{Y} Wow, er det ikke vildt mine damer og herre! \act{Dr. Y kigger glad ud mod publikum, men spotter så tvivl blandt dem}
\says{Y} Nå... Jeg kan kunne se blandt jer, at der var enkelte der også havde luret den, men det var nu også et lille tal... TeXnikken, kan vi få ET KÆÆÆÆÆMPE TAL?!
\scene{Der kommer nu et meget højt tal op på AV}
\says{Y} Husk nu Trine, du må ikke overanstrenge dig... Er dette tal... LIGE... ELLER ULIGE?!
\scene{Trine kæmper endnu mere denne gang. Hun falder faktisk ned på det ene knæ, før hun til sidst rejser sig, og atter trækker svaret til sig}
\says{T} Det er... ULIGE!
\says{Y} WOW HVOR ER DET ALTSÅ VILDT! \act{Lille kunstpause, mens folk klapper} Store UP1, det har været en fornøjelse at præsentere jer for min familie, og hvis I ikke har fået nok matemagi, så bare vent til næste års revy, hvor Trine forhåbentligt har fået færdigudviklet sin evne til at vurdere, hvorvidt et givent tal er POSITIVT... ELLER NEGATIVT!
\scene{Tæppe}
\end{sketch}
\end{document}