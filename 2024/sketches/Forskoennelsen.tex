\documentclass[a4paper,11pt]{article}

\usepackage{revy}
\usepackage[utf8]{inputenc}
\usepackage[T1]{fontenc}
\usepackage[danish]{babel}


\revyname{MatematikRevyen}
\revyyear{2024}
\version{0.1}
\eta{$5$ minut}
%\status{Snart Faerdig}

\title{Forskønnelsen}
\author{Carl '21 \& Snow '19}

\begin{document}
\maketitle

\begin{roles}
\role{I}[Sommer] Instruktør
\role{A}[Hugo] Arkitekt
\role{P}[Niklas] Projektleder
\role{I}[Louie] Institutleder
\role{N}[Philip] Ninja
\role{N}[Frederikke] Ninja
\end{roles}

\begin{props}
\item Hele HCØ
\end{props}

\begin{sketch}
\scene{P og A står på HCØ og snakker. I kommer ind.}

\says{P} Der kommer institulederen.

\scene{P og A går imod I.}

\says{P} Hej Mogens. Det her er Hugo, arkitekten der står bag nyskabelsen af HCØ. Jeg er glad for at i havde mulighed for at kommer her midt i sommerferien, før de studerende er tilbage.

\says{A} Selvfølgelig, når man arbejder med en projektleder som dig.

\says{P} Ej, du smigrer. Som jeg skrev til jer, har vi fået mulighed for at nytænke HCØ. Jeg tænker bare vi tager en tur og ser hvad der kan forbedres. Husk, vi brainstormer bare, så der er INGEN dårlige idéer. 

\says{I} Bare af ren nysgerrighed, kommer der også en studerende forbi og giver deres input?

\says{P} Nej, så omfattende kommer det slet ikke til at være. Kom med.

\scene{P, A og I går til den modsatte side af scenen. Det forestiller de grå sofabåse.}

\says{A} Hmm, det er da lidt synd, at der er stillet møbler foran kunsten på væggen.

\says{I} Men... Det er jo ikke særlig flot kunst.

\says{A} Mmjoh... Men det er underordnet, det var en del af den oprindelige \textit{arkitektoniske vision}.

\says{P}[betaget af A] Genialt! Det har du helt ret i! De er væk i morgen. 

\says{I} Woooow, vent vent vent. Er det ikke bare en visionsgåtur? Vi kan ikke bare fjerne studiepladser, der er i forvejen...

\says{P}[afbryder I] Jo jo jo. Vi snakker bare, ingen dårlige idéer. 

\says{A}[får en idé] Uhh! Apropos ikoniske kunstværker. Jeg bemærkede også en \textit{brutalistisk jernstruktur} foran mosaikken, der hvor vi kom ind.

\says{I} Mener du... handicapelevatoren ved aud1?

\says{A} JA, DEN! Det er jo lidt ærgeligt, at den skal være lige der.

\says{P}[Svinger en usynlig tryllestav] Den er væk!

\says{I} Nej nu må i...

\says{P}[afbryder og sætter sin finger på I's læber] Shhhh... \textit{(hvisker)} vi flytter den bare ud til siden. 

\scene{P og A går ind midt på scenen, I løber efter. Vi befinder os i kantinen.}

\says{A} Hmm, jeg lavede faktisk lidt hyggelæsning om bygningens historie i går. Kantinen her plejede at være et åbent rum, hvor der var plads til at tænke storslåede tanker. Lige nu stopper enhver kreativitet jo så snart, den møder sådan en skillevæg der.

\says{P} DET kan vi ikke have! Vores studerende har fortjent de bedste vilkår. \textit{(Svinger en usynlig tryllestav)} De er væk!

\scene{P og A går igang med at flytte en af skillevæggene. I står lidt nervøst i baggrunden, skal til at sige noget, men holder igen. Skillevæggene når kun at blive flyttet lidt før de stopper.}

\says{A} STOP! PRØV lige at se det gulv. Er det ægte grønlandsk granit? Er det sådan over det hele? 

\scene{P og A undersøger gulvet.}

\says{A} Ja! Så flot et gulv skal da ikke gemmes væk.

\says{P} Helt enig! Hvis vi fjerner stolene kan man se det endnu bedre.

\says{A} OG Bordene!

\says{P} OG VINDUERNE!!!!

\says{I} STOOOOOP! 

\scene{P og A falder til ro.}

\says{P}[puster ud] Nåh ja, vi lod os vist rive lidt med. Vi lader vinduerne stå.

\scene{P's telefon ringer.}

\says{P} Jeg bliver lige nødt til at tage den her.

\scene{P begynder at gå lidt væk, mens I fortsat prøver at stoppe A}

\says{P} Mmh.. JA! ... Så I kan gøre det i morgen? ... Super!

\scene{P lægger på og siger begejstret til I og A}

\says{P} Trapperne bliver fjernet i morgen!

\scene{I og A svarer i munden på hinanden}

\says{A} Genialt! De tog også virkelig meget fokus.

\says{I} HVAD?! Det kan du ikke mene? 

\scene{I tager telefonen fra P. A går mod kantinen alene og begynder at flytte ting.}

\says{I} Hør her, du kan altså ikke bare gøre alle de her ting.

\scene{I opdager at A flytter rundt på ting i kantinen og løber over for at stoppe det. Imens tager P telefonen frem og begynder endnu et opkald. Stemningen nu er en forældre, der prøver at holde styr på sine børn.}

\says{I}[forvildet] Nej, nej, nej. Så stopper vi lige med det her.

\says{P}[i telefonen] ... og du ville kunne komme i morgen og installere det? ... Ja, i må godt beholde alle de tavler i fjerner...

\scene{I løber over og tager telefonen fra P. A går strakt igang med at flytte møbler igen. A løber over for at stoppe A.}

\says{I}[til A] Mange tak for alt, men jeg tror altså ikke dine visioner passer til HCØ

\scene{P har taget en ny telefon frem. P taler og nikker lidt mens P går hen mod I og A.}

\says{P}[meget stolt] Så!

\says{I}[bekymret] Åh nej, hvad nu?

\scene{I tager igen telefonen fra P. P tager endnu én frem, som I også tager.}

\says{P} E bygningen har længe forstyrret det samlede indtryk, så i morgen kommer der nogle og flytter den ud i midten af uniparken. Ej jeg glæder mig sådan til at se de studerernes ansigter når de kommer tilbage fra ferie, og ser alle disse ændringer! Og jeg har fundet de perfekte til jobbet!  

\says{I} Hvem?

\says{P} Håndværkerne der stod bag Niels Bohr Bygningen!

\scene{Lys ned}

\end{sketch}


\end{document}