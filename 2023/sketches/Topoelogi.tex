\documentclass[a4paper,11pt]{article}

\usepackage{revy}
\usepackage[utf8]{inputenc}
\usepackage[T1]{fontenc}
\usepackage[danish]{babel}


\revyname{MatematikRevyen}
\revyyear{2023}
% HUSK AT OPDATERE VERSIONSNUMMER
\version{0.1}
\eta{$2$ minutter}
\status{Under arbejde}

\title{Topølogi}
\author{Villads '19, Sommer '17}

\begin{document}
\maketitle

\begin{roles}
\role{X}[Sommer] Instruktør
\role{F}[Niklas] Forelæser
\role{S}[Anna O] Studerende
\role{St1}[Johan] Statist
\role{St2}[Mona] Statist
\role{St3}[Thais] Statist
\end{roles}

\begin{sketch}
\scene{F afslutter forelæsning}
\says{F} ... og med det har vi defineret clopen. Det var alt vi nåede i dag i Topologi, i næste uge skal vi lære om path-connected.
\scene{Fyld forlader scenen, og S går op til F}
\says{S} Undskyld F? 
\says{F} Ja?
\says{S} Så altså jeg forstod godt det der med at closed ik?
\says{F} Ja?
\says{S} Ja altså ligesom en liter mælk man lige har købt. Og jeg forstod også godt open, fordi jeg kan jo ligesom åbne min mælk
\says{F} \act{Lidt forvirret over hvor den her sketch er på vej hen} ... øh ja?
\says{S} Men altså... jeg forstår ikke det der clopen? Jeg kan da ikke både have en åbnet og en lukket mælk
\says{F} Ahh ja, det er også et lidt forvirrende koncept, men tror ikke du skal tænke over det så praktisk
\says{S} Jamen jeg tror bare jeg har lidt brug for at se det for mig
\says{F} Okay så f.eks. KU-ITs åbningstider ik? De åbner klokken 10:30 og lukker klokken halv elleve, så de har både åbent og lukket 
\says{S} Ahh det giver lidt bedre mening, men jeg tror stadig ikke heeeelt jeg forstår det
\says{F} Okay så hvad med Bamses dør?
\scene{F peger op på AV hvor billede af Bamse i sit hus kommer ind}
\says{S} Bamse?
\says{F} Når nej det er du nok for ung til...
\says{F} Okay, nu viser jeg det, så selv en studerende som DIG kan forstå det!
\says{F}\act{Går ud og henter en øl, mens han snakker} Omg det er det samme hvert eneste år... 
\says{F}\act{Han peger på toppen af øllen} Så se hvordan den her øl er lukket
\scene{S nikker ivrigt}
\says{F}\act{Tager en nøgle eller hvordan vi nu ender med at shotgunne øllen} og se hvordan den nu også er åben!
\scene{F begynder at shotgunne øllen, mens den studerende nikker og forstår tydeligvis konceptet}
\scene{Tæppe}
\end{sketch}
\end{document}