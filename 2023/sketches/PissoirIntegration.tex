\documentclass[a4paper,11pt]{article}

\usepackage{revy}
\usepackage[utf8]{inputenc}
\usepackage[T1]{fontenc}
\usepackage[danish]{babel}


\revyname{MatematikRevyen}
\revyyear{2023}
\version{0.1}
\eta{$3.5$ minutter}
\status{Færdig}

\title{Pissoir-integration}
\author{KE '18}

\begin{document}
\maketitle

\begin{roles}
\role{X}[Jesper] Instruktør
\role{P1}[Elinborg] Gauss
\role{P2}[Kisser] Riemann
\role{P3}[Baldur] Lebesgue
\role{P4}[Thais] Dirac
\end{roles}
\begin{sketch}
%\scene{P1 kommer ind på scenen. Han pruster og stønner, løber rundt og leder}
%\says{P1}: Åh, her er toilettet! Det er fandeme svært at holde sig, når man drikker tre øl per akt.
%\scene{P1 hiver sin tissemand frem, smider den op i pissoiret og begynder at tisse}
\scene{P1 står foran et pissoire og tiser}
\scene{Imens kommer P2 ind og stiller sig ved siden af. P2 kigger over på P1}
\says{P2} Goddag Carl.
\says{P1} Goddag Bernhard.
\scene{P2 får øje på P1's ret store tissemand}
\says{P2} Sikke en flot tissemand du har dig dér
\says{P1} Tak. Jeg er også selv godt tilfreds
\says{P2} Gud ved, hvor stor den egentlig er?
\says{P1} Ja, det har jeg også tit tænkt på. Men jeg ved ikke, hvordan jeg skal \textit{måle} den.
\scene{P1 og P2 kigger på hinanden og får et eureka øjeblik}
\scene{P2 tager et stort ternet papir frem}
\scene{P1 hiver én rød og én sort tusch ud af lommen og lægger sin tissemand op på papiret}
\scene{P2 tegner - bag pissoiret - med tuschen et omrids af tissemanden og sætter prikker i alle felterne, røde prikker i det indre og sorte i det ydre}
\scene{P2 viser tegningen til publikum}
\says{P2} Så, nu har jeg sat røde prikker i alle de felter... her!
\says{P1} ...og hvad så?
\says{P2} Så ved vi, at din tissemand er mindst... 
\scene{P2 tæller ternene}
\says{P2} 10 stor!
\says{P1} 10 hvad? Æbler? Bananer?
\says{P2} Bare 10. 
\says{P2} Og med de sorte prikker, jeg har tegnet her, så ved vi, at den er maksimalt...
\says{P2} 35 stor!
\says{P1} Jamen ok, 35, den tager vi!
\says{P2} Nej nej, vent nu lidt. Vi må være stringente. Nu tager jeg et papir med 4 gange så mange tern, så gør vi det hele igen, og så dividerer vi antallet af tern med 4.
\scene{P1 ser forvirret}
\says{P2} Og for at det ikke skal tage for lang tid, så har jeg snydt lidt hjemmefra.
\scene{P2 hiver et papir mere frem, som har mindre tern, og som allerede har penisomridset tegnet}
\scene{P1 er imponeret, men studser over en ting}
\says{P1} Flot, Bernhard! Men vent lige... hvis du først har fået den tanke her nu, hvor har du så dét omrids med alle de prikker fra?
\scene{P2 har intet forsvar, kigger bare (enten Kevin James agtigt eller) slesk på publikum}
\says{P2} Nu kan du se, at vi har en øvre og nedre grænse på 100 og 55 små tern, henholdsvis.
\says{P1} Det blev vi jo ikke meget klogere af.
\says{P2} Jojo, for hvis vi nu sætter de studerende til at fortsætte den her process for evigt...
\scene{P2 bevæger sig hen mod kanten af scenen for at modtage en rekvisit}
\says{P2} ... og vi bruger \textit{klemmelemmaet}!
\scene{P2 kommer tilbage mod P1 med et ondt smil på læben og en klemme i hånden}
\scene{P2 sætter klemmen på P1s tissemand og trykker. P1 er i stor smerte}
\says{P2} ...så ved vi, jævnfør supremumsegenskaben for de reelle tal, at der findes et $\xi$, sådan at $\xi$ vil være større end enhver nedre grænse og mindre end enhver øvre grænse på arealet, uanset finheden af ternene.
\says{P1}[I smerte] Hnnnnnggggg, men hvis vi ikke har så lang tid, hvad er $\xi$ så?
%\says{P2} Altså, hvis jeg sagde det, hvad skulle de studerende så lave i al evighed?
%\says{P1}[Stadig i smerte] Hnnnnngggg, den her klemme er da godt nok grum
%\scene{P2 tager klemmen af P1s tissemand}
%\says{P1} Ja, ja, Gauss, den er god med dig og din krumning
\scene{P3 går storslået ind på scenen}
\says{P3} Jeg ville skyde på 14,2!
\says{P2}[Overrasket] Ja, det passer meget godt! Men... hvordan vidste du det? Og hvem er du?
%\scene{P3s tissemand er bred og flot, men lidt kortere end P1. P3 kigger køligt over på P1 og P2}
\says{P3}[Fransk accent] Navnet er Lebesgue. Henri Lebesgue. Jeg har selv prøvet at måle lidt af hvert, hvis du forstår hvad jeg mener.
%\says{P1} Fortæl mere!
%\says{P3} Jooo altså, nu skal jeg snart tilbage til 3. akt, men hvis I har et par ledige uger i efteråret...?
%\says{P1, P2} Det har vi!
%\says{P3} ...så kan I jo tage Lebesgue-integralet og Målteori hos Mikael og Magdalena. De har sat 9 ugers pensum af, kun til mig!
%\scene{Publikum går amok. Lebesgue er overlegen}
\scene{P4 kommer ind på scenen og hiver sin lille, meget behårede tissemand ud. P4 har et stort navneskilt på brystet, hvor der står Dirac}
\scene{P1, P2 og P3 kan ikke undgå at fnise}
\says{P1} Se, der kommer Dirac. 
\says{P4} I går dér og praler med jeres $\lim_{n \to \infty}$ og jeres $\sup$ og $\inf$. Jeg måler meget nemmere.
\scene{P4 peger på hver tissemand én gang (sig selv til sidst) og siger}
\says{P4} 0, 0, 0,  1! 
%\says{P4} 1, 1, 1, 1!
\scene{Lys ned.}












\end{sketch}

\end{document}

