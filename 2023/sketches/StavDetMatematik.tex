\documentclass[a4paper,11pt]{article}

\usepackage{revy}
\usepackage[utf8]{inputenc}
\usepackage[T1]{fontenc}
\usepackage[danish]{babel}

\revyname{Matematikrevy}
\revyyear{2023}
\version{1.0}
\eta{$4.5$ minutter}
\status{Færdig}

\title{Stav det Matematik}
\author{Sommer '17, Stine '18 og KE '18}

\begin{document}
\maketitle

\begin{roles}
\role{XS1}[Sommer] Instruktør / Studerende
\role{S2}[Elinborg] Studerende
\role{V}[Inga] TV-vært
\end{roles}

\begin{props}
\prop{Rekvisit}[Person, der skaffer]
\end{props}


\begin{sketch}

\scene{Spelling-Bee. Værten har nogle papirer i hånden, som ordene deltagerne skal stave til står på
Der har lige været en reklame-lm på AV}

\says{V} Velkommen tilbage til "Stav det matematik!". Efter reklamepausen er stillingen 7-7!

\scene{V vender sig mod S1.}

\says{V} Nå S1, det næste ord, du skal stave til, er "l'Hôspital".
 
\says{S1} Ok, øh... l. apostrof. stort H. o. s. p. i. t. a. l!

\says{V} Det er rigtigt!

\scene{Jubel!}

\says{V} Men der mangler én ting!

\scene{S1 bliver helt chokeret}

\says{V} Hvor... er hatten?!

\says{S1} Åhh nej... Øhm... Over O'et?

\says{V} \act{Peger op på AV'en hvor hatten kommer på O'et} Det er fuldstændig rigtigt!

\says{V} Over til dig S2. \act{Vender papir} Du skal stave til differentialkvotient.

\says{S2} Æøøøøøørhhhh... Kan du bruge det i en sætning?

\says{V} Ja selvfølgelig, din sætning er "Over til dig S2, du skal stave til differentialkvotient"

\says{S2} Øhm... H?

\says{V} \act{Agerer vred/"forkert" buzzer lyd} Det er desværre forkert!

\scene{Jubel!
(V Vender sig mod S1, og vender papir)}

\says{V} S1, hvis du får det næste svar rigtigt er du vinder af dagens Stav det Matematik. Er du klar?

\says{S1} Jeg er født klar V

\says{V} Det var godt. Det næste ord du skal stave til er Eisensteins irredecede... irredusilibing... irreterere.... irreducibidilili...\act{S1 ser mere og mere skræmt ud} \act{vender papir} stav til Lebesgue \act{S1 ånder lettet ud}

\says{S1} Kan du fortælle mig ordets oprindelse?

\says{V} Øhm... \act{lidt til sig selv} øhhh lé? \act{Til de andre igen} Det er fransk!

\says{S1} L. e. mellemrum b.e.c.k

\says{V} Det var meget tæt på, men det er desværre forkert!

\says{V} S2 du er nødt til at svare rigtigt på dette spørgsmål, ellers vinder S1. \act{Vender papir} Og du har trukket dagens bonus spørgsmål! Den almene hverdag! Her skal du stave til et ord, der intet har at gøre med matematik!

\says{V} Stav til: .... Hospital!

\scene{S2 er meget entusiatisk}

\says{S2}[Hurtigt] Uh uh uh det kan jeg! l, apostrof, stort H

\says{V} Det er desværre forkert! Det betyder at S1 er dagens vinder af Stav det Matematik! Se med i næste uge når vi skal Regne på Dansk!

\end{sketch}

\end{document}
