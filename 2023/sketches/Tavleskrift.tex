\documentclass[a4paper,11pt]{article}

\usepackage{revy}
\usepackage[utf8]{inputenc}
\usepackage[T1]{fontenc}
\usepackage[danish]{babel}

\revyname{Matematikrevy}
\revyyear{2023}
\version{1.0}
\eta{$3$ minutter}
\status{Færdig}

\title{Tavleskrift}
\author{Frida '21}

\begin{document}
\maketitle

\begin{roles}
\role{X}[MaWeK] Instruktør
\role{F}[Niklas] Forelæser
\role{A}[Stine] Anden forelæser
\end{roles}

\begin{props}
\prop{PowerPoint eller konferenceplange}
\end{props}


\begin{sketch}

\scene{F står på scenen og kigger i nogle noter. A kommer ind på scenen og ser lidt frustreret ud.}

\says{F} Dav A. Er du okay?

\says{A} Ej, altså. Jeg har lige holdt en forelæsning - og de studerende blev bare ved med at stille spørgsmål.. Det ødelagde bare fuldstændig mit flow!

\says{F} Ahh - det lyder som om, at de fulgte lidt for godt med?

\scene{A nikker trist.}

\says{F} Nå, men så gælder det jo bare om at forvirre dem lidt!

\says{A} Okay.. Hvordan gør man det?

\says{F} Det er sjovt du spørger, jeg har faktisk lige lavet en PowerPoint!

\scene{En PowerPoint kommer op på AV.}

\says{F} Mit ene gode råd til at forvirre de studerende mest muligt, er at få bogstaverne til at ligne hinanden.

\says{A} Ligne hinanden?

\says{F} Ja, her er nogle eksempler. \act{Slide skifter.
der står s, g og y på tavlen (Som ligner hinanden)}

\says{F} Her ser du selvfølgelig et S, et G og et Y. Hvis de kan ligne hinanden, kan alt.

\says{F} Lad os starte med noget nemt, som vi kender fra Lineær Algebra. M og N. Lad os give dem en buge mere \act{Skifter slide. Der står n og m
med en ekstre bue}

\says{F} Det er en god start. Vi kan også tilføje denne enskab til R, V og W

\says{F} Bemærk, at R kan så ligne N, men vi kan også få R til at ligne V. \act{Skifter slide. Der står r som n og som v}

\says{F} Nu kommer der et smart tip. Vi kan få H til at ligne R, og R til at ligne H, hvis vi bare justere halsen. \act{Skifter slide. Der står h og r med mellemlang hals}

\says{F} Nu har vi altså, at H ligner R, der ligner V eller N, som ligner M \act{Skifter slide. Der står listen op}

\says{F} Vi kan så tage dette bogstav. \act{Skifter slide. Der står m med lang hals}

\says{A} Ja, er det et H med dobbelbue, et M med en lang hals, eller et N med dobbelbue og lang hals? 

\says{F} Ingen ved det. Er det ikke smukt?

\scene{A nikker anerkendende.}

\says{A} Wauw! Men hvordan indfører jeg det i matematikken?

\says{F} Der er vi så heldige, at vi har X og Y. \act{Skifter slide. Der står x og y, som er ens}

\scene{A er meget imponeret.}

\says{F} Det bliver ikke nemmere end det. Til mit andet gode råd til at forvirre de studerende mest muligt, skal du huske din tavleorden.

\says{A} Okay, jeg husker min tavleorden.

\says{F} Godt, nu skal du smide den væk! Det hjælper, hvis du husker, at skrive alle de vigtige ting du underviser i, sådan lidt hulter til bulter op på tavlen.

\says{A} Men er det så ikke for tydeligt, at jeg bevidst prøver at forvirre dem?

\says{F} Ikke hvis du tilføjer pile! Pile, pile, pile, over det hele, så ligner det, at du har et system i rodet.

\says{A}[Nikkende for sig selv] Pile, pile, pile.

\says{F} Og når du ikke har mere plads på tavlen - så bare fortsæt i mellem de linjer, som du allerede har skrevet!

\says{A} Smart!

\says{F} Jeg håber, at du føler dig lidt bedre forberedt på din næste forelæsning nu. Hvis der stadig er nogle af dine elever der formår at følge med, så vend tilbage til mig -  så kan vi snakke om alt det sjove, du kan gøre med de græske bogstaver!

\end{sketch}

\end{document}
