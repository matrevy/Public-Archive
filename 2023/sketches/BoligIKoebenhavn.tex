\documentclass[a4paper,11pt]{article}

\usepackage{revy}
\usepackage[utf8]{inputenc}
\usepackage[T1]{fontenc}
\usepackage[danish]{babel}


\revyname{MatematikRevyen}
\revyyear{2023}
% HUSK AT OPDATERE VERSIONSNUMMER
\version{0.1}
\eta{$3.5$ minutter}
\status{Færdig}

\title{Bolig i København}
\author{Victoria 19'}

\begin{document}
\maketitle

\begin{roles}
\role{X}[Jesper] Instruktør
\role{S}[Simone] Rus fra Jylland, som lige er kommet til KBH. Lidt langsom i optrækket
\role{E}[Johan] Overgearet ejendomsmægler som afbryder sidste ord, som 'S' siger stort set hver gang. 
\role{C}[Thais] Sur campusservicevagt 
\end{roles}

\begin{props}
\prop{En kuffert} 
\prop{En lille container i pap}
\end{props}


\begin{sketch}

\scene{Scenen er tom, og 'S' kommer ind med sin kuffert}

\says{S}[Glad og med jysk accent] Tænk at jeg endelig skal flytte til København og starte studie! Men jeg har hørt, at boligsituationen er lidt presset herovre på djævleøen. Måske jeg skal bruge lidt hjælp... \act{tager sig til hovedet og ser tænksom ud}

\scene{'E' kommer ind på scenen i jakkesæt med store armbevægelser (måske et 'solgt' skilt i hånden)}

\says{E}[Afbryder og er begejstret] Sagde du, hjælp til bolig? \act{'S' nikker} 
Sig ikke mere, jeg har dig. Har du skrevet dig op til nogle boliger hjemmefra?

\says{S} Jojo, det har jeg da. Ventelisterne er bare meget lange. Der er 11000 foran mig til Øresundskollegiet.

\says{E} Ja, men det er meget godt. Vi har jo boligsikring i Danmark, og jeg kan se, at du står øverst til en bolig i Roskilde! 

\says{S} Men er det ikke langt væk?

\says{E}[vifter 'S' væk] Nej nej, det tager kun 20 min med offentlig. Så skal man selfølelig medregne cykelturen til toget, ventetid, forsinkelser osv. Men alt i alt bliver det nok kun en små $2$ timer!

\says{S}[Afvisende] Det skal jeg altså ikke be' om. Det er jo bare gymnasiet om igen. 

\says{E} Okay, okay. Hvad så med et kollegie? Der er flere med motiveret ansøgning!

\says{S} Jo, det kunne godt være sjovt. Så kan jeg også lære nogle at kende, som jeg ikke skal studere med.

\says{E}[Meget begejstret] Ja, lige præcis! Og så er der fester $8$ dage om ugen hele året rundt!

\says{S}[Forvirret] Såå, også når jeg skal til eksamen?

\says{E} Specielt når du skal til eksamen! På Science har I blokstruktur, så ingen af de andre på kollegiet har travlt! 

\says{S}[Ikke tilfreds] Det er da slet ikke fair! 

\says{E} Nej, men det er monster hyggeligt!

\says{S} Jeg tror du skal finde på noget andet til mig. Måske noget med lidt færre mennesker?

\says{E}[Lidt mere utålmodig] Hmmm. Okay, hvad med at få en roommate?

\says{S} Se, det lyder hyggeligt! Er det også noget med en motiveret ansøgning?

\says{E}[Med mere entusiasme] Nej, men det er endnu nemmere. Du skal bare have en god ven med en lejlighed i byen, med et ledigt værelse til dig!  

\says{S} Men duer det også, når jeg er fra Jylland? Jeg har jo ikke nogle venner fra København. 

\says{E}[Tænker] Nårh nej.. Har du tilfældigvis nogle meget rige forældre? Eller en arv i vente?

\says{S}[Mere irriteret] Nej, det har jeg da godt nok ikke. Min mor er ...

\says{E}[Afbryder] Ja, jeg syntes jo bare du skulle have muligheden. \scene{'E' tænker}
\says{E} Okay, hvad med Copenhagen Village? Du får din egen bolig med tekøkken og så reklamere de med et hyggeligt fællesskab med de andre beboere.

\says{S}[Optimistisk] Det lyder dejligt. Hvor er det henne? 

\says{E}[Også optimistisk] De har faktisk flere "Villages" i København, men den nyeste er åbnet på Nørrebro. Vi kan tage hen og se på det, hvis du er interesseret?

\says{S} Ja! Du viser vejen. (ellers farer vi bare vild)

\scene{'S' og 'E' går rundt på scenen, og en papkasse malet som en container kommer ud på scenen.}

\says{E} Så er vi er. Er her ikke hyggeligt? 

\scene{'S' kigger ind i kassen og forsøger at komme ind}

\says{S}[Forvirret] Er det det? Det er jo en container? Jeg kan jo nærmest ikke være der. 

\scene{*Lyd af S-tog, som kører forbi, så 'S' bliver nødt til at råbe lidt*} 

\says{S}[Råber] Og så er det lige ved siden af skinnerne! 

\says{E} Så er der transport lige til døren! Det er også rigtig nok, men det ligger i det hippeste postnummer i København, og så er huslejen kun lidt mere end din SU! 

\says{S} Men hvordan skal jeg få det til at løbe rundt så?

\says{E} Du kan jo få boligstøtte! Så har du stadig 600 kr til øl og mad ;) 

\says{S}[Tydeligt frustreret] Ej, det kan simpelthen ikke passe det her. Hvad vil du have fra mig? Skal jeg bare sove på uni eller hvad?

\scene{'C' Kommer ind på scenen}

\says{C}[Meget fast og højt] NEJ! Du må ALDRIG sove på campus!

\scene{Tæppe}

\end{sketch}
\end{document}