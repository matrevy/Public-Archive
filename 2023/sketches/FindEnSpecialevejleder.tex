\documentclass[a4paper,11pt]{article}

\usepackage{revy}
\usepackage{amssymb}
\usepackage{amsmath}
\usepackage{commath}
\usepackage{amsfonts}

\revyname{MatematikRevyen}
\revyyear{2023}
\version{0.1}
\eta{$3.5$ minutter}
\status{Ikke færdig}

\title{Find en Specialevejleder}
\author{Villads '19, Baldur 20', Johan 19', Stine 18'}

\begin{document}
\maketitle

\begin{roles}
\role{X}[Stine] Instruktør
\role{S}[Louise] Kommende specialestuderende
\role{V}[Maja] Ven
\end{roles}

\begin{props}
\prop{En telefon, måske bord og stol.}[Person, der skaffer]
\end{props}

\begin{sketch}
\scene S sidder med telefonen og skriver. S trykker meget voldsomt på send.
\says{S} Uhhh, det bliver godt nok spændende! Nu har jeg endelig fået skrevet til to mulige specialevejledere!
\scene{V kommer ind ad døren}
\says{V} Hej S, hvad laver du?
\says{S} Øh, jeg sidder lige og skriver med nogle forskellige. Og wow, den første har allerede svaret, jeg får helt sommerfugle i maven!
\says{V} Nåeh, så du prøver på flere på en gang?
\scene S bliver defensiv
\says{S} Det er da meget normalt?
Altså, vi har jo slet ikke aftalt noget endnu. Og desuden er der jo bedre chancer jo flere man skriver med.
\says{V} Ja, du har altid spillet på flere heste.
\scene S afbryder begejstret.
\says{S} Omg, og han sendte en glad smiley! Så er han vel interesseret?
\says{V} En glad smiley er måske ikke lige det sikreste tegn...
\says{S} Men tror du han kan lide mig? Jeg håber virkelig at han vil have mig! 
%(Jeg har allerede så mange idéer til første møde!)
%\says{V} (Hvornår skal i mødes?)
%\scene S bliver lidt trist
%\says{S} (Han siger... at han gerne lige vil tænke lidt over det? At vi først kan mødes om en uge.)
%\says{V} (Måske spiller han kostbar?)
%\says{S} (Men på den anden side, hvis de begge to siger ja, er det jo også akavet at skulle afvise den ene.)
\says{S} Uh, nu har den anden også svaret! Ej, men det er også lidt akavet, hvis jeg skal til at afvise én af dem..
\says{V} Måske skulle du bare vælge hvem du bedst an lide og så satse på dem?
\says{S} Ja, jeg tror måske også at den ene er lidt nemmere end den anden...
\says{V} Ja, du går jo altid efter de nemme!
%\says{S} (Wow, nu  har den anden også svaret. Kan jeg godt sende en emoji? Et hjerte måske?)
%\says{V} (Er det ikke lidt tidligt?)
\says{S}  Så, nu har jeg aftalt et møde men den første!
\says{V} Skal i så ud og drikke kaffe eller noget?
\says{S} Kaffe?! Han foreslår at vi mødes hos ham. Lige se hinanden i øjnene, om vi har fælles interesser og om der er kemi. Sådan om forholdet kan fungere, du ved.
\scene V er skeptisk
\says{V} Er det ikke bedst han vente lidt med at tage hjem til ham? Og hvor gamle er de egentlig?
\says{S} Altså den ene er lidt yngre så ikke så gammel, måske sådan 50?
\says{V} 50!?!
\says{S} Men altså, alder betyder ikke så meget for mig. Jeg går mere op i hvor erfaren han er. Desuden har mange af mine venner gjort det med ham, og de virkede meget tilfredse!
\scene V er tydeligt overrasket
\says{V}Har dine venner også prøvet ham??
\says{S} Ja, og Morten har jeg jo også prøvet før, for to år siden. Han var meget god, men han kørte mig også hårdt!
\scene V er tydeligt forarget.
\says{V} En mand på 50? Og han kørte dig hårdt??
\says{S} Ja altså efter et af vores møder så holdt han på mig i 15 minutter bagefter, så vi kunne nå at blive færdige. Han var tit ret hård men altså, det er jo også mega fedt.
\scene Nu begynder V at svare lummert.
\says{V} Mega fedt 
\says{S} Men jeg vil jo også gerne gøre det med ham igen. Jeg vil gerne vise ham alt hvad jeg har lært siden sidst.
\says{V} Ja! Du har godt nok været i gang, har du!
\says{S} Men altså jeg har også overvejet at gøre det i grupper. Og den ældste af dem siger, at han er interesseret i grupper!
\says{V} Ja, man skal jo prøve noget nyt engang i mellem
\says{S} Ja, altså jeg vil jo helst ikke have at det bliver alt for seriøst. Så meget tid har jeg jo slet ikke!
\says{V} Men altså er han lækker?
\says{S} Tjoh, lidt måske? 
\says{V} Okay.. Synes bare stadig det er lidt vildt, at du går i seng med folk der er så meget ældre end dig..
\says{S} Vent hvad!? Haha, nejnej du misforstår.
\scene{S viser sin mobil til V.}
\says{V} Nåårh nu forstår jeg - det er forelæsere! Ja okay, hvem ville ikke gå i seng med dem. 

\scene Lys ned
\end{sketch}

\end{document}