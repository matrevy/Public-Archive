\documentclass[a4paper,11pt]{article}

\usepackage{revy}
\usepackage[utf8]{inputenc}
\usepackage[T1]{fontenc}
\usepackage[danish]{babel}

\revyname{Matematikrevy}
\revyyear{2021}
\version{1.0}
\eta{$2.5$ minutter}
\status{Færdig}

\title{Matematisk Sikkerhedsintro}
\author{Marius '17, Sommer '17, Villads '19}

\begin{document}
\maketitle

\begin{roles}
\role{X}[Marius] Instruktør
\role{R}[Villads] Revyst
\end{roles}

\begin{props}
\prop{Rekvisit}[Person, der skaffer]
\end{props}


\begin{sketch}
\scene{P går på scenen med seriøst ansigtsudtryk, træt og gammel}

\says{P} Vi har i Matematikrevyens ledelse besluttet os for at bekæmpe et af de største problemer på matematikstudiet. Nemlig mangel af sikkerhed i matematisk praksis.

\says{P} Det er et enormt problem blandt statistikere, at de hele tiden integrerer funktioner, der slet ikke er målelige.

\says{P} Ligeledes er det vigtigt at man tjekker hvorvidt ens matrix er kvadratisk inden man udregner determinanten af den.

\says{P} Endvidere er det vigtigt, når man sidder til forelæsning ikke at falde i søvn, da det kunne blive dokumenteret på en mobil og lagt op på diverse sociale medier såsom Mat-go'nat. Dette skal I selvfølgelig ikke bekymre jer om i aften, da mobiltelefoner selvfølgelig skal være slukket under hele forestillingen.

\says{P} Senere på aftenen kunne det ske, at I stødte på nogle Mad på Fad. Her vil jeg minde jer om den matematiske 7-drinks-regel, her taler jeg selvfølgelig om det maksimale antal drinks er din alder delt med to + 7. Især hvis du er typen der drikker mere end 3 øl per akt. Lad nu være. Det er DIN skyld at der er kø til toiletterne i pausen.

\says{P} Hvis du absolut skal feste så meget, og i den forbindelse sætte ild til dansegulvet, så beder vi jer venligst om at gøre det et andet sted end her i Store UP1, da der ikke må være åben ild i salen.

\says{P} Men nu kører jeg ud af en tangent, det er jo Matematisk sikkerhed der er humlen! Du må aldrig tegne et diagram der ikke kommuterer,

\says{P}[En smule hidsig] Husk at en funktion kun er lig sin Taylorrække i den største åbne cirkel omkring et punkt,

\says{P}[Gående mod vred] Yoneda gælder kun når transformationen er na- turlig,

\says{P}[Arrig] OG TIL DE RUS OG FYSIKERE DER KAN LIDE AT DI- VIDERE MED NUL. . . . \act{Rolig} så er der nødudgange dér, dér, dér, dér, dér og dér.

\says{R} Undskyld jeg lige var oppe i det røde felt, jeg bliver bare helt tændt af Matematisk Sikkerhed. Nu er I vidst også klædt ordentligt på til revyen, god fornøjelse!

\scene{Lys ned}
\end{sketch}

\end{document}
