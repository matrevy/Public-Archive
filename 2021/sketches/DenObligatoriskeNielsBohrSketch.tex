\documentclass[a4paper,11pt]{article}

\usepackage{revy}
\usepackage[utf8]{inputenc}
\usepackage[T1]{fontenc}
\usepackage[danish]{babel}

\revyname{Matematikrevy}
\revyyear{2021}
\version{1.0}
\eta{$1$ minutter}
\status{Færdig}

\title{Den obligatoriske Niels Bohr-sketch}
\author{Sommer '17, Cecilie '17}

\begin{document}
\maketitle

\begin{roles}
\role{X}[Marius] Instruktør
\role{M}[Mikkel] Manusboss
\role{R}[Toke] Revyst
\end{roles}

\begin{props}
\prop{Rekvisit}[Person, der skaffer]
\end{props}


\begin{sketch}
\scene{Lys op}

\scene{Der står et bord på scenen hvor M og R sidder. M sidder med en stak papirer/et manus i hånden}

\says{M} Åh endnu en veludført revy, og denne gang er jeg sikker på vi ikke har glemt noget!

\says{R}[Kigger sig lidt omkring] Hmmm... Det føles bare lidt som om der er et eller andet oplagt vi har glemt.. Sådan noget kæmpe stort, som har fyldt revyen i mange år....

\says{M}[Overrasket, slår sig til panden] Fuck! Vi har sgu da glemt Niels Bohr Bygningen!

\says{R}[Trækker på skuldrene] Hm, lad os bare udskyde det et år.

\scene{Lys ned}
\end{sketch}

\end{document}
