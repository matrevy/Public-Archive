\documentclass[a4paper,11pt]{article}

\usepackage{revy}
\usepackage[utf8]{inputenc}
\usepackage[T1]{fontenc}
\usepackage[danish]{babel}

\revyname{Matematikrevy}
\revyyear{2021}
\version{1.0}
\eta{$2.5$ minutter}
\status{Færdig}

\title{Vandalyse}
\author{Sommer '17, med inspiration fra Fysikrevyen}

\begin{document}
\maketitle

\begin{roles}
\role{X}[Mads] Instruktør
\role{E}[Sommer] Ernst Vandsen
\end{roles}

\begin{props}
\prop{Tavle}[Person, der skaffer]
\end{props}


\begin{sketch}
\scene{Lys op}

\scene{Står en tavle på scenen, og ind kommer E, med blå skjorte og "vandmandnederdel" på.}

\says{E} Velkommen til Vandanalyse 0!

\says{E}[E begynder at skrive sit navn på tavlen] Mit navn er Ernst Vandsen, og det er mit job at vise jer, hvordan matematik er det vådeste studie!

\says{E} Det er godt at se et helt fyldt Hav-ditorie 1. Men det er måske ikke så overraskende efter at se karaktersnittet på Alge-bra med Nathalie Hval, at folk nu kommer STRØMMENDE til dette hval-fri kursus.

\says{E} I dette kursus vil I finde ud af hvordan alle vores matematiske beviser er defineret ud fra Axiom of Moist. Jeg er overbevist om at I vil finde dette kursus spændende, og for de rigtig fugtige, hov jeg mener dygtige, så håber jeg selvfølgelig på at se jer igen senere til enten Ål og integralteori eller VidTang 1 eller 2.

\says{E} Nå, i dag skal I lære om krebsilon-delta beviser. Nu ved jeg godt I sikkert tror I ved alt om det, efter at have haft Jesper Gro-hval og Mor-Tun Riisasger, men glem alt I ved om det! Nu hvor vi er i gang med ting I skal glemme, så glem alt om hvordan I har differentieret og integreret fra gymnasiet. Her i mit kursus skal I lære om hvordan man gør det på den smukkeste måde. . . Åhh Le-Bæk integration. . .

\scene{TeXnikken begynder at fade lyset ned}

\says{E} Hov nu kan jeg se at mine gamle 12-tals elever begynder at SEJLE igen! Jamen så må I hellere tage en pause i Vand-tinen, og når I kommer tilbage vil vores gæsteforelæser Mikael Rør-Dam fortælle jer lidt om Havs-dorff rum!

\scene{Lys ned}
\end{sketch}

\end{document}
