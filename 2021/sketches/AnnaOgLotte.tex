\documentclass[a4paper,11pt]{article}

\usepackage{revy}
\usepackage[utf8]{inputenc}
\usepackage[T1]{fontenc}
\usepackage[danish]{babel}

\revyname{Matematikrevy}
\revyyear{2021}
\version{1.0}
\eta{$3$ minutter}
\status{Færdig}

\title{Anna og Lotte og de imaginære tal}
\author{Stine Langhede}

\begin{document}
\maketitle

\begin{roles}
\role{X}[Mads] Instruktør
\role{A}[Anna] Sød og feminin Anna
\role{L}[Maja] Kejtet og maskulin Lotte
\role{G}[KE] Stor levende Georg
\end{roles}

\begin{props}
\prop{Bord}[Person, der skaffer]
\prop{To stole}[Person, der skaffer]
\prop{Børnebog om komplekse tal}[Person, der skaffer]
\end{props}


\begin{sketch}
\scene{Kendingsmelodi fra Anna og Lotte spilles}

\scene{Lys op}

\scene{Anna sidder ved et bord med en bog}

\says{A} Lotteee? Lotte, kom lige herind!

\says{L} Jamen, jeg leger lige med Georg.

\says{A} Kom nuuu!

\says{L} Okayy, Annaa.

\scene{Lotte kommer lallende ind og sætter sig ved bordet.}

\says{A} Jeg har lige lært om noget der hedder komplekse tal.

\says{L}[Lettere distraheret] Nåårh, hvad er nu det?

\says{A} Det handler om sådan nogle i.

\says{L} Sådan nogen i hvad?.

\says{A} Nej nej, Anna, bogstavet i.

\says{L} Bogstavet i hvad? Vent, aaah, i! \act{Tegner i i luften} Okay.

\says{A} Ja, så vi skriver et komplekst tal som $z = a + b \cdot i$.

\says{L} Det forstår jeg ikke.

\says{A} Jamen, så hent lige Georg.

\scene{Lotte går ud af scenen og kommer ind igen med en levende stor Georg, som har Georgs betuttede ansigtsudtryk.}

\scene{Anna rejser sig og hjælper Georg}

\says{A} Se, hvis vi nu lægger Georg ned her \act{Lægger Georg ned på maven med siden mod publikum}, så er han alle a'erne.

\says{L} Hihi, ja ae ae, ikke Georg, det er dig \act{aer Georg imens}.

\says{A} Ja, og så er i'erne opad her.

\scene{Anna stiller sig på lænden af Georg og former et i med armene, mens Georg begynder at se mere panikken ud end normalt.}

\says{L} Hvad med det der b, Anna, hvad med det?

\says{A} Nåårh ja - i'et er der b gange.

\scene{Anna hopper let nogle gange, og Georg begynder at flappe lidt med armene.}

\says{L} Aaah, så forstår jeg det!

\says{L}[Får Georg op og stå] Det var godt! Nu er Georg også blevet træt, så gå du ind og sov lidt Georg - så tager jeg dig i karbad senere.

\scene{Georg skifter hurtigt udtryk til et meget lummert smil og går så ud af scenen.}

\says{L} Men Anna, det er da meget sjovt med de der komplekse tal.

\says{A} Ja, og ved du så hvad $i^2$ er?

\says{L}[Tænker] Hmm.. iiiiiiiiiii?

\says{A} Nej, det er $-1$, er det ikke sejt?

\says{L}[Fortvivlet] Men Anna, det giver jo slet ikke nogen mening \act{ryster voldsomt på hovedet}

\says{A} Men altsa, det hedder også jo i, fordi det er imaginære tal.

\says{L} Nååårh, okay! Så det er bare noget vi leger?

\scene{Tæppe}
\end{sketch}

\end{document}
