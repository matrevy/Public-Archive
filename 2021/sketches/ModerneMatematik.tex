\documentclass[a4paper,11pt]{article}

\usepackage{revy}
\usepackage[utf8]{inputenc}
\usepackage[T1]{fontenc}
\usepackage[danish]{babel}

\revyname{Matematikrevy}
\revyyear{2021}
\version{1.0}
\eta{$4.5$ minutter}
\status{Færdig}

\title{Moderne Matematik}
\author{Ulrik '14}

\begin{document}
\maketitle

\begin{roles}
\role{X}[Marius] Instruktør
\role{K}[KE] Konsulenttype.
\role{Euk}[Hans] Matematikgren
\role{Stat}[Lise] Matematikgren
\role{Hyp}[Mikkel] Matematikgren
\role{MatHist}[Vibe] Matematikgren
\role{Par}[Magnus] Matematikgren
\role{VT}[Sophie] Matematikgren
\end{roles}

\begin{props}
%\prop{Bord}[Person, der skaffer]
\prop{Stol}
\end{props}


\begin{sketch}
\scene{Lys op.}

\scene{K sidder på sin kontorstol med benene slået op på bordet og en telefon i øret.}

\says{K}[Selvtilfreds] Bestemt, Hr. Rektor. Det var på \emph{høje} tid, at du fik rebrandet nogle af de gamle fossiler på matematisk institut. Jep. Jep. Så absolut: Hver en krone, du poster i lommen på mit bureau og mig kommer igen tifold, når I ruller noget sexet, moderne matematik ud til folket.

\scene{K nikker lidt mere og stemmer i med personen for enden af røret, mens hun snurrer lidt rundt i sin stol.}

\says{K} Jamen, Hr. Rektor, bassemand. Det er simpelt, si'r jeg dig. Bare send al' mat'matikken lidt ud på de sociale medier og lad den mingle lidt med de unge. Du ve': Møde dem i øjenhøjde.

\scene{Der bliver banket på døren.}

\says{K}[Begejstret] Uh... jeg kan høre, at vi allerede har resultater. Jeg ringer til dig senere, ik' også, snut?

\scene{Ind træder den kuglerunde Par med den hulkende og spinkle Euk under armen. K rejser sig for at tag imod dem, men kigger undrende på den hulkende Euk.}

\says{K} Wow.. stop stands. Hvad sker der? Hvor er ildebranden?

\says{Par} Altså... den euklidiske geometri og jeg prøvede bare lidt det dér Instagram dér.

\scene{Euk hulker ubehjælpeligt.}

\says{K} Nå, storartet! Hvad er så problemet?

\says{Par} Jo... altså...

\says{Euk} Jeg er hæslig!

\scene{Holder en hånd for munden for at hviske til K.}

\says{Par} De er ik' så søde.

\says{Euk}[Hulkende] Ingen kan li' én, der er helt flad.

\says{K} Pjat... du er bare kommet dårligt fra start. Måske er Instagram bare ik' lige stedet for dig, hva'?

\says{Par}[Begejstret] Ja, den hyperbolske geometri lader til at klare sig strålende, og hendes krumning er \emph{negativ}.

\says{Hyp}[Råbende] Folk, der accepterer parallelpostulatet er bogstaveligt talt værre end hvis Hitler og Corona fik et barn, og det blev opfostret af satan!

\scene{Hyp kommer vadende ind på scenen og skubber de andre til side, før hun sætter sig og snurrer rundt i Ks stol, mens hun sidder på sin telefon.}

\says{Hyp} Fedt! 100 re-tweets. Det må være en ny rekord.

\says{K} Det må jeg nok sige. Det var lige det, jeg havde forestillet mig! Mat'matikken gamer de sociale medier!

\scene{Ind vandrer en ivrig Stat med en tablet i den ene hånd og en kugleramme i den anden. Efter sig slæber han Ssh, der først ser lidt paf ud, for derefter at se sig lidt forvirret rundt og klø sig i håret.}

\says{K}[Til Stat] Nej, det bli'r endnu bedre. Statistik, min mand, du ligner én, der har regnet den ud.

\scene{K læner sig ud mod publikum og blinker kikset med det ene øje.}

\says{K} Hvis I forstår så'n en lille én.

\scene{Stat slår opgivende op med hænderne og gi'r sig til at pille febrilsk ved sin kugleramme.}

\says{Stat} Kun lige ved og næsten! Efter udførligt at ha' undersøgt algoritmen i adskillige døgn, er mine konfidensintervaller stadig ikke snævret tilpas ind til, at jeg kan konkludere, at det muligvis kunne tænkes, at det ikke kan forkastes, at min markedsføringsstrategi er korelleret med succes!

\says{K} Ja, altså at den virker?

\says{Stat}[Forvirret] Hvordan skulle man nogesinde kunne sige noget om det?

\says{Euk}[Snøftende] Hvis man er mig, ved man godt på forhånd, at intet virker.

\says{Par}[Aende] Du skal bare finde nogen, der vil ha' dig for den, du er.

\says{Hyp} Ja for helvede. Bare se her: Folk \emph{elsker} mig for min skarpe satire.

\scene{Hyp giver sig til at taste på sin telefon}

\says{Hyp} Topologi uden glat struktur er som Mette uden mink: PINLIG!

\says{K}[Akavet] Ja... øh... Sååååå... Er der i øvrigt nogen som har set Videnskabsteori?

\says{Stat} Jeg tror jeg så dem derovre tidligere, jeg kunne næsten ikke genkende dem, det er vist nogle \emph{sjove} typer de hænger ud med for tiden.

\says{K} Nå for søren, vil du ikke hente dem \emph{(peger på Stat)} så vi kan finde ud af hvad \emph{sjov} betyder?

\says{Hyp} Ja for helvede. Det lyder som \emph{MEMES}.

\says{Par} Ej, den slags er stakkels euklidisk geometri vist ikke kklar til at høre om lige nu.

\scene{Stat, Par og Euk forlader scenen,Stat kommer tilbage slæbende på VT, der er låst fast i spændetrøje og griner manisk.}

\says{VT}[Ude af sig selv] Kan I ikke se det? Beviser er en løgn! Matematikere kunne aldrig finde på at bevise noget, de ikke allerede troede på! Matematikkens ontologiske status er uklar! Det udelukkede tredjes princip er en konspiration, CIA har udtænkt for at pumpe jer alle sammen fuld af mikrochips, der en dag vil vække Præsident Kennedy op fra de døde!

\says{K} ...Måske var det her en værre idé, end jeg troede. Nej, se, Mathist. Han er så hyggelig og rar og har sikkert masse at berette om.

\scene{Mathist træder ind på scenen med sin stok.}

\says{MatHist} Hvordan opretter man sådan en... bruger?

\scene{Lys ned.}
\end{sketch}

\end{document}
