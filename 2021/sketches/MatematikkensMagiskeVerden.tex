\documentclass[a4paper,11pt]{article}

\usepackage{revy}
\usepackage[utf8]{inputenc}
\usepackage[T1]{fontenc}
\usepackage[danish]{babel}

\revyname{Matematikrevy}
\revyyear{2021}
\version{1.0}
\eta{$4$ minutter}
\status{Færdig}

\title{Matematikkens magiske verden}
\author{Victoria '19}

\begin{document}
\maketitle

\begin{roles}
\role{X}[Stig] Instruktør
\role{G}[Lise] Gymnasierettet rus
\role{M}[Toke] Mat-mat rus
\role{S}[Magnus] Statistikker rus
\role{A}[Sommer] Rus som skifter til aktuar
\role{L}[Arnvig] Lærer
\role{H}[Mona] Fordelingshatten
\end{roles}

\begin{props}
\prop{Fordelingshat, med aftagelig Possion-notation}[Rekvisitten]
\prop{Stol}[Person, der skaffer]
\prop{Gul Hufflepuf-kappe}[Rekvisitten]
\prop{Rød Gryffindor-kappe}[Rekvisitten]
\prop{Blå Rawenclaw-kappe}[Rekvisitten]
\prop{Grøn Slytherin-cardigan/trøje (tænk Thulesen Dahl med trøje over skuldrene)}[]
\end{props}


\begin{sketch}
\scene{Det er blevet tid til, at rus skal vælge deres studieretning. De skal derfor til fordelingscermoni, hvor de skal fordeles i deres huse ala Harry Potter. Der er Gymnasierettet som gule Hufflepuf, Mat-mat som blå Rawenclaw, statistikkere som røde Gryffindor og dem som skifter til øktuar som grønne Slytherin. \\
Der står en stol på midten af scenen, og 3 rus kommer ind}

\says{L}[Velkomne] Kære rus. Det er nu igen den tid på året, hvor I skal vælge jeres studieretningerne! Det er en stor og vigtg beslutning, og derfor stoler vi selvfølelig ikke på jer til at lave den... Kom herop rus nr 1.

\says{G}[Nervøs] \act{Sætter sig på stolen}

\says{L}[Smilende] Derfor har vi fået lavet en fordelingshat! \act{trækker Poissonfordelingshat frem, hatten stiller sig over hovedet af rus og ser forvirret på den}

\says{G}[Nervøs og forvirret] \act{Ser op på hatten}

\says{H} Jeg er... Poi-POISSONFORDELINGEN!!!

\says{L}[Forvirret] Hov, nej, det er jo den forkerte hat, væk med den \act{smidder hatten væk - hatten skifter fra Poission-fordelingshat til fordelingshat}

\says{L}[Entusiatisk] med denne hat fordeler vi jer ud på jeres studieretninger! \act{Sætter den rigtige hat på G}

\says{H} Nå, så du ved ikke helt hvor du hører til... Men du vil gerne gøre en forskel for unge... Der gemmer sig måske en lille skabshumanist inde i dig... GYMNASIERETTET!

\says{G}[Lettet] \act{Tager hatten af, jubler og tager sin gule kappe på}

\says{L}[Tilfreds] \act{Til rus nr 1:} hvor godt, så ses vi til næste år til fordeling af fag! Rus nr 2, kom herop!

\says{M}[Klodset] \act{Sætter sig på stolen og tager hatten på og ser op på den}

\says{M}[Nervøs] \act{hvisker} Ik' gymnasierettet, ik' gymnasierettet

\says{H} Nå, vi mangler jo faktisk gymnasielærere...

\says{M}[Forvirret] Nej, ikke det!

\says{H} Ahh sådan noget ville være alt for anvendt til en som dig... Ja, du vil langt hellere lave noget helt ubrugeligt... Haha ja, jeg ser det for mig nu, en rigtig nørd... REN MATEMATIK

\says{M}[Meget lettet] \act{Tager hatten af, går ned til gymnasierettet rus og tager en blå kappe på}

\says{L}[Smilende] Godt så, rus nr. 3, kom herop

\says{S}[Lidt selvsikker] \act{Sætter sig på stolen, tager hatten på og ser op på den}

\says{H} Hmm, rigtig modig... Klar på selv det farligste... Udover KomAn og Topologi... STATISTIK!

\says{S}[Overud lykkelig] Yes! Så kan jeg komme ud og få et rigtigt arbejde! \act{Tager hatten af, går ned til rus nr 1 og rus nr 2 og tager en rød kappe på}.

\says{L}[Smilende] Jamen så er vi jo færdige med at fordele rus

\scene{A sniger sig ind, og tager en rushat på}

\says{A} Hov hvad med mig?

\says{L} Beklager, jeg må have overset dig, kom herop.

\says{A}[Meget selvsikker] \act{Sætter sig på stolen og tager hatten på}

\says{H} Det var da højst besynderligt... Fordelte jeg ikke dig sidste år? 

\says{A}[Selvsikker] Nej nej, det må være min bror du tænker på!

\says{H} Aha... Og du er sikker på, at du er matematikrus?

\says{A} Jaja!

\says{L}\act{Undren} Vent har du ikke haft MatIntro, DisMat... Hele det første år? 

\says{A} Nej nej...

\says{H} Der er ingen tvivl, rigtig snu og vild med penge... AKTUAR!

\says{L,G,S,M}[I shock] HVAD?

\says{A}[Smørret smil] \act{Tager hatten af, rejser sig og tager en grøn trøje om skuldrene og griner}

\scene{Tæppe}
\end{sketch}

\end{document}
