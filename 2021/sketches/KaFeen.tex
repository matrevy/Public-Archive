\documentclass[a4paper,11pt]{article}

\usepackage{revy}
\usepackage[utf8]{inputenc}
\usepackage[T1]{fontenc}
\usepackage[danish]{babel}

\revyname{Matematikrevy}
\revyyear{2021}
\version{1.0}
\eta{$1$ minutter}
\status{Færdig}

\title{Ka feen?}
\author{Stine Langhede}

\begin{document}
\maketitle

\begin{roles}
\role{X}[MaWeK] Instruktør
\role{B}[Maja] Bartender
\role{F}[Jesper] Fe
\end{roles}

\begin{props}
\prop{Caféen bar}[Person, der skaffer]
\end{props}


\begin{sketch}
\scene{Caféen-bar på scenen med bartender stående bag.}

\scene{En fe kommer meget glad gående/flyvende hen til baren.}

\says{F}[Fascineret] Eeej, hvor er det dog spændende for sådan en fe som mig at være på Caféen! \act{Kigger sig fascineret omkring}

\scene{Bartenderen får tydeligvis en god idé}

\says{B}[Lumsk] Banke banke på

\says{F}[Forvirret] Hvem der?

\says{B}[Mere lumsk] Caféen \act{Peger på skiltet på baren}

\says{F}[Forvirret] Caféen hvem?

\scene{Bartenderen tager en øl op på baren og åbner den.}

\says{B}[Synger og klapper] Ka feen?, Ka feen? Ka feen drikke ud?... \act{Får publikum med}

\scene{Feen bunder fortvivlet og forvirret øllen.}

\scene{Tæppe}
\end{sketch}

\end{document}
