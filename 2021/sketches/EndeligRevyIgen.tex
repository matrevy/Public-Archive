\documentclass[a4paper,11pt]{article}

\usepackage{revy}
\usepackage[utf8]{inputenc}
\usepackage[T1]{fontenc}
\usepackage[danish]{babel}

\revyname{Matematikrevy}
\revyyear{2021}
\version{1.0}
\eta{$5$ minutter}
\status{Færdig}

\title{Endelig Revy Igen}
\author{Ulrik '14}

\begin{document}
\maketitle

\begin{roles}
\role{X}[Stig] Instruktør
\role{T}[Stine] Tjener
\role{G1}[Vibe] Den Begejstrede Gæst
\role{G2}[KE] Den Knotne Gæst
\role{D}[Nina] Performancedanser
\end{roles}

\begin{props}
\prop{Bord}[]
\prop{Elektrisk nederen stearinlys}[]
\prop{Stole}[]
\prop{Menuer}[]
\end{props}


\begin{sketch}
%\scene{Lys op.}

%\scene{På scenen ses en nydelig restaurant med et tildækket bord. I kanten står P og danser til noget kedeligt og karakterløst middagsjazz}

\scene{Spotlight går op på dør i mellemgangen. G1 og G2 åbner døren og stikker hovedet ind.}

\says{G1} Hov, folk sidder allerede ned - de er vist gået i gang!

\says{G2} Jamen så gå dog ind og find vores pladser!

\says{G1}[Begejstret] Ih, sikke dejligt det bliver endelig at se revy igen.

\says{G2}[Lidt surmulende] Tja... nu er det altså bare matrevy...

\says{G1} Ja, men føles det ikke lidt bekræftende? Sådan som om verden er ved at være helt tilbage til normal?

\says{G2} Vi ku' også bare have ventet til biorevyen i næste uge...

%\scene{De to bevæger sig ned mod scenen, mens T træder frem fra bag scenetæppet og bukker dybt.}

\scene{Lys op på scenen. T træder ud fra sidetæppet og kigger på G1 og G2.}

\says{T}[Med ophøjet mine] Godaften. Det er min udsøgte fornøjelse at byde jer hjerteligt velkommen til dette års revy. Mit navn er [navn], og jeg skal være jeres vært i aften.

\says{G2}[Til G1] Ej, skal hun absolut tale sådan?

\says{T} Så bestemt! Jeg er blevet instrueret af de fineste, mest intelligente og, ej at forglemme, ydmyge mennesker, der er at finde på hele HCØ og omegn, og de tolererer intet mindre!

\scene{G2 sukker, mens G1 smiler, klapper i hænderne og hopper på stedet.}

\says{G1} Uha, hvor fornemt, som de har oppet sig!

\says{T}[Med selvsikkert smil på læben] Jeg kan forsikre jer om, at dette års revy vil blive ganske \emph{uforlignelig}. Hvis I vil være så venlige at følge med!

\scene{G1 og G2 går op på scenen.}

\says{G1} Næ, vi får bedre pladser end Gam-Ma!

\scene{T trækker en danser ind fra sidetæppet, der begynder at danse.}

\says{T}[Gestikulerende] For at tilbyde jer den ypperste revyoplevelse i den ganske universitetspark vil vi gerne tilbyde jer, at hele jeres aften er ledsaget af kunstneriske fortolkninger fra vores performance-danser. Er det noget, der vil falde i jeres smag?

\says{G1}[Eksploderende af begejstring] Uh! Det er så kreativt, at jeg slet ikke forstår det.

\scene{T giver G1 et medlidende klap på skulderen.}

\says{T} Det er der ingen, der gør.

\says{G2}[Mens han ser surt på danseren.] Undskyld, gider du godt gå? Du distraherer mig bare fra indholdet.

\scene{Danser surmuler lidt, men går ud igen. G1 og G2 bliver sat ved bordet, hvor de tager overtøjet af, men straks tager det på igen og bider tænderne sammen af kuldegysninger.}

\says{G1} Uh... det trækker lidt.

\says{T}[Som var det selvindlysende] Ja. I trin med tidens mode har vi undladt at bygge den fjerde væg.

\scene{G2 holder sig for det øre, der er tættest på publikum.}

\says{G2} ... Det larmer.

\scene{G1 læner sig ud mod publikum og smiler}

\says{G1} ... Men sikke en udsigt!

\scene{G2 stirrer lidt mere skeptisk ud på forsamlingen.}

\says{G2} ...Tja.

\says{T} Nå... men hvis I kigger engang på vores menu, så vil I finde en treakters menu, der kan behage selv den mest kræsne kunde. Dog beklager vi at måtte informere jer om, at den først akt er udgået på grund af jeres sene ankomst.

\says{G2} Nå... Men kan vi så ikke bare starte med tredje akt?

\says{T} Den er desværre ikke færdiglavet endnu, for den skal være helt kogt!

\says{G1} Hvad med tilråb? Tilråb er altid godt at starte med!

\scene{Vent på at folk råber højt, hvorefter G2 giver sig til at slå vredt ud med armene.}

\says{G2} Så hold dog kæft! Nej, tak! Ingen tilråb! Kan vi holde støjen på et statskundskabsrevyniveau?

\scene{T skæver lidt nervøst ud til publikum, der antageligvis blot larmer endnu højere. Efter et par sekunders akavet larmhed vifter G1 ivrigt med armene.}

\says{G1} Hvad så med nogle store flotte rekvisitter?

\says{T} Der skal I vist vente til fysikrevyen.

\says{G2}[Drillende] Hvad så med jokes? Har I dem på menuen?

\says{T}[Gravalvorligt] Højst for fem mennesker i salen af gangen.

\says{G1} Hvad så med sange?

\says{T}[Atter ophøjet] Et overflødighedshorn af musik, siger jeg dig! Og for at mindske den ellers allerede minimale og samfundsukritiske risiko for smitte i salen har vi taget midler i brug for at skabe destruktiv bølgeinterferens mellem sangernes spytpartikler.

\says{G2} På dansk tak?

\says{T}[Begejstret grænsende til det maniske] De synger i hver deres tonart!

\says{G2} Det lyder som noget, man kun kan holde ud ledsaget af sundhedsskadelige mængder alkohol. . .

\says{T}[Selvtilfreds - sidste replik skal vente lidt] Det er skam også en helt bevidst markedsføringsstrategi! Så skål!

\scene{Lys ned.}
\end{sketch}

\end{document}
