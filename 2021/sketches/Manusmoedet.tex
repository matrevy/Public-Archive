\documentclass[a4paper,11pt]{article}

\usepackage{revy}
\usepackage[utf8]{inputenc}
\usepackage[T1]{fontenc}
\usepackage[danish]{babel}

\revyname{Matematikrevy}
\revyyear{2021}
\version{1.0}
\eta{$3.5$ minutter}
\status{Færdig}

\title{Manusmødet}
\author{KE '18, Stine '18, Sommer '17}

\begin{document}
\maketitle

\begin{roles}
\role{X}[Mads] Instruktør
\role{M}[Mikkel] Manusboss
\role{S1}[Niklas] Sketchskriver
\role{S2}[Toke] Sketchskriver
\end{roles}

\begin{props}
\prop{Bord}[Person, der skaffer]
\prop{Tre stole}[Person, der skaffer]
\end{props}


\begin{sketch}
\scene{Der sidder tre mennesker om et bord}

\says{M} Velkommen til manusmøde til matematikrevyen. Der er en uge til sketchudvælgelse, og vi har kun 2 sketches. Så, har I skrevet nogle sjove sketches siden sidst?

\scene{Stilhed}

\says{M} Ok... Har I skrevet nogle sketches?

\says{S1} Ja, jeg har skrevet en ting eller to.

\scene{S1} Rækker ned i sin taske efter sin computer

\says{S1} Ugh, jeg har glemt min computer!

\says{M} Kan du huske hvad det handlede om?

\says{S1} Sådan nogenlunde. Det var noget med en fysiker og en datalog

\says{M} Ja...?

\says{S1} ... Nu har jeg det! En datalog og en fysiker tager til LGBT demonstration

\says{M} Uhhh, det er lidt et følsomt emne...

\says{S1} Men de bliver smidt ud igen.

\says{M} Hvorfor...?

\says{S1} Dataloger arbejder kun i binær.

\says{M} Suk... Hvad med dig?

\says{S2} Vi kunne... lave en masse korte videoer om matematik, hvor vi danser til noget sjov musik?

\says{S1} Ja, som et socialt medie. Du kunne kalde det...

\says{S2} MatemaTik-Tok!

\act{S1} Rejser sig op.

\says{S1} Apropros musik, her i geometri 1 fik jeg en idé til en sang...

\scene{Bandet begynder at spille Errudumellerhvad? med ICEKIID}

\says{S1} Er du krum eller hvad!? Helt glat kurve, er du krum eller hvad? \\
Yo, er du krum eller hvad? Du prøver' at spille glat, men det' bar' en facade!

\says{M} Stop!

\scene{Bandet stopper}

\says{M} Jeg kan ikke høre mig selv tænke! Ellers fede rim, min ven. Vi har dog mange fede sange i år, så lad os se om den kommer med.

\scene{Publikum siger: Uhhhhhhh!}

\says{S2} Jeg har det! Vi skriver en sketch, der handlede om at punchlinen er blevet væk?

\says{M} Hvad!? Det lyder overordentligt dumt. Skulle vi så gå rundt i salen og lede efter den?

\says{S2} Jaaa, måske. Og så kunne vi fortælle en masse akavede jokes imens, der ligesom... mangler en punchline?

\says{M} Ja det lyder i hvert fald som matematikrevyen! Hvordan skulle den så slutte?

\says{S2} Det vil jeg da ikke røbe her foran publikum. Hvad hvis den kommer med i reyven?

\says{M} Bare rolig, det gør den ikke. Ikke, hvis jeg har noget at skulle have sagt.

\says{S1} Jeg har det! Vi laver en ordspilssketch om sørøvere og pirater!

\says{S2} Hvad har de med ordspil at gøre?

\says{S1} $\pi$-rater? $\pi$-stoler? $\xi$-$\rho$-vere?

\act{M} Kigger på sit ur.

\says{M} Ok folkens, nu skal jeg gå om lidt, vi må lige finde på noget. Fortæl mig jeres bedste forslag!

\says{S1} Hvad med bare at lave en sketch, der starter godt, folk har nogle sjove kostumer på, og den slutter med en fed punchline?

\act{M} Kigger olmt på S1

\says{M} Aaaaaah!

\scene{S1 bliver smidt ud af scenen af M, som tramper surt rundt}

\scene{S2 bliver bange for M}

\says{B} Ok, jeg kan mærke I tager det her seriøst. Vi har skrevet den BEDSTE joke i bandet, vil I læse den?

\scene{B rækker en seddel til M. Lyset bliver gradvist svagere.}

\act{M} Kigger på sedlen og kniber med øjnene. \says{M} Det er alt for mørkt. Hvad står der...?

\scene{S2 læner sig ind mod M for at læse sedlen, men lyset bliver gradvist svagere.}

\scene{Lys ned.}

\says{S2} Det var da en mærkelig måde at slutte en sketch på!

\end{sketch}

\end{document}
