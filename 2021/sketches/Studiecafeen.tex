\documentclass[a4paper,11pt]{article}

\usepackage{revy}
\usepackage[utf8]{inputenc}
\usepackage[T1]{fontenc}
\usepackage[danish]{babel}

\revyname{Matematikrevy}
\revyyear{2021}
\version{1.0}
\eta{$1$ minutter}
\status{Færdig}

\title{Studiecaféen}
\author{Sommer '17}

\begin{document}
\maketitle

\begin{roles}
\role{X}[Mads] Instruktør
\role{R1}[Sophie] Rus
\role{R2}[Niklas] Rus
\role{Æ}[Simone] Ældre studerende
\role{V}[Jesper] Ældre persons ven
\end{roles}

\begin{props}
\prop{Bord}[Person, der skaffer]
\prop{To stole}[Person, der skaffer]
\end{props}


\begin{sketch}
\scene{Lys op}

\scene{2 russer sidder ved et bord og arbejder på MatIntro til studiecafé (Evt. hav kalkulus på bordet), og de er tydeligvist forvirrede over noget}

\says{V}[Råber udenfor scenen] Hey Æ! Skal du med på Kassen?

\says{Æ}[Også uden for scenen] Helt sikkert, skal bare lige hente min taske, så kommer jeg!

\scene{Æ kommer ind på scenen, i lilla t-shirt, det skal dog være tydeligt at han IKKE er lektiehjælper}

\says{R1} Det giver altså ikke mening. . . Hov der er en fra studiecaféen, vi kan bare spørge ham!

\scene{R1 og R2 går over til Æ}

\says{R2} Undskyld, kan du lige hjælpe os?

\scene{Æ bliver tydeligvist bekymret. Han tvivler på at han kan hjælpe russerne, men vil heller ikke se dum ud}

\says{Æ} Øhh, er det MatIntro?

\says{R1} Jaja, altså det er bare fordi vi sk-

\says{Æ} AHH MATINTRO! Jaja det er pisse nemt, I skal bare huske at tjekke determinanten!

\scene{Russerne ser tydeligvist forvirret ud, og Æ opdager, at det nok ikke var dét som det handlede om}

\says{Æ} Øhhhh nej nej hov nej... Det jeg mente var selvfølgelig at I skal undersøge om det er måleligt!

\scene{Mere forvirring}

\says{Æ} Har I prøvet at dividere med 0? Er den binomialfordelt? \act{Bliver mere desperat} Har I tjekket wikipedia? HAR I UNDERSØGT OM DEN ER GLAT?!

\scene{Æ fortsætter lidt febrilsk med dårlige ideer}

\says{R2}[Afbryder Æ] Nej nej nej! Vi ville bare spørge hvordan man printer.

\scene{Lys ned}
\end{sketch}

\end{document}
