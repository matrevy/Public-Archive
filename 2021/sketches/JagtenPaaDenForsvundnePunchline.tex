\documentclass[a4paper,11pt]{article}

\usepackage{revy}
\usepackage[utf8]{inputenc}
\usepackage[T1]{fontenc}
\usepackage[danish]{babel}

\revyname{Matematikrevy}
\revyyear{2021}
\version{1.0}
\eta{$2.5$ minutter}
\status{Færdig}

\title{Jagten på den forsvunde punchline}
\author{Sommer '17, KE '18, Stine '18}

\begin{document}
\maketitle

\begin{roles}
\role{X}[Marius] Instruktør
\role{M}[Rune] Manusboss
\role{R}[Niklas] Revyst
\end{roles}

\begin{props}
\prop{Rekvisit}[Person, der skaffer]
\end{props}


\begin{sketch}
\scene{Lys op}

\scene{M står inde på scenen, relativt roligt, og ind styrter en revyst}

\says{R} M! M! Punchlinen er blevet stjålet!

\says{M} Hvad mener du?! Stjålet? Man kan da ikke bare stjæle punchlinen?

\says{R} Jojo den er god nok! Prøv bare at høre: Banke banke på!

\says{M} Kom ind. Vent nej!... Åh nej, jeg kan se hvad du mener. . . lad mig prøve: Hvorfor krydsede kyllingen vejen?

\says{R} Det ved jeg ikke

\says{M} Nå, det var ikke så godt, så må jeg jo selv finde ud af det. . . Fuck du har ret! Punchlinen er forsvundet!

\says{R} Hvad skal vi gøre?

\says{M} Vi må jo gå ud og prøve at finde den!

\scene{R og M forlader scenen, og går ud af tæppet op mod publikum, og får spot på}

\says{M} Kom R, måske punchlinen gemmer sig herude et sted

\scene{Mens de går kigger de sig fortvivlet omkring, bandet kan evt. spille lidt mystisk musik}

\says{R}[Nærmest en smule skræmt] Hv-hv-hvad er ligheden mellem en golden retriever og en blondine?

\says{M} \act{Venter lidt} Øhm de er begge lyshårede?

\says{R} Virker til vi stadig ikke har fundet den.

\scene{De går videre rundt i salen}

\says{M} Hvor mange fysikere skal der til for at skifte en pære?

\says{R} \act{Venter lige lidt}... Æhmm.. . Det kommer vel an på pæren?... Såååøøhh. . . En rusvejleder, en mentor og en rus går ind på en bar. . .

\scene{Akavet stilhed hvor intet sker}

\says{M} Har du hørt om ham der skiftede fra Vim til Emacs?

\says{R} Nej vi har stadig ikke fundet noget sjovt. . .

\scene{Pludselig bryder et telefon-opkald ind, og M tager telefonen, og op på AV'en kommer en video}

\scene{Videon er Fabien der står og snakker i telefon, relativt closeup}

\says{F} Hej hallo Matematik-revy? Ja hej det er Fabien. Jeg tror jeg har fundet jeres. . . øhm hvad hedder det på dansk. . . Punchline?

\scene{Kameraet paner væk fra Fabien, over på nogle matematikere der står i kø til en skål hvor der står ``Punch'' på, hvor folk hælder det op i deres glas}

\scene{Lys ned}
\end{sketch}

\end{document}
