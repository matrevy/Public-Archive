\documentclass[a4paper,11pt]{article}

\usepackage{revy}
\usepackage[utf8]{inputenc}
\usepackage[T1]{fontenc}
\usepackage[danish]{babel}

\revyname{MatematikRevy}
\revyyear{2016}
\version{1.0}
\eta{$4$ minutter $30$ sekunder}
\status{Færdig}


\title{Teoriprøve i Færdsel på Matematikstudiet }
\author{Freja '15}

\begin{document}
\maketitle

\begin{roles}
\role{X}[Freja] Instruktør
\role{K}[] Matematiker der er til prøve
\role{S}[] Speaker, taler meget neutralt
\role{KH}[] Kaffekopsbærende studerende
\role{P}[] Pige i lårkort.
\end{roles}

\begin{props}
\prop{Kaffekop}
\prop{En dør}
\prop{ES01}
\prop{Stol}
\prop{Kalkulus}
\prop{2 tallerkener, en af dem skal kunne kastes med}
\prop{Tom øldåse}
\end{props}

\begin{sketch}


\scene{Lys op. Midt på scenen står en dør.  K står på venstre side af døren. S er ren speaker. Et passende billede er på storskærmen.}




%Udformet som teoriprøve med lysbilleder/slideshow og speaker.
%En person K som ser så neutral ud som muligt agerer de forskellige dele af prøven.
%En ligeledes neutralt lydende stemme læser teksten højt så det matcher K’s agering.
%På scenen skal der være en dør (til S01) (men ikke en rigtig dør!!)


\says{S} Du går ca. 5 km/t og træder ud af S01. Hvordan vil du fortsætte? 

\scene{K starter bag døren og går igennem fra venstre med 5 km/t} 

\says{S}[Lister muligheder] 1. Jeg fortsætter med uændret hastighed, ca. 5 km/t. 

\scene{KH kommer ind fra højre med en kaffekop og går imod døren, K går ind i personen som så spilder kaffe ud over sig, kigger fornærmet op på K og råber}

\says{KH}[Vredt] Hey!

\says{S} 2. Jeg orienterer mig 
\scene{Alle går tilbage til startpunkt. K går ud af S01, kigger sig for, og støder derfor ikke ind i KH der igen kommer gående med en kaffekop}

\says{S} 3. Jeg er særligt opmærksom på de åbne vinduer. 

\scene{K vender sig om og kigger ind ad døren, krydser armene og stirrer meget skeptisk mod venstre aka i retning af vinduerne}
\says{S} 4. Hvad har det med mig at gøre? 

\scene{K sætter sig lige foran døren og kigger på sin telefon. KH kommer ud af døren fra højre, kigger fornærmet på $K$ og råber}

\says{KH}[Vredt] Hey!



\scene{KH går af scenen. Billedet på storskærmen skifter til træet i S01. Døren rykkes mod højre så man nu befinder sig inde i S01. K sætter sig på en stol. Der stilles en tavle i baggrunden med teksten Luigi = -Lug = -Gul = Lilla, Pippi = $-P^3$, Haiti = -Hat = Sko.}


\says{S} Du sidder i S01. Hvad skal du være særligt opmærksom på her? 

\scene{K sidder stille og stirrer lige ud mod publikum med hænderne på skødet}

\says{S} 1. Ruspigen i lårkort?

\scene{P kommer ind ad døren med Kalkulus i hånden og klør sig i hovedet. K stirrer lummert og smiler liiiiidt for meget}

%Indskyd måske 2. Manden i lårkort?

\says{S} 2. Det seriøse indhold på tavlen? 


\scene{K rejser sig op og vender sig om mod tavlen. Kigger på det og nikker meget velovervejende.}

\says{S} 3. Træet? 

\scene{K peger op mod skærmen med kæmpe begejstring, gerne med en smule hoppen}

\says{S} 4. Hvad har det med mig at gøre? 

\scene{K sætter sig ned på gulvet og kigger på sin telefon. KH kommer igen ind med en kaffekop, ser ikke K, snubler over ham og taber igen sin kop.}

\says{S} Hvad er pointen i at have sorte tallerkener i S01?

\scene{K går ud til scenekanten hvor en ninja rækker to tallerkner ind, en sort og en ikke-sort (en af dem fra HCØ's kantine). K tager dem og kigger forundret på dem.}

\says{S} 1. Sort er en pæn farve. 

\scene{K stiller den hvide tallerken på stolen og krammer den sorte med et smil.}

\says{S} 2. De kan let skelnes fra andre tallerkener og har derfor sværere ved at forsvinde i den blå luft. 

\scene{K samler den ikke-sorte tallerken op igen, og holder begge ud i luften. Ironisk nok er det den sorte der bliver hapset af en sceneninja, da man alligevel ikke kan se tallerkenen.}

\says{S} 3. Der er ingen pointe. 

\scene{K kigger på sin ene tilbageværende tallerken og trækker på skuldrene}

\says{S} 4. Hvad har det med mig at gøre? 

\scene{K smider tallerkenen til side og kigger på sin telefon. KH kommer ind på scenen igen, stadig med sin kaffekop, og bliver ramt af den kastede tallerken.}

\scene{Dør og KH fjernes fra scenen. Det eneste der er tilbage er stolen som vendes om og stilles helt frem på scenen. K sætter sig på den, med ryggen til publikum og stirrer i samme retning som publikum. K får en øl af en sceneninja og takker pænt. P går ind på scenen et par meter foran}

\says{S}[Imens P går ind] Du er til Matematikrevy. En sketch skal til at starte. Hvad gør du? 1. Jeg holder min kæft, nyder sketchen og drikker min øl. 

\scene{K sidder helt stille og roligt, krydser benene og tager en tår af øllen.}

\says{S} 2. Jeg drikker min øl og råber "FISSE!" 

\scene{Imens S siger sin replik tager K en usædvanligt stor tår af øllen. K rejser sig så fra stolen, vælter næsten og råber så han afslutter S replik med}

\says{K}[Fuldt] FISSE! 

\says{S} 3. Jeg starter et kor: "Tøjet! Tøjet! Tøjet! ..." 

\scene{K sætter sig ned igen og begynder at klappe og råbe “Tøjet! Tøjet! Tøjet!...” . P ser lidt ukomfortabel ud. P går af scenen.}

\says{S} 4. Jeg er allerede gået kold. 

\scene{K falder ned af stolen og ligger på gulvet i hvad der ligner en ukomfortabel stilling. AV spiller forstyrrende høje snorkelyde}

\says{S} 5. Hvad har det med mig at gøre? 

\scene{K bliver liggende men hiver sin telefon frem og kigger på den.}

\says{S} Du er til Matematikrevy. En sketch skal til at slutte. Hvad gør du? 1. Jeg sætter mig op og klapper og jubler alt hvad jeg kan! 

\scene{K springer på benene igen og klapper og  jubler efter bedste evne.) Lys ned.}

\end{sketch}
\end{document}


