\documentclass[a4paper,11pt]{article}

\usepackage{revy}
\usepackage[utf8]{inputenc}
\usepackage[T1]{fontenc}
\usepackage[danish]{babel}


\revyname{MatematikRevyen}
\revyyear{2016}
% HUSK AT OPDATERE VERSIONSNUMMER
\version{1}
\eta{40 sekunder}
\status{Færdig}

\title{Den fordomsfrie matematiker 4}
\author{Ulrik '14 \& Erik '14}

\begin{document}
\maketitle

\begin{roles}
\role{X}[Jakob] Instruktør
\role{M} Den fordomsfrie matematiker
\end{roles}



\begin{sketch}

\scene{Lys op. M er på scenen}

\says{M} Nu er den altså gal igen! Jeg bliver altså arrig oven i mit hoved. I må lære at behandle hinanden lidt ordentlig - ellers er her jo dårligt til at være. Vi må acceptere hinandens gøren og laden. Der er simpelthen for meget sondren i arbitrære kategorier, som udelukkende skaber mere i spild i vort allerede splittede nation. Jeg gik mig en tur på Amager her i går, og jeg fornemmede da en vis gruppementalitet, der besatte de unge mennesker derude. De gik for eksempel utroligt meget op i, hvilke frilæsningsbøger, folk holdte sig. Men nu må jeg altså sige, at det er noget pjat, og det bør stoppe dette øjeblik. Jeg ser hverken en filosof eller en religionssociolog. Jeg ser bare en arbejdsløs.


\scene{Lys ned}

\end{sketch}
\end{document}
