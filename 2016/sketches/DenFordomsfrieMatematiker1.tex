\documentclass[a4paper,11pt]{article}

\usepackage{revy}
\usepackage[utf8]{inputenc}
\usepackage[T1]{fontenc}
\usepackage[danish]{babel}


\revyname{MatematikRevyen}
\revyyear{2016}
% HUSK AT OPDATERE VERSIONSNUMMER
\version{1}
\eta{$45$ sekunder}
\status{Færdig}

\title{Den fordomsfrie matematiker 1}
\author{Ulrik '14 \& Erik '14}

\begin{document}
\maketitle

\begin{roles}
\role{X}[Jakob] Instruktør
\role{M} Den fordomsfrie matematiker
\end{roles}




\begin{sketch}

\scene{Lys op. M er på scenen}

\says{M} God aften. Det er blevet mig pålagt af den siddende regering og det stående studienævn, i min egenskab af fuldstændig fordomsfri matematiker, at komme her i aften og tale lidt om hvordan vi kan behandle hinanden lidt bedre. Og det har jeg med lige dele stolthed og ydmyghed valgt at tage på mig. For jeg synes virkelig ikke vi altid behandler hinanden lige godt. Her den anden dag f.eks. der var jeg ovre på Biocenteret, og der bemærkede jeg at der var sådan et lidt ubehagelig stemning. I ved, folk gik ligesom og vurderede hinanden på deres faglige tilbøjeligheder. Og så tænkte jeg altså bare: Det er altså ikke noget jeg dømmer folk på. Jeg ser sgu hverken reagensglasbiologer eller gummistøvlebiologer. Jeg ser bare et par bryster.

\scene{Lys ned}

\end{sketch}
\end{document}
