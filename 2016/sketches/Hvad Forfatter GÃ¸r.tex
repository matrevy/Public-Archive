\documentclass[a4paper,11pt]{article}

\usepackage{revy}
\usepackage[utf8]{inputenc}
\usepackage[T1]{fontenc}
\usepackage[danish]{babel}

\revyname{MatematikRevy}
\revyyear{2016}
\version{1.0}
\eta{$5$ minutter $30$ sekunder}
\status{Færdig}


\title{Hvad Forfatter Gør (er Altid det Rigtige)}
\author{Mikkel '11, Lise '14, Josefine '14 \& Jakob '11}

\begin{document}
\maketitle

\begin{roles}
\role{X}[Shake] Instruktør
\role{R}[Skuespiller 1] Redaktør
\role{H}[Skuespiller 2] H.C. Andersen
\end{roles}

\begin{props}
\prop{Tophat}
\prop{Et stort manuskript}
\prop{Bord}
\prop{2 stole}
\end{props}

\begin{sketch}


\scene{Lys op. R sidder ved et bord og kigger i nogle papirer. Der står en tom stol på den modsatte side af bordet. H kommer ind på scenen.} 

\says{R} Ah, kom indenfor Hans-Christian. \act{Signalerer at H kan sætte sig på den tomme stol} Lad som om du er hjemme.

\says{H}[Sætter sig ned] Tak, tak.

\says{R} Jeg har kigget på dit nyeste eventyr og vi bliver nødt til at tale om det. Tag nu for eksempel titlen. 

\says{H} Ja, den røber måske lidt for meget. 

\says{R} [Skeptisk] Måske lidt for meget? Altså du har kaldt eventyret: "Den Grimme Vælling; eller, En Historie om Hvordan en Ung Mand Drager i Krig, Myrder sin Bedste Ven, Vender Hjem, Bliver Tigger og Opdager at Vennen i Virkeligheden Var Hans Bror". %og Vender Hjem for at tage Vennens Plads som Fæstebonde og Derefter Opdager at Vennen i Virkeligheden Var Hans Bror." %Måske den grimme ælling eller den grimme vælling istedet for den blodige kartoffel

\says{H}[Glad] Altså du må indrømme at det er catchy! 

\says{R}  Men så er der starten af historien.

\says{H}[Små fornærmet] Hvad er der galt med den? 

\says{R}  Den er så deprimerende. Du skriver "De kolde golde marker lå øde hen. Et ensomt træ stod dødt i mellem markerne og kragerne sad på dets grene som var de dødens sendebud". Man bliver jo helt nedtrykt.

\says{H} Jeg har ellers gjort mig virkelig umage der.

\says{R}  Jamen folk vil have historier der gør dem glade. \act{Streger ud på sit papir} Hvis vi nu stryger starten og siger: "Der var så dejligt ude på landet".

\says{H}[Afbryder] Men der er jo ikke dejligt ude på landet. Der er marker alle vejne, og der stinker af gylle. 

\says{R} [Optimistisk] Jamen det er opløftende, det gør folk glade. Prøv nu at hør "Der var så dejligt ude på landet; det var sommer, kornet stod gult.. bla, bla, bla... og der gik storken på sine lange, røde ben og snakkede ægyptisk, for det sprog havde den lært af sin moder."

\says{H}[Afbryder forbløffet] Nej, hov hov hov vent et øjeblik nu er du upræcis hvornår på sommeren mener du? Og markerne er forøvrigt ikke gule. Korn er beige, og det er en uendelig kedelig farve. Og forresten har en stork overhovedet ikke lange ben.  %En stork kan jo ikke læse hieroglyffer

\says{R}  Men Hans folk er fuldstændig ligeglade med detaljerne så længe stemningen er god. De vil bare gerne være glade.

\says{H} Der er da ikke nogen der bliver glade af at læse noget der er løgn.

\says{R}  Ja okay, lad os komme tilbage til det senere. Der er alligevel også en anden ting vi er nødt til at tale om.

\says{H} Oh?

\says{R}  Historien er for lang Hans. %Den er lang, lang, lang. %Jeg bestilte et kort eventyr.

\says{H}[Småfornærmet] Jeg synes da ellers jeg begrænsede mig.

\says{R} [Skeptisk] Du har skrevet over 1000 sider... De første 200 sider handler bare om hvor koldt der er på de jyske heder. 

\says{H} Har du været i Jylland? De er virkelig kolde.

\says{R}  Og midt i historien bruger du 40 sider på at fortælle om den her sønderjyllandske købmands forfejlede investeringer i sukkerroer. Han bliver aldrig nævnt igen! %Du afbryder en vigtig scene for at en bondekone kan komme med sin opskrift på roesuppe 

\says{H} Jeg prøver at tegne et billede. Det er jo vigtigt at forstå den socioøkonomiske situation i Thy.

\says{R}  Men størstedelen af historien foregår i Roskilde. 

\says{H} \act{Kaster armene i vejret og ruller med øjnene som om S ikke forstår hans kunst}

\says{R}  Ja okay, den tager vi på et andet tidspunkt. Men senere, når hovedpersonen er i krig, der afbryder du en scene i skyttegravene for at en gammel soldat kan beskrive noget du kalder en atommodel. Du skriver om ting kaldet "Nukleoner" og "Protoner". Hver tredje side har pludselig en fodnote hvor der står "Arbeid macht frei". Det giver jo ingen mening det her.

\says{H}[Småfornærmet] Det er faktisk min yndlingspassage

\says{R} [Skeptisk] Det sagde du også om væddeløbsscenen i "Den Lille Pige med Svovlstikkerne" og de talende avocadoer i "Fyrtøjet".
%/banan/kartoffel
\says{H}[Bestemt] Det kan overhovedet ikke komme på tale. Og jeg er stadig ikke tilfreds med at du udskiftede avocadoerne med hunde. 

\says{R}  Men det er kedeligt Hans. Søvndyssende kedeligt. Jeg forsøgte at læse historien for min datter og både hende og alle mine tjenere faldt i søvn på stedet.

\says{H} Det kunne ligeså godt være din stemme som det kunne være min historie der forårsagede det.

\says{R}  Det eneste jeg siger Hansemand, er at vi måske bare kunne trimme fedtet en lille smule. 

\says{H} Hvis du skærer bare ét ord så går jeg.
%Så megen lykke drømte jeg ikke om da jeg så/spiste den grimme vælding







%Det eneste jeg siger hansemand

%


%


\says{R} [Bestemt] Jeg har bestilt et \emph{kort} eventyr med en \emph{lykkelig} slutning. 

\says{H} Historien har da en lykkelig slutning! Hovedpersonen går fra at være en sølle, sølle slave til at være tigger. Han får sin frihed.
%Hovedpersonen går fra at være en fattig, fattig dreng til at ende som fæstebonde. Han kommer ét trin højere op på den sociale rangstige!

\says{R}  Kunne han ikke blive konge eller sådan et eller andet. Folk vil gerne tro at de kan komme nogen vegne her i livet. 

\says{H} Men de kan jo ikke komme nogen vegne. Mit værk skal være realistisk. Det er kunst.


\says{R}  Men realismen er vel ikke så vigtig når det er til børn? Kunne man ikke komme lidt magi ind i historien? Hvad nu hvis vi ændrer titlen fra "Den Grimme Vælling" til "Den Grimme Ælling". Det ligger meget bedre i munden, og så kunne hovedpersonen være en and?

\says{H} \act{Rejser sig resolut op for at gå} Jeg siger op. 

\says{R} [Begejstret] Men Hans det er genialt.

\says{H} Det er absolut lort! Ænder kan jo ikke snakke. Skal jeg så også skrive en historie om en hel by kun befolket af ænder. Vi kan kalde den Andeby. Det er det jo ikke nogen der vil læse. 

\says{R}  Sæt dig nu ned Hans, så taler vi om det.

\says{H} \act{Tager en dyb indånding} Okay, lad os nu lade som om, jeg går med til, hovedpersonen er en and. Får jeg så lov til at beholde scenen hvor han bliver tævet?

\says{R}  Den får du Hans.

\says{H} Også der hvor han bliver bidt i nakken.

\says{R}  Okay.

\says{H} Jeg vil også have selvmordsforsøget med. %Men han er nødt til at være grim. 

\says{R}  Selvmordsforsøget? 

\says{H} Ja, selvmordsforsøget. Kan du ikke se det for dig? Alle ænder burde jo begå selvmord.

\says{R}  Okay, men kunne han så ikke være ved at drukne sig selv, og så se sit spejlbillede og opdage at han på magisk vis er blevet til en svane?

\says{H}[Oprevet] Det er jo vrøvl! Prøv og hør: en and er en and og en svane er en svane. Den ene bliver ikke bare lige pludselig til den anden. Det er biologisk umuligt. 

\says{R}  Men det magi, det er et eventyr, det er jo en historie til børn.

\says{H} Det er det dummeste jeg nogensinde har hørt. Jeg kunne ligeså godt skrive en historie om en stum kvinde der var halvt fisk.

\says{R} [Glad] God ide Hans. Det er kreativt tænkning. Den ta'r vi!


\scene{Lys ned}






\end{sketch}



\end{document}

%\says{R}  Vi kunne jo lade som om at historien i virkeligheden handler om dig. For at give den et lille autentisk strejf, så bliver den også populær hos de voksne.

%Hansemand

Den Grimme Ælding
Den lille havfrue
kejserens nye klæder
klodshans
Fyrtøjet
Den lille pige med svovlstikkerne
Den standhaftige tinsoldat
Det er ganske vist (fjer til fem høns)



\says{R}  Vi bliver også nødt til at finde på et lidt bedre navn til dig. 
\says{H} Hvad er der nu galt med Hans-Christian?
\says{R}  Det flyder bare ikke af tungen som f.eks. Adam Oehlenschläger. Hvad nu hvis vi i stedet bare kalder dig H.C. Andersen.
\says{H} Men det giver jo ingen mening, mit navn er Hans-Christian, med bindestreg. Mine initialer er jo bare H.A. 