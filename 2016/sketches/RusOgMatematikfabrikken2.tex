\documentclass[a4paper,11pt]{article}

\usepackage{revy}
\usepackage[utf8]{inputenc}
\usepackage[T1]{fontenc}
\usepackage[danish]{babel}

\revyname{MatematikRevy}
\revyyear{2016}
\version{1.0}
\eta{$1$ minut}
\status{Færdig}


\title{Rus og Matematikfabrikken $2$}
\author{Ulrik '14}

\begin{document}
\maketitle

\begin{roles}
\role{X}[Freja] Instruktør
\role{M}[Amalie T] Mentor
\role{R}[Mikkel F] Rus (alle replikker som Chris fra "Chris og Chokoladefabrikken")

\end{roles}

\begin{props}
\prop{2 Telefoner}

\end{props}

\begin{sketch}


\scene{Lys op. M og R står i hver deres side af scenen og taler i telefon med hinanden.}

\says{M} H.C. Ørsted Instituttet, det er Mentor.
\says{R} Hej Mentor, det er rus. Jeg kan desværre ikke komme til mentormøde i dag.
\says{M} Nå, hvorfor ikke?
\says{R} Nå, men ser du mentor, jeg var henne på Caféen? Der var nogen, der fortalte mig om det dér Tequila-tirsdag, og nu har jeg simpelthen så ondt i hovedet, mentor. Jeg tror slet ikke, at jeg kan magte at kæmpe mig hen på instituttet, mentor.
\says{M} Men rus, i går var onsdag.
\says{R} Jamen, jeg mener også var til ond onsdag, mentor.
\says{M} Mm… det har jeg aldrig hørt om rus, og jeg har altså druk… jeg mener studeret her på stedet i nogle år efterhånden.
\says{R} Jamen, det var også, fordi der var den her pige, mentor. Og hun var helt fortabt med sin MatIntro aflevering, mentor. Og hun sad lige midt i vandrerhallen, og hun havde slet ikke noget tøj på, mentor. Jeg kunne ikke lade sådan en stakkels biokemiker sidde midt i vandrerhallen og fryse, mentor. Jeg blev jo nødt til at hjælpe hende, mentor.
\says{M} Rus, du kan jo ikke hjælpe nogen med deres MatIntro aflevering, vel?
\says{R} Nej, mentor, men det er også, fordi jeg er spærret inde på DIKU, mentor. Her er mægtig uhyggelig, mentor. Jeg tror slet ikke, jeg slipper ud nogensinde igen, mentor.
\says{M} Rus, det er frokostpause. Alle døre er åbne. Nu kommer du hen på HCØ. Hvis du ikke er her om 10 minutter, så er du dumpet.
\says{R} Jeg kommer om 10 minutter, mentor.



\scene{Lys ned.}

\end{sketch}
\end{document}


