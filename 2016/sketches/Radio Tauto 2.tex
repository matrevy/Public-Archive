
\documentclass[a4paper,11pt]{article}

\usepackage{revy}
\usepackage[utf8]{inputenc}
\usepackage[T1]{fontenc}
\usepackage[danish]{babel}

\revyname{MatematikRevy}
\revyyear{2016}
\version{1.0}
\eta{$1$ minut}
\status{Færdig}


\title{Radio Tauto 2}
\author{Mikkel '11, Mathias '11, Ulrik '14 \& Jakob '11}

\begin{document}
\maketitle

\begin{roles}
\role{X}[Shake] Instruktør
\role{V}[Skuespiller 1] Vært
\role{D}[Skuespiller 2] Digteren Ver. I. Tas
\role{S1}[Skuespiller 3] Optaget stemme %Eventuelt person fra bandet
\role{SJ}[Skuespiller 4] Jinglesanger, muligvis optaget
\end{roles}


\begin{sketch}


\scene{Jingle spiller. Sketchen er ren lyd.}

\says{S1} Radio Tauto, vi taler sandt, medmindre vi lyver.

\says{V} Velkommen til Radio Tauto, sandhedenes stemme. Mit navn er Karsten Løgn. Idag har vi et lille kulturelt indslag. Vi har en gæst der er på besøg. Den gæst hedder sit navn, og hvad er det?

\says{D} Hvad er et navn andet end noget vi kalder os selv. Ofte både tænker og er jeg, undtagen når jeg ikke er. Derfor har jeg skrevet et digt: \act{rømmer sig, det næste siges langsomt med lange kunstpauser}: En rose, er en rose, er en blomst.

\says{SJ}[Bid fra jingle] Radio tautooo.

\says{V} Tak for det.  Vi slutter af med at lytte til flere dejlige ord i "Den Jeg Elsker, Den Jeg Elsker, Elsker Jeg".

\scene{Lys ned.}

\end{sketch}
\end{document}
