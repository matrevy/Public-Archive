\documentclass[a4paper,11pt]{article}

\usepackage{revy}
\usepackage[utf8]{inputenc}
\usepackage[T1]{fontenc}
\usepackage[danish]{babel}


\revyname{MatematikRevyen}
\revyyear{2016}
% HUSK AT OPDATERE VERSIONSNUMMER
\version{1.0}
\eta{$1$ minutter 30 sekunder}
\status{Færdig}

\title{Anna og Lotte}
\author{Martin '11 \& Ann-Sofie '11}

\begin{document}
\maketitle

\begin{roles}
\role{X}[Alexander] Instruktør
\role{R} Rus 
\role{H} Hunden Ernst
\end{roles}

\begin{props}
\prop{Kursus katalog (avis)}
\end{props}


\begin{sketch}

\scene{Spot på midten af scenen. R sidder på gulvet, H ligger tæt op ad R og sover. R er klædt ud som personen i Anna og Lotte-introen (se link), H er klædt ud som hunden med et navneskilt om halsen med teksten ”ERNST”. R sidder og læser i kursuskataloget (fysisk avis/blad) for at vælge fag i blok 3-4.
R hvisker evt. hele sketchen igennem.}

\says{R} Så skal vi se, hvad der er af kurser i blok 3 og 4... Hvad skal jeg vælge... Analyse 0..
\scene{H vågner, virker meget interesseret} 
\says{R} .. det gider jeg ikke. 
\scene{H hæver overkroppen: ”*huh*” (se video)} 
\says{R} Så er der Algebra 1 – det gider jeg heller ikke.
\scene{H stikker hovedet ind mellem R og kursuskataloget og tager øjenkontakt med R.}

\says{H} *huh* – du kunne vælge Stat1?
\scene{R prøver at puffe H væk og siger irriteret}
\says{R} Flyt dig! Jeg vil bare have matematik. Måske Analyse 1?
\scene{H skubber til R med skulderen for at få hans opmærksomhed.}
\says{H} *huh* – hvad med Stat2?
\scene{R får igen skubbet H væk fra kursuskataloget.}
\says{R} Lad nu være! Det kan du ikke være bekendt, jeg sagde matematik!
\scene{H stikker igen hovedet ind i kursuskataloget.}
\says{R}[Træt] Lad nu være med det. Aha! Hvad med \act{læses op langsomt og håbefuldt} Introduktion til … K-teori?
\scene{H bliver fjern i blikket.}
\says{H} *huuuhhhh* …?
\scene{H mister langsomt balancen og vælter ind i R.} 
\says{R}[Vredt]Gå nu væk!
\scene{R giver op, smider kursuskataloget på gulvet og sætter Matematikrevyen på (dette vises på en eller anden måde, f.eks. filmklip på projektoren).}
\says{H}[Fjollet, ivrig stemme] Nøøøj! Det’ for børn!
\scene{R ryster på hovedet.}
\scene{Lys ned.}
\end{sketch}
\end{document}
