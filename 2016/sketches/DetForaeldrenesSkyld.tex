\documentclass[a4paper,11pt]{article}

\usepackage{revy}
\usepackage[utf8]{inputenc}
\usepackage[T1]{fontenc}
\usepackage[danish]{babel}


\revyname{MatematikRevyen}
\revyyear{2016}
% HUSK AT OPDATERE VERSIONSNUMMER
\version{1.0}
\eta{2 minutter 30 sekunder}
\status{Færdig}

\title{Det' Forældrenes Skyld!}
\author{Jakob '14}

\begin{document}
\maketitle

\begin{roles}
\role{X}[Shake] Instruktør
\role{UM = Uddannelsesminister}[Skuespiller 1]
\role{Uni = Universiteterne}[Skuespiller 2]
\role{SI = Studieintroduktionen}[Skuespiller 3]
\role{Gym = gymnasierne}[Skuespiller 4]
\role{9kl = 9. klasse}[Skuespiller 5]
\role{GS = Grundskolen}[Skuespiller 6]
\role{BH = børnehave}[Skuespiller 7]
\end{roles}



\begin{sketch}

\scene{Lys op. Uni, UM, og SI er på scenen}
\says{Uni} Velkommen hr. uddannelsesminister.
\says{UM}[Vredt og bestemt] Ti stille! Det går forfærdeligt! Universiteterne producerer lavkvalitetskandidater, der er ingen interesse for vores forskere i udlandet, og a propos udlandet er der ingen danskere, der på nogen måde har gjort sig positivt bemærket i de globale kredse. Det er et problem, der klart kan spores tilbage til de danske universiteter. \act{kigger intenst og bebrejdende på Uni}

\says{Uni}[Forarget] Det kan du da ikke mene! Vi giver vores studerende det bedst mulige faglige udbytte, men det er bare som om de studerende ikke bliver ordentligt introduceret til livet som studerende. Det er der problemet ligger. 
\scene{Peger i sin håndflade, og kigger på SI, der smiler og svarer med løftet pegefinger.}

\says{SI}[Snarrådigt og triumferende] Men der tager du fejl, min fine ven! Hvis du kigger på vores evalueringer siger f.eks. $100\%$ af de adspurgte, at de føler sig godt introduceret til Universitets IT-systemer.

\says{Uni}[Oprigtigt imponeret] Hold da op! Det må jeg nok sige.. \act{eftertænksomt} Men hvor mange svarede på evalueringen?

\says{SI}[Skamfuldt, kigger i gulvet] Der var desværre kun to studerende der kunne finde evalueringssiden... \act{kigger trodsigt op} Men de studerende har jo slet ikke de rette kompetencer når de kommer fra gymnasiet! Mange af dem kan ikke engang... 
\scene{Bliver afbrudt af Gym, der får Uni’s opmærksomhed ved at lægge en hånd på skulderen.}

\says{Gym}[Imødekommende, men overbærende og overlegent] Jeg afbryder dig lige her. Altså jeg hører, hvad du siger, men der er lige en detalje du misser. I gymnasiet skal vi jo stå til tåls med elevernes niveau, eller rettere manglende niveau fra 9. klasse. \act{vender sig mod 9.kl, bebrejdende.}

\says{9kl}[Bøvet, taler som en fra amager] Hold da op man, eleverne kan sgu da heller ikke en skid når de kommer fra grundskolen.

\says{GS}[Hippie-agtig, pædagogisk, forsonende, Aske fra bamse og kylling stemme] Hør nu her venner, tag lige en slapper. Vi må ikke glemme at det er vigtigt at børnene lærer gennem leg. \act{spekulerende/kompromissøgende} Man kunne selvfølgelig godt begynde på det allerede i børnehaven. 

\says{BH}[Farer op, stresset, overtræt] NEJ! Nu stopper det! Jeg har 45 børn, som jeg skal tage imod, give mad, skifte bleer på og vaske hver dag. Jeg har simpelthen ikke tid til mere! Hvis I vil give børnene mere faglighed, må forældrene tage sig sammen, og engagere sig lidt i sine børn.

\says{GS}[Samme som før, bekræftende] Jaer, det er rigtigt, forældrene må også lige .. lige.. \act{9.kl fortsætter sætningen}

\says{9kl}[Grovkornet, ligefrem] Tage noget ansvar!
\says{Gym}[Høfligt, henvendt til 9.kl] Undskyld, hvad?
\says{9kl} Jeg afsluttede bare... \act{peger med tommelfingeren ud i luften hvor GS står}

\says{Gym} Hans sætning. Jamen, det lyder da også fornuftigt nok, ikke sandt? \act{Henvendt til SI}
\scene{Imens SI taler bevæger UM sig lidt væk, med en telefon ved øret}

\says{SI}[Nikker meget, glad for ikke at have ansvar] Jeg er meget enig, meget enig.
\says{Uni}[Myndigt, anerkendende] Joo, ja. \act{Vender sig mod UM, der taler i telefon. Siger strengt} Sig mig følger du overhoved med? 
\scene{(UM hører ham ikke.}
\says{UM} Nej, skat... Nej... Jamen så må vi jo aflyse eller noget. Jeg har simpelthen ikke tid til den skole-hjemsamtale… Mmh, farvel. \act{Vender sig mod de andre} Undskyld, hvad siger I?

\scene{Lys ned}

\end{sketch}
\end{document}
