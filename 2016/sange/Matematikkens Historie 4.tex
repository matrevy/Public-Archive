\documentclass[a4paper,11pt]{article}

\usepackage{revy}
\usepackage[utf8]{inputenc}
\usepackage[T1]{fontenc}
\usepackage[danish]{babel}


\revyname{MatematikRevy}
\revyyear{2016}
\version{1.0}
\eta{3 minutter 34 sekunder}
\status{Færdig}

\title{Matematikkens Historie 4}
\author{Ulrik '14}
\melody{Disney's Pocahontas - Colors of the Wind}

\begin{document}
\maketitle

\begin{roles}
\role{X}[Alexander] Instruktør
\end{roles}

\begin{song}
\sings{S} Du si’r jeg’ på det forkerte studie
Og du håner mig og kalder mig for ækel humanist
Du ved jo nok besked,
Din mat’matik er jo så fed
Men ved du det, som Jesper Lützen ved?
Lützen ved!

\sings{S} Du tror, du ved, alt godt om mat’matikken,
Bar’ fyldt med teoremer, du ka’ bru’
Men han ved, alle grene her i faget
Har en fader, har historier, li’som du

\sings{S} Du tror, de en’ste, der ka’ matematik er
Folk lænket til en tavle, støvet grå
Men fulgte du MatHist, vil’ du ha’ indsigt
I et under, du slet ikke kan forstå

\sings{S} Har du undret dig om Babylons aritmetik?
Og hvad lå bag begrebet aksiom?
Har du tænkt på pi, Euklid og Arkimedes?
Så frygt ej for Jesper Lützen ved besked!
Så frygt ej for Jepser Lützen ved besked!

\sings{S} Kom følg hans viser stier i historien,
Kom med og mærk, din fordom bliv’ forladt
Kom med og oplev alt hr. Lützens visdom
Men lad vær’ med at spørg’ om nutidspjat

\sings{S} Pythagoras og Cauchy er hans brødre
Båd’ Newton og Leibniz er hans ven
For vi er bundet sammen gennem elskov
Til et fag, hvor alt bli’r vildt på ny igen

\sings{S} Hvor gammelt er grafernes træ,
Tænk, hva’ Lützen ve’, om alt det, du ve’
For han ved, hvordan araberne bevared’ alt
Og hvilken krig, der fødte ZFC
Han ved alt om Gauss og Galois og Euler
Han er jo den mand, der altid ved besked
Mat’matik bli’r bedst lovprist,
Hvis man er lidt humanist
Så vær tryg, for Jesper Lützen ved besked




\end{song}

\end{document}