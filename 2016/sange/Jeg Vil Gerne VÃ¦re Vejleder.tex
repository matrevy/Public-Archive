\documentclass[a4paper,11pt]{article}

\usepackage{revy}
\usepackage[utf8]{inputenc}
\usepackage[T1]{fontenc}
\usepackage[danish]{babel}


\revyname{MatematikRevy}
\revyyear{2016}
\version{1.0}
\eta{3 minutter 40 sekunder}
\status{Færdig}

\title{Jeg Vil Gerne Være Vejleder}
\author{Ulrik '14}
\melody{Phil Collins - On My Way}

\begin{document}
\maketitle

\begin{roles}
\role{X}[Alexander] Instruktør
\role{S}[] Sanger
\end{roles}

\begin{song}
\sings{S} Jeg vil så gern' være vejleder
Der' russer og spas, der skal ske
En (Au-)gust fyldt med sjov, ja en vejleder
har et fællesskab, som ing' andre ve'

\sings{S} Jeg vil så gern' være vejleder
Og indbyde folk til mit hjem
Min pli den er stor, (og) jeg kan sagtens vent'
med at drikke helt indtil klokken fem

\sings{S} For jeg kæmper og maser mod system og dekan,
Så du kan få en øl på din bus
Og de ordspil, som du hør', de vækker smil
(For) du er rus på mat'matik

\sings{S} Og jeg skal vær' din vejleder
Jeg lær dig at fange idé'n!
I den stjerneklare nat, der ta'r vi på løb,
og om freda'n ta'r vi på Café'n?

\sings{S} Hverken fag eller IT skal ryste dig
(Jeg) er her for at hjælpe på vej
For jeg har jo vær't der selv, og jeg ved så mange men'sker,
og mon ik' de ved besked

\sings{S} En rusvejleder,
druk og sjov,
en rusvejleder

\sings{S} En rusvejleder
En rusvejleder
En rusvejleder

\sings{S} For jeg skal nok blive vejleder
Der' russer og spas, der skal ske
En (Au-)gust fyldt med sjov, ja en vejleder
Bli’r så glad, når de ny’ træder ti’

\sings{S} Jeg vil så gern' vær’ din vejleder
Velkommen til dit nye hjem
Min pli den er stor, (men) jeg kan knap nok vent'
Til september, hvor livet bli’r nemt

\sings{S} Som vejleder

\sings{S} Som din vejleder




\end{song}

\end{document}