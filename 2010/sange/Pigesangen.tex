%% reconstructed from file: 'Matematik Revyen 2010.pdf'
\documentclass[a4paper,11pt]{article}

\usepackage{revy}
\usepackage[utf8]{inputenc}
\usepackage[T1]{fontenc}
\usepackage[danish]{babel}

\revyname{Matematik Revyen}
\revyyear{2010}
\version{1.0}
\eta{2:58 min}
\status{Færdig}

\title{Det at være pige på matematik}
\author{Anna}
\melody{Tarzan: ``Fremmed som mig''}

\begin{document}
\maketitle

\begin{roles}
\role{S1}[NaN] Sanger
\role{S2}[NaN] Sanger
\end{roles}

\begin{song}
\sings{S1} I kigger på os
Hvad er der los?
Har i aldrig set en pige før?
For H.C. Ørsted
det er ikke vor' hjem
Der så meg't I ska' lære nu
om sko og fnis og menses-mat
Og hvis I tror vi godt vil ha' jer
Så tænk jer da lige om - tag jer sammen!

\sings{Begge} I vil ku' se
I vil røre
I vil ha' fingrene i vores kroppe
men vi vil
bar' studere
så fjern nu fokus fra vor's bryst - ja kig op!

\sings{S2} Hver bevægelse
hvert et skridt som vi tar'
bringer nye følelser frem
En frustration
som der er så gennemskuelig
Sådan her har I aldrig følt
For brysterne hopper - I bli'r nervøs'
Ja, nat.fak de for-står os ik'
Hvem der dog bar' var flygtet til KUA

\sings{Begge} I vil ku' se
I vil røre
I vil ha' fingrene i vores kroppe
men vi vil
bar' studere
så fjern nu fokus fra vor's bryst - ja kig op!

\sings{S1} Sandheden er at vi nyder det
Får alverdens opmærksomhed

\sings{S2} Hvis I gi´r en øl, kan I få lov at rør´

\sings{Begge} Bare husk, at der gælder nog´t for nog´t

\sings{Begge} Så giv en øl
Bær vor´s taske
Så skal vi gerne lave små hop for jer
Og måske
hold døren
Så vil vi også ryste numsen for jer

\sings{Begge} Så giv en øl!
\end{song}

\end{document}
