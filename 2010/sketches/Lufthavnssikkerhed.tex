%% reconstructed from file: 'Matematik Revyen 2010.pdf'
\documentclass[a4paper,11pt]{article}

\usepackage{revy}
\usepackage[utf8]{inputenc}
\usepackage[T1]{fontenc}
\usepackage[danish]{babel}

\revyname{Matematikrevy}
\revyyear{2010}
% HUSK AT OPDATERE VERSIONSNUMMER
\version{1.0}
\eta{4.5 minut}
\status{Ny tissesang mangler}

\title{Lufthavnssikkerhed}
\author{Matematikrevyen}

\begin{document}
\maketitle

\begin{roles}
\role{M}[Skuespiller] Manfred
\role{S}[Skuespiller] Sikkerhedsvagt (m/k)
\end{roles}

\begin{props}
\prop{En rekvisit}[Person, der skaffer]
\end{props}

\begin{sketch}
\scene{Plot: En rejsende (matematiker?) skal igennem sikkerhedstjekket i lufthavnen, og det går ikke så nemt.}

\scene{Opsætning: På scenen er et bord og en åben dør, som forestiller en metaldetektor. Ved siden af bordet står en stor skraldespand, og ved den anden side står en afskærmning af en slags. På bordet er den maskine, som scanner håndbagagen. Der er et skilt på scanneren, hvor der står:

Hvad er det IKKE TILLADT at have med i håndbagagen? \\
$*$ Væsker i større mængder \\
$*$ Skydevåben \\
$*$ Spidse genstande \\
$*$ Stumpe genstande \\
$*$ Eksplosive stoffer og brændbare stoffer \\
$*$ Kemiske og toksiske stoffer \\
$*$ Stoffer \\
$*$ Diverse andre ting og sager...}

\scene{M står med en rygsæk på rykken, er iklædt skjorte med metalknapper, shorts med bælte. Og sko. S har uniform på.}

\says{S} Må jeg se Deres boarding pass? \act{M når knap at tage boardingpasset op ad lommen.} Ja, det er fint. \act{ondskabsfuldt:} Så lad os komme i gang med sikkerhedstjekket.

\says{M} Okay.

\says{S} Hvis De lige starter med at tage skoene af og lægge dem på rullebåndet sammen med Deres taske.

\scene{M gør som der bliver sagt}

\says{S} Har De selv pakket Deres håndbagage?

\says{M} Ja da, det har jeg.

\says{S}[skeptisk] Det HELE?

\says{M} Øh, min kæreste har godt nok smurt min madpakke, men bortset fra det, så...

\says{S}[afbryder] Hvad!? Nej, det går virkelig ikke. Er De sindssyg?

\scene{S tager tasken og smider den bombastisk i skraldespanden.}

\says{M} Jamen, hov nej... hvad er det, De gør?

\says{S} Reglementet!

\scene{M vil til at brokke sig yderligere, men bliver afbrudt af S.}

\says{S} Må jeg se Deres negle?

\says{M} Hvad for noget?

\says{S} Kom! Frem med dem!

\scene{M rækker hænderne frem.}

\says{S} Ja, det går jo virkelig ikke. \act{kunstpause} De har jo alt for skarpe negle til at måtte medbringe neglene ombord på flyet. \act{S rækker M en neglefil} Her, så kan De lige file dem af.

\says{M} Ej, nu går det for vidt.

\says{S} Reglementet! Vil De måske ikke gerne nå Deres fly?

\says{M} Joo.

\scene{M giver sig til at file neglene.}

\says{S} Hvornår har De sidst været på toilettet?

\says{M} I morges, da jeg tog hjemmefra.

\says{S} Det er jo allerede længe siden. De ved vel godt, at man ikke må medbringe væsker i større mængder ombord på flyet.

\scene{S rækker M et plastikglas med låg.}

\says{S} Her, så kan De lige lade vandet.

\says{M} Nu bliver det altså for meget.

\says{S} Schhpp! Reglementet!

\scene{M tager imod glasset. S vinker ham om en afskærmning. M stiller sig om bag den og ``lader vandet'', dvs. bytter koppen ud med et fyldt glas med låg.}

\says{M} Jeg kan ikke lige nu.

\says{S} Joo, kom så.

\says{M}[flov] Ikke mens De kigger.

\says{S} Så vender jeg mig om.

S vender sig om, men drejer langsomt tilbage for at drille.

\says{S}[drillende] Na Na Nah. Ej, det må De undskylde. Nu skal jeg nok vende mig om.

\says{M} Jeg kan stadig ikke. Kan De ikke synge for mig?

\says{S} Hvad!?

\says{M} Den der tissesang.

\says{S} Hvad for en sang?

\says{M} Jo, kom nu. Tissesangen.

\scene{S tænker sig om.}

\says{S} Jo, okay. Hvad hedder De?

\says{M} Manfred.

\scene{S tøver lidt og begynder så at synge.}

\says{S} \act{mel.: Mester Jakob} Mester Manfred, Mester Manfred. Er De klar? Er De klar? De skal bare tisse. De skal bare tisse. Tss, tss, tss. Tss, tss, tss. Mester Manfred, Mester Manfred. Er De klar? (fortsætter)

\says{M} Ja, nu kommer det. Ah...

\says{M} Orv, der er meget. Har De et glas mere?

\says{S} Der står flere deromme.

\says{M} Årh, det er godt. Det er nok alligevel meget godt at få tisset af inden flyveturen.

\scene{M kommer frem med 4 fyldte glas med låg på. S kaster glassene over i skraldespanden. Evt. høres lyden af plastik, der knuser.}

\says{S} Sådan, det var ikke så slemt, vel. Vil De være venlig at træde igennem metaldetektoreren?

\scene{M træder igennem. Den bipper.}

\says{S} Har De ikke taget deres bælte af?

\says{M} Det vil jeg meget nødig.

\says{S} Reglementet! Det er måske ikke så vigtigt for dem, om De kommer med flyet, eller hvad?

\says{M} Joo.

\scene{M tager bæltet af, men shorts'ne falder ned, så hans underbukser kommer til syne. Underbukserne er udsmykket med formler. M rækker bæltet til S, som smider det i skraldespanden. M skal til at brokke sig, men S rækker truende pegefingeren frem.}

\says{S} Kom så igennem igen.

\scene{M træder igennem. Den bipper.}

\says{S} Hvad er nu det? Har du sølvknapper i skjorten? Det går virkelig ikke. Af med skjorten.

\says{M} Ej, nu må det stoppe.

\says{S} Schhpp! Reglementet!

\scene{M knapper skjorten op, S smider den i skraldespanden. M står nu kun iført underbukser.}

\says{S} Så prøver vi en gang til.

\scene{M træder igennem. Den bipper.}

\says{S} Hmm... De kan altså ikke komme med flyet, så længe metaldetektoreren bipper. Reglementet, De ved. Hvor kan De dog have skjult noget metal? Jamen dog, De har jo jern på!

\says{M} \act{denne replik medtages kun, hvis publikum har råbt ``Han har jern på'', inden S nåede at sige det.} Næ, det er bare en dolk (han tager hånden ned i underbukserne og tager en foldekniv/springkniv/lommekniv op).

\scene{Lys ud}
\end{sketch}

\end{document}
