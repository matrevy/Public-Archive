%% reconstructed from file: 'Matematik Revyen 2010.pdf'
\documentclass[a4paper,11pt]{article}

\usepackage{revy}
\usepackage[utf8]{inputenc}
\usepackage[T1]{fontenc}
\usepackage[danish]{babel}

\revyname{Matematikrevy}
\revyyear{2010}
% HUSK AT OPDATERE VERSIONSNUMMER
\version{0.8}
\eta{5.5 minutter}
\status{Starten skal forkortes/rettes}

\title{ESS-point}
\author{Matematikrevyen}

\begin{document}
\maketitle

\begin{roles}
\role{VO}[Skuespiller] Voice-over
\role{1}[Skuespiller] Studerende 1
\role{2}[Skuespiller] Studerende 2
\role{3}[Skuespiller] Studerende 3
\role{G}[Skuespiller] Gert
\end{roles}

\begin{props}
\prop{En rekvisit}[Person, der skaffer]
\end{props}

\begin{sketch}
\says{Voice-Over} Vi befinder os et par år fremme i tiden. De studerende opfylder nu endelig regeringens ønsker. De gennemfører studiet på højst 5 år, og får formidable karakterer. Men regeringen har imdlertid nu opdaget et andet problem. Der er tomt på gaderne, caféer og barer går nedenom og hjem. Folk har ganske enkelt haft så travlt med studiet, at de har glemt hvordan det er man er social.

I sidste måned blev Danmark endda kåret som det mindste sociale land nord for Vatikanstaten, og det var her at regeringen besluttede, at der skulle nye boller på suppen.

Man har derfor valgt at indføre ESS-point, European Socializing System-point, som man skal have 100 af inden studiet er omme foruden de 300 ECTS-point. Ordningen træder i kraft i starten af 2020.

\scene{På skærmen kommer nu kort frem med 2025. Vi ser tre VIRKELIG store nørder (ikke tykke, bare nørdede) personer stå ved siden af et podium - de skal til at modtage deres eksamensbevis for færdiggjort universitet. De står og dirrer, og er meget tændte på det hver for sig. Lige idet personen på podiet skal til at give dem deres eksamensbevis kommer der en person ind fra siden}

\scene{!!!Ret her!!! Person: Stop stop! Det er kommet os for øre, at I ikke har fået udført jeres sociale kompetencer. Faktisk har I ikke fået ét eneste ESS-point (European Socializing System-point) på hele jeres studium! Vi kan på ingen måde udlevere jeres eksamensbevis før I har fået disse.}

\scene{de tre studerende står målløse tilbage. De har ikke helt fattet hvad der lige er sket.}

\says{1} Sociale kompetencer? Hvad er nu det for noget?

\says{2} JEG har i hvert fald aldrig hørt om det før.

\says{3} Lad os tjekke på Wikipedia.

\scene{de tjekker, og deres søgeresultat kommer op på storskærmen. 3 læser højt}

\says{3} Social stammer fra det latinske ord socius, og kan i grundformen karakterisere levende organismers interaktion med hinanden.

\says{2} Hm, det lyder hårdt!

\says{1} Ja, men vi kan tilsyneladende ikke få vores bevis førend vi har fået de ESS-point. Men hvordan får vi dem hurtigst!?

\says{2} Min bror er vist ret populær blandt folket. Skal jeg prøve at få fat i ham?

\says{3} Ja, gør det!

\says{2} Ja - hej Gert. Vi har brug for din hjælp NU!

\scene{Gert skynder sig på scenen}

\says{Gert} Jeg kom så hurtigt jeg kunne. Hvad er problemet?

\says{2} Vi har fået at vide at vi ikke har været sociale nok på vores studie - og vi har nu brug for at være rigtig sociale i en fart. Kan du ikke nogle gode råd?

\says{Gert} Altså, for at være sociale, skal I først og fremmest have en masse venner.

\says{3} Årh, dét er hurtigt klaret.

\scene{Lys ned. Lys op.}

\says{3} SÅ! Nu har jeg tilføjet alle på Facebook. Hvad så?

\says{Gert} Ehm, okay. Desuden skal man tale med dem af og til.

\says{1} Årh, dét er hurtigt klaret.

\scene{Lys ned. Lys op.}

\scene{De sidder med hver sin computer og trykker hæftigt på tasterne}

\says{2} SÅ! Nu snakker vi sammen. Hvad så?

\says{Gert} I skal altså gøre det ansigt til ansigt.

\says{1} Årh, dét er hurtigt klaret.

\scene{de tager alle sammen et webcam og sætter på computeren}

\says{3} SÅ! Nu snakker vi ansigt til ansigt. Hvad så?

\says{Gert} Ehm, jeg tror vi prøver en ny indgangsvinkel. En nem måde hurtigt at blive social på er at besøge Caféen?, og drikke.

\says{2} Men, er det proportionelt?

\says{Gert} Hvad mener du?

\says{2} Jo, altså, jo mere man drikker, jo mere social bliver man.

\says{Gert} Hm, det gælder vel til en uhvis grænse. I hvert fald til man bliver for fuld.

\says{1} Okay... Fuld af hvad?

\says{Gert} Øh.. Af alkohol, selvfølgelig.

\says{3} Årh, dét er hurtigt klaret.

\scene{Lys ned. Lys op.}

\says{2} Jeg tænkte, at jo større en andel der var alkohol, så hurtigere kunne vi blive fulde, og jo hurtigere kunne vi dermed få vores eksamensbevis. Jeg har fundet finsprit! 1+

\says{3} Dét var smart tænkt!

\scene{de drikker}

\scene{Lys ned. Lys op.}

\scene{1,2,3 ligger på scenen, Gert står ved siden af og beskurer dem}

\says{2} ... Så, nu har vi været fulde. Hvad så?

\says{Gert} \act{suk} Føler I jer sociale endnu?

\says{1} Ikke rigtigt.. Har du ikke andre råd?

\says{Gert} Tjo.. I skal jo også vide hvornår der er arrangementer, så måske meld jer til nogle mailinglister?

\says{3} Årh, dét er hurtigt klaret!

\scene{Lys ned. Lys op.}

\scene{De sidder igen med deres computere. De trykker lidt hist og her, og trykker meget bestemt på enter til sidst, og siger i kor}

\says{1,2,3} Så, nu har vi meldt os til Spiltirsdag!

\scene{Lys ned. Lys op.}

\says{2} SÅ! Nu har vi meldt os til Spiltirsdag. Hvad så?

\says{Gert} Jamen.. Det er jo bare spam!

\says{3} Nå, øv..

\says{Gert} Måske burde I bare give op, eller prøve at ændre den nye lov?

\says{1} Årh, dét er hurtigt klaret. Lad os melde os ind i Socialdemokraterne!
\end{sketch}

\end{document}
