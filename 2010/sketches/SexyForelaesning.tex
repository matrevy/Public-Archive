%% reconstructed from file: 'Matematik Revyen 2010.pdf'
\documentclass[a4paper,11pt]{article}

\usepackage{revy}
\usepackage[utf8]{inputenc}
\usepackage[T1]{fontenc}
\usepackage[danish]{babel}

\revyname{Matematik Revyen}
\revyyear{2010}
% HUSK AT OPDATERE VERSIONSNUMMER
\version{1.0}
\eta{1.5 minut}
\status{Færdig}

\title{Sexy forelæsning}
\author{Matematik Revyen}

\begin{document}
\maketitle

\begin{roles}
\role{F1}[???] Forelæser 1
\role{F2}[???] Forelæser 2
\role{D}[???] Flabet dreng
\end{roles}

\begin{props}
\prop{En rekvisit}[Person, der skaffer]
\end{props}

\begin{sketch}
\scene{Der står to forelæsere på scenen. De diskuterer}

\says{F1} Hvis vi nu lader $x$ tilnærme sig uendelig, kan vi så ikke få noget brugbart ud af det?

\says{F2} Det kan vi ikke. Kurven er ikke glat.

\says{F1} Hm, hvis vi nu kun kigger på nogle af kurverne?

\scene{En dreng kommer en fra siden med et meget lumsk smil på, han ligner lidt Bølle Bob}

\says{D} Hej, hvad laver I?

\says{F1} Vi sidder og arbejder med et problem. Har du et spørgsmål?

\says{D} Ja, jeg tænkte om man ikke kunne navngive nogle algebraiske elementer selv.

\says{F2} Tjo, det kan jeg ikke se noget galt i.

\says{D} Okay, men vil I så ikke prøve at se hvad I kan få ud af følgende? Jeg er blevet givet to grupper, lad os kalde dem Kvinde og Mand.

\says{F1} Hvordan ser de grupper ud?

\says{D} Den ene gruppe består af to ortogonale vektorer, to kasser og nogle glatte kurver - og den anden består af en normeret vektor og to kugler med radius 1.

\says{F2} Okay, vi kigger lige på det om lidt.

\scene{D går grinende ud for sig selv}

\says{F1} Hm, altså, hvis man nu holder den ene vektor fast så kan vi måske tilfredsstille kasserne ved hjælp af de to kugler?

\says{F2} Er forudsætningerne opfyldt til det? Var Kvindens vektorer ikke ortogonale?

\says{F1} Ja det er de; Mandens vektor var jo enorm.
\end{sketch}

\end{document}
