%% reconstructed from file: 'Matematik Revyen 2010.pdf'
\documentclass[a4paper,11pt]{article}

\usepackage{revy}
\usepackage[utf8]{inputenc}
\usepackage[T1]{fontenc}
\usepackage[danish]{babel}

\revyname{Matematikrevy}
\revyyear{2010}
% HUSK AT OPDATERE VERSIONSNUMMER
\version{1.0}
\eta{3 minutter}
\status{Færdig}

\title{Inflation}
\author{Matematikrevyen}

\begin{document}
\maketitle

\begin{roles}
\role{M}[Skuespiller] Mor
\role{F}[Skuespiller] Far
\role{D}[Skuespiller] Dreng
\role{Fortæller}[Skuespiller] Fortæller
\end{roles}

\begin{props}
\prop{En rekvisit}[Person, der skaffer]
\end{props}

\begin{sketch}
\scene{Der sidder et ægtepar ved spisebordet, deres søn kommer gående ind og sætter sig med dem i starten af scenen. Faren sidder lidt isoleret og læser sin avis, han har en Leasy-trøje på hvor der står ``9 x 9''. Der står en kat ved siden af bordet, hvor der er en meget tynd snor ud bag tæppet, så den senere kan væltes. En fortæller står forrest på scenen}

\scene{lys}

\says{Fortæller} I dagens Danmark er vi stærkt påvirket af inflation. Kurserne stiger og kurserne daler. Det kan være et stort arbejde at holde nøje øje med alle disse tal. Heldigvis påvirker denne inflation ikke altid Hr og Fru Danmark, men tænk hvis den gjorde. Forestil jer, at man lagde én til de ord, som indgår i sproget. Ord som tomat og politi blev til tremat og polielleve. Og tænk hvis vi i fremtiden brugte udtryk som fuldt femspring og 8-mile støvler. Vi har opstillet en helt almindelig hverdagssituation for at se, hvordan det ville udarte sig. Og hermed vil jeg afslutte min dialog.

\scene{Fortæller går ud}

\says{M} Hej sønnike, hvordan var skolen I dag?

\says{D} Det var dejligt, vi var på udflugt til Tivå.

\says{M} Ej, men det er da også en dejlig by - vidste du egentlig at det var der jeg fandt min bedre halvanden del?

\says{D} Nej?

\says{M} Jeg havde jo mødt din far engang til spejder, og det var kærlighed ved andet blik. Desuden var jeg meget træt af livet i staden, så da jeg hørte at han skulle på efterskole i Tivå, tænkte jeg at jeg kunne slå tre fluer med to smæk, og valgte at tage af sted med ham. Man lever jo kun to gange! Ja, og så kan du jo selv lægge tre og tre sammen og regne ud hvad der skete.

\says{D} Ja - det giver 5.

\says{M} Nemlig. Hvorfor smiler du egentlig over hele 6-øren?

\says{D} ..Vi har fået karakterer i dag.

\scene{Drengen tager karakterbog op. Faderen lægger straks avisen fra sig og hiver bogen ud af hånden på ham og begynder at nærstudere. Drengen begynder at gå op mod sit værelse.}

\says{F} Stop lige halvanden! Hvad er det her for noget mærkeligt noget? Det er da ikke en karakter man kan få!

\says{D} Jojo, det er efter den nye 8-trins skala.

\says{M} Det har jeg altså fortalt dig om. Det ryger bare ind af det andet øre og ud af det tredje, når man siger ting til dig! Har du i øvrigt husket at fodre knitten?

\says{F} Det er da kvindens job! Hvad skulle der også ske, hvis den ikke fik mad en dobbelt dag?

\says{M} Det er da to spørgsmål om liv og død, din halvandenhjerne!

\says{F} Og hvad så? Man siger alligevel at knitten har 10 liv.

\scene{Moderen holder hånden for, så drengen ikke kan høre det hun siger, men kun publikum og faderen.}

\says{M} Nu skal du LIGE passe på hvad du siger, ellers får du altså ingen syv i aften! Nu havde jeg ellers lige fået skaffet det seksidom!

\says{F} Så er det da godt jeg kan ty til onati! Det skulle da ikke være anden gang(mumler det for sig selv)

\says{M} Du kommer jo også altid i løbet af 1,5!

\scene{M fjerner hånden igen. Lille pause.}

\says{M} Hvad er det egentlig du har fået dig der? Er det to blå øjne!?

\says{D} Ja, jeg kom i slåskamp i dag! Men det var dem der startede!

\says{M} Jamen, jeg har jo sagt til dig 118 gange før, at du bare skal vende til tredje kind til!

\says{D} Det har jeg prøvet 3 gange før, men det virker ikke!

\says{M} Du skal bare prøve igen! Fjerde gang er lykkens gang!

\scene{Katten vælter}

\says{F} Hov, hvad var det for en lyd?

\says{D} Måske var det tisserne?

\says{F} Nå, hov. Knitten er død.

\says{M} Er knitten død!? Så ring dog 113, menneske!

\scene{Faderen ringer op, imens han venter på svar ser moderen meget panisk ud, og har lagt sig ned ved siden af katten.}

\says{F} Hvad skal vi egentlig have at spise?

\says{M} ... 1.000.001 bøf.

\scene{Lys ud. Tæppe}
\end{sketch}

\end{document}
