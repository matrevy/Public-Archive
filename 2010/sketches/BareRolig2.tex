%% reconstructed from file: 'Matematik Revyen 2010.pdf'
\documentclass[a4paper,11pt]{article}

\usepackage{revy}
\usepackage[utf8]{inputenc}
\usepackage[T1]{fontenc}
\usepackage[danish]{babel}

\revyname{Matematik Revyen}
\revyyear{2010}
% HUSK AT OPDATERE VERSIONSNUMMER
\version{1.0}
\eta{0.5 minut}
\status{Færdig}

\title{Bare Rolig, jeg er skilsmisseadvokat}
\author{Matematik Revyen}

\begin{document}
\maketitle

\begin{roles}
\role{M1}[???] Mand 1
\role{K1}[???] Kvinde 1
\role{S}[???] Skilsmisseadvokat
\role{M2}[???] Mand 2
\role{K2}[???] Kvinde 2
\role{M3}[???] Mand 3
\role{K3}[???] Kvinde 3
\role{M4}[???] Mand 4
\role{K4}[???] Kvinde 4
\end{roles}

\begin{props}
\prop{En rekvisit}[Person, der skaffer]
\end{props}

\begin{sketch}
\scene{Der sidder nogle folk rundt omkring, og der er lidt hyggebelysning. De sidder i par, og snakker alle sammen parvist sammen, der er lidt cafémiljø stemning. Pludselig begynder M1 og K1 at snakke lidt mere intents sammen, skændes og til sidst giver hun ham en lussing. Derefter rejser en af de andre sig op og siger}

\says{S} Bare rolig, jeg er skilsmisseadvokat!
\end{sketch}

\end{document}
