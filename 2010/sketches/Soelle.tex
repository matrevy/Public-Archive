%% reconstructed from file: 'Matematik Revyen 2010.pdf'
\documentclass[a4paper,11pt]{article}

\usepackage{revy}
\usepackage[utf8]{inputenc}
\usepackage[T1]{fontenc}
\usepackage[danish]{babel}

\revyname{Matematikrevy}
\revyyear{2010}
% HUSK AT OPDATERE VERSIONSNUMMER
\version{1.0}
\eta{3 minutter}
\status{Færdig}

\title{Sølle}
\author{Matematikrevyen}

\begin{document}
\maketitle

\begin{roles}
\role{A}[Skuespiller] dreng
\role{B}[Skuespiller] pige \\
(replikkerne kan tilpasses til andre kønsfordelinger)
\end{roles}

\begin{props}
\prop{En rekvisit}[Person, der skaffer]
\end{props}

\begin{sketch}
\scene{To personer sidder ved et bord, drikker kaffe og spiser kage. Helst i en sofa, så det ser hyggeligt ud. Man må gerne få indtryk af, at de har kendt hinanden i lang tid og er gode venner.}

\says{A} Hvem af os tror du er bedst til at score?

\says{B} Ah, det må vist være dig. Du har altid været god til det med at underholde et selskab. Min charme er ca. lige så lille som min selvtillid og mit ulækre hår.

\says{A} Min charme er mindre end en gennemsnitlig fysikertissemand, og det siger ikke så lidt.

\says{B} Min charme er mindre end din charme.

\says{A} Hmm... Men lad nu det ligge. Jeg kommer altid galt af sted: spilder øl ud over pigerne og spilder øl ud over mig selv.

\says{B} Det er jo det jeg siger: underholdene. Og jeg kommer da også galt af sted. Sidst jeg dansede op ad stripperstangen på Caféen? forstuvede jeg mit haleben. Så er man da uheldig.

\says{A} Du nåede i det mindste hen til stripperstangen. Jeg kom aldrig længere end til baren.

\says{B} Hvor du til gengæld tilbragte det meste af aftenen i selskab med nogle dejlige piger fra fysik.

\says{A} Dejlige? De læste jo fysik... Og jeg fik jo ikke hevet nogen af dem med hjem. Hvis det havde været dig, så havde du helt sikkert fået en fyr med hjem.

\says{B} Ej. Ha... og hvis jeg endelig fik hevet en fyr med hjem, så ville han jo flygte med det samme han trådte ind ad døren. Der er jo simpelthen så ulækkert hjemme hos mig, at selv en datalog ville få lyst til at gøre rent.

\says{A} Ah, hjemme hos mig er der værre. Der er nullermændene så store, at man bare kan tage dem med hånden, når man gør rent.

\says{B} Ja ja, hos mig er nullermændene så store, at man ikke behøver tage dem med hånden. De kravler selv over til støvsugerskabet.

\says{A} Ha, du har i det mindste et støvsugerskab. Hos mig er støvsugeren nødt til at være tændt døgnet rundt, for at nullermændene ikke fylder hele lejligheden. De formerer sig jo oftere end matematikere. (Og det siger ikke så lidt.)

\says{B} Ja, et støvsugerskab har jeg, men der er da ingen støvsuger. Hos mig er vi nødt til selv at suge støvet op med munden og puste det ud af vinduet.

\says{A} Ha, vindue... hvem der bare var så heldig at have et vindue.

\says{B} Så du påstår, du ikke har nogen vinduer hjemme hos dig? Jeg var da ellers hjemme hos dig for tre år siden, hvor du havde vinduer i stuen.

\says{A} Sundhedsmyndighederne fik dem muret til.

\says{B} Det slår stadig ikke mit køkken.

\says{A} Hvad er der da galt med dit køkken? Er det ikke et godt sted at opholde sig?

\says{B} Jo, det synes kakerlakkerne i hvert fald. Og bananfluerne. De er helt vilde med at være der. Hvis de så bare betalte husleje, men det gør de jo ikke.

\says{A} I mit køkken er der ingen bananfluer.

\says{B} Der kan man bare se: du er jo god til rengøringen.

\says{A} De er nemlig blevet spist af kakerlakkerne, som til gengæld har vokset sig kæmpestore.

\says{B} Hvis jeg fik en krone for hver gang jeg havde fået noget til at vokse sig kæmpestort... Wink

\says{A} Nåh, så du scorer jo ofte...

\says{B} ... så ville jeg skylde mange penge væk!

\says{A} Åh, hvor er vi sørgelige.

\says{B} Ja, det er vi godt nok. Jeg er i hvert fald. Og så skal jeg hjem til min sørgelige seng i aften. Sengen knirker så meget, at det vækker min overbo.

\says{A} Der er måske meget lyt i lejligheden?

\says{B} Han er døv!

\says{A} Men hvis sengen knirker, så er det vel, fordi du har brugt din seng meget. (blinker med øjet)

\says{B} Nej. Jeg købte den brugt. Af en matematiker.

\says{A} I det mindste har du en seng. Jeg må sove på en madras på gulvet, fordi min seng er blevet spist af myrer. Og min overbo vækker mig hver morgen kl. 5.

\says{B} Hvilken luksus. Min madras er blevet spist af kakerlakkerne og bananfluerne, som har bredt sig helt ind i soveværelset. Og kakerlakkerne og bananfluerne vækker mig kl. halv 4 hver morgen fordi de forventer, jeg står op og laver morgenmad til dem.

\says{A} Du har da selskab. Jeg ligger mutters alene og er ved at fryse ihjel hver nat, fordi isoleringen i huset ikke virker, og mine dyner er rådnet op. Sneen fyger ind ad døren, som ikke kan lukkes helt, alt imens taget knirkende truer med at brase sammen over hovedet på mig.

\says{B}[skeptisk] Arh...

\says{A} Jo, og i de aller koldeste vinternætter ligger jeg klinet op ad radiatoren, ryster og tænker kun på en ting.

\says{B} Hvad er det?

\says{A} At viceværten ville være gavmild nok til at tænde for varmen i radiatoren!

\scene{Lys ud}
\end{sketch}

\end{document}
