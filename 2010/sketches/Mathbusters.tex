%% reconstructed from file: 'Matematik Revyen 2010.pdf'
\documentclass[a4paper,11pt]{article}

\usepackage{revy}
\usepackage[utf8]{inputenc}
\usepackage[T1]{fontenc}
\usepackage[danish]{babel}

\revyname{Matematikrevy}
\revyyear{2010}
% HUSK AT OPDATERE VERSIONSNUMMER
\version{1.0}
\eta{3 minutter}
\status{Færdig}

\title{Mathbusters}
\author{Matematikrevyen}

\begin{document}
\maketitle

\begin{roles}
\role{H}[Rolleindehaver] Hans
\role{K}[Rolleindehaver] Kurt
\role{S}[Rolleindehaver] Sune
\role{D}[Rolleindehaver] Danser
\role{B}[Rolleindehaver] Telegrambud
\end{roles}

\begin{props}
\prop{En rekvisit}[Person, der skaffer]
\end{props}

\begin{sketch}
\scene{ Værterne, Hans og Kurt, står på scenen. (Hver gang en myte bliver af- eller bekræftet skal det tilsvarende skilt vises på storskærmen (alternativt et skilt på scenen)}

\says{H} Veeeelkommen til MATHBUSTERS!

\says{K} I aften skal vi undersøge folks fordomme om matematikere. Vi skal teste sandhedsværdien af en række fordomme.

\says{H} Den mest udbredte fordom om matematikere er deres sublime evne til hovedregning. Hvordan skal vi gribe det an?

\says{K} Lad os invitere en matematiker på scenen og give ham et simpelt regnestykke.

\scene{Matematiker (Sune) kommer på scenen.}

\says{H} Hej og velkommen. Nu får du 4.2 sek til at lægge alle tallene i trekanterne sammen: \act{Viser trekanten fra Kanal Københavns quiz-program}

\says{S} Uh, nej! Det har jeg prøvet før, det kan jeg ikke.

\says{H} Nå, så er det jo afkræftet. Svaret er 428. Men hvad så med dette? \act{Viser skilt med $\sqrt{6 \sum_{n=1}^\infty \frac{1}{n^2}} + \int_{-\pi}^\pi \sin xdx$}

\says{S} \act{kigger på det et sekund og udbryder:} $\pi$

\says{H} Rigtigt! Så vores konklusion må være, at matematikeren ikke formår simpel hovedregning bedre end os andre.

\says{K} Så myten er afkræftet. Men til gengæld er de dygtige på mange andre måder.

\says{H} Helt sikkert. Hvad med næste myte?

\says{K} Det siges, at matematikere altid danser. Så lad os straks få en herop, der kan vise os det.

\scene{Testperson kommer ind på scenen og begynder at danse - ikke særlig godt}

\says{H} Ha.. som vi kan se, så har han da vist aldrig danset før. Myten er altså afkræftet.

\says{K} Ja.. Den næste er faktisk meget interessant. Myten går på, at matematikere er kedelige. Endnu engang har vi inviteret en testperson i studiet.

\scene{et bud med et telegram kommer ind på scenen med telegram til værterne}

\says{H}[læser] Nå, det ser ud til, at vores testperson har meldt afbud, fordi han skal lave faldskærmsudspring i dag, have sex i frit fald, riverrafting i Canada, safari i Zimbabwe, og derefter forberede et stand-up-show, og han skal være hjemme til aftensmad. Sikke mange spændende ting han skal nå. Men dog, det må jo egentlig afkræfte myten?

\says{K} Ja. Så matematikere er ikke kedelige. Hvad er det næste?

\says{H} Vi har hørt en fordom om at matematikere dyrker meget lidt sport.

\says{K} Ej, det kan da ikke være sandt? Jeg har set dem spille Ølkroket i Fælledparken adskillige gange efterhånden. Jeg har sågar været med engang, så vidt jeg husker.

\says{H} Nå, men så var den jo hurtigt overstået. Vores næste myte er lidt mystisk, men en seer har påstået, at da matematikere er lidt verdensfjerne, bruger de kun Windows. Det må være meget let at undersøge, hvis vi bare ...

\scene{teknikken sejler mere end normalt her, og en Windows-fejl-lyd lyder, mens Blue Screen of Death kommer op på storskærmen}

\says{K} Bekræftet! Det var jo meget let.

\says{H} Til sidst skal vi kigge på et relativt nyt fænomen indenfor matematikken.

\says{K} Ja, en ting som de færreste forstår, men som alle alligevel finder brugbart og opmuntrende i hverdagen.

\says{H} Præcis. Det er også en ting, der øger deres sexappeal, så der faktisk produceres flere matematikere til at kigge endnu nærmere på det.

\says{K} Myten hænger lidt sammen med matematikernes evindelige præcision og ordkløveri.

\says{H} Ja, det er rigtigt. Det er også rigtigt, at vi ikke engang behøver teste myten, for vi ved, at den er sand. Myten er, at matematikerne ikke kan lave en god afslutning på en sketch.

\scene{``Bekræftet'' kommer op på skærmen, og lyset går}
\end{sketch}

\end{document}
