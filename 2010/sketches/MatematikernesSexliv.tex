%% reconstructed from file: 'Matematik Revyen 2010.pdf'
\documentclass[a4paper,11pt]{article}

\usepackage{revy}
\usepackage[utf8]{inputenc}
\usepackage[T1]{fontenc}
\usepackage[danish]{babel}

\revyname{Matematikrevy}
\revyyear{2010}
% HUSK AT OPDATERE VERSIONSNUMMER
\version{0.9}
\eta{1.5 minutter}
\status{Introduktion mangler}

\title{Matematikernes sexliv}
\author{Matematikrevyen}

\begin{document}
\maketitle

\begin{roles}
\role{F}[Skuespiller] Forsker/Fremførerske
\end{roles}

\begin{props}
\prop{En rekvisit}[Person, der skaffer]
\end{props}

\begin{sketch}
\scene{En forsker fremviser sine resultater}

\says{F} Jeg har undersøgt, hvordan forskellige faggrupper indenfor matematikken arbejder. Kombinatorikere gør det på så mange måder som muligt, mens algebraikere kun gør det med grupper, og de diskrete gør det på cykler. Men der findes jo mange andre faggrupper. Prøv bare at tænke på statistikerne, de gør det, fordi det er normalt, og så gør de det med 95\% confidence! Og de anvendte svin, de gør det desværre med en rigtig model. Det gider de andre bare ikke, for analytikerne gør det næsten overalt, men de gør det dog kun lige til grænsen. Så har vi jo geometrikerne, der altid gør det fra den rigtige vinkel, og de gør det både under og over kurverne, der i øvrigt er glatte. De gamle og kendte matematikere arbejdede jo også, og Fermat gjorde det i margenen, men der var bare ikke plads til ham. Topologerne gør det helst uden huller, men hvis der er huller, kan de ikke kende forskel på dem. Nu findes der jo også matematiske fysikere, men de ved enten, hvor de gør det eller hvor hurtigt de gør det, men aldrig begge dele.

\says{F} Overordnet set kiggede jeg også på matematikerne som en helhed, og her var også fascinerende måder at arbejde på. De gør det for evigt, hvis de kan gøre det en gang, og én gang til. Det mest slående er jo bare, at de forstår teorien om, hvordan man gør det, men kan bare ikke få det til at fungere i praksis. Og HVIS det så endelig lykkes, så skal de jo bevise det, de stakkels matematikere.

\says{F} Tusind tak for i aften!
\end{sketch}

\end{document}
