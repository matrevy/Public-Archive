%% reconstructed from file: 'Matematik Revyen 2010.pdf'
\documentclass[a4paper,11pt]{article}

\usepackage{revy}
\usepackage[utf8]{inputenc}
\usepackage[T1]{fontenc}
\usepackage[danish]{babel}

\revyname{Matematikrevy}
\revyyear{2010}
% HUSK AT OPDATERE VERSIONSNUMMER
\version{1.0}
\eta{4.5 minutter}
\status{Færdig}

\title{Late Show}
\author{Matematikrevyen}

\begin{document}
\maketitle

\begin{roles}
\role{V}[Skuespiller] Vært
\role{D}[Skuespiller] Villiam Gro, datalog
\role{K}[Skuespiller] Skabskvinde
\end{roles}

\begin{props}
\prop{En rekvisit}[Person, der skaffer]
\end{props}

\begin{sketch}
\scene{Plot: Vores version af The Late Show with David Letterman. Aftenens gæst er en arbejdsløs datalog.}

\scene{Opsætning: typisk late show opstilling: bord + stol og to gæstestole.}

\scene{Jingle spiller og vildt lys på scenen. K er allerede på scenen (lidt malplaceret). V kommer gående ind og byder publikum velkommen med armene. V stiller sig foran bordet.}

\says{V} Godaften allesammen. Hvor ser I godt ud hernede fra. Når man står på scenen, kan man ikke se et eneste ansigt på grund af lyset. Men I føles godt.

\says{V} Men nok om julen; Nu får vi besøg af en gæst. Tag godt imod den top-tunede, intellektuelle, smarte datalog: Villam Gro!

\scene{Ingen kommer ind. V kigger på sit armbåndsur. Smiler. Venter. Smiler endnu mere. Kigger igen på armbåndsuret}

\says{V} \act{rømmer sig højt et par gange og råber til sidst højt} Villiam!

\scene{Så kommer D tøffende genert ind - iført julemandskostume}

\says{V} Hvem er du?

\says{D} Jeg er da Villiam. \act{De sætter sig} Du kan bare kalde mig Villie.

\says{V} Nå ja, det er vel også det, du er. Men fortæl os lidt om dit arbejde.

\says{D} Tja, min tidligere projektgruppeleder og jeg var ikke så indifferente mht. værdigrundlaget i virksomheden, så lige nu er jeg faktisk i en transitionsfase mellem stabile forhold. Så jeg har det meget godt! Jeg har masser af tid til at fokusere på mine fritidsinteresser: LAN-party, jul \act{peger på sit outfit}, forbedre min sexappeal osv.

\says{V} Prøver du på at fortælle os noget?

\says{D} \act{bryder ud i gråd} Jeg er arbejdsløs! Jeg er blevet sendt i praktik som julemand i Magasin! Hvad mere vil du vide?

\says{V} Nå ja, det var ikke sådan ment. Men så fortæl os i stedet lidt om, hvad du gør for at højne din sexappeal?

\says{D} Jeg har lige pimpet min bærbare. Der er kommet lys i understellet, og jeg har opdateret mine ubuntu-lyde. Og derudover er der rimelig mange damer, der snakker til mig, når jeg har det her tøj på. Du ved, mænd i uniform \act{blinker med øjet}

\says{V} Virker det så?

\says{D} Joo, jeg var i byen forleden og drak en masse rom. Og cola. Mmm, cola... Det gik også ret godt med damerne. Jeg fik 7 numre bare på den ene aften.

\says{V} Det var da sandelig vildt.

\says{D} Ja, et nummer mere, så havde jeg haft et helt telefonnummer.

\says{V} Vi har fået nogle læserbreve fra seere, som gerne vil stille dig nogle spørgsmål.

\says{V} ``Hvad betyder udtrykket 'en Flattenlog'?''

\says{D} Åha, det er en populær betegelse for datalogistuderende, som vejer mere i kilogram end de er høje i centimeter.

\says{V} ``Hvornår er man en rigtig datalogstuderende?''

\says{D} Det må siges at være, når man dropper studiet for at arbejde for kapitalen.

\says{V} Vores sidste spørgsmål kommer fra Oda i Herlufsmagle, som spørger ``Hvorfor er DIKU spærret inde af et kæmpe hegn, og hvorfor står der fare for nedfald på hegnet?''

\says{D} Ja, kære Oda. Vores studieleder fik hegnet sat op for at man ikke skal kunne slippe væk fra datalogistudiet, når man først er kommet ind. Det har dog ikke haft den ønskede effekt, idet de fleste dataloger nu bruger tiden på at hænge ud ved hegnet. Men skiltene det er vist en fejltagelse. Der skulle selvfølgelig have stået ``fare for frafald''.

\says{V} Og nu er det blevet tid til aftenens TOP 5.

\scene{jingle og vildt lys}

\says{V} Og vi kan måske få vores datalog til at hjælpe med at præsentere aftenens TOP 5. \act{datalogen nikker} De fem hyppigste grunde til, hvorfor en datalog får ny computer.

\scene{trommehvirvel}

\says{D} 5. Der røg cola i tastaturet.

\scene{trommehvirvel}

\says{V} 5b. Der kom sperm på hans skærm.

\scene{trommehvirvel}

\says{D} 4. Det er en overspringshandling.

\scene{trommehvirvel}

\says{V} 2. Den gamle computer kørte med Windows, og han kunne ikke selv finde ud af at installere Linux.

\scene{trommehvirvel}

\says{D} 1. Den, han havde, kan nu fås med lyserødt tastatur.

\scene{trommehvirvel}

\says{V} 0. På den nye computer ligger der et billede af en pige - måske.

\scene{jingle}

\says{V} Her til sidst skal vi kort vende tilbage til din uddannelse. Du har jo brugt... \act{kigger over på datalogen} 6.. 7.. 8 \act{datalogen nikker} år på din bachelor indtil nu. Hvornår regner du med at blive færdig?

\says{D} Jo, jeg tænker egentlig. Der er jo en ret god fremtid i det her julemandshalløj. Du ved, lange ferier... Men jeg mangler stadig kvinde i mit liv.

\says{K} \act{river overskægget og skjorten af} Bare rolig, jeg er kvinde!

\scene{Datalogen ser lykkelig ud. De tager hinanden i hånden og løber ud sammen. Værten ser forvirret ud.}

\says{V} Ja, og man kan da også sagtens blive til noget, selvom man ikke færdiguddanner sig. Tidligere statsminister Anders Fogh har også en kommentar til netop det emne. Vi har flere gange spurgt ind til Anders Foghs uddannelse. Lad os se et lille videoklip med manden selv.

\scene{Videoklip med Fogh}

\scene{Jingle og vildt lys på bandet!}
\end{sketch}

\end{document}
