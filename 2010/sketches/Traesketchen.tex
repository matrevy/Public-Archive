%% reconstructed from file: 'Matematik Revyen 2010.pdf'
\documentclass[a4paper,11pt]{article}

\usepackage{revy}
\usepackage[utf8]{inputenc}
\usepackage[T1]{fontenc}
\usepackage[danish]{babel}

\revyname{Matematikrevy}
\revyyear{2010}
% HUSK AT OPDATERE VERSIONSNUMMER
\version{1.0}
\eta{2.5 minut}
\status{Færdig}

\title{Træsketchen}
\author{Matematikrevyen}

\begin{document}
\maketitle

\begin{roles}
\role{Elm}[Skuespiller] Elm
\role{TF}[Skuespiller] Taks/Fjot
\end{roles}

\begin{props}
\prop{En rekvisit}[Person, der skaffer]
\end{props}

\begin{sketch}
\scene{To (mennesker klædt ud som) træer mødes midt på scenen.}

\says{Elm} Har du 23 kroner til en busbillet.

\says{TF} Men Elm, penge er RODEN til alt ondt.

\says{Elm} Det er lidt TRÆgt med knaster for tiden.

\says{TF} Penge vokser ikke på træerne.

\scene{pinlig tavshed}

\says{Elm} Hvor STAMMER du fra?

\says{TF} Grenaa!

\says{Elm} Jeg er fra Skovshoved, men vi flyttede da jeg var en lille spire. Vi taler ikke om det, det var meget TRÆgisk

\says{TF} Elm, tal...

\says{Elm} Vi havde idræt. Der var TRÆning i hækkeløb. Det var en af mine yndlings sportsGRENE - sammen med squash, det er nemlig PÆRElet. Bananklasen havde bueskydning, og Ask var lige ved, at skyde knoppen af mig. Men men, problemet for-for-forgrener sig.

\says{TF} Jeg synes du stammer.

\says{Elm} Undskyld, jeg er bare helt kvistet. Nå, men jeg var på vej hjem fra planteskole. Jeg krydsede vejen og så skete det. Bam, bus. Jeg var færdig som poppelsanger.

\says{TF} Årh, ringe

\says{Elm} Jeg måtte stoppe BRÆDT op for ikke at stille TRÆskoene.

\says{TF} Den oplevelse bør man ikke være ikke ENE om at BÆRe.

\says{Elm} Har du trætten kroner til en kop kaffe? Jeg er så træt, at jeg er ved, at gå i brædderne.

\says{TF} Nej, du falmer lidt hen, men det træblæser jeg på. - Nu står du vel ikke og slår rødder.

\says{Elm} Nej, det vil jeg meget nød-igt

\says{TF} Det bregnede jeg heller ikke med.

\says{Elm} Jeg leger bare "tjørnen sover".

\scene{Et klaptræ bliver kastet ind fra siden.}

\says{TF} Klaptræ!

\scene{Elm klapper}

\says{TF} It's good to be tree.

\says{Elm} Har du trækroner til en smøg?

\says{TF} Du er vidst sprunget ud som en være rod var?

\says{Elm} Skylden ligger i mit stamtræ - Jeg har det fra min granonkel.

\says{TF} Æblet falder ikke langt fra stammen.

\says{Elm} Kom nu flotte fyr.

\says{TF} Taks! (sagt lidt fornærmet)

\says{Elm} Stilk mig i det mindste en kuglepen - eller en kakTUSCH?

\says{TF} Tag og PIL af.

\scene{Lang pause}

\says{TF} GÅ UD!

\scene{Elm visner.}

\says{TF} Det skal være endnu træsorter.

\scene{Lys ud}
\end{sketch}

\end{document}
