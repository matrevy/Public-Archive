%% reconstructed from file: 'Matematik Revyen 2010.pdf'
\documentclass[a4paper,11pt]{article}

\usepackage{revy}
\usepackage[utf8]{inputenc}
\usepackage[T1]{fontenc}
\usepackage[danish]{babel}

\revyname{Matematikrevy}
\revyyear{2010}
% HUSK AT OPDATERE VERSIONSNUMMER
\version{1.0}
\eta{2.5 minut}
\status{Færdig}

\title{De nye unge russer - arbejdstitel}
\author{Matematikrevyen}

\begin{document}
\maketitle

\begin{roles}
\role{R1}[Skuespiller] Rus 1, pige
\role{R2}[Skuespiller] Rus 2, dreng
\role{Æ1}[Skuespiller] Ældre studerende 1, dreng
\role{Æ2}[Skuespiller] Ældre studerende 2, dreng
\end{roles}

\begin{props}
\prop{En rekvisit}[Person, der skaffer]
\end{props}

\begin{sketch}
\scene{Under hele scenen står der ``De Nye Unge Russer'' på skærmen. Der står to sofaerne på scenen, hver vendende ud mod den modsatte side af salen. I den ene sidder 2 russer og i den anden sidder to ældre studerende. Spottet er hele tiden på dem, der har replikker.}

\says{R1} Jeg spurgte en af de ældre studerende om noget i dag, og så kunne han ikke engang svare.

\says{Æ1} Så kom der en rus hen til mig i dag og spurgte om det dummeste. Jeg vidste slet ikke, hvad jeg skulle sige til det. Hun ville vide, hvad forskellen på et prikprodukt og et skalarprodukt var \act{tager sig til hovedet}

\says{Æ2} Det er jo virkelig dumt! Tænk hvis Ernst hørte sådan noget... Jeg snakkede forresten med ham tidligere, og han vidste slet ikke at jeg havde fået 10 i Analyse 2!

\says{R2} Hvad sker der for at ham der Ernst Hansen kunne alle navne til forelæsningen i dag!

\says{R1} Mmm... \act{tager en slurk kaffe} Ved du egentlig hvorfor kaffe er så dyrt i kantinen?

\says{Æ1} Nu har de endelig fået hængt mit billede op på 4.

\says{Æ2} Det var da også på tide. Ja, nu har jeg jo LinAlg igen, og så mødte jeg kl. 08.12, og der var de allerede gået i gang!

\says{R2} Og så mødte jeg kl. 08.00 til SaSt, men der var slet ik dukket nogen op. Efter 10min gad jeg ik mere og gik.

\says{R1} Helt sikkert, ville jeg også have gjort... Forresten, jeg fik endelig scoret ham den hotte SaSt-instruktor i fredags. Han tog mig med til det hemmeligste \act{sagt meget forelsket} sted på heeele universitetet!

\says{Æ1} Høhø, i fredags hev jeg en rus med i kemikernes bollerum...

\says{Æ2} Nice! Apropos rus, lagde du egentlig mærke til, at der rendte en rus frem og tilbage i Vandrehallen i dag?

\says{R2} Og så sagde han, at vi skulle mødes i Kantinen og så tænkte jeg bare, HVOR i Kantinen - den er jo helt vildt lang!

\says{R1} Ja, det er rigtigt! Det tager en evighed at finde folk, men det er lidt sjovt, at det egentlig altid er det samme sted man finder dem.

\says{R2} Hva', du nåede da vist lige at finde NB i fredags, hva'? \act{skubber drillende}

\says{R1} Ja, og jeg lagde ret meget an på ham, men han afviste mig bare.

\says{Æ1} Jeg hilste på NB forleden, og så begyndte han bare at snave mig helt vildt!

\says{Æ2} Det skete også for mig i fredags, men vi var på mBar, så folk kiggede virkelig. Jo, og så bestilte jeg en GT, og så fik jeg en eller anden drink!?

\says{R1} Jeg bestilte en Gin \& Tonic på Caféen? - og så gav de mig en eller anden øl!?

\scene{R1 og R2 rejser sig}

\says{R2} Kan du forresten lukke mig ind på biblioteket?

\says{R1} Nej, jeg har ikke lige fået adgang.

\scene{Æ1 og Æ2 rejser sig}

\says{Æ2} Kan du forresten lukke mig ind på biblioteket?

\says{Æ1} Nej, jeg har ikke lige fået adgang.

\scene{Lys ud}
\end{sketch}

\end{document}
