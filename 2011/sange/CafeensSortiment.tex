%% reconstructed from file: 'Matematik Revyen 2011.pdf'
\documentclass[a4paper,11pt]{article}

\usepackage{revy}
\usepackage[utf8]{inputenc}
\usepackage[T1]{fontenc}
\usepackage[danish]{babel}

\revyname{MatematikRevy}
\revyyear{2011}
\version{0.1}
\eta{$2$ minutter}
\status{Ikke færdig}

\title{Caféen?s Sortiment}
\author{Forfatter}
\melody{The Sound of Music: My Favourite Things}

\begin{document}
\maketitle

\begin{roles}
\role{I}[Maling] Instruktør
\role{Ko}[NB] Koreograf
\role{K}[Mathias] Kunde
\role{B}[Rikke] Bartender
\role{D1}[Simone] Danser
\role{D2}[Sanne] Danser
\role{D3}[Lilli] Danser
\role{D4}[Anna Munk] Danser
\end{roles}

\begin{props}
\prop{En af hver fra Cafeen?}[]
\prop{En bar}[]
\end{props}

\begin{song}
\scene{Der er en bar på scenen, kæmpe Caféen?-logo, en bartender står og kaster overlegent med en flaske eller ryster en shaker. En kunde kommer ind, bartenderen spørger hvad det skulle være og kunden siger, Ehm, det ved jeg ikke. Hvad har I?}

\sings{B} Lemon og Tonic og Cocio og Sprite og
Kindleys og Ginger Ale, og Cola Light og
Cola og Fanta og Danskvand med brus
Ting fra Caféen? der ik' gi'r en rus

Te, Lime, Citroner og Kaffe og Smøger
Alle slags Roses og Chips og så Nudler
Ananas-, æble-, og tranebærjuice
Alt på Caféen? der ik' gi'r en rus

Fuglsang ja, Pilsner og Black Bird med mere
FF, Den Blå, Limfjords Porter, GT'er
Fadøl, Specialøl, Små kolde og Mad
Druk på Caféen? det gør dig så glad

Hot'n'sweet, Fernet, Cointreau, Stolichnaya
Baileys, Passoa, Olmeca, Kahlua
(Ha)vanna Club både i mørk og i lys
Her på Caféen? får du kuldegys

Gammel dansk, Malibu, Fisk, Galliano
Jameson og Pisang Ambon, Amaretto
Bombay og Jäger, Cuba caramel
Det er den sprut som Café'n? har til salg
\end{song}

\end{document}
