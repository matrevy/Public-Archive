%% reconstructed from file: 'Matematik Revyen 2011.pdf'
\documentclass[a4paper,11pt]{article}

\usepackage{revy}
\usepackage[utf8]{inputenc}
\usepackage[T1]{fontenc}
\usepackage[danish]{babel}

\revyname{MatematikRevy}
\revyyear{2011}
\version{0.1}
\eta{$2$ minutter}
\status{Ikke færdig}

\title{Det Store Univers Matematik}
\author{Forfatter}
\melody{Yakko's Universe Song}

\begin{document}
\maketitle

\begin{roles}
\role{I}[Ronni] Instruktør
\role{S}[NB] Studerende
\end{roles}

\begin{song}
\sings{S} Alt er bygget op af de få små aktiomer
med beviser viser vi hvad der er sandt
med et lemma og en sætning teorem og et korollar
vi finder vores sandhed deriblandt
mat'matik er et spil fyldt med regler, hvis man vil det
kan man følge dem stringent og kort fortalt
har man lidt fantasi, krydderi og en smule held
så kan man jo bevise næsten alt

Det' et stort aktiomssystem
og vi er kun piloter
og vi driver mat'matik
kun med depoter af fodnoter
det' side efter side
med alt hvad man skal vide
det' et stort univers - Mat'matik

Vi er del af en stor matematisk klassedeling:
Geometrisk Analyse og Fysik
Abelskfri Geometri, og efter en Topologi
følger Algebra hermed Talteorik
og Center for Symme-metri og Deformationsmani
Statistik og så Sandsynlig regneri
Matematisk metodik og Statistisk metodik
i Forsikring og så Økono-nomi

Med e og et og pi, og nul, uendelig og i
som kvadratrod af en ener minus to
vi ved at e er Eulers tal, som har med logaritmer
og pi er diameter op i O
Nu vinker stakkels googolplexian uendelig farvel
for uendelig er mere, det er kendt
Vi ser begrebet kendt som nul, er for Babylon et hul
Og oprindelsen af et er sikkert glemt

Det' et stort aktiomssystem
og vi er kun piloter
og vi driver mat'matik
kun med depoter af fodnoter
Vid at jeg er primus motor
og I er and'npiloter
det' et stort univers - mat'matik
\end{song}

\end{document}
