%% reconstructed from file: 'Matematik Revyen 2011.pdf'
\documentclass[a4paper,11pt]{article}

\usepackage{revy}
\usepackage[utf8]{inputenc}
\usepackage[T1]{fontenc}
\usepackage[danish]{babel}

\revyname{MatematikRevy}
\revyyear{2011}
\version{0.1}
\eta{$2$ minutter}
\status{Ikke færdig}

\title{De tog stadig folk ind}
\author{Forfatter}
\melody{De bar stadig øl ind}

\begin{document}
\maketitle

\begin{roles}
\role{I}[KØK] Instruktør
\role{05}[Loke] Sanger fra '05
\role{06}[Søren] Sanger fra '06
\role{07}[NB] Sanger fra '07
\role{08}[Rasmus Bryder] Sanger fra '08
\role{09}[Camilla] Sanger fra '09
\role{10}[Alexander Jasper] Sanger fra '10
\role{11}[Simone] Sanger fra '11
\end{roles}

\begin{song}
\sings{05} Der sad ti russer i Vandrehallen
og regned' løs kun ved hjælp af skallen
de havde stole og en per næse
og det var godt når de skulle læse
men de tog stadig folk ind

\sings{06} Så året efter der blev de flere
og dette gjorde de fyldte mere
og måtte deles om deres pladser
men auditoriet havde masser
men de tog stadig folk ind

\sings{07} Så næste år der blev pladsen knap og
de måtte sidde på salens trapper
men læsepladser var der til alle
hvis blot man rykkede sig en balde
men de tog stadig folk ind

\sings{08} I ES01, der blev folket travle
for der var stole og bord og tavle
så derfor måtte man rykke sammen
og flere endte så ud' på gangen
og de tog stadig folk ind

\sings{09} Og denne årgang ja den forskyldte
at alle gangene de blev fyldte
og vandrehallen ku' ej ta' mere
men der var stadigvæk plads på fjerde
men de tog stadig folk ind

\sings{10} Professer'n fandt ikke plads ved bordet
og spiste madpakken på kontoret
og der var plads ja det sku' man mene
så hvorfor sidde der helt alene
for de tog stadig folk ind

\sings{11} med HCØ fyldt til bristepunktet
selv over loftet og under gulvet
men vi bli'r reddet af gudmorsfeen
for vi kan stadig ta' på Caféen?
og de ta'r stadig folk ind
\end{song}

\end{document}
