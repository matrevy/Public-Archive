%% reconstructed from file: 'Matematik Revyen 2011.pdf'
\documentclass[a4paper,11pt]{article}

\usepackage{revy}
\usepackage[utf8]{inputenc}
\usepackage[T1]{fontenc}
\usepackage[danish]{babel}

\revyname{MatematikRevy}
\revyyear{2011}
\version{0.1}
\eta{$6$ minutter}
\status{Ikke færdig}

\title{Bevis for Skæringssætningen}
\author{Forfatter}
\melody{Queen: Bohemian Rhapsody}

\begin{document}
\maketitle

\begin{roles}
\role{I}[Maling] Instruktør
\role{Ko}[Sanne] Koreograf
\role{R}[Morten] Rus
\role{K1}[Loke] Kor 1 / Instruktør
\role{K2}[René] Kor 2
\role{K3}[Johanna] Kor 3
\role{D1}[Freja] Danser
\role{D2}[William] Danser
\end{roles}

\begin{props}
\prop{Tal med Maling}[]
\end{props}

\begin{song}
\sings{K} Se på den tegning
Grafen går denne vej
Det er da klart nok
hvor det nulpunkt må gemme sig
Ja - her og her -
der kan være fler' end et!
Se på det største, det er da lige til
Det er supremum for mængden $D$
Hvad er $D$? Lad os se:
Det skal vær' de $x$, hvor $f(x)$ er mindre end 0 - end 0

\sings{R} Sæt $c$ - lig' $sup(D)$
vælg $x_n$ i mængden $D$
højst $\frac{1}{n}$ fra $c$
Følgen $x_n$ går mod $c$
Så derfor konkluderer vi nu at
Følgen $f(x_n)$
Konverger' mod $f(c)$
Så $f(c)$ er svagt mindre end nul og

Det var den ulighed, kan vi også få den anden?

\sings{R} $c$ + $\frac{1}{n}$
det kalder vi $x_n$
For $b > x_n$
Følgen kaldet $x_n$, den går mod $c$
så vi konluderer derfor lig'som før:
Følgen $f(x_n)$
Går mod $f(c)$
Så $f(c)$ er derfor svagt stør' end 0

\sings{R} Nu har jeg næsten fået klaret mit bevis
\sings{K} Pas nu på! Pas nu på! Du har glemt en detalje!
Du skal også vise hjælpesætning 5.1.10!

\sings{K} (Du er givet) Du er givet (Du er givet) Du er givet, Du er givet $\varepsilon > 0$(-0-0-0-0)

\sings{R} Da er vor afstand fra $f(x_n)$
\sings{K} Hen til $f(c) < \varepsilon$
Hvis blot $x_n$ kun er $\delta$ fra $c$

\sings{R} Så er jeg færdig her, følgen konverger'!
\sings{K} Bevís det! Nej - det må da være klart
Soleklart!
Bevis det! Det må da være klart!
Soleklart!
Bevis det! Det må da være klart!
Soleklart! (må da være klart!)
Soleklart! (må da være klart!)

\sings{R} TRI-VI-EEEEEELT!
\sings{K} NEJ NEJ NEJ NEJ NEJ NEJ NEJ!
\sings{R} Åh, hvorfor ikke, hvorfor ikke, hvorfor ikke trivielt?
\sings{K} Funktionen f er kontinuert i punktet c - i c - i c!

\sings{K} Da $x_n$ konvergerer mod $c$ ses igen
at $x_n$ højst er $\delta$ fra $c$ for stort $n$
Hvad mere? - Vi kan nu konkludere,
at $f(x_n)$ - har grænseværdi $f(c)$

\sings{K} Hvaaad nu? Hvaaad nu?

\sings{R} $f(c)$ er større
eller lig med nul
$f(c)$ er mindre...
$f(c)$ må være... $= 0$

\sings{K} Hvilket skulle vises...
\end{song}

\end{document}
