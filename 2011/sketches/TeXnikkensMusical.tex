%% reconstructed from file: 'Matematik Revyen 2011.pdf'
\documentclass[a4paper,11pt]{article}

\usepackage{revy}
\usepackage[utf8]{inputenc}
\usepackage[T1]{fontenc}
\usepackage[danish]{babel}

\revyname{Matematikrevy}
\revyyear{2011}
% HUSK AT OPDATERE VERSIONSNUMMER
\version{0.2}
\eta{$4$ minutter}
\status{Ikke færdig}

\title{Te$\chi$nikkens Musical}
\author{en forfatter}

\begin{document}
\maketitle

\begin{roles}
\role{I}[Ronni] Instruktør
\role{Ko}[Sanne] Koreograf
\role{Ø}[Simone] Økonomiboss
\role{T}[Anna Munk] Teknikboss
\role{R}[Christoffer] Rekvisitboss
\role{S}[René] Sponsorboss
\end{roles}

\begin{props}
\prop{2 stole}[]
\end{props}

\begin{sketch}
\scene{Hele scenen skal spilles med et klaver, eller med ét instrument pr sang, men må gerne skifte. Vi prøver at skrive på ud for sangen hvilke instrumenter der skal akkompagnere.}

\scene{Økonomibossen Ø står på scenen idet lyset tændes}

\says{Ø} Det at være Økonomiboss i revyen er slet ikke så nemt som mange tror. Folk kommer hele tiden og vil have penge, og jeg mener virkelig HELE TIDEN.

\scene{Teknikbossen T kommer ind}

\scene{Mel: \underline{En tosset sang}, orgel/klaver}

\says{T} Hør, hvorfor er min konto tom  \\
Før var den fyldt til randen \\
Vi havde flere tusind som \\
vi ku' bruge på hinanden

\scene{Mel: \underline{Musenes arbejdssang}, klaver}

\says{Ø} I kan ikke, I kan ikke, I kan ik' ha' brugt det hele! \\
Flere tusind på en konto! \\
Brugt på lys og lyd og foto!

\scene{Mel: \underline{Flyv så med!}, guitar}

\says{T} Har vi så et sponsorat? \\
Se, dét kunne være smart!

\says{Ø} : Hvis I ingen penge har \\
Må I blive aktuar \\
Vi har jo ikke mer! \\
Ikke mer! Ikke mer! Ikke mer! Ikke mer! Ikke meeer!

\scene{R kommer ind}

\scene{Mel: \underline{Bella Notte}, guitar}

\says{R} Hvaaaad? Er det slut? \\
Er revyen kaput? \\
Jamen, hvor er vores penge? \\
Vi har tjent op, \\
Vi har solgt vores krop \\
Så, hvor er vores penge?

\scene{Mel: \underline{Cruella De Ville}, klaver}

\says{Ø} Hvad har I brugt, pengene på? \\
Når de bruges rigtigt vil de ej forgå \\
Køber I et lager af den blå? \\
Men hvad bruges de penge på?

\scene{Mel: \underline{Det rent og skært nødvendige}, klaver}

\says{T} Det rent og skært nødvendige, det elementært nødvendigt, det' hvad vi bruger vores penge på. Det' nu og her nødvendige, det kræver vi' behændige, men så' det der nødvendige en \textbf{leg} )skal rime).

\scene{Ø sukker, 2 stole bliver stillet ind på scenen bag R og T.}

\scene{Mel: \underline{Vær vor gæst}, guitar}

\says{Ø} Teknik, rekvisit.. \\
Det er med en yderste stolthed og største glæde, at jeg fortæller jer
dette. Og nu beder jeg dem tage plads (skubber dem ned på deres stole) og sætte dem tilbage, mens Matematik Revyen stolt præsenterer: Budgettet.

\scene{Mel: \underline{Prins ali}, klaver}

\says{T+R} Kan vi ik' få lidt rabat Økonomiboss! \\
Vi har hørt at du er god når det er jul!

\says{R} Og det er derfor at jeg \\
Tog kassen med denne vej (vender en tom pengekasse på hovedet), \\
Så gi' en smule til mig \\
Du er så rig!

\scene{Mel: \underline{Et helt nyt liv}, intet}

\says{Ø} Prisen er sat! Og der er slet ingen rabat!

\scene{Mel: \underline{Meget snart majestæt}, intet}

\says{T} Går vi meget snart helt konkurs!?

\scene{Mel: \underline{Gør Jer klar}, klaver}

\says{Ø} Jeg ved I er meget imod det, som jeg nu fortæller jer her. \\
Men hør dog alligevel efter, hvis I godt vil tjene lidt mer'. \\
Vil du have lys til revyen? \\
Vil du have stof til at sy? \\
Vil I have alt det I ønsker (de to bosser nikker febrilsk undervejs, når de bliver adresseret) \\
Så lav jeres egen revy!

\scene{stilhed, de skammer sig}

\scene{Mel: \underline{Hakuna Matata}, guitar}

\says{R} Men matematikken. \\
Er vor eneste kick. \\
Ja Matematikken. \\
Bedre bli'r det ik'!

\scene{Mel: \underline{Føl hvordan dit liv bli'r fyldt}, klaver}

\says{T} Men vi har et enkelt krav \\
Det er ik' S.U. \\
Hr på os, det den ting vi elsker mest \\
Giv os kage nu!

\scene{Mel: \underline{Vindens Farver}, guitar}

\says{R} Bare kokos, chokolade eller med banan \\
Pistacie, nødder, æble eller bær \\
Men hvis ikke du kan give os et stykke \\
Så vil vi atter blive til besvær \\
Så vil vi atter blive til besvær

\scene{Sponsorboss S kommer ind}

\scene{Mel: \underline{Rigtige Mænd}, guitar}

\says{S} Studiesammenlægning! \\
Det er vores svar! \\
Penge fås fra DIKU! \\
Så nu er vi klar! \\
Vi har atter løn, som I kan få \\
Så bar' brug løs, I må forstå \\
Vi er rig, rigtig rig, \\

\scene{Går til omkvædet}

\says{T+Ø+R} DIKU gi'r

\says{S} Penge til Mat'matik Revyen

\says{T+Ø+R} DIKU gi'r

\says{S} Og penge til vores PR hold

\says{T+Ø+R} DIKU gi'r

\says{S} Lidt penge får også rekvisitten

\says{R+Ø+S+T} Vi' reddet, så sig' tak til KU NU!
\end{sketch}

\end{document}
