%% reconstructed from file: 'Matematik Revyen 2011.pdf'
\documentclass[a4paper,11pt]{article}

\usepackage{revy}
\usepackage[utf8]{inputenc}
\usepackage[T1]{fontenc}
\usepackage[danish]{babel}

\revyname{Matematikrevy}
\revyyear{2011}
% HUSK AT OPDATERE VERSIONSNUMMER
\version{0.2}
\eta{$5$ minutter}
\status{Ikke færdig}

\title{Harry Potter}
\author{en forfatter}

\begin{document}
\maketitle

\begin{roles}
\role{I}[Kirsten] Instruktør
\role{F}[Therkel] Fortæller
\role{HP}[William] Harry Potter
\role{PX}[Kasper Brandt] $P(X \leq x)$
\role{R}[Mathias] Ron
\role{H}[Lilli] Hermione
\role{I}[Sanne] Instruktor
\role{HB}[Kristian] Henrik Busch
\role{M}[Alexander] Malfoy
\end{roles}

\begin{props}
\prop{En kasse til fordelingsfunktionen (menneskehøj, eventuelt med trylletæppe, snak med Kirsten)}[]
\prop{6 sorte kapper}[]
\prop{Tylskørt}[]
\prop{1 farverig kapper}[]
\prop{Gummistøvler (Vaders)}[]
\prop{Manden der straffer ordspil-rustningen}[]
\prop{Julekittel (til Ron)}[]
\prop{1 stor billedramme}[]
\prop{3 nissehuer}[]
\prop{1 bord}[]
\prop{2 bægre og 2 terninger}[]
\prop{2 fezz}[]
\prop{2 pakker kridt}[]
\prop{Halstørklæder i Husfarverne fra Harry Potter}[]
\prop{2 Røde sløjfer}[]
\end{props}

\begin{sketch}
\scene{Hedwigs Theme spiller i baggrunden, der kommer lys på en mand der mest af alt ligner Mad Hatter fra alice i eventyrland. Han begynder at fortælle imens musikken fortsætter}

\scene{De fire personer Harry, Ron, Hermione og Malfoy skal ligne personerne fra "Harry Potter"så meget som muligt. Fortælleren går i midtergangen eller ude i en af siderne. Ikke på scenen.}

\says{F} Det skete en dag at Harry Potter og hans venner skulle starte på naturvidenskab. De vidste ikke hvilket studium de skulle vælge, så de startede hos studievejlederen, men hun var netop blevet fyret af ham man ikke må nævne. Sikke en skam tænkte de. Så prøvede de at kigge på science.ku.dk, men eduroam var midlertidigt nede. De endte til sidst oppe hos Ernst Hansen, hvor de klagede deres nød. Ernst Hansen smilede roligt og gav dem en fordelingsfunktion. (Fortælleren træder ud af en dør i mellem gangen og der kommer lys på scenen)

\scene{Harry, Hermione, Ron og Malfoy kommer på scenen, hvor der står en fordelingsfunktion $P(X \leq x)$. De går herefter frem til fordelingsfunktionen en efter en og bliver placeret.}

\scene{Malfoy går hen bagfordelingsfunktionen}

\says{PX} FYSIK!

\scene{Malfoy bliver smidt væk, han har nu en fez på hovedet. Så går Hermione derhen}

\says{PX} Nøj, en frækker en. Dig kunne de godt bruge på datalogi.

\says{H} NEEJ!

\scene{Hun løber grædende væk, nu med tylskørt på. Ron går derhen imod, men snubler på vejen over sine egne ben og falder}

\says{PX} BIOLOG!

\says{R} Jamen jeg har kun haft biologi på C-niveau.

\says{PX} Ja, men så har du jo mere end hvad du behøver.

\scene{Til sidst går Harry Potter forsigtigt derhen. På vejen derhen snakker de andre om ham.}

\says{R} Ej, er det ikke ham alle taler om.

\says{H} Jo, det var ham der overlevede ham der ikke må nævnes modstridsbevis. Han har stadig lynet i panden.

\says{R} Gisp!!!

\says{PX} Hm... Lad mig se.. Du er eksakt, præcis som en matematiker, Kodestærk som en datalog, har en enorm viden om svampe som en biolog og er eksperimenterende som en fysiker.

\says{HP} Ik' fysik, ik' fysik, ik' fysik.

\says{PX} Hmmmmmm... du vil ikke læse fysik, siger du. Hvor sikker er du?

\says{HP} Bombesikker.

\says{PX} Ser man det, dårlige ordspil. Du hører HELT sikkert til på Matematik!

\scene{musikken begynder så småt igen, spotlight på fortælleren igen, mens at resten af scenen gør klar til næste scene}

\says{F} Og sådan skete det at Harry og hans venner blev placeret på studierne på Naturvidenskab. De oplevede meget i deres første blok. Harry blev topscorer i Caféen?-cuppen, Hermione blev den første datalog til at bestå begge kurser i blok 1 og Rons rotte slap løs i HCØ's kantine, så den midlertidigt blev lukket - så alt var godt. Vi falder ind i historien igen en dag ved juletid.

\scene{Fortælleren tager en nissehue på og går ud af døren. I mellemtiden er alle gået ud af scenen, og der står nu kun en mand i ridderoutfit henne bagerst ved bandscenen (manden der straffer dårlige ordspil) og et "levende billede"af to personer der lægger arm. En billedramme uden noget billede i kan bruges til dette.}

\scene{Ron, Harry og Hermione kommer nu gående ind på scenen, og hen til bagtæppet, som trækkes fra og afslører en tavle til deres undervisningstime. Ron har en rød og grøn julekittel på, Hermione har røde sløjfer i sine fletninger og Harry har også en nissehue på. De går en runde på scenen og snakker på vejen hen til lokalet.}

\says{HP} Hvordan har du egentlig fået tilpasset dig på biologi, Ron?

\says{R} Jeg er faktisk blevet rigtig glad for det. Vi har undervisning ude i mudderet, det er lidt hårdt, men enormt spændende. Og efter hver forelæsning har vi 2 minutters fri leg, hvor vi HELT selv må vælge hvilket mudder vi vil se på!

\scene{lyder lidt for glad i forhold til hvad man burde, de andre to ser lidt skræmte på hinanden. De ankommer lokalet, bagtæppet trækkes fra}

\says{I} Velkommen til jeres første time i Sætninger I. I dette kursus vil I lære at bruge naturvidenskabens sætninger under de rette antagelser og på de rette tidspunkter.

\scene{Hermione rækker ivrigt hånden op}

\says{H} Hvornår må vi høre noget om de utilgivelige sætninger?

\says{I} Udvalgsaktiomet, kontinuumshypotesen,... og Pytagoras på ligesidede trekanter, det er 6. års stof! Nu skal I lære at beherske en af de mest elementære ting inden for matematik, beviser. Lad os vende tilbage til dagens hovedsætning.

\scene{Ron svinger stort med sit kridt}

\says{R} Jeg har prøvet flere formler, men det giver ingenting. Jeg har prøvet Gramm-Schmidt, ingenting. l'Hospital, ingenting! Så prøvede jeg epsilon-delta, men det kan jeg heller ikke finde ud af. Jeg forstår det ikke!

\says{H} Du gør det også alt for stort, du skal bruge små bevægelser, og desuden hedder det altså epsilon-deltaaahh..

\scene{Ron prøver at efterligne Hermione. Lyder torden i baggrunden}

\says{H} Nej!! Stop! Vent! Nu tror jeg kridtet løber af med dig. Du er jo lige ved at få en modstrid!

\scene{Hermione stopper Ron, og tordenen stilner af igen}

\says{H} Ih, du har jo også antaget det forkerte!

\says{R} Men Malfoy sagde, at jeg skulle dividere med nul!?

\says{M} Hehehe...

\says{H} Lad mig. Epsilon-deltaaahh! *svinger kridtet*

\scene{Ron efterligner hende. Hans kridt knækker.}

\says{R} Argh, nu knækkede mit kridt. Så kan jeg ikke lave noget. Jeg savner mit mudder.

\scene{Der kommer igen spotlight på fortælleren, ridder og billede går væk fra scenen, de andre går ud gennem bagtæppet og trækker for efter sig. Temaet begynder igen.}

\says{F} Og sådan fortsatte det i noget tid. Ron fandt ud af, at Hermione var mudderblod, og han inviterede hende derfor på svampetur i skoven og viste hende sit yndlingsmudder, men hun ville hellere hjem at kode. Man var vel datalog. Ingen lagde specielt meget mærke til Harry, der sad for sig selv til timerne og læste matematik, som ingen af de andre havde hørt om før. Rygterne om ham der ikke må nævnes' tilbagevenden spirede i universitetsparken, og det skulle vise sig snart at være tid til deres store prøve - måske endda med livet som indsats.

\scene{De tre kommer ind på scenen igen}

\says{R} Hvorfor er det igen at vi er taget på Caféen??

\says{H} Suk. Ernst Hansen er jo lige taget ud og rejse, så det er i aften at Henrik Busch (de andre uschsssss) ''Ham der ikke må nævnes'' slår til.

\says{HP} Men hvofor er vi så her?

\says{H} Hvor finder du ellers en fysiker en fredag aften?

\says{HB} Muhahaha!! Fjollede børn, I har fundet mit hemmelige skjulested. (tæppet fra, hvor Henrik Busch sidder bag et bord, mens han ryster et bæger. Malfoy står i baggrunden (support for Henrik Busch)). Men nu skal I dø, jeg udfordrer jer til kranen!

\says{H} Det kommer du til at fortryde!

\scene{Harry sætter sig til bords. Ron og Hermione ser på.}

\says{HB} Aha (Censur: Harry nyser, så man ikke hører meldingen)

\scene{Harry ryster og kigger på terningerne og melder med et selvsikkert smil, meget langsomt}

\says{HP} AAH, AAH, AAH. Kranen! Den bliver man grebet af!

\scene{Henrik Busch modtager Kranen, løfter, den er der.}

\says{HP} Der VAR Kranen.

\says{HB} Årh! Næste år skal I dumme russere ikke lære at spille Kranen! Jeg aflyser rusturen!

\scene{lys ud}
\end{sketch}

\end{document}
