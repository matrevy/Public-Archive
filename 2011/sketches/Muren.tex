%% reconstructed from file: 'Matematik Revyen 2011.pdf'
\documentclass[a4paper,11pt]{article}

\usepackage{revy}
\usepackage[utf8]{inputenc}
\usepackage[T1]{fontenc}
\usepackage[danish]{babel}

\revyname{Matematikrevy}
\revyyear{2011}
% HUSK AT OPDATERE VERSIONSNUMMER
\version{0.2}
\eta{$4.5$ minutter}
\status{Ikke færdig}

\title{Muren}
\author{en forfatter}

\begin{document}
\maketitle

\begin{roles}
\role{I}[KØK] Instruktør
\role{M}[Alexander] Matematik Rus
\role{RM1}[JØP] Rusvejleder på Matematik 1
\role{RM2}[Anna Munk] Rusvejleder på Matematik 2
\role{Ø}[Freja] Mat-Øk rus
\role{RØ1}[Therkel] Rusvejleder på Mat-Øk 1
\role{RØ2}[Camilla] Rusvejleder på Mat-Øk 2
\role{S1}[Ada] Statist 1
\role{S2}[Christoffer] Statist 2
\end{roles}

\begin{props}
\prop{Muren (snak med KØK)}[]
\prop{3 guldkæder}[]
\prop{3 bandanaer}[]
\prop{Et ark med en ligning}[]
\prop{Et bundt matadorpenge}[]
\end{props}

\begin{sketch}
\scene{Der skiftet til scenen, så det hele er klar til når Filmen om Muren er færdig. Der starter muren og to vejledere på scenen. En rus kommer ind på scenen efter lyset er blevet tændt}

\says{M} Hej! Tak; det var en fed rustur i arrangerede - men hvor er de andre?

\says{RM1} Hvem snakker du om?

\says{M} F.eks. dem som stiller alle de dumme spørgsmål tilforelæsningerne.

\says{RM1 + RM2} Fysikerne?

\says{M} Nej. Dem der ikke forstår de komplekse tal.

\says{RM1 + RM2} Kemikerne?

\says{M} Nej. Dem der ikke fatter $\varepsilon$ beviser.

\says{RM1 + RM2} Datalogerne?

\says{M} Nejnej. Fx hende jeg mødte på introdagene. Hende med guldkæderne og Nik og Jay-tøjet!

\says{RM2} Nåh... det må være en af dem fra...

\says{RM1 + RM2} (kigger stift på publikum og peger på muren -- og lysskift) ''Den anden side''.

\scene{Muren på scenen drejer rundt, så den nu står og gemmer for matematikerne. Der er nu Mat-Øk'ere på scenen}

\says{Ø} Hejsa.

\says{RØ1} Hej Rus!

\says{Ø} Hvad er der egentlig bag den dér mur.

\says{RØ2} (henkastet): Nå, den. Det er bare matematikerne. Der må du aldrig gå hen rus.

\says{RØ1} Så ender du bare som gymnasielærer.

\says{RØ2} Det er et farligt sted.

\says{Ø} Men vi har da mange af de samme fag som matematikerne, og vi bliver da ikke gymnasielærere.

\says{RØ1} Det er fordi, vi BRUGER det vi lærer.

\says{Ø} Jeg tror godt, jeg kunne være sammen med en gymnasielærer. I hvert fald hvis det er ham med brillerne som altid kommer 5 min for tidligt til forelæsningerne og sætter på plads nummer 3 til venstre for midten på 2. række.

\says{RØ1 + RØ2} (synges på oppe i norge): Mat-øk-rus prøver at score, for hun er en luder. Luder $x7$.

\says{Ø} (rødmer) Kan man komme til at møde matematikerne uden for forelæsninger.

\says{RØ1} Du mener vel ikke at du vil over på...

\says{RØ1 + RØ2} (kigger stift på publikum og peger på muren -- og lysskift): ''Den anden side''.

\scene{Muren på scenen drejer rundt, så den nu står og gemmer for Mat-Øk'erne. Der er nu matematikere på scenen}

\says{M} Er der egentlig nogensinde nogen matematikere, der har forsøgt at komme over muren?

\says{RM1} Jo, der var da nogle stykker tilbage i 1964. De nåede dog aldrig helskindet der over, og endte i en slags limbo.

\says{RM2} Det er det vi i dag kalder statistikstudiet. Nu skal vi til forelæsning. Hej hej rus!

\scene{RM1 og RM2 går ud, M står alene tilbage, overvejende}

\says{M} Det kan da ikke være rigtig man ikke kan komme over muren. Men er det overhovedet en god idé? Altså... på den ene side, så er jeg superglad for mit studie, og ligninger er bare total nice... men -- men - på "den anden side"...

\scene{Sceneskift til Mat-øk-rus, som sidder og læser i ØkIntro-bogen. Muren bliver løftet, Matematikrus kravler under den med en skovl i hånden.}

\says{Ø} Matematikrus?

\says{M} Mat-øk-rus?

\says{Ø} Åh, Matematikrus!

\says{M} Åh, Mat-øk-rus!

\scene{De krammer hinanden tæt}

\says{Ø} ÅH! Matematikrus!

\scene{Matematikrus trækker sig væk -- lidt skræmt}

\says{M} Wow! Det er lidt for meget virkelighed til mig, det dér.

\says{Ø} ...Matematikrus, hvad laver du her?

\says{M} Jeg var nødt til at komme og se dig. Jeg har drømt om dette øjeblik, siden jeg så dig første gang.

\scene{RM1 og RM2 kommer ind}

\says{RM1} Matematikrus! vi er kommet for at redde dig.

\says{M} Redde mig? Men jeg er taget her over helt frivilligt.

\says{RM2} HaHaHaHa! Du behøver ikke lyve, rus. Der er ingen der hører dig. Nu frelser vi dig, og tager dig tilbage til ''den anden side'', så du kan opleve hvordan det rigtigt er at studere...

\scene{sceneskift}

\says{RM2} (fra bag muren): Hov, jeg var ikke færdig, fokus skal stadig være herovre på "den anden side".

\scene{sceneskift}

\says{RM2} ...på matematik.. Og så går vi tilbage på "den anden side".

\scene{sceneskift}

\says{RM1} Hvad så du egentlig derovre?

\says{M} Ikke rigtig noget?

\says{RM2} Var der penge derovre - og pizza?

\says{M} ..Er der pizza derovre?

\says{RM1} Du har ikke brug for pizza! (rusker ham)

\says{M} Nejnej.

\says{RM1 + RM2} Puha!

\says{RM2} Vi er nødt til lige at teste dig.

\scene{RM1 holder et bundt penge og RM2 holder en blok papir frem, på blokken står der en andengradsligning. Uden tøven går Matematikrus hen og løser ligningen.}

\says{RM1} Godt! Du er stadig en af os.

\says{RM2} Nu skal vi til forelæsning. Hej hej rus!

\scene{RM1 og RM2 forlader scenen}

\says{M} Hm... pizza... Jeg kan ikke leve uden Mat-øk-rus. Jeg bliver nødt til at se hende igen. Men hvis jeg går der over, bliver jeg bare fanget. Hvad skal jeg gøre. Hvis bare den pokkers mur ikke var der. (åbenbaring) Muren! Der har vi løsningen! Hvis muren ikke er der, kan jeg gå frem og tilbage som jeg har lyst. Men hvilken betydning vil det få for studiet? Min vejleder siger, at det vil føre til matematikstudiets undergang. Men uden Mat-øk-rus bliver matematikstudiet min undergang. Den mur skal væk! Jeg må ændre verden. Jeg må have Mat-øk-rus!... Og en pizza. Jeg MÅ over på den anden side!

\scene{Han skubber muren, den drejer rund og vi ser Mat-øk-rus, og muren -- der nu er bag hende -- vælter ned over hende og Matematikrus står bag den. Mat-Øk rus skriger i lidt tid. Mat rus står stille i noget tid og kigger på publikum, hvorefter han træder sidelæns og fløjtende ud af scenen}
\end{sketch}

\end{document}
