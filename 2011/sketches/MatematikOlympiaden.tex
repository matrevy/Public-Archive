%% reconstructed from file: 'Matematik Revyen 2011.pdf'
\documentclass[a4paper,11pt]{article}

\usepackage{revy}
\usepackage[utf8]{inputenc}
\usepackage[T1]{fontenc}
\usepackage[danish]{babel}

\revyname{Matematikrevy}
\revyyear{2011}
% HUSK AT OPDATERE VERSIONSNUMMER
\version{0.2}
\eta{$2$ minutter}
\status{Ikke færdig}

\title{Matematik Olympiaden}
\author{en forfatter}

\begin{document}
\maketitle

\begin{roles}
\role{I}[Ronni] Instruktør
\role{S}[Alexander Jasper] Speaker
\role{1}[Therkel] 1
\role{exp}[JØP] Eksponentialfunktionen
\role{LV}[Anna Brusch] Logistisk Vækst
\role{NF}[Rikke] Normalfordelingsfunktionen
\end{roles}

\begin{props}
\prop{En hundeproppistol}[]
\prop{4 sæt træningstøj}[]
\end{props}

\begin{sketch}
\scene{Der er to stykker tape på scenen, som angiver start- og slutstedet. Desuden er der en mand med en pistol}

\says{S} Velkommen tilbage her til Matematik Olympiaden, vi skal nu i gang med årets Pi-meter løb, og lad os få den første kompetant på banen!

\scene{logistisk vækst kommer på scenen}

\says{S} Her har vi den garvede svenske funktion logistisk vækst, og den gør klar til start og vi er i gang.

\scene{Pistol BANG!}

\scene{LV starter sit løb meget langsomt og bliver hurtigere frem i mod midten, for derefter at blive langsommere mod slut, og går i stå lige foran målstregen.}

\says{S} Nå, det var da mærkeligt - den må have haft en asymptote inden målstregen.. Jeg hører nu en opdatering fra hækkeløbet i nærheden. Det forlyder at $\frac{1}{x}$ er blevet diskvalificeret fordi den prøvede at komme under hækkene i stedet for over. Den næste kompetant i pi-løbet er den konstante funktion 1!

\scene{1 kommer på scenen. Pistol BANG. Konstanten 1, rykker sig ikke fra start positionen men kæmper hårdt for at rykke sig}

\says{S} Og SÅ gik startskuddet. 1 ser ud til at kæmpe hårdt, men uden nytte. Desværre! Det er godt nok ikke særlig imponerende det her, men det overstiger da stadig sidste år, hvor tangens skulle forsøge sig med pi-løbet, men fejlede gevaldigt på halvvejen. Næste deltager i pi-løbet er nu den tibetanske eksponentialfunktion, lad os se hvordan den klarer sig! Den har tidligere haft stor succes inden for disse discipliner!

\scene{exp(x) kommer på scenen, stiller sig klar. Pistol BANG. exp(x) løber hurtigt den modsatte vej}

\says{S} Ej, det må skyldes en fortegnfejl. Hvor ærgerligt! Jeg hører netop nu i min øresnegl at finalen i synkronsvømning er blevet afgjort. Holdet med sinus og cosinus måtte desværre se sig slået i årets finale, da den ene hele tiden haltede bagefter den anden. Nu kommer sidste konkurrent ind og gør klar til pi-meter løbet, nemlig den franske normalfordelingsfunktion.

\scene{Den kommer ind og stiller sig klar. Pistol BANG. Normalfordelingen løber helt hen til mållinjen og vender da om}

\says{S} Øv, det så ellers lovende ud, men det var vel hvad man kunne forvente. Men hov, jeg hører at vi har fået en verdensrekord! Det drejer sig om den sure spanske parabel, som har sat ny rekord i højdespring - det antages dog stadig at eksponentialfunktionen hopper højere, men eftersom den stadig ikke er landet, tæller dens hop stadig ikke - endnu. Til slut sidder der måske nogen der ude og under sig over hvorfor vi ikke har fået en rapportage omkring ridning, men den sportsgren er blevet skåret fra, da man har indset at heste og matematikere er en dårlig kombi.
\end{sketch}

\end{document}
