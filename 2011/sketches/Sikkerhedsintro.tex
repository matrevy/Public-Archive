%% reconstructed from file: 'Matematik Revyen 2011.pdf'
\documentclass[a4paper,11pt]{article}

\usepackage{revy}
\usepackage[utf8]{inputenc}
\usepackage[T1]{fontenc}
\usepackage[danish]{babel}

\revyname{Matematikrevy}
\revyyear{2011}
% HUSK AT OPDATERE VERSIONSNUMMER
\version{0.1}
\eta{$1$ minut}
\status{Ikke færdig}

\title{Sikkerhedsintro}
\author{En forfatter}

\begin{document}
\maketitle

\begin{roles}
\role{I}[Kirsten] Instruktør
\role{F1}[William] Fysiker 1
\role{F2}[Camilla] Fysiker 2
\end{roles}

\begin{props}
\prop{2 kitler}[]
\prop{2 fez}[]
\end{props}

\begin{sketch}
\scene{2 Fysikere står på scenen.}

\says{F1} Hvad skal vi så lave i aften?

\says{F2} Det samme som vi laver hver aften, prøver at ødelægge matematikrevyen!

\says{F1} Novra! Men hvordan skal vi dog gøre det.

\says{F2} På samme måde som vi løser alle vores problemer.

\says{F1} Ej hvor smart.......... hvordan plejer vi at gøre?

\says{F2} Med bomber dit fjols

\says{F1} booiing,... krudtet og en lunte (begynder at gå ud af scenen)

\says{F2} Nej!! (F1 stopper) vi må jo ikke bruge åben ild i auditoriet (står med en løftet pegefinger ud mod publikum)

\says{F1} Kan vi så ikke lave en af de der smarte bomber, man kan ringe til?

\says{F2} har jeg da også tænkt på, men man må ikke have mobilen tændt under revyen.

\says{F1} Men, er der ikke også fysikkere herinde, dem vil vi da ikke bombe..

\says{F2} Suuk. Hvorfor tror du at der er nødudgange placeret nede bagved, i siderne og bag ved scenen (peger på udgangene mens)

\says{F1} Nå, men hvad skal vi så gøre???

\says{F2} Det er det jeg prøver at finde ud af... Tænker du det samme som jeg tænker!

\says{F1} Ja, det tror jeg nok, men hvordan finder vi en kvindelig datalog.

\says{F2} Nej dit fjols! Jeg har fundet løsningen til hvordan vi ødelægger matematikrevyen! Vi sidder bare og drikker os mega stive under hele revyen, og så komme med irriterende tilråb under hele 3. akt!

\says{F1} Årh yes mand, lad os gøre det!
\end{sketch}

\end{document}
