%% reconstructed from file: 'Matematik Revyen 2011.pdf'
\documentclass[a4paper,11pt]{article}

\usepackage{revy}
\usepackage[utf8]{inputenc}
\usepackage[T1]{fontenc}
\usepackage[danish]{babel}

\revyname{Matematikrevy}
\revyyear{2011}
% HUSK AT OPDATERE VERSIONSNUMMER
\version{0.2}
\eta{$3$ minutter}
\status{Ikke færdig}

\title{Multiple Choice Test}
\author{en forfatter}

\begin{document}
\maketitle

\begin{roles}
\role{I}[KØK] Instruktør
\role{J}[Ada] Journalist
\role{R}[Anna Munk] Rusvejleder
\end{roles}

\begin{sketch}
\says{J} (til publikum) Jeg står her med rusvejleder Conrad von Hufflebøl, som har været med til at udarbejde optagelseskravene på matematik, da der i år har været flere ansøgere, end der er plads til på studiet. Jeg har her en kopi af prøven, (henvendt til R) og Conrad; hvorfor er det vigtigt, at vide hvor god man er til norsk?

\says{R} Det er let nok at besvare, for norsk er essentielt for at kunne bestå MatIntro. For at fremme dette har vi også en studietur til Oslo efter blok 1.

\says{J} Og her står der "Kan du Kranen?". Hvorfor spørger I om det?

\says{R} Det er et trickspørgsmål, der er ingen der kan Kranen i forvejen. Vi bruger spørgsmålet til at afsløre fysikere og andre folk der allerede er droppet ud fra studiet én gang.

\says{J} Hvorfor spørger i om hvilken øl man foretrækker at drikke, når man tager på bar, med forslag som, GT, Fuglsang og den blå. det kan da på ingen måde være relevant for om man kan få lov til at gå på matematik eller ej.

\says{R} Det er det heller ikke, den eneste årsag til at spørgsmålet er med, er fordi Cafeen? meget gerne vil vide, hvilken øl de skal købe flest af til første fredagsbar.

\says{J} Nu er der jo flere drukrelaterede ting i den her test, Caféen?, Kranen og Norge. Sender det ikke et forkert signal?

\says{R} Nu har du jo også kun taget fat i en lille del af spørgsmålene. Du interesserer dig slet ikke for de spørgsmål, som handler om matematik. Vi spørger f.eks. om antallet af løsninger til en 5.-gradsligning.

\says{J} Men når nu 5.-gradsligningen er $x^5 = 0$ så er det vel ikke så svært.

\says{R} Jo, hvis man er Bio-Kemikere.

\says{J} Nå. Der er også et spørgsmål, der hedder: Er du glad for ordspil?

\says{R} Ja, her skal man svare: "Ja, jeg kan godt lide Scrabble".

\says{J} Okay. I har også spurgt ind til, om man er god til at huske adgangskoder; men hvorfor ligefrem spørge ind til det i en optagelsesprøve? Det har vel forhåbentlig ingen relevans for, om man kan blive optaget på studiet?

\says{R} Jo, det er meget relevant. Vi gider jo ikke spilde ressourcer på at hjælpe de studerende med at logge ind igen og igen. Derudover er det relevant, da unge studerende har meget få penge, og hvis du så ikke kan huske sin adgangskode, kan du ikke printe gratis, og det betyder altså, at du ikke afleverer dine opgaver, hvormed du ikke får lov til at gå til eksamen, og så dropper du ud. Derfor er det meget vigtigt at vide, hvordan de kommende studerende har det med adgangskoder.

\says{J} Nå, ja?! Nu til noget andet. Hvorfor i alverden er det et krav, at man kan lide Disney?

\says{R} Hvis man ikke kan lide Disney, så er man en særling, og sådan nogen gider vi ikke at have her.

\says{J} Nå. Og hvordan kan det være, at I har valgt at lave prøven som multiple choice, når nu der er spørgsmål som "kan du lide kage?"hvor fem af svarmulighederne er ja og det sidste er "Ja, men kun om fredagen"?

\says{R} Det skal jo heller ikke være for svært. Desuden er det simpelthen ikke muligt at lave kvantitativ analyse uden multiple choice-prøver.

\says{J} man kan da også lave kvantitativ analyse med ja/nej spørgsmål, det mener de ihvertfald i denne her test fra KUA; (... finder testen frem...) Og her er kravet for at komme ind at man skal man svare NEJ på samtlige spørgsmål for at få adgang. For eksempel \\

1. Kan du løse den her 2.-gradsligning? \\
4. Føler du dig generelt godt begavet? \\
5. Vil du gerne have et arbejde?

\says{R} Hvad ved humanister om kvantitativ analyse? Og skulle det i grunden ikke handle om vores test?

\says{J} Jo, men så har jeg et sidste spørgsmål. I skriver til sidst i jeres test: "Kan du godt lide rustur?"

\says{R} Ja, (hehe) det er fordi vi tager dem, der svarer nej og sender over til Fysik.

\scene{Lys ned}
\end{sketch}

\end{document}
