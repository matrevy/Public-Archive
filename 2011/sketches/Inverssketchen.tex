%% reconstructed from file: 'Matematik Revyen 2011.pdf'
\documentclass[a4paper,11pt]{article}

\usepackage{revy}
\usepackage[utf8]{inputenc}
\usepackage[T1]{fontenc}
\usepackage[danish]{babel}

\revyname{Matematikrevy}
\revyyear{2011}
% HUSK AT OPDATERE VERSIONSNUMMER
\version{0.1}
\eta{$1.5$ minutter}
\status{Ikke færdig}

\title{Inverssketchen}
\author{en forfatter}

\begin{document}
\maketitle

\begin{roles}
\role{I}[Maling] Instruktør
\role{M}[Kristian] Mand (læge)
\role{S}[Jenny] Kvinde
\end{roles}

\begin{props}
\prop{En indkøbspose}[]
\prop{Et lagen}[]
\prop{En lille vandpistol}[]
\end{props}

\begin{sketch}
\scene{M kommer hjem fra arbejde og så kører samtalen (Det skal siges rigtig sødt og kærligt)}

\says{M} Så er jeg smuttet, kælling.

\says{F} Ihh hvor dejligt. Jeg har virkelig gået og glædet mig til at du kom hjem.

\says{M} Det er jeg sur over at høre

\says{F} Huskede du at handle ind på vejen?

\says{M} Jeg glemte at få handlet ind. (smækker en fuld indkøbspose op på bordet)

\says{F} Nej hvor dejligt! Hvordan var det i øvrigt på arbejdet?

\says{M} Sådanehh... ad helvede til. Som det plejer. Der var en, der var lige ved at leve videre, men jeg fik gudskelov slået ham ihjel.

\says{F} Det var da godt skat. Du er altid så dygtig til dit arbejde.

\says{M} Ja, jeg er den bedste i verden.

\says{F} Ikke så beskeden skat. Er du sulten? Jeg har lavet stegt flæsk med persillesovs.

\says{M} Nej, jeg er slet ikke sulten. Det lyder vildt ulækkert. (mmmmh) Det stinker af helvede til.

\says{F} Det er glad for, at du siger. Så lad os da spise.

\says{M} Det er en dårlig ide.

\scene{De går over mod bordet. På vejen begynder M at tage F på røven}

\says{M} Du har sådan en klam, slatten røv, din so.

\says{F} (rrrauuuw) Tag mig, NU!

\says{M} Mmmhmm, hvor er du ulækker! Du har sådan nogle grimme bryster, som jeg hader at røre.

\scene{Et lagen kommer ind, og der lyses fra bagenden af scenen, så skuespillerne bliver til skyggebilleder}

\says{F} Tag mig hårdt!

\says{M} Åååh, jeg går nu!

\scene{Der sprøjtes med vandpistol}
\end{sketch}

\end{document}
