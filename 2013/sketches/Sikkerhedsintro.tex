\documentclass[a4paper,11pt]{article}

\usepackage{revy}
\usepackage[utf8]{inputenc}
\usepackage[T1]{fontenc}
\usepackage[danish]{babel}

\revyname{MatematikRevy}
\revyyear{2013}
\version{1.0}
\eta{$x$ minutter}
\status{udkast}


\title{Sikkerhedsintro}
\author{Søren, Jasmin \& Jakob}

\begin{document}
\maketitle

\begin{roles}
\role{X}[NB] Instruktør
\role{K}[William] Kaptajn
\role{S0}[Ane] Stewardesse
\role{S1}[Anne G] Stewardesse
\role{S2}[Christine] Stewardesse
\role{S3}[Cæcilia] Stewardesse
\role{S4}[Anna] Stewardesse
\role{S5}[Lilli] Stewardesse
\role{S6}[Mette] Stewardesse
\role{S7}[Nanna] Stewardesse
\role{S8}[Rikke] Stewardesse
\role{S9}[Signe] Stewardesse
\end{roles}


\begin{sketch}
\scene Lys op. K står på scenen.
\says{K} Godaften og velkommen i store UP1. \act{peger rundt}\\
Vi vil kort orientere \act{peger på ur} om proceduren i tilfælde af en formentlig katastrofe. \act{tager sig til hovedet} \\
Først og fremmest vil vi bede jer om at slukke mobiltelefonerne \act{klapper sig på lommen}, og ikke have åben ild i salen. \act{stirrer  på salen} \\
\scene To stewardesser kommer på scenen under næste replik.
\says{K} Den sikrer vi os lige at I forstod.
\says{K+Sa+Sb} Først og fremmest vil vi bede jer om at slukke mobiltelefonerne \act{klapper sig på lommen}, og ikke have åben ild i salen. \act{puster lys ud} \\
Skulle uheldet være ude, \act{tager sig til hovedet} når en eller anden vælger at sætte ild til stedet, \act{puster IKKE lys ud, men puster i en anden retning} beder vi jer bevæge jer langsomt og panisk mod nødudgangene, \act{peger på skilte under næste sætning} der er markeret med grønne exitskilte. \\
I tilfælde af længerevarende tørke, er baren i foyéren. \act{peger mod foyer} \\
\scene Yderligere to stewardesser kommer på scenen under næste replik; de to første peger tilbage på dem og velkommer dem på scenen.
\says{K} Den sikrer vi os lige at I forstod.
\says{K+Sa-d} I tilfælde af længerevarende tørke, er baren i foyéren. \act{peger mod foyer} \\
Skulle fysikerne have medbragt en bombe \act{viser bombe-tegn} eller en uran kerne \act{laver uendeligtegn}, så frygt ej, backstage er sikret \act{peger bagud} så skuespillerne nok skal overleve \act{peger på sig selv} og revyen fortsætter, med eller uden publikum \act{peger på publikum}... og band. \act{peger på band} \\
Apropos eksplosioner, vil vi orientere om, at skulle blæren være ved at sprænges \act{peger på blæren}, forefindes der ultra moderne toiletter... \act{laver træk og slip}
\scene Resten kommer ind under de næste to replikker. To fra hver dør i midtergangen, en fra under band og den sidste fra den modsatte side.
\says{K} Den sikrer vi os lige at I forstod.
\says{Sa-d} Ja, den sikrer vi os lige at I forstod.
\says{K+Sa-j}Apropos eksplosioner, vil vi orientere om, at skulle blæren være ved at sprænges \act{peger på blæren}, forefindes der ultra moderne toiletter \act{laver træk og slip}... \act{Kunstpause} på Kongens Nytorv. \act{peger mod Kongens Nytorv} \\
%Hvis der under forestillingen opstår mistanke om at din sidemand er kannibal, så sørg for at spise ham før han spiser dig.  \\
Hvis vi, mod al forventning, skulle opleve jordskælv på over 11 på richterskalaen \act{viser 11}, så sørg for at holde godt fast i jeres øl \act{holder om fiktiv øl}, så ryger den ikke nogen vegne \act{taber den}... \act{Kunstpause} Og vi har mindre rengøring imorgen. \act{mobber gulv}\\ \\
I tilfælde af tsunami i store UP1...
\scene Pigerne i salen kaster sig ned til siden, som var de skyllet væk, og bliver siddende nede under resten.
\scene K kaster armene mægtigt til siden, og pigerne på scenen trædder til siden som var de ramt af en bølge.
\says{K} ... klæd jer da af \act{K tager kasketten af, og sættern på hoved af den nærmeste pige} \\
Og påfør jer den under bordet placerede badehætte \act{K hiver en badehætte/et kondom frem og sætter den/det på hovedet}. \\
Kast dig i armene på nærmeste havfrue \act{kaster sig i armene på de fire piger, så han nu ligger vandret i luften}, og hold vejret. \\
\says{K+Sa-d} Vi ønsker alle en behagelig revy.

\scene Lys ned
\end{sketch}
\end{document}