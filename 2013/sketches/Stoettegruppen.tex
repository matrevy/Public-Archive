\documentclass[a4paper,11pt]{article}

\usepackage{revy}
\usepackage[utf8]{inputenc}
\usepackage[T1]{fontenc}
\usepackage[danish]{babel}

\revyname{MatematikRevy}
\revyyear{2013}
\version{1.0}
\eta{$x$ minutter}
\status{udkast}


\title{Støttegruppen}
\author{Beatrix, Mathias, Emil, Tobias \& Jakob}

\begin{document}
\maketitle

\begin{roles}
\role{X}[Jakob] Instruktør
\role{P}[Nanna] Psykolog 
\role{M}[Emil] Ustabil Lærer, Morten
\role{K}[Shake] Veteran Lærer, Kjeld
\role{T}[Pornokongen] Vred Lærer, Torben
\end{roles}

\begin{sketch}
\scene{Lys op.}
\says{P} Velkommen tilbage allesammen. Idag har vi et nyt medlem af gruppen.

\says{M} Hej, jeg hedder Morten.

\says{P, T \& K} Hej Morten.

\says{M} Jeg har undervist i C-niveau matematik i 1 år nu.

\says{P} Nåh Morten, fortæl os hvorfor du er her idag.

\says{M} Her på det sidste har jeg haft nogle problemer med nerverne. Så ledelsen mente jeg burde snakke med nogen om det.

\says{P} Ja, og det er jo derfor vi alle sammen er her. Så vi kan dele med gruppen. Hvad er det der trykker?

\says{M} Det er eleverne. De forstår ikke hvad jeg siger. Jeg prøver, og prøver, og prøver. De ved at minus og minus giver plus. Men -4 + -7 giver ikke 11. \act{Begynder at græde}

\says{T} Nu skal du ikke også begynde at tude. Du minder mig sgu' om mine elever.

\says{P}\act{Vender sig væk fra M og T} Kjeld, sidst diskuterede vi hvordan du skulle arbejde med at håndtere dine følelsesudsving. Hvordan er det gået med de åndedrætsteknikker vi øvede sidst?

\says{K} Det er gået rigtig godt. Når de rækker hånden op, så kigger jeg på dem og ånder tungt, så forsvinder hånden helt af sig selv. \act{Stort tandsmil}

\says{P} Ja.... \act{Taber sin kuglepen.} Årh, Kjeld gider du at række mig min kuglepen.

\says{K}[Glad] Ja. 
\scene{K giver ikke slip på kuglepenne så P er nødt til at bruge kraft for at få den uda f hænderne på ham.}

\says{P} Okay.. Morten, var der ikke noget bestemt du gerne vil tage op med gruppen i dag?

\says{M} Jo... Jeg kan ikke få dem til at lave deres afleveringer. 

\says{T}[Sammenbidt] Det har jeg prøvet før du tager bare et bat med til timen.

\says{P} Ej Torben det har vi snakket om. Det må du ikke.

\says{K}[Glad] Det er ikke noget problem. Ved du hvor de bor?

\says{M}[Grådkvalt] Ja....

\says{K} Okay, så skal du bare bruge en stor hund, femten liter tændvæske og en stormlighter. Det virker altid.

\says{P} Hvad skal hunden bruges til? 

 \says{K}[Glad] Den skal ofres til Satan...

\scene{T, P \& M kigger på K, og rykker synkront væk.}

\says{P} Men lad os arbejde videre med jer. Torben, jeg kan mærke at du har mange uløste problemer. Vi skal have skabt kontakt til den indre Torben. Lad os prøve lidt rollespil. Torben du er nu læreren og Morten er så en elev der har afleveret sin aflevering for sent.

\says{M}[Snøftende] Okay...

\says{T}[Råber] Hvordan kan du aflevere for sent?! Og så endda sådan en gang lort.

\says{M}[Tudende] Undskyld, unskyld, unskyld \act{Tuder ukontrollabelt}

\says{T}[Råber] Jeg sagde du skulle lade være med at tude! \act{Tager sit bat frem og taler til det} Kom så Bertha. \act{Gør sig klar til at slå Morten.}

\says{P} \act{Stopper Torben ved at skubbe battet væk} Nej, nej, nej, stop! \act{Forsøger at tage battet fra Torben, råber} Hvordan har i stadigvæk lov til at undervise?

\says{K}[Siger det åbenlyse] Der er jo lærermangel...

\end{sketch}
\end{document}