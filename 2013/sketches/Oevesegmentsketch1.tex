\documentclass[a4paper,11pt]{article}

\usepackage{revy}
\usepackage[utf8]{inputenc}
\usepackage[T1]{fontenc}
\usepackage[danish]{babel}

\revyname{MatematikRevy}
\revyyear{2013}
\version{1.0}
\eta{$x$ minutter}
\status{udkast}


\title{Øvesegment 1}
\author{}

\begin{document}
\maketitle

\begin{roles}
\role{X}[KØK] Instruktør
\role{H}[Søren] Henning (Soldat)
\role{M}[Ada] Monica (Monique) (Heks)
\role{T}[Therkel] Torben (Hund)
\role{D}[Ane] Desiré (Prinsesse)
\role{I}[Anne G] (Instruktør)
\end{roles}

\begin{sketch}
\scene{Lys op. }
\says{I} Hej alle sammen, jeg håber i har læst jeres replikker godt igennem. Jeg tænker at vi idag prøver for første gang at spille det hele igennem uden papir. Så jeg håber i har øvet jer og føler jer tilpas med jeres roller. Jeg har prøvet at fordele rollerne efter jeres personlighed. \act{Sagt til alle andre end M} Men gutter jeg skal lige snakke med Monica 

\says{M}[Retter ham] Monique

\says{I}[fortsat] Monique, en gang.

\scene{De andre går ud.}

\says{I} Så Monica

\says{M}[Retter ham] Monique

\says{I} Monique, hvad er det du har på?

\says{M} Hvad mener du? Det er mit kostume. Kan du li' det? Jeg tænkte det passede godt til min personlighed. 

\says{I} Jamen Monica

\says{M}[Retter ham] Monique

\says{I} Monique, du er jo en heks

\says{M} Ja og du er en kælling

\says{I} Nej, nej du spiller rollen som heksen

\says{M} Hvad siger du? Du må have husket forkert! I mit manus står der jeg er prinsessen.

\scene{D Kommer ind på scenen iført heksekostume}
\says{D} Så er jeg har. Er vi gået igang? 

\says{I} Desiré hvor har du været?

\says{D}[Lillepige glad, som om det er oplagt] Jeg har set Twilight.

\says{I} Men vi begyndte jo for 30 minutter siden

\says{D} Og filmen sluttede for fem minutter siden

\says{I} Og hvad i alverden er det du har på?

\says{D} Det er mit heksekostume, Monique sagde jeg var heksen.

\says{I} \act{Dyb indånding.} Ok, Desiré du er prinsessen, og Monica

\says{M}[Retter ham] Monique

\says{I}[Mærkbart irriteret] Monique, du spiller heksen. 

\scene{Tager deres hovedbeklædning og bytter rundt. Monique kæmper imod.} 

\says{I} Det er tydeligt at i ikke har øvet på de rigtige roller, så vi kan ikke øve scenen med heksen og soldaten. I to går hjem nu og øver jer til imorgen/næste akt. Nu tager vi scene 42 hvor soldaten møder hunden første gang. 

\scene{M og D går ud. H og S kommer ind. H er meget nervøs} 

\says{I} Så i går igang når i er klar. Og Henning, det er dig der begynder når du er klar.

\says{H}\act{ siger noget uforståeligt}

\says{I} Hvad siger du Henning?

\says{H} Hv.. hv... hv.. hvem er det der tramper på min bro?

\says{I} Henning, det er fyrtøjet vi spiller, ikke De 3 bukkebruse.

\says{H} Nåh Nej... Men er det så det græs som antiloperne spiser?

\says{I} Fyrtøjet, ikke løvernes konge.

\says{H}[Nervøst] Jamen er det så her jeg skal puste og pruste 

\says{I} Neeeeej, Henning

\says{H} Nååååh, er det den hvor jeg bliver til en svane?

\says{I} Neeeej. Fyrtøjet er den med ham soldaten der møder en heks, kravler ned i bunder af et træ og finder en stor hund på en kiste. Du siger "Nøj, sikken en stor hund".

\says{H}[Meget stille] Nøj, sikken en stor hund

\says{I} Lidt højere.

\says{H}[Meget stille] Nøj, sikken en stor hund

\says{I} Lidt højere. Du kan godt

\says{H}[Råber i skræk] NØJ, SIKKEN STOR HUND!

\says{I} Tæt nok... 

\scene{Stilhed}

\says{I} Så er det hundens replik

\says{T} \act{Stilhed}

\says{I} Torben, det er din replik

\says{T} Jeg holder en kunstpause. Lad mig lige samle mig engang

\says{I} Okay...

\says{T}[Højtideligt] Wow....

\says{I} Replikken er vov.

\says{T}\act{Spørgende} Wauw?

\says{I} Vov

\says{T} Wuf

\says{I} Vov

\says{T} Wruf

\says{I} Vov

\says{T} Vif

\says{I} Vov

\says{T} Vaf

\says{I} VOV!

\says{T} Vov? Wow, det ændrer hele min karakters opbygning, nu skal jeg lige samle mig engang. \act{Tager en dyb indånding} Okay, hvad er min motivation?

\says{I} Motivation? Du er en hund.

\says{T} Okay hvilken slags hund er jeg?

\says{I} Du er en stor hund, på en guldkiste.

\says{T} Er jeg en cockerspaniel? Jeg føler jeg er en cockerspaniel?

\says{I} En cockerspaniel siger ikke vov.

\says{T} Hvad så med en gran danoir? Nej det er måske for stort... Pudler... Pudler de er ret sådan aggressive. Jeg kunne godt være en puddel. Eller måske...

\says{I}[Afbryder] Ved du hvad, vi stopper her for idag. \act{Går ud fra scenen}

\says{T} \act{begynder at bevæge ud} Jeg kunne selvfølgelig også være en chiuaua, men det er måske for småt?... Hvad med Rottweiler? Rottweiler! Der var den kraftedme du. \act{Eftertænksomt mens han går ud} Men hvor meget skal jeg så salve?

\scene{lys ned}

\end{sketch}

\end{document}

sMeneskrækssketMh/sketMh med forskellige typer forfejlet skuespil

Instruktør der dobbler som suflør

Person der leverer samme replik i hver sMene

en blind, starter med ryggen til publikum

En der leverer alle sine replikker uden timing

udenlansk skuespiller som ingen forstår

En der tror han er i et andet stykke/leverer alle sine replikker som var det en anden genre (AMtion one-liners, reklam el. lign.)


''Vi har haft problemer med at de andre revyer har stillet brændende kors op uden for vores institut fordi vores revy er for sort.''


fraværende
blondie
dårlig dansk
tror personen er i en musical

læser i manuskriptet, læser forkert, spørger ind til manus 

fyrtårn 


Soldat - Henning 
SMenskræk
Glemmer replikker
Får sagt replikker fra et andet eventyr


Heks - 
Diva
Har klædt sig ud som prinsessen
Er fornærmet over hun ikke har fået rollen som prinsesse
'Jeg tænkte at hun kunne være en sådan lidt, prinsesse-agtig heks'
går i vrede da hun bliver prinsesse og rollen derefter Muttes


Hund(e) - 
Meget interesseret i sin rolles motivation
Har en replik... og den skal sat'me analyseres
Har en replik... læser den forkert
Leverer replikken forkert

Prinsesse - 
Blank
Blank
Blank
Går (måske fordi hun bare har planlagt noget andet)
Tror det er stykket 'fyrtårnet' (måske en rapunzel variant)

Konge - 
Sammen som hund(ene), er stand-in for en eller anden


Instruktør - 
Megaloman

standby me - standby plads

de to mest egocentrerede mennesker på date
