\documentclass[a4paper,11pt]{article}

\usepackage{revy}
\usepackage[utf8]{inputenc}
\usepackage[T1]{fontenc}
\usepackage[danish]{babel}

\revyname{MatematikRevy}
\revyyear{2013}
\version{1.0}
\eta{$n$ minutter}
\status{Version 3.0}


\title{Den Sorte Sketch}
\author{Manusgruppen}

\begin{document}
\maketitle

\begin{roles}
\role{X}[Jakob] Instruktør
\role{T1}[Anders] Revyst
\role{T2}[Jasper] Revyst
\role{T3}[Ane] Revyst
\role{T4}[Sanne] Revyst
\role{T5}[Søren] Revyst
\end{roles}

\begin{sketch}
%\says{T2}[Panisk] Sanne! Hvor skal jeg stå?!
%\says{T4} Sssh, stille!
%\says{T2} Jamen jeg kan ikke huske hvor jeg skal stå.
%\says{T1} Kan du så ikke lige hjælpe med det her akvarie?
\says{T1} Pssst! Sanne... Sanne... \act{Råber til sidst} Sanne!
\says{T4} Hvad!?
\says{T1} Kan du ikke give mig en hånd med akvariet her.
\says{T4} Kan du ikke lige selv holde det? Vi skal have de andre rekvisitter på scenen. \act{Går ind på scenen og sætter de to stole på plads} 
\scene{T2 følger efter med en palme}
\says{T2} Var det palmen vi skulle bruge?
\says{T4} Nej, Harpen.
\says{T5}[Glad] \act{Kommer ind med en kasse med knæklys} Kan vi bruge dem her? \act{bunden i kassen går i stykker og spilder knæklyst ud over scenen}. 
\scene{T1 løber bagefter og fejer op}
\says{T4} Nej!
\scene{T1\&3 bærer en kiste ind}
\says{T3} Vi kunne ikke finde harpen.
\says{T1} Men vi fandt den her kiste. Kan vi ikke bare bruge den?
\scene{Lyd af banken}
\says{T2}[Forvrenget og klynkende stemme] Hjælp! Jeg er ikke død!
\says{T4} Nej!
\says{T1} Det her akvarie er altså virkelig tungt.
\says{T2} Det har vi fattet!
\says{T5} Hold da kæft hvor er jeg fantastisk. Revyen ville være fortabt uden mig. 
\says{T1} Hvem sagde det?
\says{T5} De andre kan jo ikke finde ud af noget som helst.
\says{T3} Jeg tror Søren har glemt at slukke sin mikrofon
\scene{Lyden af en der tisser starter}
\says{T4} Nej, nej, nej, det sker bare ikke...
\scene{Tisselyden stopper halvvejs inde i sangen}
\says{T5}[Synger] Den her sang den er til dig
Åh Jan Philip Solovej
Når funktionerne de ændrer sig
Så ta'r vi det skridt for skridt 
\says{T4}[Mod backstage] Så få dog den mikrofon slukket!
\scene{Bandet går pludselig i gang med at spille. T2 løber ud på scenen bag ved T4.}
\says{T2}[På vej imod bandet] Stop, stop, stop! \act{Falder igennem scenelemmen} Aaaaaaaah!
\scene{Bandet stopper.}
\says{T5} Er du okay?
\says{T2} \act{ømmer sig} Ja, sådan nogenlunde... Hvorfor er lemmen åben?
\says{T1}[Afbryder anstrengt] Jeg kunne altså virkelig godt bruge en hånd med det her akvarie.
\says{T4} Ti nu stille med det akvarie din slapsvans! Og få den lem lukket! Aldrig har jeg set sådan en flok amatører! Man skulle tro i var fra bio-revyen. \act{fortsætter sviner}
\scene{Imens T4 taler dukker Giraffen op fra sidetæppet}
\says{T4} Og en sidste ting! \act{opdager giraffen og bevæger sig mod sidetæppet} Er det.... en giraf?
\scene{Spot på giraf}
\says{T3}[Glad] Ja!
\says{T5} Hvorfor har du taget en giraf med?
\says{T3} Gumle var ensom.
\says{T4} Hvad i al' verden skal vi med en giraf?! 
\says{T1} Kan den ikke hjælpe med mit akvarie? Det er altså virkelig tungt.
\says{T4}[Råber] Hvis du nævner det akvarie bare en gang til!
\says{T3} Ssssh ikke råbe, så bliver Simba bare utryg.
\says{T2} Simba? Hvem er Simba?
\scene{Løvebrøl}
\says{T4} En løve?! Har du taget en løve med?!
\says{T2}[Bange] Sanne...
\says{T4} Hvad nu?!
\says{T2}[Bange] Den er ikke i bur....
\scene{Lyd af løven der angriber}
\says{T2\&4} \act{Skriger i panik}
\says{T3} Lad være med at løbe. Han vil jo bare lege.
\says{T2\&4} \act{Skriger i panik}
\says{T1}[Panisk] Nej ikke den her vej!
\scene{Lyd af glas der smadrer}
\says{T1}[Tragisk] Mit akvarie!
\says{T4} Så hold dog kæft med det akvarie! 
\says{T2} Av, av, av. Der er jo glasskår over det hele... Vent hvor er løven?
\scene{Lyd af løven der angriber giraffen}
\says{T3} Stop Simba!
\scene{Lydene fortsætter. Giraffen 'dør'.} 
\says{T3}[Tragisk] Gumle! \act{Græder videre, mens de andre taler}
\says{T1}[Tragisk] Mit akvarie! \act{Taler videre om akvariet}
\says{T4} Se nu hvad i har gjort!
\says{T2} Hvad \emph{vi} har gjort?! 
\scene{T1\&3 græder, T2\&4 diskuterer med hinanden}
\says{T5}[Stolt/Højtideligt] Så er bordet klar!
\scene{Alle bliver stille}
\says{T3} Er der et bord med i den næste sketch?
\says{T1} Det har jeg ikke hørt.
\says{T2} Hvaffor et bord?
\says{T4} Nåh ja, 2 stole \emph{og} 1 bord.
\scene{Temaet til åbningen af Rumrejsen 2001 el. lign. begynder. Spot på bordet der kommer ind fra højre side. Det bæres ind og sættes midt på scenen klar til Nørden Niels 1}




%\scene{T2\&3 bærer en }
%\says{T2} Hvad med den her? 
%\says{T4} Nej!
%\says{T3} Jamen, det er jo Ernst...
%\says{T4} Tjae.... Måske kunne man... Nej! Vi skal ikke bruge en statue af Ernst! Ud med den!

%Bestilt pizza

%\says{T2} Hørte jeg rigtigt?
%\says{T1} Bordet er klar
%\says{T3} Hurrah
%\says{T4} Det var på tide

\end{sketch}


\end{document}