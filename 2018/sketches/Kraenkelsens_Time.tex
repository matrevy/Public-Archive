\documentclass[a4paper,11pt]{article}

\usepackage{revy}
\usepackage[utf8]{inputenc}
\usepackage[T1]{fontenc}
\usepackage[danish]{babel}

\revyname{Matematikrevy}
\revyyear{2018}
\version{2.1}
\eta{$2$ minutter}
\status{Færdig}

\title{Krænkelsens time}
\author{Michael '12, William '10, Kristian '10, Line '16.}

\begin{document}
\maketitle

\begin{roles}
\role{X}[Anne] Instruktør
\role{D}[Khalil] Darth Vader-agtig person fra administrationen
\role{F1}[Anders] Festgænger
\role{F2}[Daniel] Festgænger
\role{F3}[Felix] Festgænger
\role{R}[Mikkel] Mikkel
\role{K}[Taus] Festmand
\end{roles}

\begin{props}
\prop{Mexikansk fest}
\prop{Kasse med temating}
\prop{bord}
\end{props}

\begin{sketch}
\scene{Lys op.}

\scene{3 personer er ved at pynte op til fest med upassende mexikansk tema.
Der spilles mexikansk musik. (La cucharacha)}

\says{D} Stop, stop festen. Kan I ikke se at I bryder KU's nye retningslinjer?

\scene{D og hendes 2 kompagnoner tager sombreoerne af, og har business tøj/polititøj på idenunder.}

\scene{De kigger bedrøvet ned og derefter kigger op og siger}

\says{F1} Nej.

\says{F2} Men vi har jo både hvide og sorte sombreoer på.

\says{D} Prøv nu at se her. Kunne du fx. forestille dig en stor tyk mexikaner, ved navn Gonzales rent faktisk stå med sådan en kæmpe hat på? Det er tydeligvis en stereotypisering af den mexikanske kultur.

\says{F3} Nej, men jeg kender en stor tynd mexikaner, der hedder Gonzales, der...

\says{D}[Afbryder] HOLD KÆFT! 

\says{F3} Ja...

\says{D} Jeg skal videre, jeg har flere fester, som jeg skal lukke. Mikkel, du må tager over herfra.

\says{R} I kan jo nok se at...

\scene{K kommer ind med en trillebør/murespand med grønt vand i.}

\says{K} Såååååååååååå har jeg hentet GUACAMOLE gutter!!!!!!

\says{F1}  Altså kan vi ikke redde festen ved bare at lave lidt om på den?

\says{F2} Hvad er det EGENTLIGT, som er problemet?

\says{R} Problemet er jo, hvis en føler sig krænken og ekskluderet fra fælleskabet, som vedkommende er velkommen til, at være med i, fordi man er medstuderende på universitetet - så er det vigtigt at man får gjort de krænkende, som ikke er opmærksom på, at det virker krænkende, opmærksomme på, at det virkede krænkende. Så er jeg sikker på, de studerende ville sige: 'Så havde vi valgt et andet tema. Så havde vi da valgt Peter Plys.'\footnote{Citat herfra: https://www.facebook.com/Radio24syv/videos/1872658329518644/ \\ UzpfSTEwMjU0NDAxNjc1MTA3Njk6MTg1ODgwNjk1MDg0MDc0OQ/}

\says{K} Altså nu er vi jo studerende og jeg kan godt fortælle dig at vi ikke har tænkt os at alle bare skal klæde sig ud som Peter Plys.
Ja. Det ville sku' da være det mærkeligste tema, jeg nogensinde har hørt. Hvis du nu havde sagt, Peter Plys (ja, altså uden Bjørn) eller hundredemeterskoven, så nogle kunne være uglen og andre æslet. MEN æslet er måske for krænkende, er det. Laver det sjov med folk, der har depression eller hvad?

\says{R} Ja, hvis der var en deprimeret person med til festen, så kunne de måske føle sig krænkede, så er det... øhhhh... jo, et problem... og DET er det. Og det kan vi ikke have og så må festen altså aflyses.

\says{K} Altså deprimerede mennesker, de ligger jo derhjemme med bladet i badekarret og kommer slet ikke til vores fest, så der jo ikke noget problem.

\says{R} Aaaarh... Nu synes jeg lige at...

\says{F2} VI skal have et nyt tema. Jeg har snydt lidt og har et klart her. Der er hatte og lysestage og...

\scene{Hiver langsomt noget meget langt ud af en kasse.}

\says{K} Og en hel masse mønter.

\says{R} Ja, det fint. Det er ikke over grænsen.

\scene{Lys ned.}
\end{sketch}

\end{document}
