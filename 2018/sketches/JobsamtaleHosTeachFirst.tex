\documentclass[a4paper,11pt]{article}

\usepackage{revy}
\usepackage[utf8]{inputenc}
\usepackage[T1]{fontenc}
\usepackage[danish]{babel}

\revyname{Matematikrevy}
\revyyear{2018}
\version{1.0}
\eta{$1.5$ minutter}
\status{Færdig}

\title{Jobsamtale Hos Teach First}
\author{William '10}

\begin{document}
\maketitle

\begin{roles}
\role{X}[Lasse] Instruktør
\role{TF}[Anders] Teach First person
\role{L}[Taus] Fremtidig Lærer
\end{roles}

\begin{props}
\prop{Bord}
\prop{En skilt eller noget hvorpå det fremgår det er Teach First}
\end{props}

\begin{sketch}
\scene{Lys op.}

\says{A} Hej og velkommen til Teach First. Jeg kan forstå at du gerne vil være lærer.

\says{B} Ja. Det tror jeg bliver rigtig sjovt. Jeg elsker bare at arbejde med velopdragende unger.

\says{A} Ja, nu er de jo ikke ligefrem velopdragende. Det er faktisk lidt pointen at du bliver placeret der hvor de ikke er det.

\says{B} NÅH! Men det gør heller ikke så meget. Det vigtigste er at arbejde med dygtige unger.

\says{A} Altså. Nogen af dem er dygtige, men det er de fleste faktisk ikke. Jeg ville sige at halvdelen er under middel.

\says{B} Nåh! Men for mig handler det egentlig også mest om lysten. Bare de har virkelig meget lyst, så går det nok.

\says{A} Hvad var det at du ville undervise i?

\says{B} Matematik

\says{A} Ja, det kommer for manges vedkomne nok til at foregå mod deres vilje så.

\says{B} NÅH! Puha. Det kan godt være at børnene ikke opfører sig så pænt, og ikke kan finde ud af matematik, og at jeg skal lære dem det alligevel - allesammen. Og at hvis jeg så endelig finder en som har potentiale, så gider de ikke rigtig. MEEEeeeen... så må lønnen da også være derefter, ikke sandt. Det lyder som et meget velbetalt job. ER det DET???

\scene{Lys ned.}
\end{sketch}

\end{document}
