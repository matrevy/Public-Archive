\documentclass[a4paper,11pt]{article}

\usepackage{revy}
\usepackage[utf8]{inputenc}
\usepackage[T1]{fontenc}
\usepackage[danish]{babel}

\revyname{Matematikrevy}
\revyyear{2018}
\version{1.0}
\eta{$2$ minutter}
\status{Færdig}

\title{Ryger Matematik Ud?}
\author{Ulrik '14}

\begin{document}
\maketitle

\begin{roles}
\role{X}[Patrick] Instruktør
\role{R}[Daniel] Rus
\role{E}[Eigil] E-matematiker
\role{M1}[Taus] Matematiker 1
\role{M2}[Marius] Matematiker 2
\end{roles}

\begin{props}
\prop{Rekvisit}[Person, der skaffer]
\end{props}

\begin{sketch}
\scene{Lys op.}

\scene{R og E kommer ind på scenen, hvor M1 og M2 står i en tilrøget del af scenen og skriver matematik på en tavle, mens de hoster.}

\says{E} Ja... og så er der fysikbygningen, og kemi-bygningen... og ja... det var så nogenlunde hele HCØ.

\scene{E stiller sig tilfreds og kigger ud mod publikum, mens R peger over mod M1 og M2}

\says{R} Hvad så med dem derovre?

\scene{E giver et spjæt, mens de to fortsat står og hoster.}

\says{E} Ja... det er så matematikerne.

\scene{M1 og M2 kommer over med tavlen og rækker et kridt til R.}

\says{M1} Vil I have et hvæs?

\says{R}[Forvirret] Et hvæs?

\says{M2} Ja, nu skal du bare se. Du tager kridtet og tavlen, og så kører du bare udledninger derudaf, indtil du hakker, sprutter og, ja, hvæser. Det er selvfølgelig enormt \textit{*host*} vigtigt at inhalere.

\says{E}[Vredt] Ej hør lige her... Vi skal ikke have noget af jeres stads.

\says{M1} Jamen... han vil vel gerne være med i klubben.

\says{R} Det... kunne vel være okay at prøve sådan en enkelt gang.

\says{M2} Ja, ja... jeg startede også mest med at lave matematik til fester. Så nasser man lige et kridt af nogle venner, og så er det sådan en dejligt social oplevelse.

\says{E} Kom nu Rusmus. Du har ikke lyst til at være sådan én, der står med de andre matematikere og skriver på tavler, mens alle de andre hygger sig og danser, vel?

\says{R} Tja... jeg er både ung OG påvirkelig.

\says{M1} Ja, og sikkert stresset i hverdagen.

\says{R} Ja! Hele tiden!

\says{M2} Jamen... det dæmper stressniveauet sådan at få noget stimulans gennem hånden. Og der er intet bedre end at fejre et veloverstået samleje med et enkelt teorem.

\says{E}[Meget vredt] Nej, nu må det altså. Rusmus. Matematik er en styg og væmmelig ting, der både forværrer og forkorter folks liv, og så er det skadeligt for andre.

\says{M1} Hør her! Det er altså ikke rart sådan at blive forfulgt for éns livsstilsvalg. Det er folk som dig, der er skyld i, at det ikke er socialt acceptabelt at lave matematik i det offentlige rum!

\says{E} Jamen... der findes da også et alternativ, der ikke gør én mærkelig oven i hovedet og slet ikke er skadeligt for din omverden.

\says{M1+M2} Ahva?

\scene{E hiver en computer frem.}

\says{E} Har I aldrig hørt om smartboards?

\scene{Lys ned.}
\end{sketch}

\end{document}
