\documentclass[a4paper,11pt]{article}

\usepackage{revy}
\usepackage[utf8]{inputenc}
\usepackage[T1]{fontenc}
\usepackage[danish]{babel}

\revyname{Matematikrevy}
\revyyear{2018}
\version{1.0}
\eta{$0.5$ minutter}
\status{Færdig}

\title{Folk Du Helst Ikke Vil Eksamineres Af 3}
\author{Ulrik '14, Freja '15, Chris '17, Sofie '15}

\begin{document}
\maketitle

\begin{roles}
\role{X}[Lasse] Instruktør
\role{Sp}[Taus] Speaker
\role{E}[Marius] Eksaminand
\role{G}[Felix] Gandalf Den Grå
\end{roles}

\begin{props}
\prop{Rekvisit}[Person, der skaffer]
\end{props}

\begin{sketch}
\says{Sp} Matematikrevyen præsenterer: Folk, du helst ikke vil eksamineres af. Episode 3: Gandalf Den Grå.

\scene{Lys op.}

\scene{Eksaminanden styrter begejstret ind på scenen og svinger med sit notepapir.}

\says{E} Så! Nu går det fucking endelig, og det tog kun en hel revy. Hovedsætningerne for kontinuerte funktioner. Boom! Det bli'r ikke nemmere.

\scene{Gandalf træder ind i lokalet.}

\says{E} Arh... fuck!

\scene{Lys ned.}
\end{sketch}

\end{document}
