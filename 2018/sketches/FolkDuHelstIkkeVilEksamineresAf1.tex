\documentclass[a4paper,11pt]{article}

\usepackage{revy}
\usepackage[utf8]{inputenc}
\usepackage[T1]{fontenc}
\usepackage[danish]{babel}

\revyname{Matematikrevy}
\revyyear{2018}
\version{1.0}
\eta{$1.5$ minutter}
\status{Færdig}

\title{Folk Du Helst Ikke Vil Eksamineres Af 1}
\author{Ulrik '14, Freja '15, Chris '17, Sofie '15}

\begin{document}
\maketitle

\begin{roles}
\role{X}[Lasse] Instruktør
\role{Sp}[Taus] Speaker
\role{E}[Marius] Eksaminand
\role{M}[Maja] Mor
\role{F}[Khalil] Far
\end{roles}

\begin{props}
\prop{Rekvisit}[Person, der skaffer]
\end{props}

\begin{sketch}
\says{Sp} Matematikrevyen præsenterer: Folk, du helst ikke vil eksamineres af. Episode 1: Mor og Far.

\scene{Lys op.}

\scene{Eksaminanden træder ind i lokalet og hilser på eksaminator og censor. For sent går det op for eksaminanden, at de er mor og far.}

\says{F} Ej skat, det dér håndtryk er altså for slapt. Tænk på din fremtoning. Kom se skat, så skal far nok hjælpe med holdningen.

\scene{Far kommer op og prøver at ranke ryggen på eksaminanden.}

\says{E} Ej, far-ar. Hvad laver I her?

\says{M} Jamen, lille skat, vi skal da eksaminere dig.

\says{E} Det lyder ikke helt lovligt.

\says{M} Arh, pjat med dig. Mor har snakket med kontoret, så den er bare i vinkel. Det er ligesom dengang, du kun fik 10 med pil op i billedkunst.

\says{E} Nå, men hvis far gider sætte sig... så kan jeg vel... gå i gang.

\says{F} Nu er det ikke far til eksamen. Så er jeg Jørgen, ikke også skat. Du kan ikke være for familiær med dem, der skal bedømme dig.

\says{E}[Irriteret] Klart, nå, men lad os så...

\says{F} Ej skat, du mumler altså for meget. Tal tydeligt. Rank ryg. Du er ikke en baby længere vel.

\scene{Far går op og demonstrerer ved at give et mægtigt grin, hvorefter han sætter sig ned igen.}

\says{E}[Sukkende] Nå, men videre til min eksamen...

\scene{Mor går op og støver eksaminandens bukser af.}

\says{M} Nej, du bliver simpelthen nødt til at passe på med alt det kridtstøv. Du kunne få astma, mildtbrand eller beskidte bukser. Det ville ikke være første gang du smadrede en vaskemaskine, ik' os'?

\says{E}[Rullende med øjnenene] Nå, men jeg har altså trukket Greens Formel.

\says{M} Ej skat. Din far og jeg synes altså, at du øvede Den Generaliserede Kæderegel meget bedre. Syn's du ik', han skal lave den, Jørgen?

\says{F} Jo, og så fra toppen: Med fast stemme. Rank ryg. Øjenkontakt. Gør din far stolt.

\says{E} Men far...

\says{F} Jørgen, ik' skat?

\says{M} Ja, du skal holde tonen, ik' skat? Det nytter jo ikke noget, at folk tror, du er en forkælet unge, der bare tror, at du kan rende og te dig, som det passer dig. Mor og far er der jo ikke til at rydde op for dig vel.

\says{E}[Vredt] ...Så tager jeg den vel bare på re-eksaminen...

\scene{Eksaminanden forlader lokalet.}

%\says{M} Jeg synes, det var så fint. Synes du ikke, Jørgen? Lige til 10 pil op.

\scene{Lys ned.}
\end{sketch}

\end{document}
