\documentclass[a4paper,11pt]{article}

\usepackage{revy}
\usepackage[utf8]{inputenc}
\usepackage[T1]{fontenc}
\usepackage[danish]{babel}

\revyname{Matematikrevy}
\revyyear{2018}
\version{2.0}
\eta{$2$ minutter}
\status{Færdig}

\title{Nu Kan Det Være Nok}
\author{Ulrik '14}

\begin{document}
\maketitle

\begin{roles}
\role{X}[Anne] Instruktør
\role{F}[Felix] En tilfældig sur halvdårligt karikeret forsker
\role{S}[Anders] Sekretariat-agtig person
\role{Sp}[Daniel] Speaker
\end{roles}

\begin{props}
\prop{Rekvisit}[Person, der skaffer]
\end{props}

\begin{sketch}
\scene{Lys op.}

\scene{Vi ser en scene, hvor S sidder og passer sit arbejde, ganske som vedkommende plejer, mens F kommer stormende ind. Han er arrig.}

\says{F}[Arrigt] Nu kan det være nok. Meget må man bydes i denne eksistens, men dette var dog dråben!

\scene{S kigger noget uimponeret op.}

\says{S} Hvad er det denne gang?

\says{F} Altså... man skal vel have sin kaffe.

\says{S} Det skal man vel.

\says{F} Men det kan man altså ikke!

\says{S} Nå, hvorfor så ikke det?

\says{F} Fordi de studerende jo også skal have kaffe?

\says{S} Er det ikke bare et spørgsmål om at vente et minut?

\says{F} Men jeg skal altså have min kaffe nu!

\says{S} Okay, okay. Jeg løser det.

\scene{S sidder stilfærdigt og skriver. F venter 1 sekundt og siger:}

\says{F} Okay... hvad nu, hvis... vi sagde, at det kun var det videnskabelige personale, der måtte anvende kaffemaskinerne?

\says{S} Beklager, men I kan altså ikke få alle kaffemaskinerne.

\says{F} Men der render rusvejledere og andet skidtfolk rundt på fjerde og skaber køer af hidtil usete størrelsesordener.

\says{S} Det er store ord fra en mand, der ikke tror på tal større end 7.

\says{F}[Arrigt] Det her er alvor!

\says{S} Okay, okay... så kan I få den ene kaffemaskine helt for jer selv. Hvad siger du til det?

\says{F} Glimrende

\scene{F forlader scenen og lyset dæmpes, mens speakerstemmen træder til igen.}

\says{Sp} Og det var glimrende! Men mere vil have mere, og man skal aldrig række Fanden selv en lillefinger.

\scene{Lyset begynder at gå op igen, mens F bakker tilbage ind på kontoret, hvor S atter har givet sig til at arbejde.}

\scene{F rømmer sig uskyldigt, mens han går over scenen.}

\says{S}[Irriteret] Hvad er det så nu?

\says{F} Jo... de studerende kan jo booke mødelokalet på fjerde. Det er kun et enkelt lokale, og jeg synes nu engang, det er mest passende, hvis det er de ansatte, der styrer det.

\says{S}[Sukker] Det kan vi vel godt sige. Nå... Kan vi så få noget arbejdsro. Jeg har arbejde, der skal laves.

\scene{F lunter lige så stille ud af scenen, men midt som han skal til at gå ud af scenen, vender han sig om.}

\says{F} Men...

\says{S}[Vredt] Så kom dog med det.

\says{F}[Uskyldigt] Jo... ser du... jeg tænkte... de studerende fylder også forfærdeligt meget i diverse kantiner.

\says{S} Siger du, at I gerne vil have kantinerne for jer selv?

\says{F} Altså... hvis det kan lade sig gøre.

\says{S} Arh... de studerende skal altså også have en kantine.

\says{F}[Begejstret] De kan få den lige her på HCØ!

\says{S} En rigtig kantine.

\scene{Lys ned.}
\end{sketch}

\end{document}
