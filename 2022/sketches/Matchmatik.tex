\documentclass[a4paper,11pt]{article}

\usepackage{revy}
\usepackage[utf8]{inputenc}
\usepackage[T1]{fontenc}
\usepackage[danish]{babel}

\revyname{Matematikrevy}
\revyyear{2022}
\version{1.0}
\eta{$3$ minutter}
\status{Færdig}

\title{MatchMatik}
\author{Rikke '17, Nina '16, Line '16, Anna '15, Sommer '17}

\begin{document}
\maketitle

\begin{roles}
\role{X}[MaWeK] Instruktør
\role{P1}[Niklas] Ivrig person på MatchMatik
\role{P2}[Simone] Ny person på MatchMatik
\role{V}[Sommer] Voiceover/profil
\role{P2}[Baldur] Profil
\role{P3}[Nanna] Profil
\role{P4}[Stine] Profil
\end{roles}

\begin{props}
\prop{Kæmpe mobilcover}
\prop{2 stole}
\end{props}


\begin{sketch}
\scene{P1 og P2 står i hver sin side af scenen. Lys er nede}

\says{V} En gruppe dataloger mener at have øjnet en guldgrube, da de lagde mærke til hvor meget incest der foregår på Matematikstudiet. De har nu lanceret app'en MatchMatik med sloganet ``Er dit datingliv tørt? Date en nørd''. Meget tyder på, at de havde ret, da app'en allerede efter en uge har fået over 2000 brugere, hvilket underligt nok er flere end der i øjeblikket er indskrevet på Matematik.

\scene{Spot på personen til venstre på scenen der står og swiper i appen på sin mobil, mens et billede af appen dukker op på lærredet. Må gerne være en Tinder-efterligning, hvor profilerne til gengæld indeholder ting som fx årgang, måske om man er Ba, Mcs, eller phd. Man kan markere sine Interesser: Algebra, Analyse, geometri osv. På profilerne står der desuden en matematisk scorereplik af en art}

\scene{P1 sidder henslængt på en stol. P1 hiver sin telefon frem, og App'en kommer op på AV'en. P1 kigger på den første profil}

\says{P1} ``Jeg er som pi, virkelig lang og kan blive ved for evigt.'' Nååå, \act{Smiler sleskt} det må da komme an på en prøve.

\scene{Spot slukker og nyt spot kommer på P2. AV'en er nu P2's nye profil}

\says{P2} Oooog færdig! Det var da ikke så svært at sætte en profil op. Lad os se hvad vi har her.

\scene{AV skifter nu til nye profiler}

\says{P2} ``Er du en maximum likelihood estimator, for du er løsningen til min scoreligning'' Statistik? Ellers tak!

\scene{Skift over til P1}

\says{P1} \act{Begejstret} ``Pascal du med mig hjem?'' Ja, med de der kurver, kan det kun gå for langsomt!

\scene{Skift}

\says{P2} \act{Glad} Ej, han/hun ser da sød ud! \act{Skifter udtryk til skepsis} ``Hvis du har lyst, så kan jeg bevise for dig hvorfor det er vigtigt om man boller forfra eller bagfra.'' Ooooog nej tak igen

\scene{Skift}

\says{P1} ``Dine biceps er pæne, vil du se mit domæne?'' Er vild med en der er så direkte!

\scene{Skift}

\says{P2} ``Swipe til højre hvis du vil hjem og se min potensfunktion'' Det vidst en impotenspunktion!

\scene{Skift}

\says{P1} ``Hvis dit ansigt var et stationært punkt, så ville det være et saddelpunkt, for jeg har lyst til at sætte mig på det'' Ej du havde mig ellers i den første halvdel!

\scene{Skift}

\says{P2} ``Skal vi lege brøker sammen? Du kan være nævneren, for jeg kan bedst lide at være øverst'' Det eneste sted du kommer til at være øverst, er på min liste over folk jeg aldrig vil snakke med!

\scene{Skift}

\says{P1} ``Jeg havde lyst til at knep dig i går, jeg har lyst til at knep dig idag, så per induktion har jeg lyst til at kneppe dig igen i morgen'' \\
Friskt, men en skam det tydeligvis er en datalog

\scene{Skift}

\says{P2} ``$a^2 + b^2$ skulle vi to til at se hinanden?'' \\
Ej den er faktisk meget sød, og han/hun ser da faktisk også ret sød ud (Swiper til højre)

\scene{Skift}

\says{P1} Wow! Et match! Og hvaaad? Han/hun er sgu da mega lækker!

\scene{I stedet for skiftende spot, kommer der nu lys op på scenen, og P1 og P2 ser at de sidder ved siden af hinanden. De ser forelsket på hinanden og rejser sig og løber i armene på hinanden}

\scene{Lys ned}
\end{sketch}

\end{document}
