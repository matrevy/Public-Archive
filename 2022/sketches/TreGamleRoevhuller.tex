\documentclass[a4paper,11pt]{article}

\usepackage{revy}
\usepackage[utf8]{inputenc}
\usepackage[T1]{fontenc}
\usepackage[danish]{babel}

\revyname{Matematikrevy}
\revyyear{2022}
\version{1.0}
\eta{$3.5$ minutter}
\status{Færdig}

\title{Tre gamle røvhuller}
\author{Villads '19}

\begin{document}
\maketitle

\begin{roles}
\role{X}[Lise] Instruktør
\role{A}[Niklas] Gammelt røvhul 1
\role{B}[Peter] Gammelt røvhul 2
\role{C}[Thais] Gammelt røvhul 3
\end{roles}

\begin{props}
\prop{rekvisit} Tre stole
\prop{rekvisit} Et bord
\prop{rekvisit} En flaske bobler og glas
\end{props}


\begin{sketch}
\scene{Tre gamle røvhuller sidder i deres nystrøgede IT-konsulentskjorter, nyder et glas bobler og snakker om gamle dage}

\says{A} Ahhh, venner! Selvom vi efterhånden betaler alt det vi tjener tilbage i topskat, kan man nu godt unde sig en god flaske en gang i mellem!

\says{B} Ja, det kunne vi ikke engang have drømt om i vores fattige dage som matematikstuderende!

\says{A} Men det var nu tider alligevel! Så meget svær matematik vi plejede at lave! Ærgerligt at studiet ikke er det samme, som det var dengang!

\scene{Lidt forarget:}

\says{C} Ja! Har i hørt at de flyttede algebra 1 til andet år?

\says{B} En skam. Det kursus hører til på første år!

\says{C} Ja! Og da vi havde MatIntro var der ikke noget der hed terninger til MC!

\says{B} Og ingen multiple choice by night!

\scene{Pralende:}

\says{A} Dengang \emph{jeg} var rus var algebra 1 faktisk det første kursus man havde. \act{Lidt stolt} Men vi brokkede og ikke!

\says{C} Og i stedet for 70 timer sagde Ian at vi skulle studere 170 timer om ugen for at have en chance for at bestå!

\says{A} Og det passede!

\says{B} Pfff... Da jeg studerede var der slet ikke noget algebra 1. Vi startede lige på og hårdt med algebra 2. 1’eren var henvist til selvstudium!

\says{C} Ja, og det var kun det ene af de to kurser man havde!

\says{B} To kurser ad gangen?! Jeg havde 3!

\says{A} Men da jeg var rus havde man algebra 1 og analyse 0 samtidig - i blok 1 vel og mærke!

\says{C} Ha! Dengang jeg var studerende var analyse 0 slet ikke opfundet endnu! Faktisk regnede de med at vi kunne materialet fra gymnasiet!

\says{B} Hvad sker der egentlig for at operatoralgebra er ikke engang et obligatorisk kursus længere! Kan man være kandidat i matematik uden?

\says{C} I mine unge dage var det et bachelorkursus!

\says{A} Jeg husker det nu som værende et optagelseskrav?

\says{B} Ja alting er blevet så meget nemmere! Jeg plejede at have timer fra 8-17 alle dage, for man skulle overnormere og have mindst to instruktorater. Kun i oplæsningsugen kunne man sove lidt længe! Men vi var tilfredse!

\scene{Tydeligt chokeret}

\says{C} Plejede I at have oplæsningsuger? Min timer plejede at starte kl. 6:30 om morgenen 8 dage om ugen og eksamen var aller senest i kursusuge 7! Og til alle skriftlige eksaminer skulle vi skrive vores besvarelser i LaTeX med bind for øjnene og hænderne på ryggen mens forelæser stod og råbte af os!

\says{A} Luksus! Da jeg var studerende var alle eksaminer i mellemugen - FØR kurset startede. Der var afleveringer hver dag som skulle bestås med fuld point for at måtte gå til eksamen. Timerne startede klokken 3:00 om morgenen og der var mødepligt! Alle eksaminer var mundtlige og Ernst ville slagte os, hver og én. Og når vi fik -3 ville han danse rundt i eksamenslokalet og synge hallelujah!

\scene{De kigger på hinanden og nikker annerkendende}

\says{B} Og prøv nu at fortælle alt det her til de nye russer! Vi de tro på os?

\says{Alle i munden på hinanden} Nej! Nej! Nej! Nej!
\end{sketch}

\end{document}
