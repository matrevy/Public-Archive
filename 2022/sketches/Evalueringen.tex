\documentclass[a4paper,11pt]{article}

\usepackage{revy}
\usepackage[utf8]{inputenc}
\usepackage[T1]{fontenc}
\usepackage[danish]{babel}

\revyname{Matematikrevy}
\revyyear{2022}
\version{1.0}
\eta{$4$ minutter}
\status{Færdig}

\title{Evalueringen}
\author{Stine Langhede ’18}

\begin{document}
\maketitle

\begin{roles}
\role{X}[Rune] Instruktør
\role{E}[Stine] Evalueringen
\role{A}[Rasmus] Ung ny rus
\role{B}[Baldur] Ældre rus
\role{C}[Viktor] Kandidatstuderende
\end{roles}

\begin{props}
\prop{Evalueringskostume}[Stine har det på]
\prop{Bord}
\prop{3 x stole}
\end{props}


\begin{sketch}
\scene{Lys op. På scenen står Evalueringen, som ser irriterende ud i et evaluerings-kostume, hvor der står "Evalueringen"stort hen over brystet. Evalueringen står i midten af scenen og blokerer for det bord og de stole der står i den ene side (kantinen). En ung ny rus kommer ind for at komme hen til kantinen, men bliver blokeret.}

\says{E} Hey hov, du der rus, du kommer ikke bare sådan lige forbi mig!

\says{A}[Nervøst] Jamen, hvem er du?

\says{E}[Stolt og med store armbevægelser] JEG er studiestartsprøven! Og du skal igennem mig, for at komme videre i dit studie!

\says{A} Men.. Hvorfor?

\says{E} Fordi ellers DUMPER du!

\says{A} Nåh.. Ja, okay.

\scene{A står klar og Evalueringen trækker begejstret vejret dybt ind, og stiller det første spørgsmål.}

\says{E} Føler du dig i høj grad taget godt imod på uddannelsen?

\scene{Evaluering laver bevægelser og ansigtsudtryk til svarmulighederne.}

\says{E} I høj grad, i nogen grad, i mindre grad, slet ikke?

\says{A} Øhm, i høj grad!

\says{E} Føler du dig klar til at begynde på uddannelsen? I høj grad, i nogen grad, i mindre grad, slet ikke?

\says{A}[Nervøst] Øhm, i høj grad?

\scene{E løfter øjenbrynene og kigger på den unge rus.}

\says{A}[Indrømmende] I mindre grad..

\says{E} Deltog du i introugen? Deltog i det hele, deltog i det meste, deltog kun i lidt af det, deltog slet ikke.

\says{A}[Begejstret] Deltog i det hele!

\scene{E trækker vejret ind for stille næste spørgsmål, men A afbryder.}

\says{A} Er der egentlig mange spørgsmål tilbage, ville gerne ind og lave min aflevering?

\says{E}[Smiler lumsk] Nejnej, kun lidt endnu..

\scene{Lyset går kort ned, og på AV står "2 timer senere". Lyset går op, og den unge rus ser meget træt ud.}

\says{E} Har du oplevet aktiviteter på rusturen, som var ubehagelige eller grænseoverskridende for dig?

\says{A}[Udmattet] Nej, alt var altså super fint!

\says{E}[Glad] Det var bare det.

\scene{E gør plads til at komme forbi. Den unge rus ser lettet ud og går hen og sætter sig ved bordet, hvor personen tager sine bøger frem, om begynder at studere. Samtidig kommer en ældre rus ind og bliver også blokeret.}

\says{B} Undskyld, må jeg lige komme forbi.

\says{E}[Stolt og med store armbevægelser] JEG er studiestartsprøven! Og du skal igennem mig, for at komme videre i dit studie!

\says{B} Ah okay, så kom bare med det.

\says{E} Føler du dig i høj grad taget godt imod på uddannelsen? I høj grad, i nogen grad, i mindre grad, slet ikke?

\says{B}[Lidt ligeglad] I nogen grad.

\says{E} Føler du dig klar til at begynde på uddannelsen? I høj grad, i nogen grad, i mindre grad, slet ikke?

\says{B}[Lidt ligeglad] I nogen grad.

\says{E} Deltog du i introugen? Deltog i det hele, deltog i det meste, deltog kun i lidt af det, deltog slet ikke.

\says{B} Deltog slet ikke.

\says{E}[Chokeret] Hvorfor ikke?

\says{B} Øøh, jamen jeg har et arbejde, så jeg havde ikke tid..

\says{E}[Løfter øjenbrynene] Nåh.. Okay.

\scene{E trækker vejret ind og stiller næste spørgsmål.}

\says{E} Følte du dig godt taget imod i introugen?

\says{B}[Forvirret] Jamen.. Jeg var der jo ikke.

\says{E} Hvilke aktiviteter var særligt gode i introugen?

\scene{B stirrer irriteret på E og vifter med hånden for at få E til at gå videre til næste spørgsmål.}

\says{E}[Med et irriterende blink i øjet] Hovhov, du kan ikke komme videre, før du har sagt noget til spørgsmålet.

\says{B}[Irriteret] Øhhm.. Punktum!

\scene{E ser lidt irriteret ud, men begynder at stille næste spørgsmål.}

\says{E} Hvad synes du om måden..

\says{B}[Afbryder] Punktum!

\scene{E prøver at stille flere spørgsmål, men bliver afbrudt hurtigere og hurtigere, og når tilsidst kun at trække vejret ind, før der bliver sagt punktum. E ser mere og mere frustreret ud, men kan ikke gøre noget ved det.}

\says{B} Punktum, punktum, punktum, punktum!

\scene{E har ikke flere spørgsmål, men tager surt hænderne i siden. E stiller sig tilbage og lader en tilfreds B komme forbi. B sætter sig også og studerer. C kommer ind og prøver at komme forbi E, men blokerer irriterende.}

\says{C}[Irriteret] Altså hvad laver du?

\says{E}[Stolt og med store armbevægelser] JEG er studiestartsprøven! Og du skal igennem mig, for at...

\says{C} Ja, det ved jeg skuda godt - jeg tog jo studiestartsprøven for 3 år siden, da jeg var rus! Så gider du lige, at lade mig komme forbi?

\says{E}[Smiler lumsk] Men du er jo lige startet på kandidaten, ikke?

\says{C}[Bange] Jo?..

\scene{E trækker vejret dyybt ind og stiller første spørgsmål.}

\says{E} Føler du dig i høj grad taget godt imod på uddannelsen? I høj grad, i nogen grad, i mindre grad, slet ikke?

\says{C}[Truende] Ej, det gider jeg fandme ikke igennem igen, flyt dig!

\scene{E har ikke lyst til at flytte sig, men C bliver mere truende og ligner én der skal til at angribe, så E løber tilsidst bange væk.}

\says{A} Hey, se! Løbende evaluering!
\end{sketch}

\end{document}
