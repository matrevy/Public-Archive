\documentclass[a4paper,11pt]{article}

\usepackage{revy}
\usepackage[utf8]{inputenc}
\usepackage[T1]{fontenc}
\usepackage[danish]{babel}

\revyname{Matematikrevy}
\revyyear{2022}
\version{1.0}
\eta{$3$ minutter}
\status{Færdig}

\title{Træk et spørgsmål}
\author{Sommer '17}

\begin{document}
\maketitle

\begin{roles}
\role{X}[Lise] Instruktør
\role{E}[Sommer] Eksaminator
\role{S}[Baldur] Studerende
\role{C}[Simone] Censor
\end{roles}

% \begin{props}
% \prop{Rekvisit}[Person, der skaffer]
% \end{props}


\begin{sketch}
\scene{Eksamenssituation}

\says{E} Hjertelig velkommen til eksamen i Analyse0. Om lidt skal du trække et spørgsmål, så vil du have en halv times forberedelsestid, for så at fremlægge i en halv time inklusivt votering. Er det forstået?

\says{S} Selvfølgelig! Jeg er så klar, det bliver så nemt det her!

\says{E} Du virker meget sikker.

\says{S} Lad mig bare sige det sådan her: Har du nogensinde hørt om Euler? Newton? Galois?

\says{E} Ja?

\says{S} Idioter

\says{E} Wow \act{Kunstpause} I så fald, så værsgo at trække et spørgsmål \act{peger på bordet}

\says{S} Jamen det er så simpelt, alt jeg behøver at gøre er at benytte hvad jeg ved om dig, til at konkludere hvor du har placeret spørgsmålet om Hovedsætninger. En klog eksaminator ville have lagt spørgsmålet så langt væk fra en rus som muligt!

\says{E} Så du har truffet din beslutning?

\says{S} Ha! Jeg er ikke engang begyndt! Hvor var jeg?

\says{E} Rus?

\says{S} Når ja! Som alle ved, så tænker rus ikke logisk og da Analyse0 er et førsteårs kursus, så antager du sikkert at jeg ligeså tænker uklart!

\says{E} Nej for- \act{Bliver afbrudt}

\says{S} \act{Afbryder} Men jeg er jo 2. års studerende!

\says{E} Ja jeg dumpede dig sidste år

\says{S} Bevares, men ikke desto mindre er jeg blevet det klogere! Men du ved jo godt jeg er 2. års, så du har tydeligvis heller ikke lagt Hovedsætninger tættest på mig!

\says{C} Nu trækker du bare tiden

\says{S} Det kunne du lide! Jeg ved du er en sød eksaminator, fordi du bød mig hjertelig velkommen i dag, så du ville ikke have lagt Hovedsætninger et tarveligt sted. Samtidigt ved jeg at du er en streng eksaminator, da du dumpede mig sidste år, så du har nok ikke lagt det et sted jeg nemt ville vælge.

\says{E} Hvis du tror jeg røber noget om hvor spørgsmålene ligger, så kan du godt droppe det, det kommer ikke til at virke.

\says{S} Jamen det har virket! Du har allerede røbet alt! Jeg ved hvor spørgsmålet om Hovedsætninger er!

\says{E} Så træk et spørgsmål

\says{S} Selvfølgelig \act{Rykker sig tættere på bordet, men peger pludsligt væk} SE! ET FLYVENDE EGERN!

\scene{Eksaminator og Censor kigger i retningen der bliver peget, og den studerende bytter rundt på 2 af spørgsmålene}

\says{E} Hvad? Jeg kan ikke se noget?

\says{S} \act{En smule grinende} Nå, det har nok bare været et normalt egern der hoppede ned fra et træ..

\says{E} Hvad er så sjovt?

\says{S} Det vil du se om lidt! Nå lad mig trække! \act{Eksaminanten trækker et spørgsmål}

\says{E} Du har trukket Greens Sætning

\says{S} HA! Du troede jeg ville trække Greens Sætning, men mens I kiggede væk, byttede jeg rundt på spørgsmålene så jeg i stedet trak Hoved- \act{Kigger ned på papiret og stopper med at snakke, da det går op for S hvilket spørsmål der er trukket}

\says{S} \act{En smule skuffet} Jeg tror gerne jeg vil blanke...

\scene{S går ud af scenen}

\says{C} Tænk, Hovedsætninger lå forrerst hele tiden

\says{E} Der tager du fejl... \act{Smågriner lidt for sig selv} Alle spørgsmålene var Greens Sætning.

\scene{Tæppe}
\end{sketch}

\end{document}
