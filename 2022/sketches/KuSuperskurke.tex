\documentclass[a4paper,11pt]{article}

\usepackage{revy}
\usepackage[utf8]{inputenc}
\usepackage[T1]{fontenc}
\usepackage[danish]{babel}

\revyname{Matematikrevy}
\revyyear{2022}
\version{1.0}
\eta{$6$ minutter}
\status{Færdig}

\title{KU Superskurke}
\author{Sommer '17, Stine '18}

\begin{document}
\maketitle

\begin{roles}
\role{X}[Rune] Instruktør
\role{P}[Sommer] Partypooperen
\role{L}[Maja] Lyseslukkeren
\role{D}[Anna] Drømmefangeren
\role{K}[Johan] Kuldegyseren
\role{M}[Baldur] Man-Spreader
\end{roles}

% \begin{props}
% \prop{Rekvisit}[Person, der skaffer]
% \end{props}


\begin{sketch}
\scene{Lys op}

\scene{KUs ledelse sidder til møde rundt om et bord. Det viser sig de er superskurke.}

\says{P} Velkommen til vores møde i KUB. Her i Københavns Uhyrelige Bureaukrati, har vi ét formål!

\says{Alle} At ødelægge studentermiljøet!

\says{P} Man-Spreader, hvordan går dit mål om at udrydde kvinder fra undervisningen?

\says{M} Jeg har ikke fået udryddet dem alle endnu, men det er lykkedes mig at sørge for, at der blot i Analyse 0 var 3 gange så mange Rasmus'er der var instruktorer som der var kvindelige instruktorer! Jeg kan så fortælle at det virker til, at det at lave alle toiletter til Uni-sex toiletter og samtidigt fjerne samtlige pissoir på studiet, har gjort at flere kvinder end nogensinde før brokker sig over at enten er brættet ikke slået ned eller også er der pis på! \act{Griner ondt}

\says{P} Fantastisk! Kuldegyseren hvad med dig?

\says{K} Ja, så de studerende har jo længe brokket sig over vi ikke kan finde ud af vores indeklima.\act{Gør nar, fryser} "Uhh det er for koldt"\act{"Sveder"} "Nøøøj hvor er det varmt". Det er helt fantastisk! Men nu kommer mit nye onde tiltag! Vi skruer temperaturen ned til 19 grader, og bruger energikrisen som undskyldning! Hvad vi så ikke fortæller dem er, at det er 19 grader gennemsnitligt, så det kommer til at være 5 grader om vinteren og 33 grader om sommeren \act{Begynder at grine}

\says{P} Ej hvor dejligt ondt! Lyseslukkeren hvad har du af onde tiltag i dag?

\says{L} Så vi har hørt hvordan de forskellige foreninger elsker at afholde sociale arrangementer, og får et fantastisk sammenhold af at gøre rent sammen dagen efter... Det kan vi ikke have! Derfor indfører vi nu et krav om at man skal være færdig med rengøring senest kl 12 dagen efter, så der er klar til professionel rengøring. På den måde tvinger vi dem til enten at slutte festen tidligt, eller at møde op mega trætte for at gøre rent i et par timer! \act{Griner}

\says{P} Genialt!

\says{L} Jamen jeg er skam ikke færdig! Fordi hvad vi selvfølgelig ikke fortæller arrangørerne er, at vores rengøringspersonale møder op allerede kl. 8, og gør rent for dem og tager alt deres pant så de ikke kan købe pizza! Og fordi personalt er mødt op så tidligt, kræver vi også flere penge, så i stedet for nogle hundrede kroner, kræver vi nu flere tusinde kroner!

\says{P} Ha! Så kan de studerende lære ikke at prøve at skabe et sjovt og hyggeligt miljø på deres studie! Dine regler passer også perfekt sammen med de tiltag jeg selv har arbejdet på. Fremadrettet må der ikke drikkes i auditorier! \act{Kigger ud på publikum med et ondsindet smil}... Hvis der er polstrede sædder! Udover det, så skal der nu være en ædruvagt pr. 50 gæster til en fest. Vi vælger så at kalde det en \act{Gøre nar} "StUdEnTeRvAgT", så det lyder meget mere indbydende!

\says{M} Men Partypooper, er det så det samme arbejde som en ædruvagt lavede før?

\says{P} Nej nej nej Man-Spreader! Vi laver den perfekte blanding af en enormt vag beskrivelse af hvad de skal lave, samtidigt med at vi giver dem nogle enormt stramme regler såsom "at tysse på gæsterne for at tage højde for naboerne"og "at gå udenfor og samle skoder op."

\scene{Alle skurkene griner}

\says{P} Åhhh hvor er det altså rart bare at skide højt og flot på de studerende og hvordan de har det. Er der andre der har noget?

\says{D} Jeg har faktisk noget

\says{P} Drømmefangeren! De studerende er stadigvæk ikke kommet sig over vores hidtil ondeste tiltag: Fremdriftsreformen! Hvad har din onde mesterhjerne nu fundet på?

\says{D} Ja så vi har jo nogle vejledere.

\says{P} Tutorer! Ha, tænk vi fik dem til at kalde det noget nyt for ingen god grund

\says{D} Selvfølgelig! Vi har nogle tutorer, som jo elsker at lave masser af arbejde og sætte hele deres august af, bare for at sørge for at russerne får en god start. Og så gør de det helt frivilligt! Awww, hvor sødt... Det er lige til at brække sig over! Derfor indfører vi nu at der fremover skal være en hyret tutor. De skal altså søge om jobbet, og får selvfølgelig løn for det!

\says{L} Men vent hvorfor ville vi betale dem for noget de allerede gør frivilligt?

\says{D} Jo min kære lyseslukker! Ved at sørge for at én tutor bliver betalt, skaber vi en ubalance i tutorholdet. De andre tutorer vil synes det er uretfærdigt, at kun 1 bliver betalt, og der vil helt naturligt danne sig splid i tutorholdet. Derudover, så idet vi nu betaler en af dem, kan vi kræve meget mere af dem, da de nu er ansatte! Vi kan dræne et helt hold for alt deres frihed og kreativitet, ved blot at betale én af dem! \act{Griner ondt}

\says{P} Drømmefangeren du har gjort det igen! Åh de studerende kommer til at have det så forfærdeligt, jeg glæder mig allerede til at de brokker sig til deres lokale institutter og studenterforeninger, som ikke kan gøre noget som helst, som så helt desperat skriver ind til os. De studerendes gråd er som englesang på juleaften! \act{Griner ondt}

\says{K} Og tænk, det bedste er ikke engang kommet endnu!

\says{P} Nej nej, bare vent til foråret, når vi laver alle eksaminerne om til ITX-eksaminer!

\scene{Skurkene griner}

\scene{Lys ned}
\end{sketch}

\end{document}
