\documentclass[a4paper,11pt]{article}

\usepackage{revy}
\usepackage[utf8]{inputenc}
\usepackage[T1]{fontenc}
\usepackage[danish]{babel}

\revyname{Matematikrevy}
\revyyear{2022}
\version{1.0}
\eta{$1$ minutter}
\status{Færdig}

\title{Sikkerhedsoutro}
\author{Sommer '17}

\begin{document}
\maketitle

\begin{roles}
\role{X}[MaWeK] Instruktør
\role{M}[Jesper] Manusboss
\role{R}[Thais] Revyst
\end{roles}

% \begin{props}
% \prop{Rekvisit}[Person, der skaffer]
% \end{props}


\begin{sketch}
\scene{Tænkt som ekstranummer}

\scene{M og R står på scenen lidt fra hinanden}

\says{M} Ahh, endnu en veloverstået revy!

\says{R} Og tænk, vi glemte intet i år!

\scene{En mobil begynder at ringe ude blandt publikum}

\says{R} Hallo, vi bad jer sgu da om at slukke jeres telefoner!

\says{M} Ja helt ærligt, det er vitterligt det første vi fortæller jer I vores sikkerhedsintro som vi laver hvert år.

\scene{R og M ser opgivende ud, men pludseligt går det op for dem, og de kigger på hinanden}

\says{Begge} SIKKERHEDSINTROEN!

\says{M} VI VIL MEGET GERNE BEDE JER SLUKKE JERES MOBILTELEFONER UNDER HELE FORESTILLINGEN

\says{R} JA OG DER MÅ UNDER INGEN OMSTÆNDIGHEDER VÆRE ÅBEN ILD I SALEN

\says{M} NØDUDGANGENE ER DÉR!

\says{R} DÉR!

\scene{M og R råber og peger panisk på skift hvor de forskellige nødudgange er. Hint: Der er 6}

\says{M} \act{Forpustet} Og så... puuuh... Så må I have den bedste revy...

\says{Begge} \act{Forpustet men prøver at fake energi} NOGENSINDE!

\scene{Lys ned}
\end{sketch}

\end{document}
