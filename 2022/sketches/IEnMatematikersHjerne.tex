\documentclass[a4paper,11pt]{article}

\usepackage{revy}
\usepackage[utf8]{inputenc}
\usepackage[T1]{fontenc}
\usepackage[danish]{babel}

\revyname{Matematikrevy}
\revyyear{2022}
\version{1.0}
\eta{$4$ minutter}
\status{Færdig}

\title{I en matematikers hjerne}
\author{Stine Langhede ’18}

\begin{document}
\maketitle

\begin{roles}
\role{X}[Lise] Instruktør
\role{A1}[Jesper] Arbejdsom hjernecelle
\role{A2}[Rasmus] Arbejdsom hjernecelle
\role{H}[Viktor] Huskende hjernecelle
\role{F}[Maja] Flyvsk hjernecelle
\role{D}[Johan] Doven hjernecelle
\end{roles}

\begin{props}
\prop{Bord}
\prop{Tre stole}
\prop{Matematikbøger/notesbøger}
\prop{Sofa/sækkestol}
\end{props}


\begin{sketch}
\scene{Vi er inde i en matematikers hjerne. Lyset er lidt specielt, og på scenen er der 4 hjerneceller (måske i en form for arbejdstøj); F står i den ene side af scenen, med et omtåget udtryk og svajer glad fra side til side, D ligger i en sækkestol og halvsover, og de sidste to sidder ved bordet, A1 og kigger i en bog, og H med et tomt blik ud i ingenting.}

\scene{Lys op}

\scene{A2 kommer ind på scenen, med nogle papirer i hånden.}

\says{A2}[Entusiastisk] Okay, hjerneceller! Vi har lige modtaget en ny matematikaflevering, så nu er det bare med at komme i gang!

\scene{D halvvågner, kommer med et opgivende suk og ligner ikke én der har tænkt sig at hjælpe. F lader ikke til at have opdaget noget, men svajer bare videre og kigger med store øjne ud mod salen.}

\says{A1 og H}[Adlydende] Okay!

\says{D}[Dovent] Ej, er den ikke først til næste uge?

\scene{A2 sætter sig ved bordet og de tre begynder at kigge på afleveringen og bladre i lidt bøger.}

\says{A1} Skal vi måske bruge det her lemma? [Peger i bog]

\says{D}[Håbefuldt] Skal vi måske gå op på fjerde?

\scene{F begynder stille at nynne omkvædet af sang (som kort tid forinden har været med revyen)}

\says{A2} Uhh ja, måske med et induktionsbevis!

\scene{F nynner ret højt nu, helt i sin egen verden.}

\says{A1}[Halvråbende] Ja præcis, så ville det give os, at... [Mister fokus og vender sig mod F] Undskyld? Ja, hej, det er lidt forstyrende det der..

\scene{F ser halvskuffet ud, men tier stille og vender tilbage til sit omtågede svajende udtryk.}

\says{A1} Hvor kom vi fra?.. Noget med modstrid?

\says{A2} Øhmm..

\scene{A1, A2 og H kigger tænkende på hinanden i et øjeblik, men giver så op, og begynder at kigge i bøger igen. H ligner dog en, der i mellemtiden er begyndt at tænke på noget helt andet.}

\says{H} Kan I huske? Da vi tog hjemmefra i morges, låste vi så døren? [Bider sig nervøst i hånden]

\says{A2}[Sukkende] Selvfølgelig gjorde vi det, læs nu bare lidt i den her [Rækker H bog og vender sig mod A1]. Bør vi lige undersøge, om den konvergerer?

\says{D}[Opgivende] Ååårh, det er bare så kedeligt! Hallo, kan vi ikke lige spille candy crush igen, [Viser sin mobil] se, vi har 5 liv, såå..?

\says{F}[Drømmende] Uuh, candy crush!

\says{A1} Nej nej nej, vi skal altså have lavet det her! Hvad med.. determinanten! Har vi undersøgt den?

\scene{A2 bladrer begejstret i bog og finder en god sætning.}

\says{A2} Uuh, altså den kan vi vel finde, hvis vi bruger sætning..

\says{H}[Afbryder] Kan I huske hende der fra parallelklassen i gymnasiet? Hende der Clara et-eller-andet?

\says{A1}[Opgivende] Nej, altså hvad med hende?

\says{H} Bare I ved - hun var da egentlig meget sød..

\says{D} Nårh ja, det var hende der med de kæmpe..

\says{A2}[Afbryder] Nejnej! [Til D] Gider du ikke at skrive til de andre på studiet og høre, hvordan de har gjort i opgave 1?

\scene{D ligner ikke en der gider, men hiver en mobil frem og vender fokus mod den. I mellemtiden er F begyndt at nynne sangen igen.}

\says{A1}[Tænkende] Øhm, hvordan er det nu, man staver til korollar?

\says{A2}[Til F] Så nu gør du det igen, gider du ikke lige stoppe?

\scene{F lader ikke til at høre noget, men står fuldstændig svajende i sin egen verden.}

\says{A2} Hallo!? [Går hen til F og forsøger at skabe kontakt men uden held, så prøver i stedet at flytte F, men uden held]. Ej! Altså, den sidder fuldstændig fast!

\says{D} Ååårh, prøv lige at se den her video!

\scene{Cirka 7 sekunder lang video spiller oppe på storskærmen af en sød kat.}

\says{Alle}[Ramt af nuttetheden] Ååårh!

\scene{F begynder ikke at nynne igen, men svajer bare videre.}

\says{D} Vi kan lige se en til, de er jo mega korte.

\scene{Cirka 7 sekunder lang video spiller oppe på storskærmen af noget nuttet sjovt.}

\says{Alle} Hehehe.

\says{D} Vi kan lige se en til, de er jo mega korte.

\says{A1}[Ryster sig ud af det] Ejnej, det var den der aflevering, kom så!

\scene{A2 bemærker, at F er blevet stille og signalerer begejstret til de andre, at de ikke må sætte F igang igen. A2 sætter sig så ned og kigger i flere bøger. H ligner også en der rent faktisk prøver nu.}

\says{H}[Får en idé] Uuh, kan I huske det der vi lærte i Analyse 1?

\says{A1}[Begejstret] Nej?

\scene{A2 ryster også på hovedet.}

\says{H}[Skuffet] Nååh. Kan I så huske, om vi har nogle bananer derhjemme?

\scene{A1 og A2 ser opgivende ud, hvilket D ser.}

\says{D}[Lusket] Altså det kan da også være, at en lille bajselade ville hjælpe?

\scene{A1 og A2 kigger hurtigt på hinanden, men kigger så i bøgerne og prøver at ignorere D.}

\says{D}[Lokkende] De andre fra studiet skriver faktisk, at de tager på Caféen? nu.

\says{A1}[Fristet] Virkelig? [Til A2] Altså hvis vi alligevel ikke kommer så meget videre...

\scene{A2 lukker langsomt sin bog i.}

\says{A2}[Fristet] Altså vi har jo givet det et godt forsøg..

\says{A1}[Fristet] Og der er jo alligevel også genaflevering...

\says{A1 og A2}[Begejstrede] Okay, lad os tage på Caféen?!

\says{Alle andre} Wuhuuu!

\scene{A1, A2 og H begynder at pakke bøgerne sammen.}

\says{H} Kan I huske, da vi var på Caféen? i går? Der hørte vi hele tiden den der sang...

\says{A2}[Febrilsk] NEEEJ!

\scene{F begynder højlydt at nynne sangen igen og alle andre ser meget opgivende ud.}

\scene{Tæppe}
\end{sketch}

\end{document}
