\documentclass[a4paper,11pt]{article}

\usepackage{revy}
\usepackage[utf8]{inputenc}
\usepackage[T1]{fontenc}
\usepackage[danish]{babel}

\revyname{Matematikrevy}
\revyyear{2022}
\version{1.0}
\eta{$1.5$ minutter}
\status{Færdig}

\title{$F$, den kan man ikke få at se}
\author{Johan '19}
\melody{Børnesang: ``Blæsten kan man ikke få at se''}

\begin{document}
\maketitle

\begin{roles}
\role{X}[Rune] Instruktør
\role{S1}[Maja] Barn 1
\role{S2}[Sara] Barn 2
\role{S3}[Simone] Barn 3
\end{roles}

% \begin{props}
% \prop{Rekvisit}[Person, der skaffer]
% \end{props}


\begin{song}
\sings{S1 + S2 + S3} $F$, den kan man ikke få at se
Det er der ikke noget at gøre ved
Men når $n$ går mod uend'lig
Og man tager en sum af $X_i$
Så ' det let at gætte dens fordeling

$F$, den kan man ikke få at se
Det er der ikke noget at gøre ved
Men når man approksimerer
Med de tal man observerer
Så ' det let at gætte dens fordeling

$F$, den kan man ikke få at se
Det er der ikke noget at gøre ved
Men tager man en mængde data
Og beregner estimater
Så ' det let at gætte dens fordeling

$F$, den kan man ikke få at se
Det er der ikke noget at gøre ved
Den centrale grænsesætning
Si'r når $n$ går mod uend'lig
Så ' det let at gætte dens fordeling
\end{song}

\end{document}
