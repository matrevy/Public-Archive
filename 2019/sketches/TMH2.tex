\documentclass[a4paper,11pt]{article}

\usepackage{revy}
\usepackage[utf8]{inputenc}
\usepackage[T1]{fontenc}
\usepackage[danish]{babel}


\revyname{MatematikRevyen}
\revyyear{2019}
% HUSK AT OPDATERE VERSIONSNUMMER
\version{2}
\eta{$1$ minutter}
\status{Færdig}

\title{Til middag hos forelæserne 2 (Hos Søren)}
\author{Cecilie '17$\frac{1}{2}$, Peifer '17, Sommer '17}

\begin{document}
\maketitle

\begin{roles}
\role{X}[Patrick] Instruktør
\role{B}[Villads] Bergfinnur
\role{E}[Jesper] Ernst Hansen
\role{S}[Eigil] Søren Eilers
\role{Ba}[Anders] Barn
\role{N}[Ib] Ninja
\role{Br}[Johan] Brandvagt
\role{Mic}[Line] Mikrofonansvarlig
\role{BackS}[Nina] Backstage Ansvarlig
\end{roles}

\begin{sketch}

\scene{S står og laver mad og hører sin danse sang (og danser/synger lidt med). Der er lego over det hele. Der bliver ringet på}
\says{S} Velkommen Ernst, Velkommen Bergfinnur. Pas på konstruktionerne
\says{E} Ja. Du har taget dit arbejde med hjem ser jeg.
\says{B} Det dufter da godt nok godt det her. \act{ prikker til E} Ikke sandt?
\says{E} Helt Bestemt. Ja. Jo. Ja
\says{S} Det er jeg glad for at I synes, maden er også ved at være-
\scene{Søren bliver afbrudt af et barn kommer ind med hænderne fuld af lego}
\says{S} NEEEEJ! HVAD HAR JEG SAGT OM AT LEGE MED FARS LEGO!?

\scene{Tæppe}

\end{sketch}
\end{document}