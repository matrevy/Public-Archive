\documentclass[a4paper,11pt]{article}

\usepackage{revy}
\usepackage[utf8]{inputenc}
\usepackage[T1]{fontenc}
\usepackage[danish]{babel}


\revyname{MatematikRevyen}
\revyyear{2019}
% HUSK AT OPDATERE VERSIONSNUMMER
\version{2.5}
\eta{$1$ minutter}
\status{Færdig}

\title{Til middag hos forelæserne 1 (Hos Ernst)}
\author{Cecilie '17$\frac{1}{2}$, Peifer '17, Sommer '17}

\begin{document}
\maketitle

\begin{roles}
\role{X}[Patrick] Instruktør
\role{B}[Villads] Bergfinnur
\role{E}[Jesper] Ernst Hansen
\role{S}[Eigil] Søren Eilers
\role{N}[Anders] Ninja
\role{Br}[Victoria] Brandvagt
\role{Mic}[AK] Mikrofonansvarlig
\role{BackS}[Mikkel] Backstage Ansvarlig
\end{roles}

\begin{sketch}

\scene{Lys op}

\scene{Det ringer på}
\says{E} Søren. Bergfinnur. Kom indenfor.
\says{B} Det er da godt nok en smal gang det her Ernst. Ikke sandt?
\says{E} Se til højre. Se til venstre. Kun én af jer kommer igennem, ad gangen.
\says{B} Hvad skal vi have i aften?
\says{E} And og kartofler. Anden er allerede i ovnen, så hvis en af jer lige kan hjælpe med at lave kartofler.
\says{S} Den klarer jeg!
\says{B} Glem ikke salt

\says{E} Hov, det var da højst ejendommeligt. Komfuret virker pludselig ikke.
\says{B} Er det gas ?
\says{E} Nej, induktion.
\says{S} Det smarte ved et induktionskomfur er jo, at hvis vi bare koger den første kartoffel, og den næste kartoffel tilberedes mens den første er i gryden, så har vi et vilkårligt antal kogte kartofler.
\says{E} Nej. Det betyder bare at det går hurtigere end et normalt komfur.
\says{S} Rækker du mig ikke målebægeret?
\says{E} Tænker du på dirac-bægeret eller tællebægeret?
\says{B} Lebesgue-bægeret Ernst, hvad tror du selv!
\says{B} Apropos mål, hvordan går det med frafaldet på matematik?
\scene{E hiver en graf frem og forklarer}
\says{E} Ja men det ligger sådan at vi er lidt under vores mål på de 50 procent på nuværende tidspunkt. Men i og med vi i næste blok har Analyse 0 kan vi se at grafen når epsilon under vores mål!


\scene{Lys ned}

\end{sketch}
\end{document}