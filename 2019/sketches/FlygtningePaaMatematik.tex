\documentclass[a4paper,11pt]{article}

\usepackage{revy}
\usepackage[utf8]{inputenc}
\usepackage[T1]{fontenc}
\usepackage[danish]{babel}


\revyname{MatematikRevyen}
\revyyear{2019}
% HUSK AT OPDATERE VERSIONSNUMMER
\version{2}
\eta{$4$ minutter}
\status{Ikke færdig}

\title{Det sene sceneshow}
\author{Axel '17}

\begin{document}
\maketitle

\begin{roles}
\role{X}[Anne] Instruktør
\role{V}[Manchester] Vært
\role{T}[Line] Tanja Tværfaglig
\role{I}[Ib] Ib Støjberg
\role{B1}[Nina] Band
\role{B2}[Sofie] Band
\role{B3}[Gustav] Band
\role{N}[Marius] Ninja
\role{Br}[Victoria] Brandvagt
\role{Mic}[AK] Mikrofonansvarlig
\role{BackS}[Simone] Backstage Ansvarlig
\end{roles}


\begin{sketch}
\scene{Lys op}
\scene{overvej phil collins: I can feel it in the air, som introsang}

\says{V}
Godaften og velkommen til Det Sene Scene Show. \\
Flygtningekrisen er i de sidste måneder taget til.\\
Udsigten til ansættelse af færdiguddannede humanister er stadig meget lav, og det får studerende fra KUA til at flygte til Universitetsparken i stadig større mængder.\\
Rejsen er for de fleste meget farlig, da de bliver smuglet over Knippelsbro i overfyldte Christiania-cykler, uden seler eller cykelhjelm.\\
Matematisk folkeparti har for nylig foresl\aa et at \ae ndre reglerne for hvorn\aa r flygninge skal hjemsendes.\\
Tag nu rigtig godt imod aftenens første gæst: Tanja-Tværfaglig!
 \scene{TT kommer ind til nice jingle *Tanja Tværfaglig - det er fag på tværs*}
 
 \says{V} Tanja, du er jo integrationsordfører for Partiet for Ægte Inklusion, hvad mener du om denne sag?

\says{T}
Disse humanistiske stakler der flygter fra en skæbne af nyttesløshed skal naturligvis have en chance for et bedre liv her på det naturvidenskabelige fakultet.\\
Så længe disse flygtninge tilegner sig matematiske værdier, altså lærer at regne og føre stringente beviser, så ser vi det som tilstrækkeligt for at bidrage til det matematiske samfund. 

\says{V}
På studiet, undskyld, jeg mener i studiet kommer her aftenens anden gæst Ib Støjberg.
\scene{Ib kommer ind til en jingle}

\says{V}
Ib, som integrationsordfører for Matematisk folkeparti er vi også meget interesserede i at høre dit syn på sagen?

\says{I}
Vi har jo desværre allerede eksempler på samfundsvidenskabelige indvandrere på fakultetet, som det ikke er lykkedes at integrere.\\
Se bare på Mat-Øk’erne! De har grupperet sig i deres eget parallelsamfund i den anden ende af kantinen, og har deres eget fodboldhold.

\says{T} Men deres hold klarer det da bedre end ZFC..

\says{V} Ja, så måske er der slet ikke noget problem med MatØk'ere?

\says{I} Men de tager jo kun matematikkurser som er relevante for deres samfundsfaglige interesser, og de nægter at tage kurser i ren matematik.\\
Hvis man ikke har taget Topologi og KomAn er man IKKE en rigtig matematiker.

\says{T}
Vi mener ikke at MatØk’erne er en problematisk gruppe på HCØ, tvært imod.\\
Vi ser det som en succes at vores fakultet kan rumme større diversitet.\\
Datalogi har endda fulgt trop og oprettet en DatØk linie.\\
Tværfagligt samarbejde er vejen frem.

\says{I}
For det første (og eneste!) ved vi slet ikke hvordan det kommer til at gå med DatØk’erne.\\
En bekymrende stor mængde af dem har dumpet MatIntro, og det er stadig uklart om de overhovedet kan bestå ØkIntro.

\says{T}
Men det er der jo også matematikere der ikke kan…

\says{I}
Det taler vi slet ikke om !

\says{V} Jamen vi har jo netop modtaget et séerspørgmsmål om netop dette emne 
"Ernst Hansen: Jeg hører bare Ib sige Nej, Nej ! Nej! og Nej! men har han egentlig noget konkret løsningsforlag? \# NedeMedTanjaTværfaglig"

\says{T} Jeg mener at for at have et fredeligt fakultet må det være homogent.\\
Det er en skandale at Mikkel Willum, den slyngel, overhovedet får lov til at forelæse på HCØ. Det burde forbydes at sprede den slags filosofiske idéer.

\says{V} Nu stikker det her vidst lidt af. Vi går direkte videre til reklamer.

\scene{Lys ned}

\end{sketch}
\end{document}
