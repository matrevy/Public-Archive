\documentclass[a4paper,11pt]{article}

\usepackage{revy}
\usepackage[utf8]{inputenc}
\usepackage[T1]{fontenc}
\usepackage[danish]{babel}


\revyname{MatematikRevyen}
\revyyear{2019}
% HUSK AT OPDATERE VERSIONSNUMMER
\version{0.1}
\eta{$n$ minutter}
\status{Færdig}

\title{Men Hvad Laver Du Egentlig?}
\author{Ulrik '14}

\begin{document}
\maketitle

\begin{roles}
\role{X}[Mathias] Instruktør
\role{SS}[Marius]  Specialestuderende
\role{DR}[Felix] Rus
\role{Br}[Victoria] Brandvagt
\role{Mic}[AK] Mikrofonansvarlig
\role{BackS}[Simone] Backstage Ansvarlig
\end{roles}

\begin{sketch}
\scene{Lys op.}

\scene{SS står og fløjter ved en kaffemaskine, drikker af sin kop og stirrer på sit ur.}

\scene{DR kommer ind og spotter SS.}

\says{DR} Hey: Du er sådan en kandidatstuderende, ikke?

\says{SS} Altså… faktisk er jeg i gang med at skrive mit speciale. Jeg burde i virkeligheden komme tilbage til arbejdet.

\scene{SS begynder stille og roligt at bevæge sig fløjtende ud af lokalet, men er tydeligvis LIDT nølende.}

\says{DR} Klart, klart. Hvad er det egentligt du laver?

\scene{SS træder straks hen mod DR igen.}

\says{SS} Altså… det er sådan lidt teknisk, du ved.

\says{DR} Jo, klart nok, men i sådan… grove træk.

\scene{SS ser meget tænksom ud.}

\says{SS} Grove træk… klart. Så… to-tre gutter, du aldrig har hørt om, revolutionerede i 2017 et meget vigtigt resultat, du heller aldrig har hørt om, ved at svejse det sammen med et nyt buzzword. Det, jeg laver i mit speciale, er så at tage deres konstruktion og prøve at placere den i et helt urelateret univers, hvor ingen har bedt om den. Dette kræver selvfølgelig, at jeg tilføjer et nyt buzzword.

\says{DR}[entusiastisk] Altså et sexet buzzword som… machine learning eller… block chain, eller… quantum computing?”

\says{SS}[opgivende] Nej, nej. Det er sådan noget, man skriver på legatansøgninger. Nej, her er formålet med buzzwords tværtimod at eliminere enhver mulig interesse, som publikum måtte have i dit emne. Gentag efter mig: Generaliseret Boolsk quasi-$C^*$-homotopisk kompleks topologisk K-teori med koefficienter i en E-uendelig ring.

\says{DR}[forvirret] Prosit.

\says{SS}[begejstret] Præcis! Du ved, hvordan lineær algebra, det er sådan noget med at gange matricer sammen, right?

\says{DR} Klart, og så noget med baser!

\says{SS} Det her handler om den naturlige generalisering til overabstrakt hypernonsens, hvor det SLET ikke giver mening at gange.

\scene{DR ser noget mere forvirret ud og holder sin fingre op for at tælle, men SS er allerede manisk i gang med at gå videre til det næste. SS springer rundt mens vedkommende fortæller, og hiver DR med til højre og venstre.}

\says{SS}[med tiltagende mani og store armbevægelser] Forestil dig, at du har en ballon, du skal puste op, men i syv dimensioner, og al luften er firkantet. Altså, det vil sige… ikke rigtig firkantet, men generaliserede kvadranguleringer. Og hvis to af luftmolekylerne nogensinde kommer til at røre hinanden, så ryger du direkte i fængsel, og så snart, du slår to ens og kommer ud, så bliver du jagtet af kerberos, men han spyr ild og du skal hinke.”

\says{DR} Okay… men er det overhovedet en særligt god beskrivelse af det, du laver?

\says{SS} Øh… nej.

\says{DR} Okay. Så hvad laver du egentligt?

\says{SS} Altså: Primært, så drikker jeg kaffe og udskyder deadlines.”

\says{DR} Ah… så du studerer!

\scene{Lys ned.}

\end{sketch}
\end{document}
