\documentclass[a4paper,11pt]{article}

\usepackage{revy}
\usepackage[utf8]{inputenc}
\usepackage[T1]{fontenc}
\usepackage[danish]{babel}

\revyname{Matematikrevyen}
\revyyear{2019}
% HUSK AT OPDATERE VERSIONSNUMMER
% UNDLAD AT SKRIVE I TEMPLATE.TEX - KOPIÉR OG OMDØB I STEDET FOR
\version{2.718281828}
\eta{$\sim 4$ minutter}
\status{Færdig}

\title{Lommeregneren}
\author{Michael '12, William '10, Kristian '10, Line '16}

\begin{document}
\maketitle

\begin{roles}
    \role{X}[Anne] Instruktør
	\role{S}[Felix] Studerende
	\role{L}[KE] Lommeregner
\role{Br}[Anders] Brandvagt
\role{Mic}[Peifer] Mikrofonansvarlig
\role{BackS}[Sommer] Backstage Ansvarlig
\end{roles}


\begin{sketch}

\scene Lys op

\says{L}
Jeg er T I 89! En lommeregner! Jeg er ikke bare mester i addition, substraction, multiplication, division, kvadrering, logaritmetagning. Jeg kan løse komplicerede ligninger med mit computer algebra system. 

En af mine nye features er at jeg kan fortælle dig at sin(60 grader) er kvadratrod 3 halve, og ikke bare 0,86603. \act{0.86603 siges meget monotont og computeragtigt}

Jeg kan finde grænser af funktioner - inklusiv dem gående mod uendelig og grænser fra en enkelt retning.

Jeg kan differentiere bestemte integraler eksakt og.... hov, nu kommer der en bruger!!!! YES!

\says{S}
Hvad er 12 gange 13.

\says{L}
156, men det kan man jo regne i hovedet.

\says{S}
minus 10.

\says{L}
146. Ej, det mener du ikke!!! Hvad bilder du dig ind?! Jeg nægter at tro på at du ikke kan regne det ud selv. Det er som at skyde gråspurve med klimaforandringer!...

\scene S får en ide.

\says{S}
1 divideret med 0

\says{L}
.mf..f. uendelig! Ha ha så let får du ikke RAM på mig. Hvad ville du sige til at jeg satte dig til at lægge tal sammen i et eller andet programmeringssprog efter at du havde brugt 6 år på en svær universitetsuddannelse?

\scene S trykker på sluk knappen.

\scene Lys ned.


\end{sketch}
\end{document}