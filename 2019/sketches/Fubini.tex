\documentclass[a4paper,11pt]{article}

\usepackage{revy}
\usepackage[utf8]{inputenc}
\usepackage[T1]{fontenc}
\usepackage[danish]{babel}

\revyname{Matematikrevyen}
\revyyear{2019}
% HUSK AT OPDATERE VERSIONSNUMMER
% UNDLAD AT SKRIVE I TEMPLATE.TEX - KOPIÉR OG OMDØB I STEDET FOR
\version{2.718281828}
\eta{$\sim 4$ minutter}
\status{Færdig}

\title{Fubinis nummer}
\author{Michael '12, William '10, Kristian '10, Line '16}

\begin{document}
\maketitle

\begin{roles}
    \role{X}[Michael] Instruktør
	\role{F}[Sofie] Fubini
	\role{A}[Villads] Assistent slash manager (kan skrives ind i sketchen)
\role{N}[Felix] Ninja
\role{N}[Marius] Ninja
\role{N}[Rikke] Ninja
\role{Br}[Johan] Brandvagt
\role{Mic}[Line] Mikrofonansvarlig
\role{BackS}[Nanna] Backstage Ansvarlig
\end{roles}


\begin{props}
	\prop{stol}	
\end{props}

\begin{sketch}

\scene Lys op

\scene En cirkusagtig tryllekunstner kommer ind på scenen. 

\says{F} (buongiorno til italienierne?)Bonjouur til franskmændene. Wiiiilkommen, det er tysk. og ----- til datalogerne! Jeg er den store Fubini og i dag skal I være vidne til et stort stort nummer!!! I vil se fantasier blive til realitet, drømme blive virkelighed og umuligheder forblive umulige. \act{Sagt på en meget karakteristisk cirkusagtig måde}

\says{F}
Fubinis første nummer er det teleskoperende lommetørklæde. Først er det kort, men nu er det langt.

\scene{ Her sker der noget helt vildt vha. en rekvisit.}

\says{F}
Og nu! Skal jeg bruge... min assistent! Tunelli! Tunelli!

\scene{ Enter: Villads / Tunelli}

\says{F}
Her på bordet ser I tre hatte.  Fubini vil fremvise en kanin under en af hattene. Da jeg kan trylle, ved vi at der er en kanin under én af hattene. Der er ingen kanin under denne hat og ingen kanin under denne hat. Altså ved vi nu vha. skuffeprincippet at jeg har tryllet en levende kanin frem under den midterste hat. HAHAHA.
\scene{ Tunelli løfter yderste hatte. Uden at publikum lægger mærke til det, placeres en bold under hat nr. 3, når den lægges på plads. \\
Tunelli forsøger at løfte midterste hat, men får skældud.
}
\says{F} Nej, nej, nej. Nu har jeg ikke brug for dig mere. Smut Tunelli. 

\says{F}
``Oooog til mit næste nummer'' skal jeg bruge en frivillig fra publikum. Nøj, hvor er der mange hænder. Jeg går lige tilbage så jeg kan se jer alle. Ja, hvad meeeed den unge persooon .....  her.

\scene{ Får publikum til at række hænderne op hvorefter tryllekunstneren peger på bagtæppet og en Tunelli / Villads træder frem - nu med overskæg}

\says{F}
og for en god ordens skyld, skal jeg lige spørge: Vi da ikke mødt hinanden før?

\says{A}
nej, -

\says{F}
Lige præcis.

\says{F}
Og med den frivillige vil vi nu lave et fantastisk nummer:
Tænk på et helt tal. Kan du huske det?  

\says{A} Nej

\says{F}
Frivillige er da også uduelige! Kom, vi henter tavlen.

\scene{henter tavlen (den der lille en med papir)}

\says{F}
Tænk på et tal... OG SKRIV DET NED! Kig på det og husk det. Har du skrevet det ned? Gang det med 9. Tag den reducerede tværsum. Træk 2 fra. Var det 7?

\says{A}
det giver minus 2. \act{Fubini skynder sig videre}

\says{F} Fubini vil nu udføre et nummer der aldrig tidligere er set ! Han vil trylle denne øl væk ! og såå 1, 2, 3! PUF ! \act{Fubini bunder en bajer} 
Puuha, det er hårdt at trylle !
Og så kan vi jo trylle øllen frem igen ! Men det nummer vil han overlade som en øvelse til tilskueren.

\says{F}
Vi vil nu trylle denne bold fra den første hat over i den tredje hat. Nummeret vil deles op i 3 dele. En såkaldt tretrinsraket. Læg først mærke til at vi kan flytte alle boldene fra den første hat over i den anden. For enhver bold i hat 1, gælder der at den nu er i den midterste hat. Tilsvarende kan vi se at det gælder fra anden til tredje hat. (Her løfter F hat nummer 2 og viser en tom hat, F har ikke løftet nogen af de andre.) Det var andet led i raketten. Til slut kan vi nu placere denne bold under den første hat, og så PUF har vi bolden i den tredje hat !

\scene{F løfter hat 3, hvor der ligger en bold under.}

\says{F}
Og til mit næste nummer får jeg brug for - ab-so-lut... Stilhed. (lang pause)
Fubini vil nu med hjælp fra sin kyndige assistent lade sig selv binde i lænker og disse hænder spændes fast med håndjern, hvorefter han til sidst låses ind i denne kasse af armeret jernbeton. Fubini har nu fem-og-tredive sekunder til at bryde ud af denne tilsyneladende kropumulige hovedpine... men grundet tidsmangel vil vi overlade de sidste skridt som en øvelse til publikum.

\scene Lys ned.

\scene Ideer til mere:

\says{F}
Kan trylle skrald om til penge...  Hviiiis der er pantmærker på ;)

\says{F}
Jeg kan trylle bukserne af ALLE pigerne. Den er btw altid et hit til børnfødselsdagen!

\says{F}
"Nårh, nej, det kan man ikke - så må man låne"


\says{F}
Slutning: Tryller sketchen væk. Står eventuelt og anstrenger sig og langsom går lyset.
SLutning: Men det når vi ikke nu.




Man har snydt lidt hjemmefra. 

Meeeen før vi kan gøre det, så må vi lige vise det her resultat, men vi nøjes med at vise den ene vej. 

Overlades som en øvelse til tilskueren



Fin assistent

Lang anekdote, som ikke leder op til noget som helst

Tryller en øl væk... og så kan vi trylle vi den frem igen! ... men det gør vi senere



Tryller BH frem

Bruge en ufrivilig ...






\end{sketch}
\end{document}