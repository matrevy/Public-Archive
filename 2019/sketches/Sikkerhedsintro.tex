\documentclass[a4paper,11pt]{article}

\usepackage{revy}
\usepackage[utf8]{inputenc}
\usepackage[T1]{fontenc}
\usepackage[danish]{babel}


\revyname{MatematikRevyen}
\revyyear{2019}
% HUSK AT OPDATERE VERSIONSNUMMER
\version{0.3}
\eta{$n$ minutter}
\status{Færdig}

\title{Sikkerhedsintro}
\author{Frederik '16, Sommer '17}

\begin{document}
\maketitle

\begin{roles}
\role{X}[Buchter] Instruktør
\role{T1}[Felix] Lysmand fra TeXnikken
\role{T2}[Ib] Lydmand fra TeXnikken
\role{T3}[Toke] Lydmand fra TeXnikken
\role{D}[Marius] Person fra publikum
\role{E}[Manchester] Revyst
\role{Br}[Victoria] Brandvagt
\role{Mic}[AK] Mikrofonansvarlig
\role{BackS}[Mikkel] Backstage Ansvarlig
\end{roles}

\begin{sketch}

\scene{Lys er nede. D er ude hos publikum, tæt på lys el. lyd i TeXnikken. TeXnikken snakker sammen.}
\says{T1} Hallo gutter (Evt. gentag dette eller andet tilsvarende, hvis publikum larmer)
\says{T1} Hvad fanden er cuen til den her sketch?
\says{T2} Jeg tror det er noget med at vi skulle kunne se en kæmpe rekvisit på scenen?
\scene{Lys op på scenen}
\says{T1} Men der sker sgu da ikke noget dernede?
\says{T3} Du tror vel ikke de andre revyster har glemt det vel?
\says{T1} Ej det er første sketch jo, den kan de da umuligt have glemt!
\says{T3} Det ville da ikke være det dummeste der er sket i MatRevyen.
\says{T1} Altså der er vel nogen der skal på scenen? 
\says{T2} Fri!
\says{T3} Jeg skal ikke der op. 
\says{T1} Altså jeg skal slet ikke. Hey du der?
\scene{Spot på D. D peger på sig selv og er tydeligvist forvirret.}
\says{T1} Ja dig! Hør kan du ikke smutte ned på scenen og lige improvisere noget, mens vi lige finder ud af hvor scenefolkene er blevet af?
\scene{T2 stikker D sin mikrofon}
\says{D} Ej, jeg skal jo slet ikke være med i revyen i år.
\says{T1} Ej, kom nu!
\says{D} det gider jeg altså ikke
\says{T2} Ej G kom nu!
\says{T1-3} Marius, Marius, Marius, $n \times $Marius
\says{D} Jo, jo det kan jeg vel godt
\scene{D skynder sig ned på scenen}
\says{D} Heeeey hva så store UP1? Hey I bliver ikke sure hvis jeg lige tager et selfie vel?
\scene{D begynder at hive sin mobil ud af lommen}
\says{T2} Hov hov nej, du må faktisk ikke have din mobil tændt under revyen.
\says{D} Nåååår, ah ja okay...
\act{D begynder at trippe lidt}
\says{D}[Hvisker] Hvad fanden plejer de at lave i de her sketches.
\scene{D står og tænker lidt videre, og får lige pludseligt en idé}
\says{D}Okay så jeg har den her syge historie. Okay så mig og min fætter ik? - Ej, jeg tænder lige en smøg først..
\scene{Mens D forklarer, hiver han cigaretpakke og lighter ud af lommen for at imitere sin historie}
\says{T2} Nej nej nej, du må jo slet ikke have åben ild herinde.
\says{D} Ahh pis, ej så gider jeg slet ikke. Hvordan kommer jeg ud herfra?!
\says{T1}  Det er simpelt, der er nødudgange der, der, der, der, der og der
\scene{Spot lyser på de forskellige udgange som B “udpeger” dem}
\says{D} Fuck jer, så må I sgu lave jeres egen revy!
\scene{D skal til at gå ud af scenen, men støder ind i E som kommer løbende ind på scenen klædt fint på}
\says{E} Hva så UP1 er I klar til den bedste mat-
\scene{E stopper op og ser overrasket på D}
\says{E} D? Hvad fanden laver du her?
\says{D} Altså publikum var trætte af at høre datalogerne synge "I morgen er verden vor", og I satte jo aldrig revyen i gang, så TeXnikken bad mig improvisere noget...
\says{E} Nånå men jeg er altså klar nu, så du kan bare smutte
\scene{D smutter ud af scene}
\says{E} Mine damer og herre, først og fremmest det er yderst vigtigt at I har jeres mobiller slu-
\scene{TeXnikken cutter E af}
\says{T3} E? Det er allerede blevet sagt
\says{E} Hvad? J-j-jamen så er det vigtigt at I ikke har åben-
\says{T2} Vi har også taget den med ingen åben ild
\scene{E bliver tydeligvist lidt ked af det}
\says{E} Også den med nødudgangene?
\says{T1} Jep E. Også den med nødudgangene.
\scene{E sukker skuffet. Et par sekunders akavet stilhed}
\says{T3} Altså vi har ikke sagt at så kan revyen gå i gang
\says{E}[Lysner op] Jamen så kan revyen gå i gang! 
\scene{Tæppe}

\end{sketch}
\end{document}