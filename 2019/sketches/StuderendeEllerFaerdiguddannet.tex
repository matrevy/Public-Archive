\documentclass[a4paper,11pt]{article}

\usepackage{revy}
\usepackage[utf8]{inputenc}
\usepackage[T1]{fontenc}
\usepackage[danish]{babel}

\revyname{Matematikrevyen}
\revyyear{2019}
% HUSK AT OPDATERE VERSIONSNUMMER
% UNDLAD AT SKRIVE I TEMPLATE.TEX - KOPIÉR OG OMDØB I STEDET FOR
\version{2.718281828}
\eta{$\sim 4$ minutter}
\status{Færdig}

\title{Studerende eller færdiguddannet?}
\author{Michael '12, William '10, Kristian '10, Line '16}

\begin{document}
\maketitle

\begin{roles}
    \role{X}[Michael] Instruktør
	\role{S}[Peifer] Simon som studerende
	\role{A}[Johan] Simon som arbejdende
	\role{P}[KE] Simon som pensionist
\role{T}[Eigil] Tjener
\role{N}[Aurora] Ninja
\role{N}[Nanna] Ninja
\role{N}[Villads] Ninja
\role{Br}[Victoria] Brandvagt
\role{Mic}[AK] Mikrofonansvarlig
\role{BackS}[Simone] Backstage Ansvarlig
\end{roles}


\begin{props}
	\prop{2x bord og stol}
	\prop{Fad med lækker mad}
	\prop{Tallerken med 8 skiver rugbrød}
	\prop{Laptop og headset}
	\prop{Diverse matematikbøger/notesblokke}
	\prop{Skilt hvor der står RIP}
	\prop{Rollator}
\end{props}

\begin{sketch}

\scene Lys op

\scene De to Simon står i hver sin side af scenen. (Bevæger sig måske på samme måde)

\says{S} 
Jeg er Simon Jensen og jeg er studerende på andet år.
\says{A} 
Jeg er Simon Jensen 6 år senere og jeg har fuldtidsarbejde.

\says{S} 
Årrrh... Jeg har mange lektier for.
\says{A} 
Aaaah... jeg har ikke nogen lektier for.

\says{S} 
Det er da alligevel rimelig mange penge jeg skal betale for rimelig lav kvalitet af mad i den her kantine
\says{A} 
Det er da alligevel rimelig få penge jeg skal betale for rimelig høj kvalitet af mad i den her kantine

\says{S} 
Jeg elsker bare ren matematik! Når jeg bliver færdig, så gider jeg ikke at kode.
\says{A} 
Det er egentligt meget fedt at arbejde i netcompany - og så koder man bare lige dét bedre i skjorte.

\says{S} 
Når jeg er færdig med studiet, så kommer jeg måske ikke til at arbejde med C*-algebraer, men jeg regner med stå ved et whiteboard og regne på nogle komplicerede matricer.
\says{A} 
Jeg arbejder med matricer, fx. så arbejder jeg på en rapport med hele to rækker OG to kolonner.

\says{S}
Jeg HADER bare store globale profitmaksimerende virksomheder. Fx har red bull for 5. år i træk spurgt, om de må komme ud på rusturen - men næ nej - ikke imens jeg er rusvejleder
\scene{A tager en 6-pack red bull frem og knapper en op}
\says{A} 
Ahh - Jeg elsker bare gratis red bull

\says{S} 
Jeg er lige blevet veganer for klimaets skyld - og så
regner jeg med at blive CO2-neutral i blok 3
\says{A} 
Jeg går også meget op i klimaet. Altså dermed med mener jeg indeklimaet her på mit kontor, der skal nemlig være præcist 19.5 grader

%\says{S} 
%Hun er ikke helt dum, hende Pernille Skipper.
%\says{A} 
%Han er ikke helt dum, ham Alex Vanopslagh.

%Mmmmh... jeg elsker leverpostejsmad. 

%Mmmmh... jeg elsker stadigvæk leverpostejsmad. 

\says{S} 
Så var jeg i oslohegnet her i weekenden og jeg tror at jeg fik brækket mig på alle etager på den fucking færge.
\says{A} 
Så var jeg til et kulturelt arrangement, kaldet oslomuren her i weekenden og jeg tror at jeg... øøøh... det var fint.

\says{S} 
Wooow jeg har godt nok mange ugers sommerferie!
\says{A} 
Det har jeg så ikke.

\says{S} 
Så nu er jeg færdig med aflevering, ah hvor dejligt, men nu er det tid til forelæsning mandag morgen. Jeg føler godt nok jeg har travlt her for tiden. Jeg glæder mig til, at jeg i fremtiden får lidt mere energi og tid til at lave nogle af de ting, jeg ikke får plads til i min hverdag
%Jeg glæder mig til, at jeg virkelig for tid til at lave de ting, jeg gerne vil.
\says{A} 
Så nu har jeg ikke længere tømmermænd - ahh, hvor skønt, men nu er det mandag og jeg skal på arbejde. Jeg glæder mig til, at jeg virkelig har tid til at lave de ting, jeg gerne vil.
\says{P} 
Nu har endelig fået en masse tid. Men jeg er sååå træt.
\scene{P dør}
%Så nu er klokken kun 18. Jeg må hellere tage en lur. God nat, sov godt!

\scene Lys ned.


\end{sketch}
\end{document}