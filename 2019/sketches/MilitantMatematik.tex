\documentclass[a4paper,11pt]{article}

\usepackage{revy}
\usepackage[utf8]{inputenc}
\usepackage[T1]{fontenc}
\usepackage[danish]{babel}


\revyname{MatematikRevyen}
\revyyear{2019}
% HUSK AT OPDATERE VERSIONSNUMMER
\version{0.2}
\eta{$n$ minutter}
\status{Færdig}

\title{Militant Matematik}
\author{Ulrik '14}

\begin{document}
\maketitle

\begin{roles}
\role{X}[Buchter] Instruktør
\role{I}[Eigil] Instruktor
\role{R1}[Toke] Rus 
\role{R2}[Felix] Rus 
\role{R3}[Anders] Rus 
\role{N1}[Aurora] $\varepsilon$--Ninja
\role{N2}[Nanna] $\varepsilon$--Ninja
\role{Br}[Johan] Brandvagt
\role{Mic}[Line] Mikrofonansvarlig
\role{BackS}[Nina] Backstage Ansvarlig
\end{roles}

\begin{props}
\prop{Donuts som kaffekopper til hver rus}[Person, der skaffer]
\prop{$\delta$-sværd til hver rus}[Person, der skaffer]
\prop{Kaffemaskine}[Person, der skaffer]
\prop{Skilte med teoremer}[Person, der skaffer]
\prop{Flæskesvær, bøllehat og dannebrogsflag}[Person, der skaffer]
\prop{Fløjte}[Person, der skaffer]
\end{props}


\begin{sketch}
\scene{Lys op.}

\scene{I træder ind på scenen og stirrer olmt ud over publikum. I er klædt i en militariseret instruktoruniform. Der går et øjebliks akavet stilhed, før I samler sin fløjte op og pifter. Russerne stimler ind i militariseret hverdagstøj med $\delta$-sværd og donuts ved siden.}

\says{I} Øvelseshold! Præsentér… Kaffekop!

\scene{Russer holder donuts op.}

\says{I} Hold… kaffepause!

\scene{En kaffemaskine dukker op fra bag/side-tæppet (med en spand i bunden, så vi ikke gør scenen våd), og russerne stimler i kø op til kaffemaskinen, der hælder kaffe lige ned i hullet på deres donuts. Derefter går de tilbage én efter én, gør sig nogle krampagtige bevægelser, hvorefter de hver hiver et skilt frem med en matematisk sætning på. Når de er færdige, fløjter I igen.}

\says{I} Godt så. Øvelseshold: Permutér!

\scene{Russerne giver sig til at bytte plads sådan lidt på må og få.}

\says{I}[arrigt] JEG SAGDE PERMUTÉR!

\scene{Russerne gisper og giver sig i al mulig hast til at bytte tøj. Gør det helst lidt fjollet.}

\says{I}[endnu mere arrigt] KALDER I DET FJOLLET?! MIN GAMLE MOR KUN FINDE PÅ EN SJOVERE PERMUTATION I SØVNE!

\scene{I står og pruster i et par sekunder, før denne tager sig sammen.}

\says{I}[roligt] Okay… Træk jeres $\delta$-sværd… Som straf skal I… parere ti $\varepsilon$’er!
\scene{Russerne klynker.}

\says{I}[voldsomt] EKSPLICIT! Godt så… kom i gang!

\scene{Russerne prøver febrilsk at trække deres $\delta$-sværd, mens I, fra sin plads, og N1 og N2 fra sidetæppet kaster $\varepsilon$’er efter dem. Se på, hvor længe det er sjovt, og stop derefter.}

\says{I} Godt så… øvelseshold… Random… walk!

\scene{Russerne giver sig til at vandre ud i salen på må og få på nær en enkelt, der primært bare render rundt i en firkant. I får øje på dette og tramper langsomt derover med et vredt udtryk i ansigtet. R1 fortsætter rundt i en firkant i lidt tid, inden denne indser, at I er vred. R1 giver et gips.}

\says{I}[vredt] Generisk rus nr. 42! Hvad tror du så lige, du bestiller?

\says{R1}[forvirret] Øh… tilfældigt?

\says{I} Og… hvor langt bør du så være fra start efter $n$ skridt?

\scene{Beat. R1 stirrer blankt ud i luften.}

\says{I}[arrigt] $\sqrt{n}$ din mide! Vor Herre til hest, du går jo praktisk talt deterministisk!

\scene{I sukker og kigger ned.}

\says{I} Der er kun én afstraffelse i mit arsenal, der slår an her.

\scene{Det giver et sæt i rus.}

\says{R1}[desperat] Nej… du mener ikke…

\says{I} Jo… Du skal… integrere!

\scene{R1 giver sig til at hulke.}

\says{R1} Nej… nej…

\scene{R1 tager bøllehat på og giver sig til at dele dannebrogsflag og flæskesvær ud til publikum, mens denne, under hulken og tuden, synger ‘Der Er Et Yndigt Land’.}

\scene{Der går lidt tid, og så fløjter I.}

\says{I} Godt så, øvelseshold. Sidste øvelse er…

\scene{I modtager et skilt med et kommutativt diagram fra sidetæppet.}

\says{I} Et diagram chase!

\scene{I løber af scenen, mens russerne jagter ham. Følges af spot til de forlader døren i midtergangen}


\end{sketch}
\end{document}
