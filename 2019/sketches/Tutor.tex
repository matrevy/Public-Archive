\documentclass[a4paper,11pt]{article}

\usepackage{revy}
\usepackage[utf8]{inputenc}
\usepackage[T1]{fontenc}
\usepackage[danish]{babel}
\usepackage{wasysym}

\revyname{MatematikRevyen}
\revyyear{2019}
% HUSK AT OPDATERE VERSIONSNUMMER
\version{0.1}
\eta{$2$ minutter}
\status{Færdig}

\title{Tutor}
\author{Anne G og Buchter}

\begin{document}
\maketitle

\begin{roles}
\role{X}[Patrick] Instruktør
\role{SE}[Anders] Simon Emil Ammitzbøll-Bille
\role{G}[Gustav] Gordon Ramsey
\role{BS}[Johan] BS Christiansen
\role{BU}[Mikkel] Bubber
\role{S1}[Felix] Studerende
\role{S3}[Toke] Studerende
\role{S2}[Ib] Sort studerende
\role{N}[Manchester] Ninja
\role{N}[Stig] Ninja
\role{N}[Villads] Ninja
\role{Br}[Victoria] Brandvagt
\role{Mic}[AK] Mikrofonansvarlig
\role{BackS}[Simone] Backstage Ansvarlig
\end{roles}

\begin{props}
    \prop{Bord x 2}
    \prop{Stol x 4}
    \prop{Rulletavle}
    \prop{Lampe x 2} som dem i mat-kantinen
\end{props}

\begin{sketch}
\scene{Multiple Choice by night, de kendte (pånær Bubber) er lektiehjælpere. Studerende sidder og venter på hjælp.}
\says{BS} Kom nu, bubber. Det er ikke så svært. Det lærte du på Mat A.
\says{BU} Jamen BS jeg ved ikke hvordan man integrere så stort en ting...
\says{BS} Jamen for helvede, Bubber. Hvis du ikke ved det, så bare slå med terningerne! Det er MC for fanden!
\says{BU} Jamen jeg har ikke nogen terning...
\says{BS} SÅ GÆT BUBBER! GÆÆÆÆT!!!!
\scene{G går hen og smager på noget pizza. Spytter det ud igen.}
\says{G} FØJ for helved - det er jo ikke lavet på stenovn! Og den "kebab"... Er den skidt ud af en ko?
\says{S1} \act{Vinker G hen til sig} Gordon! Så $(a+b)^{2}$ er bare $a^{2}+b^{2}$ ikke?
\scene{G tager to stykker pizza og holder op for S1's ører}
\says{G} HVAD ER DU???
\says{S1} Hvad mener du?
\says{G} HVAD. ER. DUUU?
\says{S1} En idiot (pizza)sandwich.... Jeg tager noget kaffe (ish)
\scene{S1 tager sin kaffekop og prøver at komme ud, men møder så G og ændrer retning. S2 rækker hånden oppe, mens han siger:}
\says{S2} Hey Simon Emil. Kan du lige hjælpe mig med de her Radikaler
\says{SE} Er det det her til højre
\says{S2} Nej, venstre
\says{SE} Hmm. Det skal du bare droppe og gå videre.
\says{S3} Simon Emil så kan du måske hjælpe mig?
\says{SE} Ja hva så min borgerlige ven
\says{S3} Jeg mangler mit centrum
\says{SE} Ja det skal du ikke bruge tid på. Bare skynd dig videre til det næste.
\says{SE}  Bubber! Kan jeg hjælpe dig?
\says{BU} Jeg føler mig bare så alene. 
\says{SE} Ja det kender jeg godt \smiley
\says{BU} Det er også om at alliancen ikke går så godt derhjemme. Måske har det noget at gøre med barnepigen.
\says{SE} Nå men jeg kan sige det samme til dig, som jeg har sagt til dem. Du skal bare gå videre. Det gør jeg hele tiden. Du skal bare tænke FREMAD!

\scene{Tæppe}

\end{sketch}
\end{document}