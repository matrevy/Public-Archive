\documentclass[a4paper,11pt]{article}

\usepackage{revy}
\usepackage[utf8]{inputenc}
\usepackage[T1]{fontenc}
\usepackage[danish]{babel}


\revyname{MatematikRevyen}
\revyyear{2019}
% HUSK AT OPDATERE VERSIONSNUMMER
\version{2}
\eta{$1$ minutter}
\status{Færdig}

\title{Til middag hos forelæserne 3 (Hos Bjergfigur}
\author{Cecilie '17$\frac{1}{2}$, Peifer '17, Sommer '17}

\begin{document}
\maketitle

\begin{roles}
\role{X}[Patrick] Instruktør
\role{B}[Villads] Bergfinnur
\role{E}[Jesper] Ernst Hansen
\role{S}[Eigil] Søren Eilers
\role{N}[Gustav] Ninja
\role{N}[KE] Ninja
\role{N}[Manchester] Ninja
\role{N}[Stig] Ninja
\role{Br}[Anders] Brandvagt
\role{Mic}[Emma] Mikrofonansvarlig
\role{BackS}[Sommer] Backstage Ansvarlig
\end{roles}

\begin{sketch}

\scene{Hjemme hos Bergfinnur. Der står et bord og Bergfinnur er i gang med at dække det da det ringer på døren. Evt. en sjov ringeklokke "Ikke sandt. Ikke sandt. Ikke sandt". Ernst og Søren kommer ind}
\says{B} God dag Ernst, God dag Søren
\says{E} Her bor du da godt nok fint Bergfinnur.
\says{B} Ja det har været et Dyrhuus
\says{S} Sikke et flot billede. Hvad forestiller det?
\says{B} Det er da bare en Bjergfigur.
\says{B} Kan du ikke lige række mig brød\textbf{kurven} Ernst.
\scene{E rækker ham den, og B går ud af bagtæppet}
\says{E} Det er godt nok et lækkert bord.
\scene{B råber ude fra bagtæppet af}
\says{B} Ja jeg har lige købt det. Det er helt glat
\scene{B kommer ind med brød i kurven og sætter sig. S brækker et stykke af og tager en bid}
\says{S} Mhm hvor lækkert. Virkelig sprødt, det har en god krum-
\says{B} KRUMNING. Ja! Ja det en god Gauss krumning. 
\says{S} Ejj, nu er du vidst på vej ud af en tangent
\scene{Lys ned}


\end{sketch}
\end{document}