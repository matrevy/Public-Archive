\documentclass[a4paper,11pt]{article}

\usepackage{revy}
\usepackage[utf8]{inputenc}
\usepackage[T1]{fontenc}
\usepackage[danish]{babel}


\revyname{MatematikRevyen}
\revyyear{2019}
% HUSK AT OPDATERE VERSIONSNUMMER
\version{0.1}
\eta{$n$ minutter}
\status{Færdig}

\title{Den Svære Samtale}
\author{Erik '15, Ulrik '14}

\begin{document}
\maketitle

\begin{roles}
\role{X}[Patrick] Instruktør
\role{F}[Peifer] Far
\role{S}[Marius] Søn
\role{P}[Stine] Lækker pige
\role{N}[Rikke] Ninja
\role{N}[Simone] Ninja
\role{Br}[Johan] Brandvagt
\role{Mic}[Line] Mikrofonansvarlig
\role{BackS}[Nina] Backstage Ansvarlig
\end{roles}




\begin{sketch}
\scene{Lys op.}
\scene{F og S sidder. F ser meget alvorlig ud.}

\says{F} Min kære søn, din mor og jeg har snaget lidt i din søgehistorik, og vi har talt om, at du nu er kommet i en alder, hvor der er nogle ting, vi skal have taget en snak om.

\says{S} Ej, faar.

\says{F} Og det er faktisk det første, jeg ville sige: Det er en smuk og helt naturlig ting, som man ikke skal skamme sig over.

\says{S}[Irriteret] Ååh, du lyder som en pædagog. Kan vi ikke droppe det her?!

\says{F} Nej, det er meget vigtigt at have tænkt det ordentligt igennem før man begiver sig ud i sådan noget. Det er meget… at have lyst og det kan nærmest tilføje en ekstra dimension til livet.

\says{S} Så det, du siger, er, at man bare skal gøre det så meget, man kan og lige, når man har lyst?

\says{F} Nej, min søn. Under de rigtige omstændigheder er det en meget smuk ting, men du må ALDRIG ombytte en dobbeltsum, der ikke er absolut konvergent!!!

\scene{Lys ned.}

\end{sketch}
\end{document}
