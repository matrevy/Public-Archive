\documentclass[a4paper,11pt]{article}

\usepackage{revy}
\usepackage[utf8]{inputenc}
\usepackage[T1]{fontenc}
\usepackage[danish]{babel}


\revyname{MatematikRevyen}
\revyyear{2019}
% HUSK AT OPDATERE VERSIONSNUMMER
\version{0.1}
\eta{$3-4$ minutter}
\status{færdig}

\title{Eva Eventyren}
\author{Anne '11, Michael '12, Line '16, Patrick '13}

\begin{document}
\maketitle

\begin{roles}
\role{X}[Michael] Instruktør
\role{E}[Line] Dora the explorer typen
\role{V}[Peifer] Evas ven
\role{A}[Stig] Gut der køber en billet
\role{N}[AK] Ninja
\role{N}[Felix] Ninja
\role{N}[Jasmin] Ninja
\role{N}[Simone] Ninja
\role{Br}[Julie] Brandvagt
\role{Mic}[Emma] Mikrofonansvarlig
\role{BackS}[Marius] Backstage Ansvarlig
\end{roles}

\begin{props}
\prop{Hammer}[Patrick]
\prop{Paraply}
\prop{Penisattrap}
\prop{Pengekasse} med penge i
\end{props}


\begin{sketch}

\scene{Lys op}

\says{E}[Alt for glad] Ih hvor jeg dog er glad for der snart er Pulefrokost. Jeg skal bare lige hamre et søm i plakaten.

\says{E}Hvad skal jeg bruge til at hamre det her søm i? Er det: Min hammer? Min paraply? Min penisattrap? \act{viser hammeren, paraplyen og penisattrappen}

\says{E}
Tusind tak! Jeg prøver! Det virkede! We did it!
Nu kan vi rigtig nok sælge billetter. Kom med mig - så sælger vi!

\scene{Eva sætter sig hen ved et bord for at sælge billetter. A kommer ind og vil købe en billet.}

\says{E} Så nu kommer der en og køber en billet.

\says{A} Nu kommer jeg og køber en billet

\says{E}
Åh åh - billetten koster kun 80 kr., men manden gav mig 100! Hvor meget skal han have tilbage? Er det 10 kr.? er det 20 kr.? eller er det alle pengene i kassen? \act{laver fagter med pengene}

\says{E} Sådan! Tak! Thank you! Nu er manden glad!

\says{A} Nu er jeg glad

\scene{A går af scenen. Eva går videre og møder sin ven; Vinter.}

\says{E}
Hej Vinter !

\says{V} Hello Eva !

\says{E}Hvad er nu det? Kan du ikke finde vej ud? 

\says{V} I cannot find way

\says{E} Hjælp mig med at vise Vinter vej! Er det denne vej?
\act{Eva peger på en af indgangene til scenen.}
\says{E} Er det denne vej? \act{Eva peger på udgangen af scenen}
\says{E} Eller er det denne vej ?
\act{Eva peger ud over publikum}

\says{E} Super! 
\says{V} Precisely !

\act{Vinter begynder at gå ud af den valgte vej.}

\says{E}  Se børn, så længe vi hjælper hinanden, så skal det nok gå !

\says{E} Vi ses, Vinter!
\says{V} We cees !

\act{Vinter går ud}

\says{E} Åh åh ! Manden her ser jo helt bedrøvet ud. Hvordan skal vi gøre ham i godt humør igen?
\act{peger på en gut i publikums øl}

\says{E} Skal jeg klappe ham på hovedet? Eller kilde ham? Eller drikke hans øl?
\act{Eva drikker hans øl (formentlig)}

\says{E} Nammenammenam! Det gjorde godt i mavsen.
Men hov ! Nu har manden jo ingen øl. Hvad skal vi så gøre ? Jeg har det ! Så giver vi ham bare sidemandens øl ! 
\act{Eva giver ham sidemandens øl og går tilbage på scenen}

\says{E}Åh nej hvor er det svært. Hvordan skal man nu slutte det her eventyr? Hjælp mig alle sammen. Skal jeg bruge et flot buk? Eller min penisattrap?

\scene{Lys ned}

\end{sketch}
\end{document}
