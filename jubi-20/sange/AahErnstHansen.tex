\documentclass[a4paper,11pt]{article}

\usepackage{revy}
\usepackage[utf8]{inputenc}
\usepackage[T1]{fontenc}
\usepackage[danish]{babel}

\revyname{Matematikrevy}
\revyyear{2019}
\version{1.0}
\eta{$2.5$ minutter}
\status{Færdig}

\title{Åh Ernst Hansen}
\author{Sommer '17}
\melody{The Chordettes -- Mr. Sandman}

\begin{document}
\maketitle

\begin{roles}
\role{S1}[Jeppe] Sanger
\role{S2}[Line] Sanger
\role{S3}[Rikke] Sanger
\role{S4}[Sommer] Sanger
\role{E} [Jesper] Ernst
\end{roles}

\begin{song}
\sings{Alle} Åh Ernst Hansen, giv mig nu 10
Eller et 12-tal, det dét jeg kan li'
Analyse 0, det klarede jeg så flot
og synes selv bacheloren gik ret godt

\sings{Alle} Censor, synes jeg var god
jeg havde øvet, og samlet mig mod
Please udlev min fantasi
Åh Ernst Hansen, giv mig nu 10
 
\sings{Alle} Åh Ernst Hansen, giv mig nu 10
For jeg er flittig, og mangler ik' pli
Jeg tøvede ik', og nailede mit forsvar
Jeg klarede det hele uden at du skulle hjælp mig

\sings{Alle} Hansen, vil du ik' nok?
Jeg ryster jo, fra hoved til sok
så Jeg vil bare høre dig sige
"Tillykke her får du 10"

\sings{Alle} Åh Ernst Hansen, giv mig nu 10
Elsker forresten, din bog i MI
Jo uden tvivl, den bedste mat-bog
Man læser den og råber "Wow hvor' han klog!"

\sings{Alle} Kom nu, kan ik' vente mer'
Er I ik' færdige med at votér?
Jeg har ikke mere at sige
Åh Ernst Hansen, giv mig, please, please giv mig
Åh Ernst Hansen, giv mig nu 10
\end{song}

\scene{Kvartetten er færdige og står med jazzhands. Først glade, så nervøse og ryster i barbershop-pose. E kommer ind med en stor blok papir. Skriver først 0, lader som om han skal til at skrive skrive 10, og skriver så 00, holder skiltet op til publikum og går ud. Kvartetten bliver kede af det, bandet spiler “wham wham wham whaaaaam” - kvartetten tager hattene ned en af gangen på hvert “wham” og går slukøret ud af scenen. S2 stopper op, tager sin telefon. Bagtæppet går op, og AN0-Eksamen starter med dialog mellem S2 og Sanger.}

\end{document}
