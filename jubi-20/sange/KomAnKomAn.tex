\documentclass[a4paper,11pt]{article}

\usepackage{revy}
\usepackage[utf8]{inputenc}
\usepackage[T1]{fontenc}
\usepackage[danish]{babel}

\revyname{Matematikrevy}
\revyyear{2006}
\version{1.0}
\eta{$3.5$ minutter}
\status{Færdig}

\title{KomAn KomAn}
\author{Ukendt}
\melody{Frank Sinatra -- New York, New York}

\begin{document}
\maketitle

\begin{roles}
\role{YD}[Nanna] Koreograf / Danser
\role{S1}[Jeppe] Sanger / Danser
\role{S2}[Simone] Sanger / Danser
\role{D2}[Frida] Danser
\role{D3}[Rune] Danser
\role{D4}[Sanne] Danser
\role{D5}[Stig] Danser
\end{roles}

\begin{song}
\sings{S1} Cauchy er for sej,
han er lige mig.    
Det er derfor jeg si'r til ham:
KomAn! KomAn!

Vores verden er sær,
og imaginær.
Syng med på den her, allesammen:
KomAn! KomAn!

\sings{S1 + S2}
Vi differentierer hver en holomorf funktion.
Vi går uden om en pol, ved integration.      

\sings{S2}
De reelle tal 
--- en akse der' smal.
Vi vil ha' mere rod end det.
KomAn! KomAn!  

Gi' mig en... Akse til,
for jeg' den... Slags der vil
ha' meget mer'
KomAn! KomAn!   

KomAn! KomAn!

\sings{S1 + S2}
Vi differentierer hver en holomorf funktion.
Vi går uden om en pol, ved integration.
Smider den væk! (Sig det ik' til nogen!) 

\sings{S1}
Det er elegant, 
at den er konstant
--- er den reel og holomorf,
KomAn! KomAn!

En række... Med potens,
ka' også ha'... Konvergens.
Det' trivielt,
KomAn! KomAn! 
\end{song}

\end{document}
