\documentclass[a4paper,11pt]{article}

\usepackage{revy}
\usepackage[utf8]{inputenc}
\usepackage[T1]{fontenc}
\usepackage[danish]{babel}

\revyname{MatematikRevy}
\revyyear{2016}
\version{1.0}
\eta{$2$ minutter}
\status{Færdig}


\title{Radio Tauto 1}
\author{Mikkel '11, Mathias '11, Ulrik '14 \& Jakob '11}

\begin{document}
\maketitle

\begin{roles}
\role{X}[NB] Instruktør
\role{V}[Kasper] Vært
\role{S}[Lui] Alle andre roller
\end{roles}

\begin{sketch}

\says{Band}[Bid fra jingle] Radio tautooo.

\says{S} Radio Tauto, vi taler sandt, medmindre vi lyver.

\says{V} Godaften, medmindre det er morgen. Velkommen til Radio Tauto, sandhedes stemme. Mit navn er Karsten Løgn og vi har et program. Vi vil starte med det første, så derefter tale om det mellemste, for til sidst at slutte af med det sidste. Og nu til sporten der idag stiller skarpt på fysiske aktiviteter.

\says{Band}[Bid fra jingle] Radio tautooo.

\says{S} Vinderen af superligaen kom på førstepladsen, topscoren scorede flest mål og Caroline Wozniacki har tabt endnu en tennis finale. \act{Kunstpause} Udsnit fra dagens golfturnering, de små kugler landede på greenen, medmindre de ikke gjorde. Videre til bowling hvor vinderen tog kejler. Vinderen af det danske agility mesterskab var en hund og skak er fortsat ikke en rigtig sport. %Og ingen fulgte med i VM i hundrede meter fri.

\says{Band}[Bid fra jingle] Radio tautooo.

\says{V} Tak til sporten. Og nu har vi et lille kulturelt indslag. Vi har en gæst der er på besøg. Den gæst hedder sit navn, og hvad er det?

\says{S} Hvad er et navn andet end noget vi kalder os selv. Ofte både tænker og er jeg, undtagen når jeg ikke er. Derfor har jeg skrevet et digt: \act{rømmer sig, det næste siges langsomt med lange kunstpauser}: En rose, er en rose, er en blomst.

\says{Band}[Bid fra jingle] Radio tautooo.

\says{V} Tak til vores gæst, der sagde ting. Vi slutter af med at lytte til klassikeren "Den Jeg Elsker, Elsker Jeg".

\scene{Slut.}

\end{sketch}
\end{document}




