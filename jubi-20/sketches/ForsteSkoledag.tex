\documentclass[a4paper,11pt]{article}

\usepackage{revy}
\usepackage[utf8]{inputenc}
\usepackage[T1]{fontenc}
\usepackage[danish]{babel}

\revyname{MatematikRevy}
\revyyear{2014}
\version{1.0}
\eta{$5$ minutter}
\status{Udkast færdigt}

\title{Første Skoledag}
\author{William, Sofie, Peter og Jasmin}

\begin{document}
\maketitle

\begin{roles}
\role{X}[Mads] Instruktør
\role{M}[Maja] Mentor
\role{R1}[Frigg] Rus 1 - Siger ikke noget
\role{R2}[Toke] Rus 2 - Har hikke
\role{R3}[Sommer] Rus 3 - Stiller kloge/svære spørgsmål
\role{R4}[Elinborg] Rus 4 - Ultra dum rus, ikke så vild med x, vil sætte tal ind
\end{roles}

\begin{props}
\prop{Bord}[done]
\prop{4 Stole}[done]
\prop{Evt. Reol}[Fyldt med MatIntro-bøger]
\end{props}
  
\begin{sketch}
\scene{Det er et bord med mad og drikke og 4 stole på scenen. Lys op. Mentor alene på scenen. M går rundt og er perfektionistisk. Han er klædt ud i tema.}

\says{M} Boller - tjek. Te og kaffe - tjek. En leg, så det ikke bliver akavet - tjek. Det kan bare ikke gå galt.

\scene{DING DONG. M oppe at køre, klapper i hænderne.}

\says{M} Så går det i gang!!

\says{M (til dørtelefonen)} VELKOMMEN TIL!!! Det er hoooos Maja - skal DU have noget morgenmad?

\scene{Stilhed.}

\says{R1} Er det her der er matematik...?

\says{M} JA! Jeg lukker dig ind nu.

\scene{R1 kommer ind.}

\says{M} Velkommen! Kaffe?

\scene{M anbringer R1 på den dertilhørende stol 3}

\says{R1} Øhh, øhh, nej. Nej tak.

\says{M} Vi har også lækker Juice?

\says{R1} Ja, øhhja, nej tak.

\says{M} Men du kan også bare tage nogle boller, hvis du ikke er tørstig!

\says{R1} JA TAK! - Men der er vel ikke gluten i? 

\scene{Pinlig tavshed. Efter lidt tid ringer det på døren.}

\says{M} Hov, hvem kommer der?

\says{M (til dørtelefonen, lidt panisk)} Hej, godt du kom! Kom ind, kom ind!

\says{R2} *Hik* Ja tak *hik*

\says{M (til R1)} Så kommer en af dine medstuderende, som du skal møde!

\scene{R1 virker ikke begejstret og læner sig væk fra R2. R2 kommer ind. Prøver at sætte sig på stol 4, men M anbringer hende på stol 2.}

\says{R2} *hiik* hej *hiik*

\says{M} Hov, du har da vist fået hikke, tag du noget at drikke her så burde det gå væk!

\scene{R2 forsøger at drikke, og spilder imens hun hikker.}

\says{R2} Hov... *hik*

\says{M} Øhh, hvad med at holde vejret?

\scene{R2 holder vejret men hikker stadig.}

\says{M} Nå, men øhh, så må du jo, øh, op og have hovedet mellem benene.

\scene{M hjælper med stor besvær R2 op med hovedet mellem benene. Det virker ikke.}

\says{M (fortsat)} Ehh, jamen så IMENS, at du drikker!

\scene{M giver R2 glasset med vand, og hjælpe ham med at drikke. DING DONG.}

\says{M} Hov, øhh, og så husk også at holde vejret...

\scene{M skynder sig hen til døren. R2 vælter. Tager dørtelefonen.}

\says{M (til dørtelefonen)} Det er hos Maja.

\scene{R2 er lige kommet op.}

\says{R1} BØH!!!!

\scene{R1 forskrækker R2, som stivner af skræk.}

\says{M} Argh!

\scene{Tydeligt også forskrækket, lukker bare R3 ind, og går hen for at hjælpe R2. R3 kommer ind og sætter sig på stol 4. M er tydeligt generet over, at R3 sidder på den forkerte stol.}

\says{R3} Du har læst matematik et stykke tid, ikke?

\says{M} Jo.

\says{R3} Og du har haft en masse kurser, ikke?

\says{M} Jo, jo!

\says{R3} Har du haft et kursus om uendelige rækker?

\says{M} Ja!

\says{R3} Og du er god til det der matematik, ikke?

\says{M} Jo, rimelig!

\says{R3} Hvad er så potensrækken for $\sin(x)$?

\scene{Mentor går i panik og begynder at finde bøger frem fra reolen og bladrer febrilsk i dem.}

\says{M} $\sum_{n=0}^\infty \frac{(-1)^n\cdot x^{2n+1}}{(2n+1)!}$

\says{M} Den uendelige sum over minus et i n'te - gange x i to n plus første - delt med to n plus en fakultet. Du kan selv se det hér!

\scene{M giver R3 en bog.} 

\says{M} Og jeg har da vidst glemt, at I skal have hatte på.

\scene{M uddeler stråhatte til alle.}

\says{M (til R2)} Og der er da vist kommet styr på din hikke!

\says{R2} *hik*

\says{R1} *atju*

\says{R2} *hik*

\says{R1} *atju*

\says{R2} *hik*

\says{R1} *atju*

\says{R3} Hvad så med $\sin^2(x)$?

\says{R1} Der er ikke hø i den her stråhat vel?

\scene{Mentor panikker igen. DING DONG.}

\says{M (til dørtelefonen)} Øh, hø, halm, jeg ved det ikke; du sætter bare det hele i anden... Hov, undskyld, kom ind!

\says{M} Du sætter dig bare ned.

\says{R4} Det der matematik vi skal til at læse, ik’? Det er ikke sådan noget abstrakt noget, vel? - Med mange ubekendte og sådan?

\scene{M cringer, laver høj lyd.}

\scene{R4 sætter sig på stol 1}

\says{R1} Hvor gammel er den her fabrik egentlig?

\says{R3} Gælder det for alle x?

\says{R4} Er det et problem, at jeg ikke kan norsk?

\says{R2} Er det et problem, at jeg kun kan norsk.

\says{R3} Hvad så med $\sin^2(y)$?

\scene{R2 hikker.}

\says{R1} Altså Maja, var de her boller glutenfri eller ej???

\says{R4} What, Maja? Jeg skulle jo have været hos Michael! 

\scene{R4 går ud. M er i panik. DING DONG. M begynder at bryde sammen.}

\says{M (til dørtelefonen)} Lad mig være!!! Gå væk.... Men husk at meld dig ind i facebookgruppen.

\end{sketch}

\end{document}
