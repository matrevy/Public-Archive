
\documentclass[a4paper,11pt]{article}

\usepackage{revy}
\usepackage[utf8]{inputenc}
\usepackage[T1]{fontenc}
\usepackage[danish]{babel}

\revyname{MatematikRevy}
\revyyear{2013}
\version{1.0}
\eta{$4$ minutter}
\status{udkast}


\title{Rho I Retten}
\author{William \& Michael Staal-Olsen}

\begin{document}
\maketitle

\begin{roles}
\role{X}[MaWeK] Instruktør
\role{H}[Kasper] Høje Censor
\role{S}[Louise] Studerende
\role{E}[Frigg] Eksaminator
\role{F}[Niklas] Funktion
\end{roles}

\begin{sketch}

\says{S} Jah… øhh… Jeg har så trukket spørgmål 1: Hovedsætningerne om kontinuerte funktioner. Og øhh… Den første siger jo at en kontinuert funktion, der er defineret på en lukket og begrænset mængde har en mindste og en største værdi.

\says{E} \act{Ernst lyd} Men kan du bevise det?

\scene{Dramatisk overgang, hvor censor tager dommertøj på og scenen bliver som en retsal.}

\says{H} Rho i RETTEN!

\says{H} Lad F være givet!

\scene{F sætter sig i skranken.}

\says{S} Høje censor – anmoder tilladelse til at undersøge F

\says{H} Givet.

\says{S} Hvad er dit navn?

\says{F} Tilde, … F. Tilde.

\says{S}  Hvad laver du til hverdag?

\says{F} Jeg arbejder bl.a. på afsluttede og begrænsede mænder.

\says{S} Så du er kirurg

\says{F} Nej, nej. Jeg operer ikke kun i snit.

\says{S} okay, men er du veldefineret.

\says{F} Selvfølgelig tror du jeg er ny inden for feltet? %Ny på settet

\says{S} Hvad er din forskrift?

\says{E} Protest! Det er ikke relevant for rettergangen.

\says{H} Godtaget.

\says{S} Hvor var du den 21. Oktober i tidsrummet mellem a og b?.

\scene{Vis graf med intervalpunkter.}

\says{F} Jeg var lidt rundt omkring. Måske antog jeg mit supremum, eller også gjorde jeg ikke. Jeg kan ærligt tal ikke huske det.

\says{S} Du antog dit supremum, gjorde du ikke. Du tog supremummet og så antog du det – I KOLDT BLOD – Indrøm det!

\says{E} Protest! 

\says{H} Du må ikke tale sådan til vidnet.

\says{S} Betragt bevismateriale A, computer, zoom ind på gerningsstedet. Computer, forbedr billedkvaliteten! Computer, gentag processen.

\scene{På storskærmen: billede af en funktion, der går højt op, så den måske når uendeligt. På $[a,b]$, zoomer ind på $[a_1,b_1]$, hvor det høje sted er, men grafen er uklar, bliver derefter klar. Zoomer så ind på $[a_2,b_2]$ og forbedrer og så på $[a_n,b_n]$ og forbedrer.}

\says{S} Der findes altså et $x_0$, som ligger i alle disse intervaller og…

\says{H} Hvilket $x$ er det?

\says{S} Det kan jeg desværre ikke afsløre pga. Vidnesbeskyttelsesprogrammet.

\says{F} Hvem er det? Er det $x_i$, det lille stikker-svin.

\says{S} Der eksisterer altså et $x_0$, som…

\says{E} Protest, der bliver brugt lemma!

\says{H} Hvad har du at sige til det?

\says{S} Høje censor, det… øhh… det har vi vist til en øvelsestime!

\says{H} Godtaget.

\scene{E, F surmuler}

\says{S} Og da F er kontinuert, vil … Er du egentlig kontinuert?

\says{F} … Ja, det er jeg faktisk, det ligger til min familie.

\says{S} (konkluderende) Så F er kontinuert, lukket, begrænset. Og vi kan jo tydeligt se vha. materiale A og Ruselemmaet er F er skyldig. Det var i øvrigt alt det jeg havde på disse her papirer, som jeg ikke må kigge på og vi kunne ikke være mere færdige og er F antage supremum i $x_0$ og dermed er vi færdige og sætningen er bevist! Quod Erat Demonstrandum!

\says{H} Unge dame. Det var jo kun den ene halvdel af beviset.'

\says{S} (Falder fortvivlet sammen)

\end{sketch}
\end{document}
