\documentclass[a4paper,11pt]{article}

\usepackage{revy}
\usepackage[utf8]{inputenc}
\usepackage[T1]{fontenc}
\usepackage[danish]{babel}

\revyname{Matematikrevy}
\revyyear{2018}
\version{2.0}
\eta{$1$ minutter}
\status{Færdig}

\title{En Velordnet Sætning}
\author{Ulrik '14}

\begin{document}
\maketitle

\begin{roles}
\role{X}[Mads] Instruktør
\role{Sp}[Mikkel] En meget, meget træt semi-ansvarlig
\end{roles}

\begin{props}
\prop{Rekvisit}[Person, der skaffer]
\end{props}

\begin{sketch}
\scene{Lys op.}

\scene{En enlig person træder ind på scenen. Et teorem bliver sat op på projektor. Den enlige person har et Adidas tracksuit på, hvor der tydeligt står AADDIS. På scenen ligger en bunke rekvisitter i regnbuens farver. Personen sorterer dem efter bølgelængde og er tydeligvis tilfreds bagefter. Personer ser så teoremet på projektor og er utilfreds igen.}

\says{P} $A$ af Antag antisymmetrisk at betingelser delmængde element en en Enhver er er et et følgende første har i ikke-tom Lad lig-med medfører mindre-end mindre-end mindre-end mindre-end mindre-end mindre-end mængden og opfylder relation Så to velordning være $x$ $x$ $x$ $X$ $X$ $y$ $y$ $y$

\scene{Det velordnede teorem dukker op på projektor. Den enlige person bukker.}

\scene{Lys ned.}
\end{sketch}

\end{document}
