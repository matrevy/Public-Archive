\documentclass[a4paper,11pt]{article}

\usepackage{revy}
\usepackage[utf8]{inputenc}
\usepackage[T1]{fontenc}
\usepackage[danish]{babel}

\revyname{Matematikrevy}
\revyyear{2021}
\version{1.0}
\eta{$3.5$ minutter}
\status{Færdig}

\title{Vandalyse}
\author{Sommer '17, med inspiration fra Fysikrevyen}

\begin{document}
\maketitle

\begin{roles}
\role{X}[Mads] Instruktør
\role{E}[Niklas] Ernst Vandsen
\end{roles}


\begin{sketch}
\scene{Lys op}

\scene{Ind kommer E, med hawaiiskjorte, surfer-halskæde, shorts og slippers.}

\says{E} Velkommen til Vandalyse 0!

\says{E}[E begynder at skrive sit navn på tavlen] Mit navn er Ernst Vandsen, og det er mit job at vise jer, hvordan matematik er det vådeste studie!

\says{E} Det er godt at se et helt fyldt Hav-ditorie 1. Men det er måske ikke så overraskende efter at se karaktersnittet på Alge-bra med Nathalie Hval, at folk nu kommer STRØMMENDE til dette hval-fri kursus.

\says{E} Jeg er overbevist om at I vil finde dette kursus spændende, og for de rigtig fugtige, hov jeg mener dygtige, så håber jeg selvfølgelig på at se jer igen senere til enten VidSand 1 eller 2 eller til Ål og integralteori.

\says{E} Nå, i dag skal I lære om krebsilon-delta beviser. Og så tænker I sikkert "jamen Ernst Vandsen, det har vi jo styr på! Det fik lært af Jesper Gro-hval og Mor-Tun Riisasger". Men det kan I godt glemme, for her vil vores gæsteforelæser Mikael Rør-Dam nemlig præsentere jer for alt godt fra Havs-dorff rum!

\says{E} Nu hvor vi er i gang med ting I skal glemme, så glem alt om hvordan I har differentieret og integreret fra gymnasiet. Her i mit kursus skal I lære om hvordan man gør det på den smukkeste måde. . . Åhh Le-Bæk integration. . .

\scene{Lys ned}
\end{sketch}

\end{document}
