\documentclass[a4paper,11pt]{article}

\usepackage{revy}
\usepackage[utf8]{inputenc}
\usepackage[T1]{fontenc}
\usepackage[danish]{babel}

\revyname{Matematikrevy}
\revyyear{2023}
\version{1.0}
\eta{$7$ minutter}
\status{Færdig}

\title{Studenterservice}
\author{Sommer '17, Stine '18 og Snow '19}

\begin{document}
\maketitle

\begin{roles}
\role{X}[Mads] Instruktør
\role{S1}[Stine] Legeglad studenterservice
\role{S2}[Sommer] Legeglad studenterservice
\role{Std1}[Kasper] Studerende der ringer (CBS'er der skal have merit)
\role{Std2}[Elinborg] Studerende der ringer (Problemer med KU-mail)
\role{Std3}[Louise] Rus/Studerende der ringer (Ked af det)
\end{roles}

\begin{props}
\prop{Lykkehjul}
\prop{"Studenterservice"-skilt}
\prop{"$+$ Ku IT-support"-skilt}
\prop{"$+$ Administration"-skilt}
\prop{2 borde + stole}
\prop{2 drejeskivetelefoner}

\end{props}


\begin{sketch}

\scene{S1 og S2 sidder ved hvert deres bord med en fastnetstelefon. Der står et studenterservice-skilt foran det ene bord. En af
telefonerne ringer}
\says{S1}[Tager telefonen] Studenterservice, det er S1
\scene{Når der er én der ringer, så står de oppe på bandscenen og der kommer først spot på dem når de kommer igennem, og lyset fjernes i mens de er muted.}
\says{Std1} Hej du snakker med Std. Så jeg har taget nogle kurser over på CBS,
og ville høre om jeg kunne få merit for dem?
\says{S1} Aha, merit siger du.... 2 sekunder, jeg spørger lige min kollega (Muter
telefonen, og lægger den ned)
\says{S1} Hey S2, jeg har en studerende der spørger om noget merit...
\scene{Begge bryder ud i grin}
\says{S2} Okay okay, hvad skal vi gøre med ham?
\says{S1} Skal vi hente hjulet?
\says{S2} Hent hjulet!
\scene{S1 løber ud og henter Ulykkeshjulet ind på scenen}
\says{S1} Vil du gøre det?
\says{S2} Det ville være min ære
\says{Begge} SPIN. THAT. WHEEL!
\scene{S2 spinner hjulet, og det lander på en vilkårlig farve}
\says{S2} *Indsæt farven de landte på*! Det betyder
\says{Begge} BORGERSERVICE!
\says{S1}[Tager telefonen op igen] Det er faktisk slet ikke os du skal snakke med, men derimod Borgerservice
\says{Std1} Borgerservice???
\says{S1} Ja frygt ikke, jeg viderestiller dig med det samme, jeg håber du har
din lægeerklæring klar
\says{Std1} LÆGEERKLÆRING?
\says{S1} (Lægger telefonen på)
\says{S2} Ej... (Lidt fnisende) Lagde du bare på?
\says{S1} (Opfører sig som en 5-årig der har gjort noget de ikke måtte og er
sluppet afsted med det. Nikker tilfreds) Mhm!
\says{S2} Ej du så slem! Ej du skal lige høre hvad der skete i sidste uge. Så jeg havde den her kandidatstuderende der ringende ind. Så mens han
skrev speciale var han kommet til at tabe sit studiekort, og ville høre om han kunne få et nyt.
\says{S1} Ej, hvad gjorde du så?
\says{S2} Jamen jeg spinnede da hjulet!
\says{S1} Ej hvad landte det på?!
\says{S2} Drop ud!
\says{S1} Jeg troede, vi havde fjernet det felt!
\says{S2} (Kunstpause, så lidt fnisen) Jeg tilføjede det igen!
\says{S1} Og du siger JEG er slem?
\says{S2} Ja det er 6 år af hans liv han aldrig får igen!
\scene{De griner, og telefonen ringer}
\says{S1}(Griner færdigt, og putter sit seriøse ansigt på og tager telefonen)
Studenterservice, det er S1

\says{Std2} Ja hej du snakker med Std2. Så jeg har nogle problemer med KU-mail, jeg kan faktisk ikke komme ind på det?

\says{S1} Aha, KU-mail, ja så skal du snakke med IT Support (Sender videre
over til S2)
\scene{S2 sætter et "$+$ IT-supportskilt" op ved siden af studenterservice-skiltet.}
\says{S2} (Laver fake dyb stemme) IT Support det er Lars
\says{Std2} Ja jeg kan ikke komme ind på min KU-mail

\says{S2} Yes, klassisk problem, jeg spørger lige... øhm.. Serveren... (Muter te-
lefonen og lægger den ned)

\says{S2} Hey S1
\says{S1} S2?
\scene{De kigger hen på hjulet og så på hinanden og udbryder}
\says{Begge} SPIN THAT WHEEL
\scene{S1 spinner hjulet, og det lander på et nyt felt. Hvis det lander på feltet
fra før, så rykker de det lige 1 felt på en lidt sjov måde}
 *Indsæt farven de lander på*!
\says{Begge} ABSURDE ÅBNINGSTIDER
\says{S2} Hej igen... Så det viser sig faktisk, at vi slet ikke har åbent lige nu, så
vi kan desværre ikke hjælpe dig.
\says{Std2} Jamen, jeg snakker da med dig lige nu!

\says{S2} Beklager virkeligt, prøv igen næste torsdag mellem 10:30 og halv elleve, hvis det da ellers er fuldmåne. Ellers kan du kontakte os på mail, men du skal være klar over, at vi kun svare hvis du sender fra din
KU-mail
\says{Std2} Jamen det jo derfor jeg ringer, jeg kan jo ikke komme ind!

\says{S2} Håber vores hjælp har været tilstrækkelig og tilfredsstillende hej he-
eeeej (Lægger telefonen på mens der bliver sagt hej heeej)

\says{S2} Ahhh...
\says{S1} Apropos åbningstider, klokken er snart 12
\says{S2} Og det er tirsdag... Du ved hvad det betyder!
\says{Begge} Weekend!
\says{S1} Så hvad skal du i weekenden?
\says{S2} Så jeg skal-(Bliver afbrudt af telefonen der ringer)
\scene{Telefonen ringer}
\says{S2} Ej helt seriøst (Vinker S1 over til sig og tager telefonen op, og bruger
fake computerstemme) Du er nummer... (Peger telefonen over til S1)
\says{S1} (Med anderledes, men også "fake"stemme) ... 20... (Telefonen kommer
tilbage til S2)
\says{S2} ... i køen. (Lægger langsomt telefonen fra sig, mens han synger/nynner
hvad der minder om dårlig ventetids musik)
\says{S2} Nå, hvor kom jeg fra? Når ja, weekenden! Så jeg skal-(Bliver afbrudt
af telefonen igen)
\scene{Telefonen ringer}

\says{S1} (Lidt opgivende) Vi må nok hellere... (Tager telefonen) Studenterservice, det er S1.

\says{Std3} (Hulkende og gerne lidt stammende, lidt 5-årig der ikke kan finde sin mor) He-he-heeeej, så det bare fordi ik, at jeg er lige startet på matematik, og jeg er mega glad for det ik? Men så glemte jeg at svare på min studiestartsprøve, og nu siger min vejleder at jeg bliver smidt
uuuuuuud (Græder lidt videre)
\says{S1} Nååårhhh
\says{Std3} (Stadig hulkende) Vil I ikke nok hjælpe miiiig?
\says{S1} Jo da, du har ringet det helt rigtige sted. Jeg spørger lige min kollega
om hvad vi skal gøre med dig (Lægger telefonen fra sig, men glemmer
at mute)
\says{S1} O. M. G. Det er en rus! Og han græder!
\says{S2} No waaaay!
\says{S1} Skal vi introducere ham til hjulet?
\says{S2} (Går over til hjulet, og kigger tilbage på S1) Vi skal jo tage imod de nye
med manér!
\says{Begge} SPIN. THAT. WHEEL!
\scene{S2 skal til at spinne hjulet, men bliver afbrudt af, at det viser sig
russen har hørt med hele tiden, og nu prøver at bryde igennem}
\says{Std3} Hallooooo? Hvad er det for et hjul? Skulle I ikke hjælpe mig???
\scene{S1 og S2 kigger panisk på hinanden. Det her er ikke sket før. S1 går over og tager halvfnisende telefonen op igen}
\says{S1} Øhm..
\says{Std3} Det slipper I ikke afsted med det her! Jeg klager til administrationen!
\scene{S1 og S2 kigger bedrevidende på hinanden. S2 tager et "$+$ administration"-skilt frem samtidig med at S1 svarer:}
\says{S1}[Smilende] Hvem tror du, at du snakker med?

\scene{Lys ned}


\end{sketch}

\end{document}
