\documentclass[a4paper,11pt]{article}

\usepackage{revy}
\usepackage[utf8]{inputenc}
\usepackage[T1]{fontenc}
\usepackage[danish]{babel}

\revyname{MatematikRevy}
\revyyear{2014}
\version{1.0}
\eta{$3.5$ minutter}
\status{Udkast færdigt}

\title{Historien om Rusmus}
\author{Jakob, Jasmin og Rasmus}

\begin{document}
\maketitle

\begin{roles}
\role{I}[Jasper] Instruktør
\role{S}[Marcus] Speaker
\role{R}[Alexander] Rusmus - Ser dum ud
\role{O}[Kristian] Den Onde Rusvejleder
\role{G}[Marianne] Den Gode Rusvejleder
\role{N1}[MaWeK] Ninja med Auditorium 1 skilt
\role{N2}[Sofie] Ninja med Auditorium 1 skilt
\role{N3}[Johanna] Ninja med Auditorium 1 skilt
\role{F}[Brandt] Forelæser
\role{C1}[Peter] Caféen?-gænger
\role{C2}[Malthe] Caféen?-gænger
\end{roles}

\begin{props}
\prop{Skolerygsæk til R}[]
\prop{Kommer senere...}[]
\end{props}
  
\begin{sketch}
\says{S} Her er Rusmus. Han er lige startet og skal til sin første forelæsning i Matintro.

\says{R} Nøøøøøj, jeg har glædet mig til det her heeeeeele sommerferien.

\says{S} Rusmus gik lidt rundt i vandrehallen.

\says{R} Gad vide hvor Auditorium 1 ligger?

\scene{R går direkte forbi Aud 1. R kigger ud på publikum og peger.}

\says{R (fortsat)} Der er kantinen. Der er godt nok mange mennesker; der er slet ikke plads til mig. Det er lidt, som om, de alle sammen kigger på mig.

\scene{R Står og kigger lidt rundt.}

\says{S} Rusmus gik lidt rundt og ledte efter Auditorium 1.

\scene{R går rundt og leder på scenen. Han kigger ovre ved bandscenen, under en monitor og bag et tæppe i højre side af scenen.}

\says{R (mens han kigger bag tæppet)} Nej, der står et træ.

\scene{Mens R siger sin replik, sniger O sig ind fra venstre og stiller sig bag Rusmus.}

\says{S} Pludselig så Rusmus en af sine rusvejledere.

\scene{R vender sig om, opdager O og bliver forskrækket.}

\says{O} Hej rus, leder du efter noget?

\says{R (lidt skræmt)} Hej vejleder. Ja, jeg skal have min første forelæsning. Men jeg kan ikke finde Auditorium 1.

\says{O} Nåååååh. Nu skal jeg vise dig hvor du skal gå hen.

\scene{Griber fat i R og trækker ham med.}

\says{S} Men Rusmus vidste ikke at han var blevet fundet af en ond vejleder.

\scene{Der lyder torden, og O ler ondt. Pludselig tavshed}

\says{O} Jeg er ond.

\says{S} Vejlederen tog Rusmus med hen til en lille og mørk hule, hvorfra der kom mærkelige lyde.

\scene{O leder R over imod venstre side af scenen, hvor Cafeen?-baren står.}

\says{O} Velkommen til Caféen?.

\scene{Ond latter fra O. Træder om bag ved R. (evt. torden-lydeffekt.) R kigger ind i hulen.}

\says{R (nervøst)} Ligger Auditorium 1 derinde?

\scene{O skubber R ind på Caféen.}

\says{R} Hvorfor er der sære hulemalerier på væggen? Hvad er det for nogle syge mennesker der bor her?

\says{O} Det er her de rigtige studerende holder til.

\says{S} Rusmus blev meget bange og løb skrækslagent tilbage imod HCØ.

\says{R} Det var godt nok et uhygge(r)ligt sted. Der vil jeg aldrig tilbage igen.

\scene{O sniger sig ud i venstre side.}

\says{S} Rusmus ledte hele vandrehallen igennem.

\says{R} Jeg kan bare ikke finde Auditorium 1.

\says{S} Men pludselig opdagede han et skilt.

\says{R} Der står noget med Auditorie.

\scene{R går langsomt mod skiltet.}

\says{S} Men den onde rusvejleder forsøgte igen at forhindre Rusmus i at komme til forelæsning.

\scene{I højreside er en ninja gået ind og holder et skilt, hvor der står Auditorium 10. O sætter en mærkat for 0’et, så der istedet står Auditorium 1. Ond latter fra O. O går ud i højre side, og R når hen til skiltet.}

\says{R} Næææææh, jeg har fundet det!

\scene{R stiller sig ud bag tæppet i højre side.}

\says{S} Men i virkeligheden var Rusmus endt i Auditorium 10, hvor en entusiastisk algebraiker stod og øvede sig.

\says{S} Og deraf kan vi konkludere, at $0+0=0$...

\says{R} Det forstår jeg altså ikke, det lyder alt for svært.

\says{S} Rusmus ledte og ledte og ledte og ledte og ledte.

\says{S} Men pludselig så Rusmus skiltet til Auditorium 1. Og han skyndte sig hen imod det.

\scene{R løber efter Auditorium 1 skiltet der flytter sig væk.}

\says{S (fortsat)} Men det var næsten, som om, at han ikke kunne komme tættere på skiltet.

\scene{Yakety Sax spiller. R Jagter N1 rundt på scenen i en cirkel. De løber rundt et par gange. Til sidst løber N1 ud i venstre side. N2 ind i højre side. R kigger rundt og opdager N2. R løber over mod N2, der går ud igen. N1 kommer ind igen i venstre side, og R løber over mod ham. Det fortsætter 2 gange. R stopper op i midten, og N1 og N2 står begge på scenen i hver deres side. R kigger forvirret rundt, og N3 kommer op ad lemmen. N1, N2 og N3 løber rundt på scenen, R jagter dem. N3 går ned gennem lemmen igen og lukker den. N1, N2 og R ender med at løbe rundt i en cirkel. Til sidst løber N1 og N2 ud over bordene blandt publikum. Musikken stopper. R sætter sig ned på scenen og græder.}

\says{R (grådkvalt)} Jeg finder aldrig Auditorium 1.

\says{S} Stakkels Rusmus sad og græd. Men lige pludselig kom den gode rusvejleder forbi.

\says{G} Hvad er der galt, lille rus?

\says{R} Jeg kan ikke finde hen til MatIntro-forelæsning.

\says{G} Men Auditorium 1 ligger jo lige der...

\scene{G peger på N1, der kommer ind midtfra. R når hen til bagtæppet.}

\says{F} ... Det var så første forelæsning i MatIntro. Husk, at vi næste gang er i Store UP1.

\scene{R kigger fortvivlet mod publikum. Lys ned.}
\end{sketch}

\end{document}
