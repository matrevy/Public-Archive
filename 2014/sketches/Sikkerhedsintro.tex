\documentclass[a4paper,11pt]{article}

\usepackage{revy}
\usepackage[utf8]{inputenc}
\usepackage[T1]{fontenc}
\usepackage[danish]{babel}
\usepackage{amssymb}

\revyname{MatematikRevy}
\revyyear{2014}
\version{1.0}
\eta{$1.5$ minutter}
\status{Udkast færdigt}

\title{Sikkerhedsintro}
\author{Jakob}

\begin{document}
\maketitle

\begin{roles}
\role{I}[NB] Instruktør
\role{A1}[Agent A.] Agent A
\role{A2}[Lasse] Agent B
\role{P}[Marianne] Præsident
\role{S}[Michael] Speaker
\end{roles}

\begin{props}
\prop{Podie}[Med en masse mikrofoner på.]
\end{props}
  
\begin{sketch}
\says{S} Det er med lige dele stolthed og ydmyghed i stemmen, at jeg kan fortælle jer,
at jeg har fået den utvivlsomme ære at skulle præsentere jer alle for
Præsidenten af Matematikrevyen.

\says{S} Af hensyn til Præsidentens sikkerhed vil vi dog gennemgå de vigtigste sikkerhedsforanstaltninger.

\scene{Mere seriøst.}

\says{S} Først og fremmest må der ikke være åben ild i salen, da sikkerhedspersonalet er
trænet til øjeblikkeligt at åbne ild - imod åben ild.

\says{S} Dernæst er det forbudt at have mobiltelefoner tændte, da signalerne fra disse ellers vil påvirke vor
sikkerhedssystemer i uhåndterbar grad.

\says{S} Til sidst bør I vide, at der er nødudgange der - og der \ldots

\scene{A1 og A2 markerer 3 udgange hver med netop en arm.}

\says{S (fortsat)} Skulle uheldet dog være ude bør i HOLDE JER FRA NØDUDGANGENE da disse er forbeholdt præsidenten.

\says{S} Lad mig med de ord liggende trygt bag mig sige: Giv en stor hånd til Præsidenten af Matematikrevyen.

\scene{Hail to the chief el. lign. spilles.}

\scene{Præcidenten kommer ind. Hun går op til publikum, og kysser en baby og evt. en GT.}

\scene{Præcidenten stille sig bag pulten, og slår på den første mikrofon.}

\scene{Hun slår på den anden mikrofon.}

\scene{Hun spiller en melodi på de to mikrofoner,}

\says{S} Hmm...

\says{P} Nå ja.

\says{P} God aften. Et revyår er gået, siden det sidst var et år siden at der sidst var matematikrevy. Matematikrevyen havde sin spæde start i året $2^2 \cdot 3 \cdot 167$. Nu skriver vi året $2 \cdot 19 \cdot 53$ og som enhver jo så kan regne ud er det 10 år siden matematikrevyen startede. 

\scene{Der kommer 5 røde laserpointerlys frem på P’s brystkasse under forrige replik. Da replikken er færdig opdager hun dem, og leger med dem.}

\scene{Agent A kaster sig over præcidenten og Agent B skyder vildt i alle retninger.}

\scene{P synes det er sjovt.}

\says{P} Tihi fnis.

\says{P (fortsat)} Jeg ville ønske jeg kunne sige at dette års russer var de bedste nogensinde. På samme måde ville jeg ønske jeg kunne sige at heltalsringen til en given legemsudvidelse af $\mathbb{Q}$ var eksplicit givet, men sådan er virkeligheden bare ikke.

\scene{Her bliver præsidenten træt af at kigge på sin tale, og smidder den væk.}

\scene{(fortsat) Når nu det er jubilæum, så er der en strid jeg synes vi skal tage hånd om. Det er en strid mellem to kæmper, og den har stået på for længe. Jeg tror på at vi vil stå stærkere sammen, i stedet hver for sig. Jeg sankker om striden mellem matematikkere og aktuarer. For at overbevise jer om at en forsoning er i alles interesse har jeg lavet en liste over nogle af de fordele der er ... ved at læse aktuar.}

\scene{Præsidenten griber en rulle i baglommen, og lader den løbe hele vejen ned på gulvet.}

\says{P (fortsat)} Opgjort i alfabetisk rækkefølge. 'A' for Arbejdsløshed lig 0, 'B' for Bedrevidende indenfor alt, 'C' for Creditrisk er sjovt, 'D' for Død er godt, 'E' for Estimationer en mass, 'F' fordi vi er bedst i fodbold: 10-1 VANDT VI I ÅR MOD MATEMATIK!!, 'G' for alle de grin vi har, ...


\scene{Lys ned.}
\end{sketch}

\end{document}
