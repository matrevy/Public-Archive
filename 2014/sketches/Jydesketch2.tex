\documentclass[a4paper,11pt]{article}

\usepackage{revy}
\usepackage[utf8]{inputenc}
\usepackage[T1]{fontenc}
\usepackage[danish]{babel}

\revyname{MatematikRevy}
\revyyear{2014}
\version{1.0}
\eta{$0.5$ minutter}
\status{Udkast færdigt}

\title{Jydesketch 2}
\author{Jakob og Lotte}

\begin{document}
\maketitle

\begin{roles}
\role{I}[Shake] Instruktør
\role{J}[Ann G] Jyde
\end{roles}

\begin{props}
\prop{Kommer senere...}[]
\end{props}

\begin{sketch}
\scene{Tavle, hvor der står udvalgsaksiomet.}

\says{J} Mojn. Idav skal vi så snak’ om det der hedder ud-valgs-ak-si-o-met - men lad os bar’ drop’ det ski’e kjøvenhavnerord -  vi ska snakk om at vælge ting. Hvis’ du har en stald. Og der’ gris’ i. Så ka’ du ta’ en gris og luk’ den u’. Hvis du har to stald'. Og der’ gris’ i dem beg’. Så ka’ du luk’ en gris ud af dem hver. Hvis du har tre stalde. Og der’ gris’ i dem alle sammen. Så ka’ du å’ luk’ en gris ud af hver af dem. Og så’n ka’ man bli’ ve'. Men hvis du kommer hær å si'r at du har fire stalde, eller derover. Så tror a et på dig. Det er løvn i din hals. Så ri’ er der kraft'e'me ikk’ nogen der er. (Alternativ: Så er jeg kraft’ed’me sur på dig, for så har du flere stalde end mig.) Og så’n er det.
\end{sketch}

\end{document}
