\documentclass[a4paper,11pt]{article}

\usepackage{revy}
\usepackage[utf8]{inputenc}
\usepackage[T1]{fontenc}
\usepackage[danish]{babel}

\revyname{MatematikRevy}
\revyyear{2014}
\version{1.0}
\eta{$1.5$ minutter}
\status{Udkast færdigt}

\title{Problemer på Tallinjen}
\author{Maling}

\begin{document}
\maketitle

\begin{roles}
\role{I}[NB] Instruktør
\role{T}[Sofie] Tal
\role{T}[William] Tal
\role{T}[Anna M] Tal
\role{T}[Alexander] Tal
\role{T}[Brandt] Tal
\role{T}[Marianne] Tal
\role{T}[Patrick] Tal
\role{T}[Michael] Tal
\role{T}[Ann G] Tal
\role{T}[Bettina] Tal
\role{T}[Kristian] Tal
\role{T}[Malthe] Tal
\role{T}[Marcus] Tal
\role{T}[Sanne] Tal
\role{T}[Jeppe] Tal
%\role{1} Naturligt tal
%\role{2} Naturligt tal
%\role{3} Naturligt tal
%\role{4} Naturligt tal
%\role{5} Naturligt tal
%\role{0} Nul
%\role{-1} Heltal
%\role{-2} Heltal
%\role{-3} Heltal
%\role{-4} Heltal
%\role{-3/2} Rationalt tal
%\role{3/2} Rationalt tal
%\role{e} Reelt tal
%\role{$\pi$} Reelt tal
%\role{i} Imaginært tal
\end{roles}

\begin{props}
\prop{Kommer senere...}[]
\end{props}
  
\begin{sketch}
\scene{$1$, $2$, $3$, $4$ og $5$ står på scenen ved bagtæppet, vipper stille og roligt med til musikken. $1$ står helt over ved bandscenen.}

\scene{$0$ kommer ind fra højre, kigger sig lidt omkring og går derefter hen og skubber $1$ væk, så den kan tage pladsen fra ham. De andre bliver lidt misfornøjede. $0$ vipper med til musikken, de andre er fortsat småirriteret.}

\scene{$-1$, $-2$, $-3$ og $-4$ kommer ind fra højre, løber alle hen til $0$ og skubber hele rækken til højre. 5 ryger af scenen, $1$, $2$, $3$ og $4$ er irriterede, $0$ er forvirret og $-1$, $-2$, $-3$, $-4$ danser til mere aktivt med til musikken.}

\scene{To personer kommer ind arm i arm, de har \emph{et} skilt der viser $00$. De opdager tallinjen, bliver glade indtil de opdager at der allerede er et $0$. De bliver triste, men kigger så på hinanden og giver slip, ændrer deres skilte til $\frac{3}{2}$ og $-\frac{3}{2}$ og stiller sig ind på deres plads. Alle står nu meget klemt og det går dem tydeligvis på, men de danser vipper lidt videre.}

\scene{To personer med lidt for meget attitude kommer nu ind på scenen, $e$ og $\pi$. De går direkte hen på deres pladser og skubber folk væk, $4$ og $-4$ ryger af scenen. De står lidt tid sådan.}

\scene{Sketchen sluttes af med at $i$ kommer ind på scenen og ser der ikke er plads overhovedet, så han begynder at tænke, hvorefter han skynder sig om bag scenen og kravler op så man kan se ham bag bagtæppet over $0$. Sangen slutter.}
\end{sketch}

\end{document}
