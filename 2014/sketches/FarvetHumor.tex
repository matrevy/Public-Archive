\documentclass[a4paper,11pt]{article}

\usepackage{revy}
\usepackage[utf8]{inputenc}
\usepackage[T1]{fontenc}
\usepackage[danish]{babel}

\revyname{MatematikRevy}
\revyyear{2014}
\version{1.0}
\eta{$3$ minutter}
\status{Udkast færdigt}

\title{Farvet Humor}
\author{Jasmin}

\begin{document}
\maketitle

\begin{roles}
\role{I}[aDA!] Instruktør
\role{B}[Malthe] Manusboss
\role{M1}[Anna M] Rationelt menneske
\role{M2}[Alexander] Opgivende og sur
\role{M3}[Christine] Komplet idiot
\role{M4}[Marianne] Kagespisende
\end{roles}

\begin{props}
\prop{1 bord, 5 stole, en slinky, kage i flere halvtomme fade - eller andet der kan give indtrykket af at der har været meget kage.}[]
\end{props}
  
\begin{sketch}
\scene{Fire manusgruppemedlemmer til (M1, M2, M3, M4) et manusmøde. Der er kage. M1 og M2 sidder ved bordet. M3 sidder på gulvet henne mod udgang og leger med slinky. M4 sniger sig til kage og står henne mod bandet. M1 sidder og kigger på nogle papirer, og nikker og streger ting ud og retter. M2 sidder med armene over kors og kigger hele tiden på sit ur.}

\says{M2} Ej, det her, det der, det er simpelthen for dumt, det er værre end at smide en enhjørning på scenen i et boybandnummer. Hvad fanden tænkte forfatteren på?!?!

\says{M3} Nå, gud, var det en enhjørning, jeg troede det var et æsel!!

\scene{Manusboss (B) kommer farende ind på scenen med et officielt udseende brev i hånden.}

\says{B} Ja altså, nej, undskyld jeg kommer for sent, jeg ved godt at det var mig der indkaldte til manusmøde, men den er simpelthen helt gal!!

\says{M1} Hvad så boss?

\says{B} Jo, jeg har i dag modtaget et brev fra rektor på KU, og nu skal I bare høre...

\scene{Læser op.}

\says{B (fortsat)} Det er blevet os bevidst, at Matematikrevyen udleder alt for meget sort humor. Det må derfor nødvendigvis påkræves, at Matematikrevyen finder mere miljøvenlige udveje og mere grøn humor.

\says{M3} Som i træhumor?!?!

\says{M2} Ej, det kan da ikke være rigtigt at vi skal droppe sort humor! Hvad skal vi så gøre? Alt det her er jo ubrugeligt, vi når ikke at blive klar til deadline...

\scene{River nogle af papirerne på bordet over og smider dem væk, nogle af papirerne bliver kastet over på M5, der i smug prøver at samle de iturevne papirer og putte dem i lommerne.}

\says{M1} Men vi kan vel godt prøve at finde på alternativer? Hvad med at bruge genbrugshumor - sådan ligesom SaTyR?

\says{B} Jo da, det kunne man vel godt, men nej, altså, ihh....

\says{M1} Eller måske 2. generations biorevyhumor?

\scene{M4 tager et nyt stykke kage.}

\says{M2} Ej, det er simpelthen for dumt, matematikrevyen er intet uden sort humor.

\says{B} ......Du har måske ret. Vi må kæmpe for det her.

\scene{Rejser sig op.}

\says{B (fortsat)} Lad os ikke svælge i dalen af fortvivlelse, mine kære manusgruppe. Selvom vi står ansigt til ansigt med vanskelighederne af i dag og i morgen, så har jeg stadig en drøm. Det er en drøm med rødder dybt nede i revyernes drøm. Jeg har en drøm om, at revyerne en dag vil rejse sig og udleve den sande trosbekendelse: Det er en selvfølge, at al humor er skabt lige.

\scene{Her har M4 spist al kagen, og sidder og roder rundt for at prøve at finde de sidste krummer.}

\says{B (fortsat)} Jeg har en drøm om, at man har frihed til at lave alle de jokes, man føler for, at vi kan leve i et land...

\says{M4} Er der mere kage?

\says{M3} Shh, bossen snakker!!!!

\scene{M1 har en banan med i sin taske til M4}

\says{M3 (fortsat)} Det kan være vigtigt!!

\says{B} ....Ahem, i et land, hvor tilhængere af alle slags humor kan sidde sammen i et rum, og at mine fire små børn kan leve i et land, hvor en joke ikke vil blive dømt på hvor meget mening den giver, men på hvor mange grin den får. Jeg har en DRØM i dag!

\scene{Revysterne er begejstrede! Lys ned. Skub til træet begynder! Spot på træ, der kommer ind fra bagtæppet.}
\end{sketch}

\end{document}
