\documentclass[a4paper,11pt]{article}

\usepackage{revy}
\usepackage[utf8]{inputenc}
\usepackage[T1]{fontenc}
\usepackage[danish]{babel}

\revyname{MatematikRevy}
\revyyear{2014}
\version{1.0}
\eta{$0.5$ minutter}
\status{Udkast færdigt}

\title{Jydesketch 3}
\author{MaWeK og Shake}

\begin{document}
\maketitle

\begin{roles}
\role{I}[Shake] Instruktør
\role{J}[Ann G] Jyde
\end{roles}

\begin{props}
\prop{Kommer senere...}[]
\end{props}

\begin{sketch}
\says{J} Mojn. Ja, så står vi her sgu igen. Og idav skal vi så snakke om længder og vinkler i en trekant, igå? Hvis du har en trekant. Og du kender den ene sid’. Og du kender så og’ en anden sid’. Og så ved du oven i det, at de hersens to sider, de laver så'n ret vinkel ved hinanden igå. Lig'som hjørnern' i en god grisestald.  Jamen, så ved du jo og’ hvad den sidste side er... For du måler den. Og så’n er det.
\end{sketch}

\end{document}
