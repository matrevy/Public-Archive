\documentclass[a4paper,11pt]{article}

\usepackage{revy}
\usepackage[utf8]{inputenc}
\usepackage[T1]{fontenc}
\usepackage[danish]{babel}

\revyname{MatematikRevy}
\revyyear{2014}
\version{1.0}
\eta{$3$ minutter}
\status{Udkast færdigt}

\title{Revyforberedelse}
\author{Jakob og Alexander}

\begin{document}
\maketitle

\begin{roles}
\role{I}[aDA!] Instruktør
\role{R1}[Marianne] Revyst
\role{R2}[Alexander] Revyst
\role{R3}[Lise] Revyst
\role{OR}[Lasse] Overrevyst
\end{roles}

\begin{props}
\prop{Kommer senere...}[]
\end{props}

\begin{sketch}
\scene{Lys op. R1 og R2 står på scenen og laver englehop.}

\says{OR} 497, 498, 499, 500! Godt klaret revyaspiranter. Så tager i 500 mavebøjninger!

\scene{R2 Begynder at tage mavebøjninger og bliver ved mens de to andre taler.}

\says{R1} Du Lasse... skal vi ikke snart...

\says{OR (afbryder og retter)} Husk! Jeg er Scenens Betvinger, Hr. Overrevyst i revysammenhæng

\says{R1 (skeptisk)} Så Hr. Overrevyst, hvorfor er det egentlig vi laver alt det her?

\says{OR} Det er jo ikke bare hvem som helst der kan være revyst! Man skal være forberedt på at skulle bære et pumpeorgel. Sådan et er skide tungt har jeg hørt.

\says{R1} Kunne vi ikke lave noget der var bare en lille smule mere relevant?

\says{OR (dramatisk)} Så du tror i er klar til at lave nogle rigtige revyøvelser?

\says{R1 (tørt)} Ja...

\says{OR} Så gør jer klar, for nu skal i...

\scene{OR bemærker at R2 stadig ligger og tager mavebøjninger.}

\says{OR (fortsat)} Henning hvad i helvede har du gang i?

\says{R2} Jam... jamen, du sagde at vi skulle tage 500 mavebøjninger.

\says{OR} Det er jo fuldstændig latterligt, sådan noget kunne jeg jo aldrig finde på. Rejs dig nu op. Har du dit kostume?

\says{R2} Ja boss. Det er lige her.

\scene{Hiver et kostume frem fra bag- eller sidetæppet.}

\says{OR (meget dramatisk)} Så gør jer klar, for nu skal i øve: Kostumeskift på tid!...

\says{R1 (indskyder)} Seriøst?

\says{OR (forsat)} ...Gør jeres kostumer klar. 3-2-1 Begynd!

\scene{R1 går om bag scenetæppet og trækker det for. R3 kommer ind med flot gallatøj, samme frisøre som R1, men ellers tydeligt en anden person. Henning løber ud og kommer ind igen med et sæt bukser på hovedet eller noget lignende.}

\says{OR} Flot rus det er sådan et skift skal se ud. Henning, du var rigtig hurtig, men jeg må desværre fortælle dig at bukser ikke skal tages på hovedet. Jeres næste øvelse er et af de vigtigeste evner en revyst kan beside, kunsten at blive... usynlig.

\says{R3} Usynlig? Er sceneninjaerne ikke normalt klædt i sort?

\says{OR} Det sorte tøj er kun første skridt! Den fuldendte revyst er i stand til at gå fuldstændig i et med sine omgivelser, uanset påklædning. Nu lukker jer øjnene og tæller ned. Og når jeg så er færdig så er i blevet fuldstændig usynlige.

\scene{OR lukker øjnene og begynder at tælle ned. R3 bander af OR og stiller sig ud bag et af scenetæpperne. R2 gemmer sit hoved, muligvis ved at åbne scenelemmen og stikke sit hoved ned i hullet. OR åbner øjnene og kigger rundt.}

\says{OR} Ja det er smukt i går fuldstændig i et med omgivelserne. Jeg kan overhovedet ikke se jer.

\scene{OR opdager R2 og går over til ham. Mens OR og R2 snakker går R1 tilbage ind på scenen og stiller sig bag ved OR, så OR ikke kan se ham.}

\says{OR (opgivende)} Henning, hvad laver du?

\says{R2} Jeg er jo usynlig, ig’ås boss? Jeg så et program på Animal Planet om strudser. Jeg aflurede deres strategi chef. De stikker bare hovedet ned i jorden, så bliver de helt usynlige.

\says{OR} Henning, Hvis du kunne se den, var den så usynlig?

\says{R2} Det selvfølgelig en rigtig god pointe boss.

\says{OR} Henning forhelvede... Hvor blev ham russen af?

\scene{OR vender sig rundt og opdager at R3 står lige bag ham. OR bliver tydeligt forskrækket.}

\says{R1 (sarkastisk)} Undskyld! Det var ikke min mening at afbryde jeres ellers så betagende zoologiske samtale, men nu har vi været igang i 4 timer, har vi ikke snart lavet nok åndssvage øvelser til at være med i revyen?

\says{OR} Nej, på ingen mulig måde.

\scene{OR vender ryggen til R1 og R2. R1 skal til at hoppe på OR, men bliver stoppe af R2.}

\says{OR (fortsat)} Jeg tror vi skal prøve noget mere grundlæggende. Et af de vigtigste tricks i en revysts arsenal er exittet fra scenen. Rus, jeg synes du skal starte.

\says{R1} Med glæde.

\scene{R1 går om bag bagtæppet og videre ud i salen.}

\says{OR} Rus, hvor er du på vej hen?

\says{R1 (råber)} Jeg går!

\says{OR} Du kan da ikke bare gå.

\says{R1} Og dog, er jeg i fuld gang!

\says{OR (en lille snert af desperation)} Kan vi ikke i det mindste snakke om det?

\says{R1} Der er ikke noget at snakke om. Jeg tager på Caféen?.

\says{OR (desperat)} Jamen hvem skal så lave revy?

\says{R1} Du kan selv lave din åndssvage revy.

\scene{R1 smækker sidedøren. OR kigger lidt forvirret rundt og vender sig så mod R2.}

\says{OR (højtideligt)} Henning, du er den eneste der har klaret alle prøvelserne. Du har gjort mig stolt.

\says{R2} Er jeg så revyst nu?

\says{OR (grandiøst)} Ja. Sammen vil vi lede revyen ind i en gylden periode, vis lige aldrig før er set.

\scene{OR lægger kammeratligt en arm omkring R2.}

\says{OR (fortsat)} Henning min ven, jeg tror vi går store ting imøde.

\says{R2} Du boss, jeg har egentligt et spørgsmål.

\says{OR} Ja Henning?

\says{R2} Hvad er en revy?

\says{OR (råber, vredt)} HENNING FORHELVEDE!

\scene{Lys ned.}
\end{sketch}

\end{document}
