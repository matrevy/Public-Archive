\documentclass[a4paper,11pt]{article}

\usepackage{revy}
\usepackage[utf8]{inputenc}
\usepackage[T1]{fontenc}
\usepackage[danish]{babel}

\revyname{MatematikRevy}
\revyyear{2014}
\version{1.0}
\eta{$3.5$ minutter}
\status{Udkast færdigt}

\title{Matematiksangen}
\author{Jasmin og Rasmus}
\melody{Svedbanken $-$ Rollespils sangen}

\begin{document}
\maketitle

\begin{roles}
\role{I}[Jasper] Instruktør
\role{S}[Patrick] Sanger
\role{D1}[Christine] Person der driller 1
\role{D2}[Johanna] Person der driller 2
\end{roles}

\begin{props}
\prop{Kommer senere...}[]
\end{props}

\begin{song}
\sings{S} Det er ikke sjovt, så’n at træng’ til kaffe
Og være løbet tør
Hvor ku' jeg godt drik' en god kop kaffe
Men ingen kaffe til mig

Mine briller er grå og store
Og min ryg er krum
Jeg har ik’ hørt om ham der Gustav
Så alle synes jeg er dum


Men når jeg’ så regner en uend'lig sum
Så bliver enhver der driller pluds'lig stum


For jeg læser mat’matik her på HCØ
Jeg vil ikk’ holde op, så heller' dø
Med ortonormale følger i Hilbertrum
Får alle andre klø

Det er ikke sjov at folk syn’s jeg lugter
og siger min sang er grum
Folk forstår ikk’ de sjove ordspil
en aftenringnigs skum ?

\sings{S} Men når jeg’ så regner en uend'lig sum
Så bliver enhver der driller pluds'lig stum


Ja jeg læser mat’matik, mens jeg spiser kag’
Du må kun være med, hvis du ka’ bag’
(?) Og jeg bruger theoremer, de viser vej
(?) - så vis respekt for mig


Kom nu, bare en lille smule respekt
Lyt nu - fordomme væk - se mig!
Kom nu!


For jeg læser mat’matik her på HCØ
Jeg vil ikk’ væk herfra, så hel’re dø
Jeg læser mat’matik,
Læs med mig, vær med, vær med, vær med


Jeg læser mat’matik!
Kom nu og skriv med kridt!! Skriv det! Skriv det! Kom med og skriv….
\end{song}

\end{document}
