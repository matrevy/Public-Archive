\documentclass[a4paper,11pt]{article}

\usepackage{revy}
\usepackage[utf8]{inputenc}
\usepackage[T1]{fontenc}
\usepackage[danish]{babel}

\revyname{MatematikRevy}
\revyyear{2014}
% HUSK AT OPDATERE VERSIONSNUMMER
\version{2.0}
\eta{$3.5$ minutter}
\status{Udkast færdigt}

\title{Primtal}
\author{Jasmin, Jakob, Shake og Rasmus}
\melody{Robbie Williams $-$ Candy}

\begin{document}
\maketitle

\begin{roles}
\role{I}[aDA!] Instruktør
\role{K}[Jasmin] Koreograf
\role{S}[MaWeK] Sanger
\role{P}[Marcus] Person
\role{K1}[Bettina] Kor
\role{K2}[Lise] Kor
\role{D1}[Laerke] Danser
\role{D2}[Jasmin] Danser
\end{roles}

\begin{props}
\prop{6 stole}[]
\prop{Tavle}[]
\prop{Øl}[]
\prop{2 vimpler}[]
\prop{$p$ skilte}[]
\prop{Meget stort Dankort(?)}[]
\end{props}

\begin{sketch}
\scene{Alle sidder på stole i en halvcirkel, hygger med øl og spiller Buffalo!}

\says{K2 (peger på K1)} 18!

\says{K1 (peger på S)} 19!

\says{S (peger på P)} 20!

\says{P (peger på S)} BUFFALO!

\says{S} Skål.

\scene{S bunder. Alle jubler!}

\says{S (fortsat)} Okay, okay... ny regel: "Man skal sige 'BUM' på alle primtal!"

\says{P} Ej, ikke en primtalsregel nu igen! Jeg hader primtal!

\scene{Alle pånær P er chokeret!}

\says{S} Ej, du kan da ikke hade primtal!... Ved du hvad, vi skal nok lære dig at elske primtal!... Band - HIT IT!

\scene{Musikken til Primtal begynder. Alle pånær P skifter til deres faverige tøj. Alle stole pånær én bæres ud. P som er frustreret og opgivende bliver sat på den sidste stol i siden af scenen.}

\says{P} Åh nej, ikke nu igen... (det sker hver gang jeg spiller med matematikere.)
\end{sketch}

\begin{song}
\sings{S} Lad os definere
De tal der fascinerer
For primtal skal der gælde
Faktorer trivielle
Elegant og simpelt, det kan man let forstå
Men trods det er det fulde billed’
Umuligt at opnå
Men hør nu
 
Euler og Euklid de viste
Blandt primtal er der slet intet sidste
Givet $n$ kan primtal skrives
Helt entydigt, vi har uniqueness
 
\sings{S+Kor} Skriv nu $p$ og $q$
Der’ så meg’t der ikk’ er vist endnu
Uden dem går teori itu
For alt er smukt ved primtal
Skriv nu $p$ og $q$
De er overalt, det vides jo
Også selvom det er svært at tro
For alt er smukt ved primtal

\sings{S} Deres heltalsringe
Dem kan man let frembringe
For vilkårligt $p$
Er der meget at indse
Man kan se tendenser
Og smukke kongruenser
Perfekt at anerkende
Forbundet med Mersenne

Goldbach havde en formodning
Og kryptologer brug’r dem i kodning
Dankortkøb forløber sikkert fordi
Vi slipper primtal fri
Hvad gjord’ vi uden dem?
Så kom nu!!

\sings{S+Kor} Skriv nu $p$ og $q$
Der’ så meg’t der ikk’ er vist endnu
Uden dem går teori itu
For alt er smukt ved primtal
Skriv nu $p$ og $q$
De er overalt, det vides jo
Også selvom det er svært at tro
For alt er smukt ved primtal

\sings{S} $\mbox{Pi}(x)$ er asymptotisk
Distribueringen dog ej logisk
Sylows sætning hjælper os til at se
Grupper af orden $p$
Hvad gjord’ vi uden dem?
Hvad gjord’ vi uden dem?
Hvad gjord’ vi uden dem?
Hvad gjord’ vi uden dem?
\sings{S+Kor} Hvad gjord’ vi uden dem?
Hvad gjord’ vi uden dem?
Hvad gjord’ vi uden dem?
Hvad gjord’ vi uden dem?
Hvad gjord’ vi uden dem?

\sings{S+Kor} Skriv nu $p$ og $q$
Der’ så meg’t der ikk’ er vist endnu
Uden dem går teori itu
For alt er smukt ved primtal
Skriv nu $p$ og $q$
De er overalt, det vides jo
Også selvom det er svært at tro
For alt er smukt ved primtal

\sings{S+Kor} Skriv nu $p$ og $q$
Der’ så meg’t der ikk’ er vist endnu
Uden dem går teori itu
For alt er smukt ved primtal
Skriv nu $p$ og $q$
De er overalt, det vides jo
Også selvom det er svært at tro
For alt er smukt ved primtal
\end{song}

\end{document}
